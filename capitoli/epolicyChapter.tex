%Andrea Borghesi
%Università degli studi di Bologna

%capitolo dedicato alla descrizione del progetto e-policy e ai meccanismi incentivanti 

\documentclass[12pt,a4paper,openright,twoside]{report}
\usepackage[italian]{babel}
\usepackage{indentfirst}
\usepackage[utf8]{inputenc}
\usepackage[T1]{fontenc}
\usepackage{fancyhdr}
\usepackage{graphicx}
\usepackage{titlesec,blindtext, color}
\usepackage[font={small,it}]{caption}
\usepackage{subfig}
\usepackage{listings}
\usepackage{color}
\usepackage{url}
\usepackage{textcomp}

%impostazioni generali per visualizzare codice
\definecolor{dkgreen}{rgb}{0,0.6,0}
\definecolor{gray}{rgb}{0.5,0.5,0.5}
\definecolor{mauve}{rgb}{0.58,0,0.82}
 
\lstset{ %
  basicstyle=\footnotesize,           % the size of the fonts that are used for the code
  backgroundcolor=\color{white},      % choose the background color. You must add \usepackage{color}
  numbers=left,                   % where to put the line-numbers
  numberstyle=\tiny\color{gray},  % the style that is used for the line-numbers
  numbersep=5pt,  
  showspaces=false,               % show spaces adding particular underscores
  showstringspaces=false,         % underline spaces within strings
  showtabs=false,                 % show tabs within strings adding particular underscores
  rulecolor=\color{black}, 
  tabsize=2,                      % sets default tabsize to 2 spaces
  breaklines=true,                % sets automatic line breaking
  breakatwhitespace=false,        % sets if automatic breaks should only happen at whitespace
  title=\lstname,                   % show the filename of files included with \lstinputlisting;
  frame=single,                   % adds a frame around the code
                                  % also try caption instead of title
  keywordstyle=\color{blue},          % keyword style
  commentstyle=\color{dkgreen},       % comment style
  stringstyle=\color{mauve},         % string literal style
  escapeinside={\%*}{*)},            % if you want to add LaTeX within your code
  morekeywords={*,...},              % if you want to add more keywords to the set
  deletekeywords={...}              % if you want to delete keywords from the given language
}

%per avere un bordo intorno alle figure
\usepackage{float}
\floatstyle{boxed} 
\restylefloat{figure}

%per poter poi impedire che certe parole vadano a capo
\usepackage{hyphenat}
\usepackage{listings}

%ridefinisco font per fancyhdr, per ottenere un'intestazione pulita
\newcommand{\changefont}{ \fontsize{9}{11}\selectfont }
\fancyhf{}
\fancyhead[LE,RO]{\changefont \slshape \rightmark} 	%section
\fancyhead[RE,LO]{\changefont \slshape \leftmark}	%chapter
\fancyfoot[C]{\changefont \thepage}					%footer

%titolo capitolo con "numero | titolo"
\definecolor{gray75}{gray}{0.75}
\newcommand{\hsp}{\hspace{20pt}}
\titleformat{\chapter}[hang]{\Huge\bfseries}{\thechapter\hsp\textcolor{gray75}{|}\hsp}{0pt}{\Huge\bfseries}


\oddsidemargin=30pt \evensidemargin=20pt

%sillabazioni non eseguite correttamente
\hyphenation{sil-la-ba-zio-ne pa-ren-te-si si-mu-la-to-re ge-ne-ra-re pia-no}

%interlinea
\linespread{1.15}  
\pagestyle{fancy}

%cartelle contenenti le immagini
\graphicspath{{/media/sda4/tesi/immagini/grafici/}{/media/sda4/tesi/immagini/grafici/incCompare/}{/media/sda4/tesi/immagini/grafici/rawData/}{/media/sda4/tesi/immagini/grafici/regressionAnalysis/}{/media/sda4/tesi/immagini/schemi/}{/media/sda4/tesi/immagini/simulazione/}{/media/sda4/tesi/immagini/epolicy/}}

%in modo che dopo il titolo di un paragrafo il testo vada a capo
\newcommand{\myparagraph}[1]{\paragraph{#1}\mbox{}\\}

\begin{document}
\chapter{\nohyphens{QUADRO GENERALE}}

I problemi delle politiche pubbliche sono estremamente complessi, avvengono in ambienti che cambiano rapidamente caratterizzati da incertezza e coinvolgono conflitti tra diversi interessi. La nostra società è sempre più complessa a causa della globalizzazione, ampliamento e cambiamento delle situazioni geopolitiche. Questo implica che l'attività politica e la sua area di intervento si siano estese, rendendo più difficili da determinare gli effetti di tali interventi, mentre al tempo stesso diventa sempre più importante assicurarsi che le azioni intraprese affrontino in maniera efficacie le sfide reali che la crescente complessità comporta.\\*
Da questo consegue che coloro responsabili di creare, implementare e far rispettare le politiche devono essere in grado di giungere a delle decisioni nel caso di problemi mal definiti e non pienamente compresi, senza una singola risposta corretta, che coinvolgono diversi interessi in competizione e interagiscono con altre politiche su multipli livelli.; è quindi necessario trattare con coerenza tali problematiche e ricercare tecniche, metodologie e strumenti per affrontare la complessità in questo settore.\\*\\*
Con questo scopo in mente è stato ideato il progetto ePolicy che verrà ora introdotto; nel resto del capitolo verranno quindi fornite una descrizione di questo progetto in termini generali, seguito dalla presentazione del caso di studio con cui si è deciso di testare le tecniche sviluppate - la regione Emilia Romagna - e per passare infine a descrivere le strategie implementative adottabili per la messa in atto delle politiche studiate.


\section[E-POLICY]{PROGETTO E-POLICY}

Il progetto europeo \emph{ePolicy} (dall'inglese Engineering the Policy Making Life Cycle, cioè ingegnerizzare il processo di creazione delle politiche) ha come obiettivo la creazione di un sistema di supporto alle decisioni per la pianificazione regionale e la valutazione degli impatti sociali, economici e ambientali. Con l'espressione \emph{sistema di supporto alle decisioni} (a cui in seguito ci riferiremo anche utilizzando l'acronimo \emph{DSS}, dall'inglese Decision Support System) si intende una classe molto ampia di sistemi software che hanno come scopo aiutare a prendere decisioni in caso di gestione di problemi complessi, facilitando l'analisi di grandi quantità di dati e suggerendo strategie e  politiche da adottare.\\*
Avviato nell'Ottobre del 2011, il progetto è coordinato dall'Università di Bologna e coinvolge nove partner tra mondo dell'accademia e della ricerca, governi regionali e settore privato, distribuiti in cinque paesi diversi dell'Unione Europea.\\*\\*
I decisori politici devono prendere decisioni complesse valutando un notevole numero di variabili e vincoli, tenendo conto quindi degli impatti che le loro scelte avranno su diversi aspetti ambientali, economici e sociali. Al tempo stesso, si è osservata negli anni un sempre crescente desiderio da parte dei cittadini di contribuire alla creazione delle politiche attraverso mezzi come i social network e i blog.\\*\\*
L'intenzione del progetto, una volta concluso, è quella di permettere a coloro che effettuano le decisioni di disporre di un sistema integrato e user-friendly, in grado di creare e valutare piani alternativi altamente ottimizzati tra i quali poter scegliere sulla base di una dettagliata analisi dei costi e benefici degli stessi.\\*\\*
Oltre a esaminare gli aspetti teorici, il progetto ePolicy mira a applicare i suoi risultati a un caso pratico: la pianificazione energetica nella regione Emilia Romagna. In particolare, il governo regionale si è posto l'obiettivo di incrementare la produzione di energia da fonti rinnovabili, concentrandosi soprattutto sulle tecnologie fotovoltaiche (PV) e a biomassa. Di conseguenza, ePolicy punta a sviluppare un modello che fornirà supporto ai decisori politici della regione che stanno cercando di mettere in pratica il miglior meccanismo incentivante per stimolare la crescita della produzione energetica da alcune tecnologie rinnovabili.

\begin{figure}[hbt]
	\centering
	\includegraphics[scale=0.55]{epolicyLifeCycle}
	\caption{Processo di decisione delle Politiche}
	\label{epolicyLifeCycle}
\end{figure}

In Figura ~\ref{epolicyLifeCycle} osserviamo in che modo sia strutturato il ciclo di vita del processo di creazione delle politiche all'interno del progetto ePolicy: \begin{itemize}
\item il livello di ottimizzazione globale, che prende in considerazione gli obiettivi, gli aspetti finanziari e gli impatti socio-ambientali su larga scala (produce dei piani e degli scenari per le politiche);
\item il livello individuale delle simulazioni ad agenti, con il quale si intendono simulare  comportamenti sociali riguardanti le nuove politiche sulla base delle opinioni e desideri personali (per ottenere le strategie implementative);
\item  l'integrazione tra la prospettiva globale e quella individuale, ad esempio con tecniche mutuate dalla teoria dei giochi;
\item l'individuazione degli impatti sociali e le reazioni delle persone attraverso l'uso di tecniche di opinion mining , cioè estrazione delle opinioni, con i dati raccolti in rete (servendosi di blog, forum, social network,...);
\item la visualizzazione dei risultati attraverso strumenti appositamente ideati per aiutare i decisori politici.
\end{itemize}


\begin{figure}[hbt]
	\centering
	\includegraphics[scale=0.55]{epolicyScheme}
	\caption{Schema generale del sistema}
	\label{epolicyScheme}
\end{figure}

In Figura ~\ref{epolicyScheme} è mostrato lo schema generale del progetto ePolicy. Si possono osservare le varie componenti del sistema come l'ottimizzatore che lavora a livello globale o il simulatore per il livello individuale e le interazioni tra loro e con gli utenti, ovvero i decisori politici che specificano vincoli, obiettivi e impatti e i cittadini dai quali ottenere informazioni per poter meglio pianificare (ex ante opinion mining) e osservarne le reazioni alle strategie implementative (ex post opinion mining).\\*\\*
Un aspetto importante da tenere in considerazione per fornire supporto ai decisori politici è la definizione formale dei modelli delle politiche. In letteratura la maggioranza dei modelli politici è basata su simulazioni ad agenti \cite{AgentBaseLandUseModel,Nigel,socialScienceMicrosim} dove gli agenti rappresentano le parti coinvolte nel processo decisionale e implementativo. L'idea è che modelli ad agenti e relative simulazioni siano adatti per sistemi complessi. In particolare, questi modelli permettono di effettuare esperimenti computazionali per garantire una migliore comprensione della complessità dei sistemi economici, sociali e ambientali, cambiamenti strutturali e adattamenti reattivi endogeni in riposta ai cambi di politiche.
\\*\\*
Per riassumere, i principali obiettivi del progetto ePolicy sono i seguenti:
\begin{itemize}
\item supportare i decisori politici nel loro lavoro, ovvero uno sforzo multidisciplinare mirato a ingegnerizzare il ciclo di vita del processo di creazione delle politiche;
\item integrare le prospettive globale e individuale all'interno del processo decisionale;
\item valutare gli impatti sociali, economici e ambientali durante lo sviluppo delle politiche (sia a livello globale che individuale);
\item stabilire i probabili effetti sociali attraverso opinion mining;
\item aiutare tutti coloro che sono coinvolti nei processi decisionali e i cittadini interessati con degli strumenti di visualizzazione efficaci.
\end{itemize}
Una volta realizzati questi obiettivi, è possibile aspettarsi alcuni benefici sociali ed economici, tra i quali una migliore previsione degli impatti delle politiche attuate in grado di condurre a una più efficiente implementazione delle politiche regionali e migliore identificazione degli effetti positivi per cittadini e imprese; o ancora, un aumentato impegno dei cittadini e un più ampio uso degli strumenti informatici e di telecomunicazione (ITC), che possono risultare in iterazioni innovative tra cittadini e governi. In secondo luogo si punta a ottenere una maggiore trasparenza delle informazioni sull'impatto delle decisioni economiche sulla società e una migliorate capacità di reagire alle principali sfide poste alla società e maggiore fiducia pubblica verso le attività governative e burocratiche.


\section{PIANIFICAZIONE REGIONALE}

Il caso di studio scelto per sperimentare le metodologie sviluppate con il progetto ePolicy è la creazione del Piano Regionale dell'Energia per la Regione Emilia Romagna (d'ora in poi abbreviata anche con l'acronimo RER).\\*\\*
La pianificazione regionale è lo studio della disposizione efficiente delle attività e delle infrastrutture territoriali per una crescita sostenibile della regione. I piani regionali sono classificati in base all'ambito che considerano, come ad esempio Agricolo, Forestale, Energia, Industria, Trasporti, Risorse Idriche, Urbano, Ambientale, etc.\\*
Nonostante i diversi piani differiscano per obiettivi e tipo di attività, essi condividono alcune caratteristiche comuni che consentono un trattamento uniforme in termini di requisiti per un sistema di supporto alle decisioni.\\*
A grandi linee, i piani regionali sono organizzati secondo quanto segue: \begin{itemize}
\item analisi della situazione e dei piani precedenti, nella quale vengono considerati aspetti sociali, economici e ambientali e i risultati degli strumenti implementati in passato sono identificati e valutati;
\item obiettivi e strategie, possono essere derivati dalle linee guida europee o nazionali, leggi e norme esistenti, opinioni dai cittadini, specifiche necessità regionali;
\item priorità e linee di intervento, la parte decisionale del piano durante la quale vengono allocate le risorse mirata a soddisfare gli obiettivi rispettando determinati vincoli;
\item implementazione e monitoraggio, definendo strumenti che possono essere economici, come tasse o sussidi , regolatori, cooperativi, ad esempio accordi volontari o tra produttori e consumatori, informativi, come campagne informative e pubblicitarie o trasferimenti tecnologici.   
\end{itemize}
L'approccio di ePolicy un piano consiste in un insieme di attività che dovrebbero essere effettuate per raggiungere certi obiettivi. Con il fine di aiutare durante la pianificazione, la modellazione delle politiche deve tenere conto di alcuni aspetti, descritti in modo più esteso nei prossimi paragrafi e capitoli. Innanzitutto ogni piano presenta un certo numero di differenti obiettivi (anche diversi a seconda dell'aspetto del funzionamento della regione che affrontano); durante la creazione di un piano, essi devono essere tenuti contemporaneamente in considerazione, compito non semplice poiché potrebbero essere in conflitto tra loro. In secondo luogo, l'implementazione di un piano è limitata da un insieme di vincoli finanziari ed economici, tipicamente espressi nei termini dei fondi disponibili e dei costi privati stimati. Ancora, gli effetti positivi o negativi in grado di influenzare aspetti sociali o ambientali devono essere considerati durante la pianificazione. Infine, un'altra attività fondamentale per la creazione di un piano è la definizione di strategie implementative, cioè i meccanismi usati per portare a compimento le attività previste, i quali hanno ovviamente un impatto sulle possibilità di conseguire gli scopi prefissati.

\subsection[VINCOLI FINANZIARI]{\nohyphens{VINCOLI FINANZIARI}}
La realizzazione di un piano comprende due tipi di costi: costi pubblici, ad esempio quelli sostenuti dagli enti regionali, e costi privati, come i cittadini interessati e coinvolti in qualche attività relativa alla pianificazione.\\*
I costi pubblici sono in genere coperti dal budget allocato per l'implementazione del piano;  in Emilia Romagna i fondi sono allocati tramite il Programma Operativo Regionale \cite{POR} (POR), parzialmente finanziato dall'Unione Europea. Le strategie del Programma Operativo sono basate principalmente sulle direttive regionali, identificate previa analisi delle potenzialità della regione, e i contesti strategici nazionali ed europei, che stabiliscono i principi per l'erogazione dei fondi provenienti dalla Comunità Europea.\\*Il Programma è diviso in cinque priorità: ricerca industriale e trasferimento tecnologico, sviluppo imprenditoriale e innovazione, sviluppo sostenibile e miglioramento dell'efficienza ambientale ed energetica, maggior sfruttamento dell'eredità ambientale e culturale, assistenza tecnica.\\*\\*
I vincoli fiscali derivanti dall'allocazione dei fondi sono modellati all'interno dell'approccio di ePolicy per mezzo di vincoli sui costi previsti, calcolati sulla base delle attività identificate come appartenenti al piano, insieme con i costi delle strategie implementative. \\*\\*
I costi privati possono avere un notevole impatto sul conseguimento degli obiettivi prefissati: costi elevati potrebbero scoraggiare i potenziali investitori dalla partecipazione alle attività pianificate, mentre costi troppo bassi potrebbero dare luogo a eccessi nel coinvolgimento dei privati; ePolicy tiene conto di questi costi considerandoli opportunamente all'interno del simulatore sociale.
 
\subsection[IMPATTI]{\nohyphens{IMPATTI ECONOMICI, SOCIALI E AMBIENTALI}}
Ogni piano avrà delle conseguenze in termini ambientali, sociali ed economici. Per raggiungere gli scopi prefissati un piano prevede l'esecuzione di un certo numero di attività; due categorie sono state individuate con l'aiuto di esperti della regione Emilia Romagna.\\*
Le \emph{attività primarie} sono quelle direttamente legate al conseguimento delle finalità del piano, ad esempio producendo esiti misurabile che influenzando direttamente gli obiettivi (nel caso di un piano energetico, la costruzione di una nuova centrale elettrica, per un piano relativo ai trasporti, la costruzione di una strada).\\*
Le \emph{attività secondarie} sono quelle che non agiscono direttamente sul valore degli obiettivi ma sono necessarie per l'implementazione delle attività primarie, cioè attività di supporto che non producono effetti misurabili sugli scopi del piano (sempre per un piano energetico, operazioni strettamente legate alla creazione di una centrale sono la realizzazione di strade per raggiungerla e di linee elettriche per collegarla alla rete nazionale).\\*\\*
Tra attività primarie e secondarie c'è una relazione diretta, pienamente inserita dentro il modello di ePolicy, in particolare gli esperti del dominio forniscono delle stime di ''quanto'' ogni attività secondaria sia richiesta per realizzare un  certa ''quantità'' di una certa attività primaria; più precisamente, gli esperti possono fornire una funzione per ogni coppia di attività primaria/secondaria, la quale ha in ingresso la quantità desiderata di attività primaria e restituisce una stima dell'attività secondaria necessaria. Riassumendo, ePolicy prende come input una matrice quadrata $N_a \times N_a$, $D$, dove ogni elemento $d_{ij}$ e una funzione che relazione l'attività $j$ con quella $i$ ($N_a$ è il numero totale di attività considerate).

\myparagraph{VALUTAZIONE DEGLI IMPATTI AMBIENTALI}
Per effettuare la valutazione ambientale sono stati usati diversi metodi e strumenti, tra i quali la metodologia adottata in Emilia Romagna; essa è basata sulle matrici coassiali \cite{coaxMatr} sviluppata a partire dai ''metodi a rete'' \cite{networkMethod}. In questa metodologia ogni attività influenza l'ambiente  in termini di \emph{pressioni negative} e \emph{pressioni positive} - tra le prime, la produzione di agenti inquinanti, mentre le seconde annoverano la maggior disponibilità di energia. Le pressioni stesse sono legate ai recettori ambientali, come la qualità dell'aria o delle acque superficiali; sia sulle pressioni che i recettori sono imposti dei vincoli (ad esempio ci sono limite per la massima emissione di gas serra per il piano complessivo).\\*\\*
Una matrice $M$ definisce le dipendenze tra le sopra menzionate attività contenute in un piano e le pressioni sull'ambiente. Ogni elemento $m_{ij}$ della matrice rappresenta una dipendenza qualitativa (con valori alto, medio, basso o nullo) tra l'attività $i$ e l'impatto positivo o negativo $j$.\\*
Una seconda matrice $N$ stabilisce come gli impatti/pressioni influenzano i recettori ambientali e in questo caso ogni elemento $n_{ij}$ lega l'impatto positivo o negativo con $i$ il recettore ambientale $j$.\\*
A partire da queste matrici vengono calcolati gli impatti del piano sui recettori ambientali.

\myparagraph{VALUTAZIONE DEGLI IMPATTI SOCIALI}
Conoscere l'opinione delle persone a riguardo di determinate politiche è di importanza fondamentale per chi deve prendere decisioni; spesso dunque i decisori politici propongono un processo partecipativo prima di iniziare la pianificazione, per raccogliere le opinioni di tutte le parti coinvolte durante incontri e workshop. Da ciò segue che la prima parte di un piano contiene il risultato della raccolta di opinioni svolta durante la fase partecipativa; ePolicy si propone di trattare questo lavoro in maniera automatica.\\*\\*
L'estrazione delle opinioni viene effettuata ricorrendo all'analisi di risorse testuali liberamente frequentate in rete, come blog, social network e forum; ad ogni messaggio rilevante viene assegnato un punteggio che indica se l'opinione riguardante un certo tema sia positiva o negativa e in che grado. Questo tipo di procedimento sfortunatamente non è molto generalizzabile, in quanto la scelta dei siti web rilevanti dipende dalla tipologia di piano considerata e inoltre il modello appreso tramite opinion mining dipende dall'argomento considerato.

\myparagraph{VALUTAZIONE DEGLI IMPATTI ECONOMICI}
Nella struttura di ePolicy gli impatti economici sono valutati sfruttando tecniche e metodi sviluppati all'interno del progetto RAMEA \cite{ramea}, un sistema di contabilità per l'ambiente utile per valutare le prestazioni economiche e ambientali delle regioni e per garantire alle politiche/strategie regionali informazioni circa lo sviluppo sostenibile, in linea con gli strumenti sviluppati a livello nazionale (NAMEA). Gli obiettivi di questi studi sono stati mirati soprattutto per definire strumenti in grado di collegare la conoscenza economica sulla produzione e le attività di consumo con le emissioni inerenti all'ambiente, costruire uno strumento utile per compiere studi, esaminare scenari,realizzare piani e dare comunicati, fornire indicatori per misurare, controllare e prevedere le prestazioni regionali e infine identificare in che modo una regione possa ottenere sviluppo economico e sociale senza causare ripercussioni sull'ambiente.\\*
Oltre a ciò le i metodi RAMEA possono essere utilizzati per diverse altre analisi, ad esempio monitorare le emissioni nell'aria e la eco-efficienza, comparando quella regionale con quella nazionale e comprendendo gli effetti e le responsabilità delle catene di produzione e consumo sull'ambiente. \\*\\*
Il progetto RAMEA, oltre ad essere utile per la raccolta di dati, mira anche a fornire i seguenti strumenti:
\begin{itemize}
\item un sistema di monitoraggio in grado di esaminare le pressioni imposte sull'ambiente da settori economici e infrastrutture, aiutando a identificare i ''punti caldi'' per quanto riguarda gli impatti ambientali, e consentire la costruzione di indici per valuare l'efficienza ecologica - ad esempio è possibile capire quali sono i settori chiave della regione per l'emissione di anidride carbonica, stabilire un collegamento diretto con le loro prestazioni economiche, comprendere se esiste una relazione tra crescite economica e inquinamento e creare indici per l'eco-efficienza;
\item uno strumento per effettuare previsioni permette di fare analisi degli scenari - dopo aver identificato i settori chiave per la $CO_2$ è possibile valutare e e quantificare gli effetti di diverse politiche/strategie mirate a ridurre le emissioni, includendo anche lo scenario base (nessuna azione intrapresa);
\item uno strumento per valutare le prestazioni che consente di confrontare differenti regioni. 
\end{itemize}
 
\subsection[OBIETTIVI]{\nohyphens{OBIETTIVI DEL PIANO}}
La definizione degli obiettivi di un piano richiede di prendere  in considerazione molte informazioni provenienti da diverse fonti. Innanzitutto, i programmi operativi nazionali e della Comunità Europea che identificano le finalità e i campi d'intervento a un livello generale. Secondariamente, vi sono le esigenze specifiche della regione: gli scopi generici sono elaborati in obiettivi dettagliati adattati alla situazione locale; durante questa fase le scelte politiche giocano un ruolo importante, insieme ai cittadini e a tutti i soggetti coinvolti. Infine i piani precedenti e i risultati ottenuti influenzano la determinazione degli obiettivi per il nuovo piano.\\*\\*
Come conseguenza, ogni piano presenta molteplici obiettivi, i quali devono essere presi in considerazioni dai decisori politici e riuniti in un'unica funzione, prestando anche attenzione ai possibili conflitti, con la necessità quindi di adottare criteri di ottimizzazione multipli.\\*\\*
Quando un problema ha un solo criterio di selezione per scegliere tra diverse soluzioni si ottiene un singolo valore ottimo (che potrebbe corrispondere a più soluzioni equivalenti). Invece quando sono presi in esame diversi obiettivi si ricavano più soluzioni ottime (definite di Pareto) - soluzioni che non sono dominate da altre. Una soluzione $x$ non è dominata da altre soluzioni con rispetto a un numero di funzioni obiettivo ($f_1, f_2, ...f_n$) se non esiste una soluzione che migliori rispetto a $x$ almeno una funzione obiettivo e presenti lo stesso valore per le restanti.\\*\\*
Solitamente i piani regionali devono rispettare un certo numero di obiettivi, tra i quali uno potrebbe essere il costo. Sia i fondi pubblici che privati vanno considerati, con i primi impiegati nella realizzazione del piano e l'implementazione delle strategie.Altri obiettivi interessanti sono i recettori ambientali, cioè gli indicatori della qualità di un preciso aspetto ambientale, tra i quali possono esserne ricordati alcuni, come qualità delle falde acquifere, qualità dell'acqua marina, qualità del suolo, qualità dell'aria, limitazione della subsidenza, stabilità degli argini e dei letti fluviali, qualità del clima, benessere della fauna selvatica, della vegetazione terrestre e degli animali acquatici, valore dell'eredità storica/culturale, disponibilità di terreni fertili, disponibilità di acqua, accessibilità delle risorse ricreative, benessere e salute della popolazione, disponibilità di energia, etc.\\*\\*
Chiaramente, a seconda della regione, il piano regionale potrebbe considerare diverse combinazioni di recettori da ottimizzare - ad esempio in Emilia Romagna, la qualità dell'aria è in genere piuttosto scarsa poiché il territorio piatto e circondato dalle Alpi a Nord e gli Appennini a Sud permette pochi modi di dissipare gli inquinanti immessi nell'atmosfera, quindi un recettore spesso tenuto in considerazione è la qualità dell'aria.\\*\\*
Da tutto ciò segue che potremmo avere un piano che cerchi di minimizzare il costo complessivo e massimizzare la qualità dell'aria; disponendo di due funzioni obiettivo, possiamo visualizzare in un diagramma cartesiano piani alternativi non dominati. In Figura ~\ref{paretoOpt} sono mostrate le curve ottime di Pareto contenenti piani non dominati.

\begin{figure}[hbt]
	\centering
	\includegraphics[scale=0.55]{paretoOpt}
	\caption{Frontiera ottima di Pareto per i piani energetici e due funzioni obiettivo}
	\label{paretoOpt}
\end{figure}


\section{\nohyphens{STRATEGIE IMPLEMENTATIVE}}

\subsection[INCENTIVI]{\nohyphens{TIPOLOGIE DI INCENTIVI}}

\myparagraph{MECCANISMI INCENTIVANTI}

\myparagraph{CONFRONTO DEI MECCANISMI D'INCENTIVAZIONE}

\subsection[INCENTIVI EUROPEI]{\nohyphens{INCENTIVI IN EUROPA}}

\myparagraph{TARIFFE DI INCENTIVAZIONE}

\myparagraph{QUOTE OBBLIGATORIE DA RINNOVABILI}

\myparagraph{SUSSIDI AGLI INVESTIMENTI}

\myparagraph{INCENTIVI O ESENZIONI PER LE TASSE}

\myparagraph{INCENTIVI FISCALI}


\subsection[INCENTIVI ITALIANI]{\nohyphens{INCENTIVI IN ITALIA}}

\myparagraph{TARIFFA INCENTIVANTE}

\myparagraph{TARIFFA INCENTIVANTE ONNICOMPRENSIVA}

\subsection[INCENTIVI REGIONALI]{\nohyphens{INCENTIVI IN EMILIA ROMAGNA}}

\myparagraph{MECCANISMI INCENTIVANTI REGIONALI}

\myparagraph{INCENTIVI FISCALI}


\nocite{*}
\bibliographystyle{plain}
\bibliography{bibliography}

\end{document}
