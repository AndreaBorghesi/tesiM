%Andrea Borghesi
%Università degli studi di Bologna

%capitolo dedicato alla descrizione (breve) dell'ottimizzatore

\documentclass[12pt,a4paper,openright,twoside]{report}
\usepackage[italian]{babel}
\usepackage{indentfirst}
\usepackage[utf8]{inputenc}
\usepackage[T1]{fontenc}
\usepackage{fancyhdr}
\usepackage{graphicx}
\usepackage{titlesec,blindtext, color}
\usepackage[font={small,it}]{caption}
\usepackage{subfig}
\usepackage{listings}
\usepackage{color}
\usepackage{url}
\usepackage{textcomp}

%impostazioni generali per visualizzare codice
\definecolor{dkgreen}{rgb}{0,0.6,0}
\definecolor{gray}{rgb}{0.5,0.5,0.5}
\definecolor{mauve}{rgb}{0.58,0,0.82}
 
\lstset{ %
  basicstyle=\footnotesize,           % the size of the fonts that are used for the code
  backgroundcolor=\color{white},      % choose the background color. You must add \usepackage{color}
  numbers=left,                   % where to put the line-numbers
  numberstyle=\tiny\color{gray},  % the style that is used for the line-numbers
  numbersep=5pt,  
  showspaces=false,               % show spaces adding particular underscores
  showstringspaces=false,         % underline spaces within strings
  showtabs=false,                 % show tabs within strings adding particular underscores
  rulecolor=\color{black}, 
  tabsize=2,                      % sets default tabsize to 2 spaces
  breaklines=true,                % sets automatic line breaking
  breakatwhitespace=false,        % sets if automatic breaks should only happen at whitespace
  title=\lstname,                   % show the filename of files included with \lstinputlisting;
  frame=single,                   % adds a frame around the code
                                  % also try caption instead of title
  keywordstyle=\color{blue},          % keyword style
  commentstyle=\color{dkgreen},       % comment style
  stringstyle=\color{mauve},         % string literal style
  escapeinside={\%*}{*)},            % if you want to add LaTeX within your code
  morekeywords={*,...},              % if you want to add more keywords to the set
  deletekeywords={...}              % if you want to delete keywords from the given language
}

%per avere un bordo intorno alle figure
\usepackage{float}
\floatstyle{boxed} 
\restylefloat{figure}

%per poter poi impedire che certe parole vadano a capo
\usepackage{hyphenat}
\usepackage{listings}

%ridefinisco font per fancyhdr, per ottenere un'intestazione pulita
\newcommand{\changefont}{ \fontsize{9}{11}\selectfont }
\fancyhf{}
\fancyhead[LE,RO]{\changefont \slshape \rightmark} 	%section
\fancyhead[RE,LO]{\changefont \slshape \leftmark}	%chapter
\fancyfoot[C]{\changefont \thepage}					%footer

%titolo capitolo con "numero | titolo"
\definecolor{gray75}{gray}{0.75}
\newcommand{\hsp}{\hspace{20pt}}
\titleformat{\chapter}[hang]{\Huge\bfseries}{\thechapter\hsp\textcolor{gray75}{|}\hsp}{0pt}{\Huge\bfseries}


\oddsidemargin=30pt \evensidemargin=20pt

%sillabazioni non eseguite correttamente
\hyphenation{sil-la-ba-zio-ne pa-ren-te-si si-mu-la-to-re ge-ne-ra-re pia-no}

%interlinea
\linespread{1.15}  
\pagestyle{fancy}

%cartelle contenenti le immagini
\graphicspath{{/media/sda4/tesi/immagini/grafici/}{/media/sda4/tesi/immagini/grafici/incCompare/}{/media/sda4/tesi/immagini/grafici/rawData/}{/media/sda4/tesi/immagini/grafici/regressionAnalysis/}{/media/sda4/tesi/immagini/schemi/}{/media/sda4/tesi/immagini/simulazione/}{/media/sda4/tesi/immagini/epolicy/}}

%in modo che dopo il titolo di un paragrafo il testo vada a capo
\newcommand{\myparagraph}[1]{\paragraph{#1}\mbox{}\\}

\begin{document}
\chapter{\nohyphens{OTTIMIZZAZIONE}}

%vedi D3.1, D2.2 , GavanelliEtAl

\section{STRUMENTI}

\subsection{ECLiPSe}

\subsection[APPROCCIO A VINCOLI]{\nohyphens{PERCHÉ UN APPROCCIO BASATO SUI VINCOLI}}
L'attività di pianificazione regionale è al momento svolta da esperti umani che costruiscono un singolo piano, considerando gli obbiettivi strategici regionali che seguono le direttive nazionali ed europee. Dopo che il piano è stato ideato l'ente per la protezione ambientale è chiamata a valutarne la sostenibilità dal punto di vista ambientale. In genere non c'è nessuna retroazione, la valutazione può solamente stabilire se il piano sia ecocompatibile o meno ma senza poterlo per modificare; in rari casi può proporre alcune misure correttive, le quali possono però solamente mitigare gli effetti negativi di decisioni di pianificazione sbagliate.\\* 
Oltre a ciò, sebbene le normative prevedano che una valutazione ambientale significativa debba confrontare due o più opzioni (piani differenti), questo è fatto raramente in Europa poiché la valutazione è tipicamente fatta a mano e richiede un lungo lavoro; anche nei pochi casi in cui due opzioni vengano considerate, solitamente una è il piano e l'altra è l'assenza di pianificazione.\\*\\*
La modellazione a vincoli supera le limitazioni dei processi manuali per diversi motivi. In primo luogo, essa fornisce uno strumento che automaticamente prende decisioni di pianificazione, tenendo in considerazione il budget allocato sulla base sia del piano operativo regionale che delle linee guida nazionali/europee.\\*
Secondo, gli aspetti ambientali sono considerati durante la costruzione del piano, evitando di procedere per tentativi ed errori.\\*
Come terza ragione, il ragionamento con i vincoli è uno strumento potente nelle mani di un decisore politico in quanto la generazione di scenari alternativi è estremamente semplificata ed il confronto e valutazione seguono naturalmente. Nel caso in cui i risultati non soddisfino coloro che stabiliscono le politiche o gli esperti ambientali gli aggiustamenti possono essere introdotti molto facilmente all'interno del modello; ad esempio, nel settore della pianificazione energetica regionale, cambiando i limiti della quantità di energia che ogni fonte può fornire, possiamo correggere il piano considerando l'andamento del mercato e anche la potenziale ricettività della regione.



\end{document}
