%Andrea Borghesi
%Università degli studi di Bologna

%capitolo dedicato alla descrizione (breve) dell'ottimizzatore

\documentclass[12pt,a4paper,openright,twoside]{report}
\usepackage[italian]{babel}
\usepackage{indentfirst}
\usepackage[utf8]{inputenc}
\usepackage[T1]{fontenc}
\usepackage{fancyhdr}
\usepackage{graphicx}
\usepackage{titlesec,blindtext, color}
\usepackage[font={small,it}]{caption}
\usepackage{subfig}
\usepackage{listings}
\usepackage{color}
\usepackage{url}
\usepackage{textcomp}
\usepackage{eurosym}
\usepackage{amsmath}

%impostazioni generali per visualizzare codice
\definecolor{dkgreen}{rgb}{0,0.6,0}
\definecolor{gray}{rgb}{0.5,0.5,0.5}
\definecolor{mauve}{rgb}{0.58,0,0.82}
 
\lstset{ %
  basicstyle=\footnotesize,           % the size of the fonts that are used for the code
  backgroundcolor=\color{white},      % choose the background color. You must add \usepackage{color}
  numbers=left,                   % where to put the line-numbers
  numberstyle=\tiny\color{gray},  % the style that is used for the line-numbers
  numbersep=5pt,  
  showspaces=false,               % show spaces adding particular underscores
  showstringspaces=false,         % underline spaces within strings
  showtabs=false,                 % show tabs within strings adding particular underscores
  rulecolor=\color{black}, 
  tabsize=2,                      % sets default tabsize to 2 spaces
  breaklines=true,                % sets automatic line breaking
  breakatwhitespace=false,        % sets if automatic breaks should only happen at whitespace
  title=\lstname,                   % show the filename of files included with \lstinputlisting;
  frame=single,                   % adds a frame around the code
                                  % also try caption instead of title
  keywordstyle=\color{blue},          % keyword style
  commentstyle=\color{dkgreen},       % comment style
  stringstyle=\color{mauve},         % string literal style
  escapeinside={\%*}{*)},            % if you want to add LaTeX within your code
  morekeywords={*,...},              % if you want to add more keywords to the set
  deletekeywords={...}              % if you want to delete keywords from the given language
}

%per avere un bordo intorno alle figure
\usepackage{float}
\floatstyle{boxed} 
\restylefloat{figure}

%per poter poi impedire che certe parole vadano a capo
\usepackage{hyphenat}
\usepackage{listings}

%ridefinisco font per fancyhdr, per ottenere un'intestazione pulita
\newcommand{\changefont}{ \fontsize{9}{11}\selectfont }
\fancyhf{}
\fancyhead[LE,RO]{\changefont \slshape \rightmark} 	%section
\fancyhead[RE,LO]{\changefont \slshape \leftmark}	%chapter
\fancyfoot[C]{\changefont \thepage}					%footer

%titolo capitolo con "numero | titolo"
\definecolor{gray75}{gray}{0.75}
\newcommand{\hsp}{\hspace{20pt}}
\titleformat{\chapter}[hang]{\Huge\bfseries}{\thechapter\hsp\textcolor{gray75}{|}\hsp}{0pt}{\Huge\bfseries}


\oddsidemargin=30pt \evensidemargin=20pt

%sillabazioni non eseguite correttamente
\hyphenation{sil-la-ba-zio-ne pa-ren-te-si si-mu-la-to-re ge-ne-ra-re pia-no}

%interlinea
\linespread{1.15}  
\pagestyle{fancy}

%cartelle contenenti le immagini
\graphicspath{{/media/sda4/tesi/immagini/grafici/}{/media/sda4/tesi/immagini/grafici/incCompare/}{/media/sda4/tesi/immagini/grafici/rawData/}{/media/sda4/tesi/immagini/grafici/regressionAnalysis/}{/media/sda4/tesi/immagini/schemi/}{/media/sda4/tesi/immagini/simulazione/}{/media/sda4/tesi/immagini/epolicy/}{/media/sda4/tesi/immagini/ottimizzazione/}}

%in modo che dopo il titolo di un paragrafo il testo vada a capo
\newcommand{\myparagraph}[1]{\paragraph{#1}\mbox{}\\}

\begin{document}
\chapter{\nohyphens{OTTIMIZZAZIONE}}
In aggiunta al modello di simulazione ad agenti, con la relativa prospettiva individuale, il progetto ePolicy ritiene necessario considerare anche una prospettiva globale (regionale nel caso della Regione), in grado di affrontare il problema della pianificazione e dalla valutazione degli impatti dei piani da un punto di vista più ampio, mantenendo al tempo stesso una stretta integrazione con il livello individuale.

La pianificazione delle attività regionale può essere vista come un complesso problema di ottimizzazione combinatoria; i decisori politici devono prendere decisioni e soddisfare un insieme di vincoli, tentando al tempo stesso di realizzare un certo numero di obiettivi, come ad esempio ridurre gli effetti negativi e incrementare i positivi su ambiente, società ed economia. La fase di valutazione - valutare quali siano gli impatti delle politiche scelte sull'ambiente ed in misura minora in ambito economico e sociale - è ora in genere effettuata in sequenza dopo la creazione di un piano, con lo svantaggio che se questo contenesse impatti negativi sull'ambiente potrebbero venire applicate solo delle contromisure correttive; per evitare ciò, nell'approccio di ePolicy la valutazione e la pianificazione sono condotte allo stesso tempo.\\*\\*
Per la valutazione ambientale sono stati proposti diversi metodi: un modello probabilistico \cite{logicDSSstrategicAss}, un modello fuzzy (la logica fuzzy prevede che si possa attribuire a una proposizione un grado di verità compreso tra 0 e 1) \cite{fuzzyLogicstrategicAss} e un modello lineare a vincoli (\emph{Constraint Logic Programming}, programmazione logica a vincoli, chiamata in seguito anche CLP) \cite{GavanelliEtAl}. Il motivo per sperimentare diversi tipi di modello è che le matrici usate dagli esperti ambientali si prestano a differenti interpretazioni, quindi era importante capire quale fosse la migliore scelta possibile. Il modello CLP è risultato essere il più veloce - a livello computazionale - in quanto per la programmazione lineare esistono tecniche di risoluzione molto efficienti. In secondo luogo, questo modello può essere facilmente esteso aggiungendo nuovi vincoli, per risolvere nuovi tipi di problemi; ad esempio se le attività da pianificare fossero variabili decisionali (invece che valori fissi) potremmo svolgere la pianificazione contemporaneamente alla valutazione ambientale. Dal momento che questo era uno degli scopi del progetto si è scelto di utilizzare l'approccio CLP.\\*\\*
In questo capitolo introdurremo molto brevemente la programmazione logica a vincoli, con una breve panoramica e citando gli strumenti software di cui ci siamo serviti, passeremo poi a presentare il modello sviluppato che incorpora al suo interno le attività di pianificazione e valutazione e infine mostreremo i risultati ottenuti applicando il modello al caso di studio scelto, ovvero il piano energetico 2011-2013 per la regione Emilia-Romagna.

\section[CLP]{\nohyphens{PROGRAMMAZIONE LOGICA A VINCOLI}}
Come abbiamo già ampiamente spiegato in precedenza il problema della creazione di un piano regionale può essere considerato come un problema caratterizzato da un insieme di vincoli e una funzione obiettivo. Nell'ambito dell'Intelligenza Artificiale i problemi per cui è richiesto soddisfare un insieme di vincoli sono definiti come \emph{Problemi di Soddisfacimento di Vincoli} (in inglese Constraint Satisfaction Problem, da cui l'acronimo \emph{CSP}). Un CSP è definito da una terna $<X,D,C>$, dove $X$ è un insieme di variabili $X=\{X_1,X_2...X_n\}$, $D$ è un dominio discreto per ogni variabile $D=\{D_1,D_2...D_n\}$ e $C$ è l'insieme di vincoli - un vincolo è una relazione tra variabili che definisce un sottoinsieme del prodotto cartesiano dei domini $D_1 \times D_2 \times ... \times D_n$; con queste premesse un la soluzione di un problema di soddisfacimento di vincoli è data da un assegnamento di valori alle variabili consistente con i vincoli \cite{cspFoundations}. Analogamente, un problema di ottimizzazione con vincoli (Constraint Optimization Problem, \emph{COP}) è definito da $<X,D,C,f>$, cioè un CSP più una \emph{funzione obiettivo} $f(X_1,X_2...X_n)$, la cui soluzione è un assegnamento di valori alle variabili compatibile con i vincoli del problema che ottimizza la funzione obiettivo. Le metodologie risolutive impiegate per risolvere problemi a vincoli attingono in buona parte alle tecniche di Ricerca Operativa e Intelligenza Artificiale, in particolare noi considereremo la \emph{Programmazione Logica a Vincoli}.\\*\\*
La programmazione logica a vincoli \cite{clpSurvey} (in inglese \emph{Constraint Logic Programming} da cui CLP) è una classe di linguaggi di programmazione che estendono la classica Programmazione Logica - il paradigma di programmazione basato sulla logica del primo ordine implementato da linguaggi come ad esempio Prolog \cite{Colmerauer,Kowalski,clocksin2003programming}, sviluppato nei primi anni settanta e ampiamente diffuso ancora oggi. Alle variabili possono essere assegnati sia termini (i tipi di dato e le strutture riconosciute in Prolog) sia valori interpretati, appartenenti a una determinata \emph{classe}, un parametro caratteristico dello specifico linguaggio CLP; per esempio è possibile avere CLP(\emph{R}) \cite{clpR}, in grado di operare sui valori reali, oppure CLP(FD), in cui le variabili appartengono a domini finiti. All'interno di una classe sono definite le funzioni interpretate (che possono essere nei domini numerici i soliti operatori $+$, $-$, $\times$, etc.) e i predicati (ad esempio, $<$, $\neq$, $\geq$, etc.), che sono chiamati \emph{vincoli}. La semantica dichiarativa consente l'interpretazione intuitiva per i vincoli e i termini, relativamente al dominio considerato: ad esempio, $1.3+2<5$ è \emph{vero} in CLP(\emph{R}); ciò è un'estensione molto significativa rispetto alla programmazione logica standard in quanto i linguaggi logici operano in domini non interpretati (''Universo di Herbrand'') e quindi le relazioni tra variabili possono essere solamente verificate a posteriori e non trattate come vincoli veri propri. La semantica operativa somiglia a quella di Prolog per atomi costruiti sui predicati usuali - quelli definiti da un insieme di clausole - ma conserva quelli da interpretare, i vincoli, in una struttura dati speciale, chiamata \emph{constraint store}; essa è in seguito interpretato e modificato da un meccanismo esterno, il \emph{constraint solver}, il risolutore dei vincoli. Il risolutore è in grado di controllare se una combinazione di vincoli è soddisfacibile o meno, e può modificare lo store, sperabilmente per semplificarne lo stato. Generalmente il solver non effettua una propagazione \emph{completa}: se una valutazione ha come risultati \emph{falso}, allora sicuramente la soluzione è impossibile, anche se in certi casi può capitare che il risolutore non rilevi l'impossibilità di un problema anche se non esistono soluzioni.\\*\\*  
CLP(\emph{R}) è una classe di programmazione logica a vincoli in cui le variabili appartengo all'insieme dei numeri reali, i vincoli disponibili sono uguaglianze e disuguaglianze lineari e generalmente il risolutore è implementato tramite l'algoritmo del simplesso, molto veloce e completo per (dis)equazioni lineari (cioè è sempre in grado di restituire \emph{vero} o \emph{falso}). In alcuni sistemi, grazie alla disponibilità di risolutori efficienti per la programmazione lineare intera, certi vincoli non lineari sono accettati nel linguaggio, in particolare è possibile imporre che certe variabili assumano esclusivamente valori interi; in casi come questo la complessità del problema passa da P a NP-hard e il risolutore deve spesso ricorrere a tecniche di branch-and-bound. Ad ogni modo, l'utente può specificare anche una funzione obiettivo, un termine lineare che dovrebbe essere massimizzato o minimizzato garantendo al tempo stesso che tutti vincoli siano soddisfatti.\\*\\*


\section{STRUMENTI}
Al giorno d'oggi esistono numerose implementazioni di CLP(\emph{R}) \cite{inclpR} e diverse versioni di Prolog dispongono di una propria libreria per CLP(\emph{R}). Per il progetto ePolicy si è scelto di adottare il software open source  ECL$^i$PS$^e$ \cite{clpEclipse,fromLPtoCLPeclipse}.

\subsection{ECL$^i$PS$^e$}
ECL$^i$PS$^e$ è un sistema software per lo sviluppo di applicazioni di programmazione a vincoli e indicata per lo studio di aspetti relativi alla risoluzione di problemi combinatori, come appunto la programmazione vincolata, modellazione di problemi, programmazione matematica, tecniche di ricerca di soluzioni,... Al suo interno sono contenuti librerie per risolutori a vincoli, un linguaggio di alto livello (derivato da Prolog), interfacce per risolutori esterni e altre funzionalità. In Figura~\ref{eclipseUI} mostriamo come si presenta l'interfaccia utente di ECL$^i$PS$^e$.

\begin{figure}[h]
	\centering
	\includegraphics[scale=0.28]{eclipseUI}
	\caption{ECL$^i$PS$^e$, Interfaccia Utente}
	\label{eclipseUI}
\end{figure}

Tra le diverse librerie disponibili, ne esiste una denominata \emph{Eplex} \cite{eplex} che interfaccia ECL$^i$PS$^e$ a un risolutore lineare intero, il quale può essere sia uno strumento commerciale, come CPLEX o Xpress-MP, che open source. A default Eplex nasconde molti dei dettagli del risolutore, ma nondimeno, quando richiesto, l'utente può regolare diversi parametri per migliorare le prestazioni ed esaminare lo stato interno del solver. Nell'ambito del progetto ePolicy ci siamo serviti di questa libreria, insieme alle altre funzionalità offerte da ECL$^i$PS$^e$, per modellare e risolvere i problemi relativi alla pianificazione energetica regionale.
\\*\\*
Illustreremo ora due esempi di modellazione di problemi a vincoli sfruttando il linguaggio ECL$^i$PS$^e$ (nel primo caso considerando domini finiti e nel secondo valori reali, avvalendoci anche della libreria Eplex), anche per mostrare come possono essere strutturati i problemi di programmazione logica a vincoli; in questa trattazione supporremo noti i concetti elementari della programmazione logica (procedimenti risolutivi, definizioni di un termine, etc.), la cui discussione esula da questo lavoro.

\myparagraph{ESEMPIO CLP(FD)}
Il cosiddetto \emph{Send More Money} puzzle è un esempio classico di programmazione a vincoli; le variabili $[S,E,N,D,M,O,R,Y]$ rappresentano cifre da 0 a 9 e lo scopo è assegnare alle variabili valori diversi in modo che l'operazione aritmetica di Figura~\ref{SendMoreMoney} risulti corretta - inoltre i numeri devono essere ben formati, da cui $S>0$ e $M>0$. 

\begin{figure}[h]
	\centering
	\includegraphics[width=0.215\textwidth]{sendMoreMoney}
	\caption{Send More Money puzzle}
	\label{SendMoreMoney}
\end{figure}

Con la programmazione convenzionale si avrebbe necessità di esprimere una strategia di ricerca in modo esplicito (senza contare possibili ottimizzazioni come cicli innestati), mentre con linguaggi logici come Prolog verrebbe sfruttata la ricerca fornita dal risolutore interno (il motore inferenziale), con il vantaggio di una programmazione estremamente facilitata ma col rischio di un'efficienza non elevata - a meno di programmi ottimizzati, i quali richiederebbero comunque maggiori tempo e abilità. 

Questo è in effetti il campo di applicazione ideale della programmazione logica a vincoli, in particolare nell'ambito  dei domini finiti (CLP(FD)): le variabili possono assumere valori appartenenti ad un insieme finito di numeri interi, i vincoli sono facilmente esprimibili formalmente e occorre effettuare una certa quantità di ricerca nello spazio delle soluzioni. In questo problema sarebbe naturale usare le variabili del programma per rappresentare le diverse cifre e la soluzione finale dovrà essere un assegnamento di un valore unico per ogni variabile. 

Risolvere questo problema con Prolog comporta l'utilizzo della strategia di ricerca chiamata \emph{Generate and Test}, che prevede che prima la generazione di una soluzione e poi la verifica della consistenza dei vincoli e, nel caso che questa dia esito negativo, l'assegnamento di nuovi valori alle variabili seguita da nuova verifica e così via. In questo modo l'esplorazione dello spazio delle soluzioni è chiaramente inefficiente - per esempio la possibile implementazione in Prolog mostrata qui sotto, per quanto suscettibile a miglioramenti, deve gestire $\frac{10!}{2}$ possibili assegnamenti di valori alle variabili. 

\lstset{language=Prolog}
\begin{lstlisting}
% Send More Money puzzle in Prolog
smm :-
        X = [S,E,N,D,M,O,R,Y],           % variabili
        Digits = [0,1,2,3,4,5,6,7,8,9],	 % domini
        
        % predicato che assegna una soluzione
        assign_digits(X, Digits),
       	
       	%  verifica dei vincoli vincoli
        M > 0, 
        S > 0,
                  1000*S + 100*E + 10*N + D +
                  1000*M + 100*O + 10*R + E =:=
        10000*M + 1000*O + 100*N + 10*E + Y,
        write(X).

select(X, [X|R], R).
select(X, [Y|Xs], [Y|Ys]):- select(X, Xs, Ys).

assign_digits([], _List).
assign_digits([D|Ds], List):-
        select(D, List, NewList),
        assign_digits(Ds, NewList).
\end{lstlisting}

L'implementazione realizzata con ECL$^i$PS$^e$ presenta i vantaggi di semplificare ulteriormente la modellazione del problema e di appoggiarsi all'efficiente risolutore interno per l'esplorazione dello spazio delle soluzioni, in modo particolare il fatto che ogni volta che una variabile viene istanziata i vincoli vengono propagati per eliminare a priori strade inconsistenti, riducendo gli spazi delle soluzioni e prevenendo fallimenti sicuri. 

\begin{lstlisting}
% Send More Money puzzle in ECLiPSe
smm :-
     X = [S,E,N,D,M,O,R,Y],		% variabili
     X :: [0 .. 9],				% domini finiti
     
     % vincoli
     M #> 0,
     S #> 0,
               1000*S + 100*E + 10*N + D +
               1000*M + 100*O + 10*R + E #=
     10000*M + 1000*O + 100*N + 10*E + Y,
     alldistinct(X),
     
     % ricerca della soluzione
     labeling(X),
     write(X).

\end{lstlisting}

\myparagraph{ESEMPIO CLP(R) - EPLEX}

Presentiamo ora un esempio di un problema (Fig.~\ref{clpR_example}) che rientra nell'ambito dei CLP(R) e che fa uso della libreria Eplex, tratto dal manuale di ECL$^i$PS$^e$ \cite{eclipseTut}. Ci sono tre impianti, o fabbriche, (1-3) in grado di produrre un certo prodotto con capacità diverse e i cui prodotti devono essere trasportati a quattro clienti (A-D) con quantità richieste diverse; anche il costo unitario di trasporto ai clienti è variabile. L'obiettivo del problema è minimizzare i costi di trasporto soddisfacendo le esigenze dei clienti. 

\begin{figure}[h]
	\centering
	\includegraphics[scale=0.7]{clpR_example}
	\caption{Esempio di un problema CLP(R). Fonte \cite{eclipseTut}}
	\label{clpR_example}
\end{figure}

Per formulare il problema definiamo la quantità di prodotto trasportata dall'impianto $N$ al cliente $p$ come variabile $N_p$ - ad esempio $A_1$ rappresenta il costo di trasporto dalla fabbrica $A$ al cliente $1$. I vincoli da considerare sono di due tipi (sempre facendo riferimento alla Figura~\ref{clpR_example}):
\begin{itemize}
\item La quantità di prodotto consegnata da tutti gli impianti a un cliente deve deve essere uguale alla domanda del cliente, ad esempio per il cliente $A$ che può essere rifornito dagli impianti 1-3, abbiamo che $A_1+A_2+A_3=21$
\item La quantità di prodotto in uscita da una fabbrica non può essere superiore alla sua capacità produttiva, ad esempio per l'impianto $1$ che invia prodotti ai clienti A-D si ha che $A_1+B_1+C_1+D_1 \leq 50$
\end{itemize}   
Poiché lo scopo è minimizzare i costi di trasporto, la funzione obiettivo è di minimizzare i costi combinati del trasporto dei prodotti dai tre impianti a tutti e quattro i clienti.

La formulazione del problema è quindi la seguente.\\*
Funzione obiettivo: 
\begin{equation}  \label{exampleClpRObiett}
	\min(10A_1+7A_2+200A_3+8B_1+5B_2+10B_3+5C_2+5C_2+8C_3+9D_1+3D_2+7D_3)
\end{equation}
Vincoli:
\begin{equation}  \label{exampleCons1}
	A_1+A_2+A_3=21
\end{equation}
\begin{equation}  \label{exampleCons2}
	B_1+B_2+B_3=40
\end{equation}
\begin{equation}  \label{exampleCons3}
	C_1+C_2+C_3=34
\end{equation}
\begin{equation}  \label{exampleCons4}
	D_1+D_2+D_3=10
\end{equation}
\begin{equation}  \label{exampleCons5}
	A_1+B_1+C_1+D_1 \leq 50
\end{equation}
\begin{equation}  \label{exampleCons6}
	A_2+B_2+C_2+D_2 \leq 30
\end{equation}
\begin{equation}  \label{exampleCons7}
	A_3+B_3+C_3+D_3 \leq 40
\end{equation}

Mostriamo ora come questo problema venga modellato sfruttando la libreria Eplex. In primo luogo occorre caricare la libreria Eplex di cui si dispone (in questo caso abbiamo sfruttato un risolutore esterno open source) e ottenerne un'\emph{istanza}, la quale rappresenta un singolo problema sotto forma di modulo, a cui possono essere riferiti vincoli e funzione obiettivo consentendo quindi al solver esterno di risolvere il problema. Il codice che segue mostra come il problema di Figura~\ref{clpR_example} sia stato trasposto all'interno di ECL$^i$PS$^e$.

\begin{lstlisting}
:- lib(eplex).		% caricamento della libreria Eplex
:- eplex_instance(prob).		% definizione dell'istanza - chiamata 'prob'

main(Cost, Vars) :-
		% dichiarazione delle variabili e definizione del loro dominio		
		Vars = [A1,A2,A3,B1,B2,B3,C1,C2,C3,D1,D2,D3],
		prob: (Vars $:: 0.0..1.0Inf),  % valori maggiori o uguali a 0
		
		% definizione dei vincoli applicati all'istanza eplex
		prob: (A1 + A2 + A3 $= 21), 
		prob: (B1 + B2 + B3 $= 40),
		prob: (C1 + C2 + C3 $= 34),
		prob: (D1 + D2 + D3 $= 10),

		prob: (A1 + B1 + C1 + D1 $=< 50),
		prob: (A2 + B2 + C2 + D2 $=< 30),
		prob: (A3 + B3 + C3 + D3 $=< 40),

		% inizializza il solver esterno con la funzione obiettivo
		prob: eplex_solver_setup(min(10*A1 + 7*A2 + 200*A3 + 
			8*B1 + 5*B2 + 10*B3 +
		 	5*C1 + 5*C2 + 8*C3 +
		 	9*D1 + 3*D2 + 7*D3)),

		% ---------- Fine Modellazione ----------

		% risoluzione del problema
		prob: eplex_solve(Cost).
\end{lstlisting}

Per usare un'istanza Eplex occorre prima dichiararla con \emph{eplex\_instance/1}; una volta dichiarata, l'istanza viene riferita tramite il nome specificato. 

Come primo passo creiamo le variabili del problema e imponiamo che possano assumere solamente valori non negativi e rendiamo noti all'istanza il loro dominio (\emph{\$::/2}). Successivamente imponiamo i vincoli che modellano il problema sotto forma di uguaglianze e disuguaglianze aritmetiche; per via del solver esterno scelto, gli unici tipi di vincoli accettati sono quelli lineari - che ovviamente consentono una maggiore efficienza nella risoluzione.

Occorre poi inizializzare il risolutore esterno con l'istanza eplex creata, in modo che questa possa essere risolta. Questo è fatto dal \emph{eplex\_solver\_setup/1}, che prende come argomento la funzione obiettivo, la quale può essere di minimizzazione o massimizzazione. Infine è possibile risolvere il problema modellato attraverso \emph{eplex\_solve/1}.

Quando un'istanza viene risolta, il solver prende in considerazione tutti i vincoli ad essa relativi, i valori che le variabili del problema possono assumere e la funzione obiettivo specificata. In questo caso è possibile ottenere una soluzione ottimale pari a 710.0: 
\begin{lstlisting}
?-	main(Cost, Vars).

Cost = 710.0
Vars = [A1{0.0 .. 1e+20 @ 0.0}, A2{0.0 .. 1e+20 @ 21.0}, ....]
\end{lstlisting}

\section{MODELLAZIONE PROBLEMA}

Passeremo ora ad occuparci dell'implementazione dell'ottimizzatore impiegato nell'ambito del progetto ePolicy per fornire supporto alle decisioni per lo sviluppo del piano energetico regionale.

In Figura~\ref{ioDSS} è presentata una visione generale e di alto livello dei fattori che entrano in gioco durante la creazione la fase di pianificazione, come i costi e la capacità produttiva degli impianti, gli impatti economici e ambientali delle azioni intraprese, i vincoli di budget e gli obiettivi di produzione energetica, etc. (benché non siano oggetto di discussione in questo capitolo sono inclusi per completezza nella figura anche gli impatti sociali e le modalità di incentivazione, gestite rispettivamente dal componente di Opinion Mining e dal simulatore).

\begin{figure}[h]
	\centering
	\includegraphics[scale=0.45]{ioDSS}
	\caption{Panoramica di Input e Output del DSS di ePolicy}
	\label{ioDSS}
\end{figure}

\subsection[APPROCCIO A VINCOLI]{\nohyphens{PERCHÉ UN APPROCCIO BASATO SUI VINCOLI}}
L'attività di pianificazione regionale è al momento svolta da esperti umani che costruiscono un singolo piano, considerando gli obbiettivi strategici regionali che seguono le direttive nazionali ed europee. Dopo che il piano è stato ideato l'ente per la protezione ambientale è chiamata a valutarne la sostenibilità dal punto di vista ambientale. In genere non c'è nessuna retroazione, la valutazione può solamente stabilire se il piano sia ecocompatibile o meno ma senza poterlo per modificare; in rari casi può proporre alcune misure correttive, le quali possono però solamente mitigare gli effetti negativi di decisioni di pianificazione sbagliate.

Oltre a ciò, sebbene le normative prevedano che una valutazione ambientale significativa debba confrontare due o più opzioni (piani differenti), questo è fatto raramente in Europa poiché la valutazione è tipicamente fatta a mano e richiede un lungo lavoro; anche nei pochi casi in cui due opzioni vengano considerate, solitamente una è il piano e l'altra è l'assenza di pianificazione.\\*\\*
La modellazione a vincoli supera le limitazioni dei processi manuali per diversi motivi. In primo luogo, essa fornisce uno strumento che automaticamente prende decisioni di pianificazione, tenendo in considerazione il budget allocato sulla base sia del piano operativo regionale che delle linee guida nazionali/europee.

Secondo, gli aspetti ambientali sono considerati durante la costruzione del piano, evitando di procedere per tentativi ed errori.

Come terza ragione, il ragionamento con i vincoli è uno strumento potente nelle mani di un decisore politico in quanto la generazione di scenari alternativi è estremamente semplificata ed il confronto e valutazione seguono naturalmente. Nel caso in cui i risultati non soddisfino coloro che stabiliscono le politiche o gli esperti ambientali gli aggiustamenti possono essere introdotti molto facilmente all'interno del modello; ad esempio, nel settore della pianificazione energetica regionale, cambiando i limiti della quantità di energia che ogni fonte può fornire, possiamo correggere il piano considerando l'andamento del mercato e anche la potenziale ricettività della regione.

\subsection{MODELLO CLP}
Il Piano Regionale è il risultato della principale attività di definizione delle politiche in cui le regioni europee siano coinvolte. Ogni regione dispone di un budget distribuito sulla base del Programma Operativo, il quale specifica le priorità di ogni regione per l'assegnamento dei fondi (nel campo dell'energia una esempio di priorità è l'incremento della produzione da fonti rinnovabili).

Partendo dalla struttura per il sistema di supporto alle decisioni descritta nel primo capitolo, ci accingiamo ora a illustrare il il modello per l'ottimizzazione globale ideato per prendere in considerazione la prospettiva regionale. 
\\* \\*
Nella fase di pianificazione occorre decidere quali attività debbano essere svolte, distinguendo a grandi linee sei tipologie: infrastrutture e impianti, edifici e trasformazioni sul territorio, estrazione delle risorse, modifiche del regime idrico, trasformazioni industriali, gestione dell'ambiente. Per ogni attività dovrebbe essere inoltre deciso un ordine di grandezza che descriva in quale quantità un'attività venga portata a termine.

Abbiamo quindi $N_a$ attività rappresentate con $a_i (i=1..N_a)$ e distinguiamo tra \emph{attività primarie} e \emph{attività secondarie}, nel caso del piano energetico esse sono rispettivamente quelle che producono direttamente energia e quelle che supportano le prime fornendo le necessarie infrastrutture. Possiamo quindi immaginare i seguenti tipi di input, collegati alla nozione di attività:
\begin{enumerate}
\item lista delle \emph{attività primarie} (impianti a biomassa, impianti fotovoltaici, impianti eolici, etc.) e \emph{quantità minime/massime} per ogni attività (es., aumento del 10\% la produzione di energia fotovoltaica);
\item \emph{funzione di costo}: consideriamo un vettore di costi $C=(c_1,...,c_{N_a})$ dove ogni elemento è associato ad una specifica attività e rappresenta il costo unitario della stessa - sostanzialmente si tratta di determinare il costo per MWatt di energia prodotta in base all'attività\footnote{Il costo di un impianto dipende principalmente dalla potenza installata: il costo di installazione di un impianto solare dipende dalla dimensione in metri quadri dei pannelli installati, da cui deriva di conseguenza una certa potenza massima. Notare come il costo considerato sia il costo totale per il sistema regionale, che non è lo stesso costo sostenuto dai contribuenti della regione Emilia-Romagna, poiché la regione può attuare le proprie politiche in diversi modi, convincendo i privati a investire nella produzione energetica};
\item \emph{funzione di efficienza}: tramite un vettore per le produzioni energetiche $Out=(out_1,...,out_{N_a})$ determiniamo l'energia generata per ogni MWatt di potenza installata di ogni attività - questa funzione dipende dalla località geografica (ad esempio nell'Italia meridionale per via del clima soleggiato l'efficienza del fotovoltaico è maggiore rispetto alle aree settentrionali); 
\item \emph{la matrice delle attività primarie-secondarie} $D$, usata per stabilire quali attività secondarie e in quale misura sono necessarie per realizzare un determinato ammontare di attività primaria, ovvero una matrice di dipendenze tra le attività. 

\end{enumerate}
Ogni regione ha i propri obiettivi, alcuni intrinsecamente determinati dalla regione stessa altri conseguenze di decisioni politiche.
\begin{enumerate}
\item \emph{produzione richiesta/attesa}: in un piano energetico la produzione energetica attesa all'interno  dei confini regionali è basata sulle stime dei consumi energetici;
\item \emph{vincoli di budget}: ogni regione dispone di una certa quantità di fondi da destinare a incentivi con i quali indirizzare il mercato dell'energia nella direzione desiderata - dato un budget disponibile per il piano, questo è vincolato in termini di costo totale (questo vincolo può essere applicato sia al piano complessivo sia a parti di esso);
\item \emph{obiettivi politici}: sono forniti dai decisori politici, come ad esempio la priorità di conformarsi alla linea guida europea come l'iniziativa 20-20-20\footnote{L'iniziativa 20-20-20 mira a raggiungere tre obiettivi entro il 2020: ridurre del 20\% l'emissione di gas serra, produrre il 20\% dell'energia consumata attraverso fonti rinnovabili e migliorare del 20\% l'efficienza energetica.}, e possono essere traslati in vincoli che specificano la minima quantità di energia prodotta da fonti rinnovabili.
\end{enumerate} 

Infine occorre tenere in considerazione gli input relativi alla configurazione geofisica della regione e alla buona norma.
\begin{enumerate}
\item \emph{diversificazione delle fonti energetiche}: l'allocazione dei fondi non dovrebbe essere diretta verso un'unica fonte di energia ma coprire diverse risorse energetiche. Questo requisito è considerato come buona norma e supportato da numerose considerazioni, per esempio per assicurare ``robustezza`` nei confronti di fluttuazioni di prezzo e disponibilità delle varie risorse - queste indicazioni possono essere implementate sotto forma di vincoli che impongano che percentuali minime della produzione totale siano soddisfatte da ogni fonte energetica;
\item \emph{limiti sociali e geografici}: ogni regione ha le sue caratteristiche geofisiche (ad esempio, l'essere o meno ventose), che possono porre limiti alla quantità massima di energia generate da determinate risorse - le centrali idroelettriche possono essere costruite con un'attenta valutazione degli impatti ambientali che possono avere sul territorio, come, ovviamente, l'allagamento di vaste aree.
\end{enumerate} 

Un altro aspetto che i decisori politici devono considerare sono le ripercussioni su ambiente, società ed economia. Come già descritto nel primo capitolo uno degli strumenti per effettuare questa valutazione in Emilia-Romagna prevede l'impiego di matrici coassiali che definiscono le relazioni tra le attività primarie di un piano e gli impatti ambientali (o pressioni) positivi o negativi e in che modo tali attività influiscano sui recettori ambientali (anche essi introdotti in precedenza)\footnote{La matrice correntemente utilizzata per la valutazione ambientale contiene 93 attività, 29 impatti negativi, 19 positivi e 23 recettori e valuta 11 tipi di piano. Queste matrici prendono in esame anche in che modo le attività secondarie richieste impattino sui recettori}. 
\\*\\*
Lo scopo finale di un piano energetico può essere definito, più o meno, come il tentativo di guidare il mercato libero della produzione di energia verso la copertura del fabbisogno energetico regionale e il soddisfacimento di alcuni obiettivi politici. Per questo fine un piano consiste in un insieme di \emph{decisioni} riguardanti i seguenti aspetti:
\begin{enumerate}
\item quali tipi\footnote{Per la precisione i tipi di energia sono influenzati anche da alcuni obiettivi politici, presi come ingressi} (attività) di produzione energetica sia necessario impiegare e in che ordine di grandezza; 
\item quanti fondi debbano essere assegnati per spingere il mercato versa la produzione desiderata (attraverso l'implementazione di meccanismi incentivanti). Questa decisione può essere separata in due sotto-problemi:
	\begin{enumerate}
	\item la quantità di finanziamenti da allocare per promuovere ogni sistema di generazione di energia diverso, sotto forma di incentivi;
	\item i meccanismi per assegnare tali incentivi. 
	\end{enumerate}
\end{enumerate}

\subsection{IMPLEMENTAZIONE MODELLO}
Illustreremo ora come sono stati modellate le problematiche introdotte precedentemente all'interno del paradigma della Programmazione Logica a Vincoli, in particolare sul dominio dei numeri reali.

\myparagraph{MATRICI COASSIALI IN CLP(R)}
%D.3 6.2 
Le matrici coassiali possono essere interpretate facilmente come modelli di programmazione lineare. Più in dettaglio, rappresentiamo le attività e i loro ordini di grandezza attraverso il vettore $A = (a_1,...,a_{N_a})$, dunque gli impatti ambientali causati dall'attività $i$ possono essere stimati col sistema di equazioni lineari:
\begin{equation}
\label{eq:impattiAtt}
	\forall j \in \{1,...,N_p\}  \quad p_j = m_j^i a_i
\end{equation}
Quando consideriamo un intero piano regionale, sommiamo i contributi di tutti le attività e otteniamo la stima dell'influenza esercitata su ogni impatto/pressione:
\begin{equation}
\label{eq:impattiAttTot}
	\forall j \in \{1,...,N_p\}  \quad  p_j = \sum_{i=1}^{N_a} m_j^i a_i
\end{equation}
Allo stesso modo, dato il vettore delle pressioni ambientali $P = (p_1,...,p_{N_p})$, è possibile calcolare l'influenza sul recettore ambientale $r_i$ per mezzo della matrice $N$, che collega pressioni e recettori:
\begin{equation}
\label{eq:pressRec}
	\forall j \in \{1,...,N_r\}  \quad  r_j = \sum_{i=1}^{N_p} n_j^i p_i
\end{equation}

L'obiettivo finale della valutazione di una politica ambientale è stimare l'impronta ambientale di un piano. Poiché il piano è definito da un insieme di valori che rappresentano l'ordine di grandezza delle attività previste, è possibile calcolare l'impronta ambientale $R = (r_1,...,r_{N_r})$ semplicemente applicando le equazioni (\ref{eq:impattiAttTot}) e (\ref{eq:pressRec}).
Un altro importante interrogativo che potrebbe interessare l'utente del sistema di supporto alle decisioni è sapere quali attività, tra quelle possibili, abbiano un maggiore impatto sul recettore $r_i$. Dal momento CLP(R) prevede di poter massimizzare o minimizzare una funzione obiettivo, il modello diventa:
\begin{equation}
\label{eq:impactOnGivenRec}
	\begin{aligned}
		max(r_i) \\
		s.t.  \quad (\ref{eq:impattiAttTot}) (\ref{eq:pressRec}) \\
		\sum_j a_j = 1 \\
		\forall j, a_j \mbox{ è intero} 
	\end{aligned}
\end{equation}

Infine, se ci sono leggi che impongono limiti per alcuni recettori (ad esempio sulla $CO_2$), è immediato imporre tali restrizioni sui recettori (es. $r_{CO_2} \leq limite_{CO_2}$), scoprendo quindi se un'attività può essere portata a termine o è necessario intraprendere misure correttive (ad esempio stimolare la crescita di nuove foreste per il $r_{CO_2}$).
\\* \\*
Nel casi in cui più attività alternative possano contribuire a soddisfare gli stessi bisogni, la regolamentazione prevede che le alternative debbano essere prese in esame e confrontate. Ad esempio, la necessità di energia elettrica aggiuntiva è soddisfatta dalla costruzione di una nuova centrale; in ogni caso, è possibile scegliere il tipo di impianto da realizzare in base alle condizioni ambientali. In un'area dall'elevato inquinamento atmosferico la costruzione di una centrale termoelettrica rischierebbe di aumentare il livello di inquinamento oltre la soglia imposta dalla legge, quindi sarebbe necessario progettare un tipo di impianto differente, ad esempio uno a energia solare. D'altro canto, un impianto solare potrebbe rivelarsi molto costoso e precludere dunque la realizzazione di altre attività indispensabili nell'aerea (costruire scuole, ospedali, etc.). In una simile circostanza il responsabile del piano potrebbe imporre un vincolo che afferma che sono necessari almeno $k$ MW di energia elettrica:
\begin{equation}
\label{eq:kEner}
	\sum_{i \in ImpiantiEnergetici} a_i \geq k
\end{equation}
(dove $ImpiantiEnergetici$ è l'insieme degli indici del vettore $A$ corrispondenti agli impianti che producono energia elettrica) e poi ottimizzare per uno dei recettori, es. $r_{CO_2}$, o qualche somma pesata dei ricettori d'interesse. O ancora, sarebbe possibile chiedere al DSS quale sia la massima produzione energetica possibile nella regione senza violare il limiti imposti dalla legge:  
\begin{equation}
\label{eq:maxOutLimLaw}
	\begin{aligned}
		max \sum_{{i \in ImpiantiEnergetici}}  a_i\\
		s.t. \quad (\ref{eq:impattiAttTot}) (\ref{eq:pressRec}) \\
		\forall i \in \{ 1,...,N_r \} \quad r_i \leq limite_i
	\end{aligned}
\end{equation}
In questo modo è possibile trovare il massimo numero di MW che possono essere prodotti e anche l'energia elettrica generata da ogni impianto. Il risolutore del problema CLP può inoltre trovare una soluzione che imponga l'esecuzione di attività di compensazione, come suggerito in precedenza.

\myparagraph{ANALISI DELLA SENSIBILITÀ}
%D.3 6.3

\myparagraph{PIANIFICAZIONE CON CLP(R)}
%gavanelliEtAl 1.3 D.3 8

\section{IL PIANO REGIONALE 2011-2013}
%gavanelliEtAl 1.4

\subsection{VALORE AGGIUNTO DEL CLP}
%gavanelliEtAl 1.5


%  ?gavanelliEtAl 1.6  D.3 10

\nocite{*}
\bibliographystyle{plain}
\bibliography{bibliography}

\end{document}
