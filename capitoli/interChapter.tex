%Andrea Borghesi
%Università degli studi di Bologna

%capitolo dedicato all'interazione tra ottimizzatore e simulatore

%\documentclass[12pt,a4paper,openright,twoside]{report}
%\usepackage[italian]{babel}
%\usepackage{indentfirst}
%\usepackage[utf8]{inputenc}
%\usepackage[T1]{fontenc}
%\usepackage{fancyhdr}
%\usepackage{graphicx}
%\usepackage{titlesec,blindtext, color}
%\usepackage[font={small,it}]{caption}
%\usepackage{subfig}
%\usepackage{listings}
%\usepackage{color}
%\usepackage{url}
%\usepackage{textcomp}
%\usepackage{eurosym}
%\usepackage{amsmath}
%\usepackage{url}

%%impostazioni generali per visualizzare codice
%\definecolor{dkgreen}{rgb}{0,0.6,0}
%\definecolor{gray}{rgb}{0.5,0.5,0.5}
%\definecolor{mauve}{rgb}{0.58,0,0.82}
% 
%\lstset{ %
%  basicstyle=\footnotesize,           % the size of the fonts that are used for the code
%  backgroundcolor=\color{white},      % choose the background color. You must add \usepackage{color}
%  numbers=left,                   % where to put the line-numbers
%  numberstyle=\tiny\color{gray},  % the style that is used for the line-numbers
%  numbersep=5pt,  
%  showspaces=false,               % show spaces adding particular underscores
%  showstringspaces=false,         % underline spaces within strings
%  showtabs=false,                 % show tabs within strings adding particular underscores
%  rulecolor=\color{black}, 
%  tabsize=2,                      % sets default tabsize to 2 spaces
%  breaklines=true,                % sets automatic line breaking
%  breakatwhitespace=false,        % sets if automatic breaks should only happen at whitespace
%  title=\lstname,                   % show the filename of files included with \lstinputlisting;
%  frame=single,                   % adds a frame around the code
%                                  % also try caption instead of title
%  keywordstyle=\color{blue},          % keyword style
%  commentstyle=\color{dkgreen},       % comment style
%  stringstyle=\color{mauve},         % string literal style
%  escapeinside={\%*}{*)},            % if you want to add LaTeX within your code
%  morekeywords={*,...},              % if you want to add more keywords to the set
%  deletekeywords={...}              % if you want to delete keywords from the given language
%}

%%per avere un bordo intorno alle figure
%\usepackage{float}
%\floatstyle{boxed} 
%\restylefloat{figure}

%%per poter poi impedire che certe parole vadano a capo
%\usepackage{hyphenat}

%%ridefinisco font per fancyhdr, per ottenere un'intestazione pulita
%\newcommand{\changefont}{ \fontsize{9}{11}\selectfont }
%\fancyhf{}
%\fancyhead[LE,RO]{\changefont \slshape \rightmark} 	%section
%\fancyhead[RE,LO]{\changefont \slshape \leftmark}	%chapter
%\fancyfoot[C]{\changefont \thepage}					%footer

%%titolo capitolo con "numero | titolo"
%\definecolor{gray75}{gray}{0.75}
%\newcommand{\hsp}{\hspace{20pt}}
%\titleformat{\chapter}[hang]{\Huge\bfseries}{\thechapter\hsp\textcolor{gray75}{|}\hsp}{0pt}{\Huge\bfseries}


%\oddsidemargin=30pt \evensidemargin=20pt

%%sillabazioni non eseguite correttamente
%\hyphenation{sil-la-ba-zio-ne pa-ren-te-si si-mu-la-to-re ge-ne-ra-re pia-no}

%%interlinea
%\linespread{1.15}  
%\pagestyle{fancy}

%%cartelle contenenti le immagini
%\graphicspath{{/media/sda4/tesi/immagini/grafici/}{/media/sda4/tesi/immagini/grafici/incCompare/}{/media/sda4/tesi/immagini/grafici/rawData/}{/media/sda4/tesi/immagini/grafici/regressionAnalysis/}{/media/sda4/tesi/immagini/schemi/}{/media/sda4/tesi/immagini/simulazione/}{/media/sda4/tesi/immagini/epolicy/}{/media/sda4/tesi/immagini/ottimizzazione/}
%{/media/sda4/tesi/immagini/interazione/}}

%%in modo che dopo il titolo di un paragrafo il testo vada a capo
%\newcommand{\myparagraph}[1]{\paragraph{#1}\mbox{}\\}

%%per scrivere bene CLP(R) e CLP(FD)
%\newcommand{\clpr}{CLP({\ensuremath{\cal R}})}
%\newcommand{\clpfd}{CLP({\ensuremath{\cal FD}})}

%\begin{document}
\clearpage{\pagestyle{empty}\cleardoublepage}
\chapter{\nohyphens{Interazione componenti}}
Nei capitoli precedenti abbiamo descritto gli elementi principali che concorrono a definire l'architettura del sistema ePolicy (certamente all'interno del progetto sono presente ulteriori componenti, come quello dedicato all'opinion mining, ma in questo caso ci riferiamo a quelli considerati in questo lavoro). Da un punto di vista generale essi sono il modello a vincoli che garantisce una ottimizzazione a livello globale e il simulatore che studia il comportamento delle modalità di incentivazione a livello locale/regionale. Un aspetto molto importante è quindi capire come gestire l'interazione tra ottimizzatore e simulatore in modo ottimale, in modo da integrare le prospettive globale e locale.

Possiamo illustrare la necessità di comprendere a fondo questa interazione con un esempio. Supponiamo che la fase di ottimizzazione abbia prodotto due scenari alternativi, il primo concentrandosi sulla creazione di impianti a biomassa e il secondo sostenendo la costruzione di centrali idroelettriche; entrambi i piani avrebbero un impatto non indifferente sui cittadini. La produzione energetica con la biomassa comporta un impatto sostanziale sulle aree boschive, potenziale inquinamento del suolo e delle coltivazioni, inquinamento dell'aria nelle aree urbane vicine alla centrale; d'altra parte, le centrali idroelettriche prevedono l'allagamento vaste porzioni di territorio. In ogni caso le strategie per implementare il piano, studiate tramite il simulatore, dovrebbero tenere in considerazione questi effetti sugli individui; le attività di implementazione implicherebbero quindi costi aggiuntivi che dovrebbero essere inseriti come nuovi vincoli all'interno dell'ottimizzatore, il quale potrebbe poi effettuare nuovamente la fase di pianificazione, con la possibilità di ottenere risultati diversi. 

Un approccio molto basilare sarebbe il semplice scambio di risultati tra i due livelli di pianificazione delle politiche, svolgendo anche diverse iterazioni, ma questo metodo rischierebbe di non garantire la convergenza. A un certo punto le iterazioni possono essere fermate quando un equilibrio è stato raggiunto o quando il decisore politico valuta che ulteriori aggiustamenti non siano più necessari o richiesti. Citando Clement Attlee\footnote{Fonte: A. Sampson, \emph{Anatomy of Britain}, Hodder \& Stoughton, 1962} \emph{``Democracy means government by discussion but it is only effective if you can stop people talking.''} - democrazia significa governo fondato sulla discussione, ma funziona solamente se si riesce a far smettere la gente di discutere.
\\*

Il tema dell'integrazione efficacie tra pianificazione regionale e simulazione è oggetto di intensa ricerca per ottenere una soluzione ottimale all'interno del progetto ePolicy (in modo particolare servendosi di metodologie appartenenti alla \emph{teoria dei giochi}); nel resto del capitolo verrà mostrato un possibile approccio, da noi implementato e messo alla prova con il problema dell'assegnazione dei fondi regionali per l'incentivazione della tecnologia fotovoltaica in Emilia-Romagna.

\section{Integrare DSS e simulazioni}
Un primo approccio a cui è possibile pensare, consiste nello sfruttare i metodi e le tecniche dell'\emph{Apprendimento Automatico} (noto in letteratura anche come \emph{Machine Learning}), il quale rappresenta un'area dell'Intelligenza Artificiale che si occupa della realizzazione di sistemi e algoritmi che si basano su serie di osservazioni e dati per la sintesi di nuova conoscenza. Senza voler entrare nel dettaglio, possiamo comunque citare una definizione comunemente accettata di Apprendimento Automatico: \emph{``Un programma apprende da una certa esperienza E se, con rispetto a una classe di compiti T e una misura delle prestazioni P, le prestazioni P nello svolgere un compito dell'insieme T sono migliorate dall'esperienza E''} \cite{Mitchell}.
 
Nel nostro caso abbiamo sfruttato le tecniche di regressione viste nei capitoli precedenti per ricavare le relazioni che legano i fondi destinati agli incentivi regionali con la produzione elettrica di energia elettrica fotovoltaica, partendo dalle serie di dati forniti dal simulatore, affinché fosse poi possibile inserirle all'interno dell'ottimizzatore sotto forma di ulteriori vincoli, da tenere in considerazione per la fase di pianificazione. Il nostro fine è stato quello di estrarre dalla grande quantità di dati generata dal simulatore delle informazioni utili per migliorare il modello del problema da ottimizzare.

La Figura~\ref{schemaIterLearn} mostra lo schema dell'interazione tra il livello globale e il livello locale realizzata tramite l'approccio dell'Apprendimento Automatico. Nella parte alta osserviamo il sistema di supporto alle decisioni, l'ottimizzatore, che, a fronte delle possibili decisioni (l'allocazione di risorse per lo svolgimento di attività per la produzione di energia energetica nel rispetto dei diversi vincoli), produce un piano (oppure un insieme di piani o scenari). Il modello del DSS è arricchito con i vincoli che vengono appresi nella fase di \emph{Learning} a partire dai risultati prodotti dal simulatore; in ingresso al simulatore troviamo un insieme di piani interessanti per la relazione che stiamo tentando di apprendere - ad esempio per studiare il rapporto tra i fondi investiti nel metodo di incentivazione Conto Interessi è stato necessario effettuare simulazioni per un ampio numero di valori di budget. 
\begin{figure}[htb]
	\begin{center}
	\includegraphics[scale=0.65]{schemaIterLearn}
	\end{center}
	\caption{Modello di interazione basato su Apprendimento Automatico}
  	\label{schemaIterLearn}
\end{figure}

La fase di apprendimento, e quindi le simulazioni, devono essere effettuate prima della fase di pianificazione (\emph{offline}), in quanto occorre inserire all'interno del modello i nuovi vincoli appresi, i quali non saranno più modificati (se non nel caso in cui vengano sostituiti da altri ricavati da un nuovo processo di apprendimento). \`E necessario effettuare un grande numero di simulazioni per garantire un valore statistico alle relazioni apprese e fornire un buon insieme di dati tramite cui effettuare l'apprendimento; questo rappresenta sicuramente il maggiore limite di questo tipo di approccio, in quanto comporta che i vincoli appresi non possano essere modificati facilmente - effettuare un gran numero di simulazioni richiede molto tempo - e l'interazione avvenga sostanzialmente in una sola direzione, dall'simulatore verso il DSS. 
\\*\\*
Una seconda metodologia per integrare ottimizzazione e simulazione da noi considerata, nonostante non sia stata concretamente implementata a differenza della prima, è una tecnica classica di decomposizione dei problemi presa in prestito dall'ambito della Ricerca Operativa, la cosiddetta \emph{decomposizione di Benders} \cite{bendersDec}. Essa consiste in un metodo per risolvere problemi di ottimizzazione combinatoria che possono essere scomposti in due componenti, un problema master e un sotto-problema. Originariamente era stata concepita per il campo della Programmazione Lineare Intera ma è stata in seguito estesa per trattare risolutori più generali, \emph{Logic-Based Benders Decomposition} (decomposizione di Benders basata sulla Logica) \cite{Hooker95logic-basedbenders}. 

Nel caso da noi preso in esame il problema master è la definizione del piano energetico regionale che partizioni l'energia necessaria tra le diverse fonti energetiche rinnovabili e viene risolto tramite il modello a vincoli descritto nel capitolo precedente. Il sotto-problema consiste nella definizione della strategia di incentivazione per raggiungere la produzione energetica desiderata, in modo consistente con i vincoli regionali sul budget. Partendo dalle soluzioni ottenute con l'ottimizzazione, ovvero la produzione energetica attesa, per comprendere quale sia il budget corretto da allocare per gli incentivi vengono portate a termine diverse simulazioni - in numero comunque molto inferiore rispetto all'interazione basata sull'Apprendimento Automatico. Nel caso in cui gli incentivi non siano compatibili con il budget regionale viene generato un cosiddetto \emph{taglio di Benders} (chiamati anche \emph{no-good}), cioè un vincolo che va ad aggiungersi al modello del problema master, e successivamente una nuova soluzione viene generata dal DSS. 

A differenza del primo metodo, con questo approccio la comunicazione tra i due componenti qui considerati viene estesa ad un ciclo, come si può facilmente notare in Figura~\ref{schemaIterBend}.

\begin{figure}[htb]
	\begin{center}
	\includegraphics[scale=0.65]{schemaIterBend}
	\end{center}
	\caption{Modello di interazione basato su Decomposizione di Benders}
  	\label{schemaIterBend}
\end{figure}

L'interazione inizia dall'ottimizzatore che fornisce una soluzione per il problema master, soluzione che contiene la produzione energetica attesa da fotovoltaico e dei valori ipotetici della dimensione dei fondi da destinare agli incentivi regionali. Questi valori ipotetici sono passati al simulatore, il quale esegue delle simulazioni esclusivamente con tali parametri forniti dal DSS e produce le corrispondenti statistiche (il tempo di calcolo è di qualche ordine di grandezza minore rispetto all'approccio basato sull'apprendimento); queste ultime possono confermare o meno i valori ipotizzati in fase di ottimizzazione: se il valore (medio) di produzione energetica ottenuto dalle simulazioni è maggiore o uguale di quello atteso, l'iterazione può concludersi è il risultato è probabilmente ottimale \cite{bendersDec}. Viceversa se invece il valore atteso è maggiore di quello simulato un'altra iterazione è necessaria, quindi  all'ottimizzatore è comunicato un nuovo vincolo, il quale può essere visto come spiegazione del fatto che non è possibile ottenere la produzione energetica richiesta con i fondi agli incentivi ipotizzati. A questo punto il DSS inserisce il vincolo all'interno del modello del problema, ricerca nuovamente una soluzione ottimale e ipotizza nuovi valori da fornire in input al simulatore.

La sfida principale consiste nel determinare l'insieme dei vincoli che vengono trasferiti tra le due componenti: se venissero esclusi dall'insieme dei valori ammissibili solamente quelli ipotizzati - e trovati non adatti grazie alle simulazioni - si correrebbe il rischio di effettuare troppe iterazioni, arrivando al caso limite di effettuare una simulazione esaustiva per tutti i parametri (in pratica verrebbero nuovamente fatte delle simulazioni per ogni valore del budget per gli incentivi regionali); se invece dall'insieme dei valori  venissero esclusi (troppi) valori ulteriori il pericolo sarebbe quello di scartare delle soluzioni promettenti. Questo tema e l'implementazione effettiva di questo secondo approccio sono attualmente oggetto di ricerca.

\section{Regressione lineare a tratti}
Passiamo ora a discutere del modo in cui l'approccio basato sull'Apprendimento Automatico sia stato implementato nel nostro modello a vincoli.
\\*\\*
Come è stato descritto nel terzo capitolo, tramite un grande numero di simulazioni è stato possibile ottenere una grande quantità di dati dalla quale abbiamo successivamente ricavato le relazioni che legano i fondi per gli incentivi regionale alla produzione energetica da fotovoltaico e quest'ultima alla forza dell'interazione sociale tra gli agenti. Tali relazioni sono state espresse sotto forma di funzioni e corrispondenti curve, ottenute attraverso l'applicazione di tecniche di regressione. A questo punto la nostra intenzione è stata quella di integrare queste funzioni all'interno modello a vincoli del problema di ottimizzazione, aggiungendo cioè i nuovi vincoli appresi grazie alle simulazioni svolte; è sorto quindi un problema, poiché, come descritto nel capitolo precedente, il risolutore dai noi utilizzato l'ottimizzazione gestisce esclusivamente equazioni lineari - per motivi di efficienza. Dal momento che modificare questa caratteristica, ovvero impiegare un risolutore in grado di trattare le funzioni quadratiche e di grado anche superiore ricavate dalla regressione, avrebbe richiesto cambiamenti radicali nella struttura generale e nel codice dell'ottimizzatore, abbiamo ritenuto che fosse meglio procedere in un altro modo, che ci consentisse di preservare la linearità del modello a vincoli sviluppato. Per questo motivo abbiamo deciso di tentare di rendere lineari le relazioni  ottenute con la regressione sfruttando una tecnica matematica definita \emph{approssimazione lineare a tratti} \cite{piecewiseApprox,Cattafi} (dall'inglese, \emph{piece-wise linear approximation}), che consiste nell'approssimare un'arbitraria funzione con un insieme di equazioni lineari con la massima accuratezza possibile. 
Possiamo ad esempio osservare in Figura~\ref{piecewiseApprox_example} l'approssimazione di una semplice funzione quadratica (in blu) attraverso cinque funzioni lineari (in rosso). 

\begin{figure}[htb]
	\begin{center}
	\includegraphics[scale=0.8]{piecewiseApprox_example}
	\end{center}
	\caption{Una funzione (in blu) e la sua approssimazione lineare a tratti (in rosso). Fonte {\tt http://commons.wikimedia.org/wiki/File:Finite\_element\_method\_1D\_illustration1.svg}}
  	\label{piecewiseApprox_example}
\end{figure}

Illustreremo adesso il funzionamento di questo metodo. Data una funzione (anche non lineare) $y=f(x)$, campioniamo la curva $g$ in $k$ punti $x_1,...,x_k$ e l'approssimazione lineare a tratti $y'=g'(x)\simeq g(x)$ è definita come
\begin{equation} 
\label{eq:aprroxEqX}
	x = \sum_{i=1}^k \lambda_i \cdot x_i,
\end{equation}
\begin{equation} 
\label{eq:aprroxEqY}
	y = \sum_{i=1}^k \lambda_i \cdot y_i,
\end{equation}
dove $\lambda_i \in [0..1]$ sono variabili continue soggette ai vincoli:
\begin{equation} 
\label{eq:aprroxEqLambda}
		\sum_{i=1}^k \lambda_i = 1
\end{equation}
Al massimo due $\lambda_i$ possono essere diverse da zero e in tal caso queste devono essere adiacenti.

Chiaramente queste ultime due condizioni non sono lineari, ma potrebbero essere modellate in un problema di Programmazione Logica Intera introducendo nuove 0-1 variabili intere, ma esiste una opzione più efficiente. In molti risolutori - compreso quello da noi impiegato - è possibile dichiarare $(\lambda_1,...,\lambda_k)$ come Special Order Set del secondo tipo (SOS2) \cite{bealeTomlin}, cioè un insieme ordinato di variabili utilizzato per specificare  determinate condizioni in problemi di ottimizzazione, e il risolutore sfrutterà questa informazione per ricercare una soluzione ottimale in modo più efficiente (in pratica, sapere che una variabile appartiene ad un certo insieme ordinato consente di usare in modo più intelligente gli algoritmi di branch-and-bound del solver).

\subsection{Implementazione in R}

Le informazioni necessarie per poter inserire le equazioni (\ref{eq:aprroxEqX}), (\ref{eq:aprroxEqY}) e (\ref{eq:aprroxEqLambda}) all'interno del modello a vincoli sono le coordinate dei punti di campionamento. Per trovarle ci siamo serviti del precedentemente introdotto R e in particolare del pacchetto software \emph{Segmented} \cite{segmentedPackage}. Grazie ad esso è stato molto semplice trovare un'ottima approssimazione lineare per le funzioni che legavano il budget alla produzione energetica (una per ogni tipo di incentivo), come si può facilmente osservare nel codice qui presentato.

\lstset{language=R}
\begin{lstlisting}
> library(segmented)
> #  ...
> # inserisci i dati delle simulazioni in apposite strutture 
> #  ...
> # operazioni varie (ordina dati, etc.)
> #  ...
> # estrai un modello lineare a tratti per l'incentivo Conto Interessi
> modelloLineareATratti_CI <- segmented(modelloGrezzo_CI,seg.Z=~ Budget,psi=c(3))
> # estrai un modello lineare a tratti per l'incentivo Fondo Garanzia
> modelloLineareATratti_FG <- segmented(modelloGrezzo_FG,seg.Z=~ Budget,psi=c(12,30))
> #  ...
> # incentivi restanti
> #  ...
\end{lstlisting}

Una volta ricavate le approssimazioni lineari a tratti delle funzioni, è possibile visualizzare il risultato ottenuto, come riportato in Figura~\ref{incentCompare_piecewise}.

\begin{figure}[hbt]
	\centering
	\includegraphics[scale=0.6]{incentComparePiecewise}
	\caption{Confronto tra i diversi incentivi - Approssimazione lineare a tratti}
	\label{incentCompare_piecewise}
\end{figure}

\section{Integrazione modello}

Possiamo ora descrivere in che modo abbiamo inserito le relazioni approssimate all'interno del modello a vincoli dell'ottimizzatore. In particolare mostreremo quali estensioni siano state aggiunte al modello per permettere al risolutore di calcolare l'assegnazione ottima dei fondi disponibili ai vari tipi di incentivo, con lo scopo di massimizzare la produzione di energia.

\subsection{Variabili}

Innanzitutto per ogni tipologia di incentivo regionale sono state introdotte due variabili che rappresentano il costo associato e la relativa produzione energetica. Per i costi le variabili sono chiamate $costo_A$, $costo_{CI}$, $costo_R$, $costo_G$, con $costo_j \in[0..50]$ (dominio espresso in milioni di euro) - rispettivamente incentivo in conto capitale (denominato in precedenza anche Fondo Asta), Conto Interessi, Fondo Rotazione e Fondo Garanzia. In modo simile, alle produzioni energetiche garantite dagli incentivi abbiamo associato delle variabili chiamate $prod_A$, $prod_{CI}$, $prod_R$, $prod_G$, con $prod_j \in [0..10]$ (dominio espresso in MWatt); il codice relativo alla creazione di tali variabili è qui riportato.

\lstset{language=Prolog}
\begin{lstlisting}
%creazione istanza eplex
:- eplex_instance(eplex_instance).

%predicato che modella il problema dell'assegnazione ottimale dei fondi agli incentivi regionali
assegna_fondi(...):-
	%crea le variabili per i costi  (una per ogni tipo di incentivo specificato nella lista TipiInc)
	crea_var_names(TipiInc,Costi,0,50),
	%crea le variabili per le produzioni
	crea_var_names(TipiInc,Prods,0,60),
	...
\end{lstlisting}

Occorre fare subito un'importante precisazione, ovvero spiegare che per produzione energetica associata ad ogni metodologia incentivante abbiamo in questo caso inteso la produzione di energia non in termini assoluti, bensì la differenza (aumento) di produzione energetica che si ottiene finanziando un tipo di incentivo rispetto al caso in cui nessun meccanismo incentivante sia supportato (il valore di produzione energetica associato a nessuna incentivazione è definito dalla costante $PR_{base}$). Questa scelta è stata dettata dal fatto che in questa maniera fosse molto facile valutare il guadagno in termini di produzione offerto dai diversi incentivi.

Abbiamo poi introdotto le variabili ausiliarie necessarie all'approssimazione lineare chiamate $\lambda_i$ nell'equazione (\ref{eq:aprroxEqLambda}) e ugualmente nominate all'interno del problema e sempre con $\lambda_i \in [0..1]$; per ciascun tipo di incentivo e per ogni punto scelto per la campionatura sulle curve budget-produzione è stata introdotta una variabile ausiliaria $\lambda_{ji}$, con $j$ indice per il tipo di incentivo e $i$ per il punto campionato. Come scelta iniziale i punti utilizzati per la campionatura sono il minor numero possibile tramite cui rappresentare correttamente la funzione approssimata, cioè i punti in cui si osserva un cambiamento del gradiente della curva; in particolare abbiamo usato tre punti per gli incentivi Fondo Asta, Conto Interessi e Fondo Rotazione e quattro per il Fondo Garanzia. Per maggiore precisione, osserviamo che le coordinate dei punti per la campionatura sono quelle ricavate tramite le semplici righe di codice scritte in \emph{R} e descritte nella sezione precedente. 

\subsection{Vincoli}

Dopo le variabili occorre esprimere i vincoli che consentono di modellare il problema. Come prima cosa è necessario trasporre sotto forma di vincoli le equazioni descritte in precedenza (\ref{eq:aprroxEqX}), (\ref{eq:aprroxEqY}) e (\ref{eq:aprroxEqLambda}), utilizzate per approssimare le curve delle relazioni con delle funzioni lineari a tratti; nell'equazione \ref{eq:vincoloAuxY} il termine $y_{ji}-PR_{base}$ serve per calcolare la differenza di produzione energetica rispetto al caso di nessun incentivo.

Per chiarezza ricordiamo che gli indici $j$ e $i$ selezionano rispettivamente metodo incentivante e punto di campionamento e quindi nei vincoli ~\ref{eq:vincoloAuxX}, ~\ref{eq:vincoloAuxY} e ~\ref{eq:vincoloAuxLambda} si ha $j \in [1..4]$.

\begin{equation} \label{eq:vincoloAuxX}
	costo_j = \sum_{i=1}^k \lambda_{ji} \cdot x_{ji} 
\end{equation}
\begin{equation} \label{eq:vincoloAuxY}
	prod_j = \sum_{i=1}^k \lambda_{ji} \cdot (y_{ji}-PR_{base}) 
\end{equation}
\begin{equation} \label{eq:vincoloAuxLambda}
	\sum_{i=1}^k \lambda_{ji} \leq 1
\end{equation}

Nel codice seguente è mostrata la creazione delle variabili ausiliarie e l'imposizione di questi vincoli.
\begin{lstlisting}
%predicato che modella l'approssimazione lineare a tratti
%viene invocato da assegna_fondi tante volte quanti sono i tipi di incentivo
piecewise_linear_model(TipoIncentivo,Costo_inc,Prod_inc):-
	%creo tante variabili ausiliarie quanti sono i punti che caratterizzano l'incentivo
	crea_var_sub(Punti,Auxs,0,1),
	...
	%vincoli per le variabili ausiliarie
	eplex_instance:(sum(Auxs) $=< 1),
	eplex_instance:(Costo_inc $= (Auxs*Xs) ),
	eplex_instance:(Prod_inc $= (Auxs*YsSub) ),
	...
\end{lstlisting}

A questo punto le funzioni approssimate (quelle rappresentate in Fig.~\ref{incentCompare_piecewise}) sono state inserite all'interno del modello e il passo successivo consiste nello sfruttamento di questa conoscenza per la ricerca della distribuzione ottimale dei fondi da parte del risolutore del problema.

Per questo motivo, imponiamo che la somma dei costi dei vari incentivi sia minore del budget totale dedicato agli incentivi regionali per l'energia fotovoltaica, indicato dalla costante $BUDGET_{PV}$.

\begin{equation} \label{eq:vincoloCosti}
	\sum_{j=1}^4 costo_j \leq BUDGET_{PV}
\end{equation}

In un secondo momento, specifichiamo la funzione obiettivo fornita al risolutore del problema a vincoli, il quale dovrà di massimizzare la produzione energetica assegnando il budget disponibile ai vari incentivi nel modo più efficacie possibile.

\begin{equation} \label{eq:vincoloCosti}
	\max ( \sum_{j=1}^4 prod_j )
\end{equation}

Riportiamo quindi l'implementazione.

\begin{lstlisting}
	(assegna_fondi)
	...
	%la somma dei fondi assegnati ad ogni incentivo deve essere minore del budget per il PV
	eplex_instance:(sum(Costi) $=< BudgetPV ),
	%inserisco in liste apposite le variabili ausiliarie che devono formare SOS2 (un SOS2 per tipo di incentivo)
	get_aux_sos(Auxs,AuxA,AuxCI,AuxR,AuxG,"A"),
	%la funzione obiettivo cerca di massimizzare la produzione energetica
	eplex_instance:(VarObiettivo $= (sum(Prods))),
	%inizializzo il solver con la funzione obiettivo specificata indicando anche di sfruttare SOS2 per la risoluzione ottimizzata 
	eplex_instance:eplex_solver_setup(max(sum(Outs)),VarObiettivo,[...],[...]),
	...
\end{lstlisting}

\section{Assegnazione fondi}

Per valutare ora l'interazione tra il DSS e il simulatore analizzeremo ora in che modo vengono finanziati le diverse tipologie di incentivazione sulla base del problema a vincoli presentato in precedenza.

Per come il sotto-problema dell'assegnazione dei fondi è stato definito, il risolutore procede distribuendo i fondi a partire dall'incentivo che risulta essere più efficace e prosegue poi finanziando gli incentivi restanti fino a esaurimento del budget previsto. Per determinare quale sia il tipo di incentivo più efficace vengono sfruttate le funzioni lineari a tratti che approssimano le relazioni tra budget e produzione energetica. In Figura~\ref{assegnFondi} osserviamo un esempio di distribuzione dei fondi, con un budget pari a dodici milioni di euro: le quattro funzioni lineari a tratti sono le relazioni inserite all'interno del modello e i punti individuati su di esse rappresentano le combinazioni di spesa (la spesa viene mostrata sull'asse delle ascisse) migliori dal punto di vista della produzione (asse delle ordinate); ad esempio, notiamo come la massima produzione energetica possibile per il Conto Interessi (in rosso) sia raggiunta con una spesa relativamente contenuta e minore del budget totale, consentendo di distribuire i fondi rimanenti ad altri tipi di incentivo.

\begin{figure}[hbt]
	\centering
	\includegraphics[scale=0.7]{assegnFondi}
	\caption{Assegnazione Fondi - Budget \euro12M }
	\label{assegnFondi}
\end{figure}

Nelle tabelle seguenti sarà mostrato in che modo vengono assegnati i fondi ai vari incentivi a partire da un determinato budget, evidenziando anche l'aumento di produzione energetica ottenibile rispetto al caso in cui nessun incentivo venga finanziato; per avere un riferimento ricordiamo che nel nostro simulatore, quindi non adeguatamente scalato, il valore di produzione energetica ottenibile in assenza di meccanismi incentivanti è di circa 21.664MW. Ricordiamo inoltre che, per come è stato progettato il simulatore, il budget destinato agli incentivi è distribuito lungo un arco temporale quinquennale. 

\myparagraph{Fondo incentivi – \euro2.5M}

Con questo fondo per gli incentivi viene finanziato totalmente il Conto Interessi consumando interamente il budget disponibile, impedendo quindi di stanziare fondi anche agli altri incentivi, e ottenendo un aumento di produzione energetica di 3.563MW.

\begin{table}[h]
\centering
	\begin{tabular}{ p{0.3\textwidth}  | p{0.3\textwidth} | p{0.3\textwidth}  }
		\hline \hline 
		\nohyphens{\emph{Tipologia Incentivo}} & \nohyphens{\emph{Costo (M\euro)}} & \nohyphens{\emph{Produzione Energetica Differenziale(MW)}} \\ \hline
		Asta &  0.00 & 0.000 \\ 
		Conto Interessi & 2.50 & 3.563 \\ 
		Rotazione & 0.00 & 0.000 \\ 
		Garanzia & 0.00 & 0.000 \\ \hline 
		Totale & 2.50 & 3.563 \\
		\hline \hline 
	\end{tabular}
	\caption{Assegnazione Fondi - \euro2.5M}
	\label{tab:assegnFondi2M}	
\end{table}

\myparagraph{Fondo incentivi – \euro5M}

In questo caso è possibile finanziare interamente il Conto Interessi (il quale, come atteso, richiede decisamente meno risorse degli altri, ovvero potendo essere completamente soddisfatto con 2.53 milioni di euro) e distribuire il budget restante agli incentivi Fondo Rotazione e Garanzia, ottenendo così un miglioramento della produzione energetica di 4.121MW.

\begin{table}[h]
\centering
	\begin{tabular}{ p{0.3\textwidth}  | p{0.3\textwidth} | p{0.3\textwidth}  }
		\hline \hline 
		\nohyphens{\emph{Tipologia Incentivo}} & \nohyphens{\emph{Costo (M\euro)}} & \nohyphens{\emph{Produzione Energetica Differenziale(MW)}} \\ \hline
		Asta &  0.00 & 0.000 \\ 
		Conto Interessi & 2.53 & 3.606 \\ 
		Rotazione & 1.00 & 0.286 \\ 
		Garanzia & 1.00 & 0.170 \\ \hline 
		Totale & 5.00 & 4.121 \\
		\hline \hline 
	\end{tabular}
	\caption{Assegnazione Fondi - \euro5M}
	\label{tab:assegnFondi5M}	
\end{table}

\myparagraph{Fondo incentivi – \euro10M}

Con un ulteriore aumento del budget l'assegnazione dei fondi ottimale prevede di distribuire prima al Conto Interessi seguito sempre dai Fondi Garanzia e Rotazione, ottenendo un incremento della produzione energetica pari a 4.708MW; in realtà il scendo miglior incentivo è il Fondo Garanzia e quindi esso riceve finanziamenti maggiori rispetto al Rotazione, a quest'ultimo viene comunque assegnato un budget di un milione di euro (anche nel caso precedente) in quanto, per come sono stati campionati i punti che definiscono l'approssimazione lineare a tratti, a tale budget è associata una produzione energetica migliore rispetto a quella del Fondo Garanzia.

\begin{table}[h]
\centering
	\begin{tabular}{ p{0.3\textwidth}  | p{0.3\textwidth} | p{0.3\textwidth}  }
		\hline \hline 
		\nohyphens{\emph{Tipologia Incentivo}} & \nohyphens{\emph{Costo (M\euro)}} & \nohyphens{\emph{Produzione Energetica Differenziale(MW)}} \\ \hline
		Asta &  0.00 & 0.000 \\ 
		Conto Interessi & 2.53 & 3.606 \\ 
		Rotazione & 1.00 & 0.286 \\ 
		Garanzia & 6.47 & 0.816 \\ \hline 
		Totale & 10.00 & 4.708 \\
		\hline \hline 
	\end{tabular}
	\caption{Assegnazione Fondi - \euro10M}
	\label{tab:assegnFondi10M}	
\end{table}

\myparagraph{Fondo incentivi – \euro15M}

Aumentando i fondi stanziati dalla regione di altri cinque milioni, si ottiene come risultato la distribuzione dei finanziamenti aggiuntivi esclusivamente al Fondo Garanzia, mentre per le restanti tipologie di incentivo sono previste le stesse spese del caso di budget pari a dieci milioni; ad ogni modo, con un fondo per l'incentivazione di quindici milioni di euro l'incremento della produzione energetica è di 5.34MW.

\begin{table}[h]
\centering
	\begin{tabular}{ p{0.3\textwidth}  | p{0.3\textwidth} | p{0.3\textwidth}  }
		\hline \hline 
		\nohyphens{\emph{Tipologia Incentivo}} & \nohyphens{\emph{Costo (M\euro)}} & \nohyphens{\emph{Produzione Energetica Differenziale(MW)}} \\ \hline
		Asta &  0.00 & 0.000 \\ 
		Conto Interessi & 2.53 & 3.606 \\ 
		Rotazione & 1.00 & 0.286 \\ 
		Garanzia & 11.47 & 1.447 \\ \hline 
		Totale & 10.00 & 5.340 \\
		\hline \hline 
	\end{tabular}
	\caption{Assegnazione Fondi - \euro15M}
	\label{tab:assegnFondi15M}	
\end{table}

\myparagraph{Fondo incentivi – \euro20M}

Con un budget di venti milioni di euro notiamo due differenze. 

La prima è che dopo aver assegnato circa 13 milioni e mezzo al Fondo Garanzia diventa più utile destinare i fondi rimanenti al Fondo Rotazione, ricavando un aumento della produzione di 5.923MW; questo accade perché in corrispondenza di quel budget la pendenza della curva che descrive l'incentivo Fondo Garanzia diminuisce sensibilmente (Fig.~\ref{incentCompare_piecewise}), rendendo quindi l'incentivo meno efficace in confronto al Fondo Rotazione.

La seconda differenza è che conviene anche destinare una minima parte (un milione di euro) del budget anche al Fondo Asta e questo è probabilmente dovuto nuovamente al modo in cui sono stati campionati i punti per l'approssimazione.

\begin{table}[h]
\centering
	\begin{tabular}{ p{0.3\textwidth}  | p{0.3\textwidth} | p{0.3\textwidth}  }
		\hline \hline 
		\nohyphens{\emph{Tipologia Incentivo}} & \nohyphens{\emph{Costo (M\euro)}} & \nohyphens{\emph{Produzione Energetica Differenziale(MW)}} \\ \hline
		Asta &  1.00 & 0.125 \\ 
		Conto Interessi & 2.53 & 3.606 \\ 
		Rotazione & 2.73 & 0.458 \\ 
		Garanzia & 13.74 & 1.734 \\ \hline 
		Totale & 20.00 & 5.923 \\
		\hline 
	\end{tabular}
	\caption{Assegnazione Fondi - \euro20M}
	\label{tab:assegnFondi20M}	
\end{table}

\myparagraph{Fondo incentivi – \euro40M}

Con un budget di 40 milioni di euro i risultati sono molto simili al caso precedente, con la differenza che i maggiori finanziamenti disponibili vengono interamente indirizzati al Fondo Rotazione, poichè il Conto Interessi è in grado di rendere al massimo anche con una spesa minima e il Fondo Garanzia diventa meno efficace con budget maggiori di 13 milioni e mezzo di euro (come anche il Fondo Asta con qualsiasi budget); l'incremento di produzione energetica rispetto alla mancanza di metodologie incentivanti è di 7.913MW.

\begin{table}[h]
\centering
	\begin{tabular}{ p{0.3\textwidth}  | p{0.3\textwidth} | p{0.3\textwidth}  }
		\hline \hline 
		\nohyphens{\emph{Tipologia Incentivo}} & \nohyphens{\emph{Costo (M\euro)}} & \nohyphens{\emph{Produzione Energetica Differenziale(MW)}} \\ \hline
		Asta &  1.00 & 0.125 \\ 
		Conto Interessi & 2.53 & 3.606 \\ 
		Rotazione & 22.73 & 2.448 \\ 
		Garanzia & 13.74 & 1.734 \\ \hline 
		Totale & 40.00 & 7.913 \\
		\hline 
	\end{tabular}
	\caption{Assegnazione Fondi - \euro40M}
	\label{tab:assegnFondi40M}	
\end{table}

%\nocite{*}
%\bibliographystyle{plain}
%\bibliography{bibliography}

%\end{document}
