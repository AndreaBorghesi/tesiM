%Andrea Borghesi
%Università degli studi di Bologna

%capitolo dedicato all'interazione tra ottimizzatore e simulatore

\documentclass[12pt,a4paper,openright,twoside]{report}
\usepackage[italian]{babel}
\usepackage{indentfirst}
\usepackage[utf8]{inputenc}
\usepackage[T1]{fontenc}
\usepackage{fancyhdr}
\usepackage{graphicx}
\usepackage{titlesec,blindtext, color}
\usepackage[font={small,it}]{caption}
\usepackage{subfig}
\usepackage{listings}
\usepackage{color}
\usepackage{url}
\usepackage{textcomp}
\usepackage{eurosym}
\usepackage{amsmath}

%impostazioni generali per visualizzare codice
\definecolor{dkgreen}{rgb}{0,0.6,0}
\definecolor{gray}{rgb}{0.5,0.5,0.5}
\definecolor{mauve}{rgb}{0.58,0,0.82}
 
\lstset{ %
  basicstyle=\footnotesize,           % the size of the fonts that are used for the code
  backgroundcolor=\color{white},      % choose the background color. You must add \usepackage{color}
  numbers=left,                   % where to put the line-numbers
  numberstyle=\tiny\color{gray},  % the style that is used for the line-numbers
  numbersep=5pt,  
  showspaces=false,               % show spaces adding particular underscores
  showstringspaces=false,         % underline spaces within strings
  showtabs=false,                 % show tabs within strings adding particular underscores
  rulecolor=\color{black}, 
  tabsize=2,                      % sets default tabsize to 2 spaces
  breaklines=true,                % sets automatic line breaking
  breakatwhitespace=false,        % sets if automatic breaks should only happen at whitespace
  title=\lstname,                   % show the filename of files included with \lstinputlisting;
  frame=single,                   % adds a frame around the code
                                  % also try caption instead of title
  keywordstyle=\color{blue},          % keyword style
  commentstyle=\color{dkgreen},       % comment style
  stringstyle=\color{mauve},         % string literal style
  escapeinside={\%*}{*)},            % if you want to add LaTeX within your code
  morekeywords={*,...},              % if you want to add more keywords to the set
  deletekeywords={...}              % if you want to delete keywords from the given language
}

%per avere un bordo intorno alle figure
\usepackage{float}
\floatstyle{boxed} 
\restylefloat{figure}

%per poter poi impedire che certe parole vadano a capo
\usepackage{hyphenat}
\usepackage{listings}

%ridefinisco font per fancyhdr, per ottenere un'intestazione pulita
\newcommand{\changefont}{ \fontsize{9}{11}\selectfont }
\fancyhf{}
\fancyhead[LE,RO]{\changefont \slshape \rightmark} 	%section
\fancyhead[RE,LO]{\changefont \slshape \leftmark}	%chapter
\fancyfoot[C]{\changefont \thepage}					%footer

%titolo capitolo con "numero | titolo"
\definecolor{gray75}{gray}{0.75}
\newcommand{\hsp}{\hspace{20pt}}
\titleformat{\chapter}[hang]{\Huge\bfseries}{\thechapter\hsp\textcolor{gray75}{|}\hsp}{0pt}{\Huge\bfseries}


\oddsidemargin=30pt \evensidemargin=20pt

%sillabazioni non eseguite correttamente
\hyphenation{sil-la-ba-zio-ne pa-ren-te-si si-mu-la-to-re ge-ne-ra-re pia-no}

%interlinea
\linespread{1.15}  
\pagestyle{fancy}

%cartelle contenenti le immagini
\graphicspath{{/media/sda4/tesi/immagini/grafici/}{/media/sda4/tesi/immagini/grafici/incCompare/}{/media/sda4/tesi/immagini/grafici/rawData/}{/media/sda4/tesi/immagini/grafici/regressionAnalysis/}{/media/sda4/tesi/immagini/schemi/}{/media/sda4/tesi/immagini/simulazione/}{/media/sda4/tesi/immagini/epolicy/}{/media/sda4/tesi/immagini/ottimizzazione/}
{/media/sda4/tesi/immagini/interazione/}}

%in modo che dopo il titolo di un paragrafo il testo vada a capo
\newcommand{\myparagraph}[1]{\paragraph{#1}\mbox{}\\}

\begin{document}

\chapter{\nohyphens{INTERAZIONE COMPONENTI}}
Nei capitoli precedenti abbiamo descritto gli elementi principali che concorrono a definire l'architettura del sistema ePolicy (certamente all'interno del progetto sono presente ulteriori componenti, come quello dedicato all'opinion mining, ma in questo caso ci riferiamo a quelli considerati in questo lavoro). Da un punto di vista generale essi sono il modello a vincoli che garantisce una ottimizzazione a livello globale e il simulatore che studia il comportamento delle modalità di incentivazione a livello locale/regionale. Un aspetto molto importante è quindi capire come gestire l'interazione tra ottimizzatore e simulatore in modo ottimale, in modo da integrare le prospettive globale e locale.

Possiamo illustrare la necessità di comprendere a fondo questa interazione con un esempio. Supponiamo che la fase di ottimizzazione abbia prodotto due scenari alternativi, il primo concentrandosi sulla creazione di impianti a biomassa e il secondo sostenendo la costruzione di centrali idroelettriche; entrambi i piani avrebbero un impatto non indifferente sui cittadini. La produzione energetica con la biomassa comporta un impatto sostanziale sulle aree boschive, potenziale inquinamento del suolo e delle coltivazioni, inquinamento dell'aria nelle aree urbane vicine alla centrale; d'altra parte, le centrali idroelettriche prevedono l'allagamento vaste porzioni di territorio. In ogni caso le strategie per implementare il piano, studiate tramite il simulatore, dovrebbero tenere in considerazione questi effetti sugli individui; le attività di implementazione implicherebbero quindi costi aggiuntivi che dovrebbero essere inseriti come nuovi vincoli all'interno dell'ottimizzatore, il quale potrebbe poi effettuare nuovamente la fase di pianificazione, con la possibilità di ottenere risultati diversi. 

Un approccio molto basilare sarebbe il semplice scambio di risultati tra i due livelli di pianificazione delle politiche, svolgendo anche diverse iterazioni, ma questo metodo rischierebbe di non garantire la convergenza. A un certo punto le iterazioni possono essere fermate quando un equilibrio è stato raggiunto o quando il decisore politico valuta che ulteriori aggiustamenti non siano più necessari o richiesti. Citando Clement Attlee\footnote{Fonte: A. Sampson, \emph{Anatomy of Britain}, Hodder \& Stoughton, 1962} \emph{``Democracy means government by discussion but it is only effective if you can stop people talking.''} - democrazia significa governo fondato sulla discussione, ma funziona solamente se si riesce a far smettere la gente di discutere.
\\*

Il tema dell'integrazione efficacie tra pianificazione regionale e simulazione è oggetto di intensa ricerca per ottenere una soluzione ottimale all'interno del progetto ePolicy (in modo particolare servendosi di metodologie appartenenti alla \emph{teoria dei giochi}); nel resto del capitolo verrà mostrato un possibile approccio, da noi implementato e messo alla prova con il problema dell'assegnazione dei fondi regionali per l'incentivazione della tecnologia fotovoltaica in Emilia-Romagna.

\section{INTEGRARE DSS E SIMULAZIONI}
Un primo approccio a cui è possibile pensare, consiste nello sfruttare i metodi e le tecniche dell'\emph{Apprendimento Automatico} (noto in letteratura anche come \emph{Machine Learning}), il quale rappresenta un'area dell'Intelligenza Artificiale che si occupa della realizzazione di sistemi e algoritmi che si basano su serie di osservazioni e dati per la sintesi di nuova conoscenza. Senza voler entrare nel dettaglio, possiamo comunque citare una definizione comunemente accettata di Apprendimento Automatico: \emph{``Un programma apprende da una certa esperienza E se, con rispetto a una classe di compiti T e una misura delle prestazioni P, le prestazioni P nello svolgere un compito dell'insieme T sono migliorate dall'esperienza E''} \cite{Mitchell}.
 
Nel nostro caso abbiamo sfruttato le tecniche di regressione viste nei capitoli precedenti per ricavare le relazioni che legano i fondi destinati agli incentivi regionali con la produzione elettrica di energia elettrica fotovoltaica, partendo dalle serie di dati forniti dal simulatore, affinché fosse poi possibile inserirle all'interno dell'ottimizzatore sotto forma di ulteriori vincoli, da tenere in considerazione per la fase di pianificazione. Il nostro fine è stato quello di estrarre dalla grande quantità di dati generata dal simulatore delle informazioni utili per migliorare il modello del problema da ottimizzare.

La Figura~\ref{schemaIterLearn} mostra lo schema dell'interazione tra il livello globale e il livello locale realizzata tramite l'approccio dell'Apprendimento Automatico. Nella parte alta osserviamo il sistema di supporto alle decisioni, l'ottimizzatore, che, a fronte delle possibili decisioni (l'allocazione di risorse per lo svolgimento di attività per la produzione di energia energetica nel rispetto dei diversi vincoli), produce un piano (oppure un insieme di piani o scenari). Il modello del DSS è arricchito con i vincoli che vengono appresi nella fase di \emph{Learning} a partire dai risultati prodotti dal simulatore; in ingresso al simulatore troviamo un insieme di piani interessanti per la relazione che stiamo tentando di apprendere - ad esempio per studiare il rapporto tra i fondi investiti nel metodo di incentivazione Conto Interessi è stato necessario effettuare simulazioni per un ampio numero di valori di budget. 
\begin{figure}[htb]
	\begin{center}
	\includegraphics[scale=0.65]{schemaIterLearn}
	\end{center}
	\caption{Modello di interazione basato su Apprendimento Automatico}
  	\label{schemaIterLearn}
\end{figure}

La fase di apprendimento, e quindi le simulazioni, devono essere effettuate prima della fase di pianificazione (\emph{offline}), in quanto occorre inserire all'interno del modello i nuovi vincoli appresi, i quali non saranno più modificati (se non nel caso in cui vengano sostituiti da altri ricavati da un nuovo processo di apprendimento). \`E necessario effettuare un grande numero di simulazioni per garantire un valore statistico alle relazioni apprese e fornire un buon insieme di dati tramite cui effettuare l'apprendimento; questo rappresenta sicuramente il maggiore limite di questo tipo di approccio, in quanto comporta che i vincoli appresi non possano essere modificati facilmente - effettuare un gran numero di simulazioni richiede molto tempo - e l'interazione avvenga sostanzialmente in una sola direzione, dall'simulatore verso il DSS.
\\*\\*
Una seconda metodologia per integrare ottimizzazione e simulazione da noi considerata, nonostante non sia stata concretamente implementata a differenza della prima, è una tecnica classica di decomposizione dei problemi presa in prestito dall'ambito della Ricerca Operativa, la cosiddetta \emph{decomposizione di Benders} \cite{bendersDec}. Essa consiste in un metodo per risolvere problemi di ottimizzazione combinatoria che possono essere scomposti in due componenti, un problema master e un sotto-problema. Originariamente era stata concepita per il campo della Programmazione Lineare Intera ma è stata in seguito estesa per trattare risolutori più generali, \emph{Logic-Based Benders Decomposition} (decomposizione di Benders basata sulla Logica) \cite{Hooker95logic-basedbenders}. 

Nel caso da noi preso in esame il problema master è la definizione del piano energetico regionale che partizioni l'energia necessaria tra le diverse fonti energetiche rinnovabili e viene risolto tramite il modello a vincoli descritto nel capitolo precedente. Il sotto-problema consiste nella definizione della strategia di incentivazione per raggiungere la produzione energetica desiderata, in modo consistente con i vincoli regionali sul budget. Partendo dalle soluzioni ottenute con l'ottimizzazione, ovvero la produzione energetica attesa, per comprendere quale sia il budget corretto da allocare per gli incentivi vengono portate a termine diverse simulazioni - in numero comunque molto inferiore rispetto all'interazione basata sull'Apprendimento Automatico. Nel caso in cui gli incentivi non siano compatibili con il budget regionale viene generato un cosiddetto \emph{taglio di Benders} (chiamati anche \emph{no-good}), cioè un vincolo che va ad aggiungersi al modello del problema master, e successivamente una nuova soluzione viene generata dal DSS. 

%ecms12 pag 4 dopo la figura 6

\section{REGRESSIONE LINEARE A TRATTI}
%a33-cattafi -> pag. 8-9

\subsection{IMPLEMENTAZIONE IN R}

\section{INTEGRAZIONE MODELLO}

\subsection{VARIABILI}

\subsection{VINCOLI}

\section{ASSEGNAZIONE FONDI}

\myparagraph{FONDO INCENTIVI – 2.5 MILIONI EURO}

\myparagraph{FONDO INCENTIVI – 5 MILIONI EURO}

\myparagraph{FONDO INCENTIVI – 10 MILIONI EURO}

\myparagraph{FONDO INCENTIVI – 15 MILIONI EURO}

\myparagraph{FONDO INCENTIVI – 20 MILIONI EURO}

\myparagraph{FONDO INCENTIVI – 40 MILIONI EURO}

\nocite{*}
\bibliographystyle{plain}
\bibliography{bibliography}

\end{document}
