%Andrea Borghesi
%Università degli studi di Bologna

%introduzione

%\documentclass[12pt,a4paper,openright,twoside]{book}
%\usepackage[italian]{babel}
%\usepackage{indentfirst}
%\usepackage[utf8]{inputenc}
%\usepackage[T1]{fontenc}
%\usepackage{fancyhdr}
%\usepackage{graphicx}
%\usepackage{titlesec,blindtext, color}
%\usepackage[font={small,it}]{caption}

%\begin{document}

\clearpage{\pagestyle{empty}\cleardoublepage}
\chapter*{Introduzione} 
\markboth{Introduzione}{Introduzione}
\addcontentsline{toc}{chapter}{Introduzione}

La definizione delle politiche pubbliche a livello nazionale, regionale o locale è un compito complesso, in quanto occorre operare in ambiti caratterizzati da dinamicità e incertezza, tentando di risolvere diverse problematiche e conciliando interessi conflittuali. Fattori come la globalizzazione o la sostenibilità ambientale rendono ancora più difficili le scelte che i decisori politici sono tenuti ad effettuare per l'ideazione e l'implementazione di strategie in grado di affrontare le sfide reali della società odierna, senza sottovalutare il fatto che l'elevata complessità dei sistemi considerati non consente di determinare facilmente gli effetti relativi alle decisioni prese.

Da tutto ciò consegue che sia profondamente avvertita l'esigenza di sviluppare metodologie e strumenti di cui i decisori politici si possano avvalere per gestire le problematiche di questo settore. In questa direzione procede lo sviluppo di modelli matematici e computazionali alla base dei sistemi di supporto alle decisioni politiche; tali sistemi devono essere in grado di fornire una serie di scenari decisionali alternativi, con i quali è possibile aiutare il politico a svolgere il proprio compito, ma certamente senza sostituirvisi. Prendere le decisioni senza un supporto informatico è estremamente difficile poiché sia esse che le loro interconnessioni, ovvero impatti e conseguenze che ne derivano, sono moltissime e anche perché occorre prendere in considerazione diversi aspetti, da quelli economici a quelli ambientali e sociali che hanno un grado di complessità intrinseca molto elevato.

Come esempio, basti pensare alle valutazioni da fare per l'ottimizzazione di uno o più recettori ambientali, come la qualità dell'aria o delle acque. Invece, per quanto riguarda gli aspetti sociali, è necessario tenere conto di come la società reagirà alle politiche che si vogliono implementare: ad esempio se a fronte di determinati meccanismi incentivanti i cittadini o gli imprenditori investiranno in impianti di energia da fonti rinnovabili. Disporre quindi di un sistema che modelli dal punto di vista matematico le decisioni di un piano (locale, regionale, nazionale e così via) permette di prendere in considerazione tutti questi aspetti contemporaneamente, in modo da generare politiche che abbiano impatti economici, sociali e ambientali accettabili e controllati. 
\\*

Il lavoro da noi svolto è che ci accingiamo a illustrare rientra nell'ambito sopra esposto. In particolare rientra all'interno del progetto e-Policy, VII Programma quadro dell'Unione Europea, dedicato allo sviluppo di sistemi di supporto ai decisori per produrre politiche sostenibili dal punto di vista ambientale e socialmente accettate; la regione Emilia-Romagna è partner di questo progetto e lo sviluppo del piano regionale energetico ha fornito il caso di studio per e-Policy e il lavoro in seguito presentato. Da un punto di vista molto generale, il sistema per il supporto alle decisioni sviluppato è costituito da componenti che si avvalgono di metodi provenienti da settori diversi come l'intelligenza artificiale, la ricerca operativa, sociologia, economia, etc. Per quanto riguarda il lavoro qui descritto, l'ambito considerato è quello dell'intelligenza artificiale e i componenti studiati sono un simulatore ad agenti per la comprensione del comportamento dei cittadini in reazione alle politiche che si desidera implementare e un ottimizzatore che si occupa di modellare matematicamente e individuare un piano regionale energetico ottimo. 

Per questi motivi, nel primo capitolo di questa trattazione forniremo un quadro dettagliato del progetto e-Policy e delle problematiche relative alla pianificazione regionale. 
Successivamente nei capitoli secondo e terzo saranno mostrati rispettivamente il simulatore economico-sociale e l'analisi statistica dei risultati delle simulazioni. 
Nel quarto capitolo discuteremo la modellazione matematica e la fase di ottimizzazione, mentre nel quinto capitolo parleremo dell'interazione tra quest'ultima fase e quella di simulazione.

%\end{document}
