%Andrea Borghesi
%Università degli studi di Bologna

%brevissimo report sui risultati ottenuti inserendo la regressione lineare a tratti nel modello a vincoli

\documentclass[12pt,a4paper,openright,twoside]{report}
\usepackage[italian]{babel}
\usepackage{indentfirst}
\usepackage[utf8]{inputenc}
\usepackage{graphicx}
\usepackage{hyphenat}
\usepackage[T1]{fontenc}
\usepackage{fancyhdr}
\usepackage{titlesec,blindtext, color}
\usepackage[font={small,it}]{caption}

\graphicspath{{/media/sda4/tesi/immagini/grafici/}{/media/sda4/tesi/immagini/grafici/incCompare/}{/media/sda4/tesi/immagini/grafici/rawData/}{/media/sda4/tesi/immagini/grafici/regressionAnalysis/}{/media/sda4/tesi/immagini/schemi/}{/media/sda4/tesi/immagini/simulazione/}{/media/sda4/tesi/immagini/epolicy/}}

\begin{document}
\chapter{TITOLO CAPITOLO}


\section{REGRESSIONE LINEARE A TRATTI}

In precedenza abbiamo visto come le relazioni che legano il budget fornito alla produzione energetica ottenibile con i diversi tipi di incentivi non siano esclusivamente di natura lineare. Questo in generale non rappresenterebbe un problema, ma l'ottimizzatore realizzato gestisce esclusivamente vincoli lineari; per ovviare a questa difficoltà abbiamo deciso di approssimare le relazioni non lineari con funzioni lineari a tratti,‌ affinché fosse possibile inserirle in modo naturale all'interno del problema a vincoli.
\\* \\* 
Data una funzione $y=f(x)$, campioniamo la curva $g$ in $k$ punti $x_1,...,x_k$ e l'approssimazione lineare a tratti $y'=g'(x)\simeq g(x)$ è definita come
\begin{equation} \label{eq:aprroxEqX}
	x = \sum_{i=1}^k \lambda_i \cdot x_i,
\end{equation}
\begin{equation} \label{eq:aprroxEqY}
	y = \sum_{i=1}^k \lambda_i \cdot y_i,
\end{equation}
dove $\lambda_i \in [0..1]$ sono variabili continue soggette ai vincoli:
\begin{equation} \label{eq:aprroxEqLambda}
	\sum_{i=1}^k \lambda_i = 1
\end{equation}
Al massimo due $\lambda_i$ possono essere diverse da zero e in tal caso queste devono essere adiacenti.\\*
Queste ultime due condizioni potrebbero essere modellate introducendo nuove 0-1 variabili intere, ma nel risolutore del problema a vincoli è possibile dichiarare $(\lambda_1,...,\lambda_k)$ come Special Order Set del secondo tipo (SOS2)[inserire riferimento], e il risolutore sfrutterà questa informazione per efficienti euristiche di branching.  
\\* \\*
Con riferimento alle equazioni ~\ref{eq:aprroxEqX} e ~\ref{eq:aprroxEqY}, nel nostro caso $x$ indica il budget fornito all' incentivo e $y$ rappresenta la produzione energetica; le varie $x_i$ e $y_i$ sono le coordinate dei punti che sono stati campionati nelle curve delle relazioni tra budget e produzione energetica.\\*
Per calcolare questi punti ci siamo serviti dello strumento usato in precedenza per effettuare l'analisi statistica sui dati ottenuti dalle simulazioni e ricavare l'andamento delle funzioni cercate.
In figura ~\ref{incentCompare_piecewise} è possibile osservare quale sia l'approssimazione lineare a tratti delle relazioni mostrate in precedenza.
\begin{figure}[hbt]
	\centering
	\includegraphics[scale=0.6]{incentComparePiecewise}
	\caption{Confronto tra i diversi incentivi - Approssimazione lineare a tratti}
	\label{incentCompare_piecewise}
\end{figure}


\section{MODELLO PROBLEMA}

Dopo aver introdotto la tecnica generale per approssimare una funzione non lineare con una lineare a tratti, vediamo ora come questa è stata implementata all'interno del modello a vincoli.

\subsection{VARIABILI}

Innanzitutto per ogni tipologia di incentivo sono introdotte due variabili che rappresentano il costo associato e la relativa produzione energetica; per i costi le variabili sono chiamate $b_A, b_{CI}, b_R, b_G$, con $b_j \in [0..50]$ (dominio espresso in milioni di euro), mentre per le produzioni energetiche associate agli incentivi le variabili sono chiamate $o_A, o_{CI}, o_R, o_G$, con $o_j \in [0..10]$ (dominio espresso in megawatt). Occorre fare subito un'importante precisazione, ovvero spiegare che per produzione energetica associata ad ogni metodologia incentivante abbiamo in questo caso inteso la produzione di energia non in termini assoluti, bensì la differenza (aumento) di produzione energetica che si ottiene finanziando un tipo di incentivo rispetto al caso in cui nessun meccanismo incentivante sia supportato (il valore di produzione energetica associato a nessuna incentivazione è definito dalla costante $Y_{BASE}$)
\\* \\*
Abbiamo poi introdotto le variabili ausiliarie necessarie all'approssimazione lineare chiamate $\lambda_i$ nell'equazione ~\ref{eq:aprroxEqLambda} e ugualmente nominate all'interno del problema e sempre con $\lambda_i \in [0..1]$; per ciascun tipo di incentivo e per ogni punto scelto per la campionatura sulle curve budget-produzione è stata introdotta una variabile ausiliaria $\lambda_{ji}$, con $j$ indice per il tipo di incentivo e $i$ per il punto campionato. Come scelta iniziale i punti utilizzati per la campionatura sono il minor numero possibile tramite cui rappresentare correttamente la funzione approssimata, cioè i punti in cui si osserva un cambiamento del gradiente della curva; in particolare abbiamo usato tre punti per gli incentivi Fondo Asta, Conto Interessi e Fondo Rotazione e quattro per il Fondo Garanzia. 

\subsection{VINCOLI}

I vincoli seguenti sono necessari per approssimare la curva con una lineare a tratti e sono la diretta trasposizione delle equazioni ~\ref{eq:aprroxEqX}, ~\ref{eq:aprroxEqY} e ~\ref{eq:aprroxEqLambda} all'interno del nostro modello a vincoli; nel equazione ~\ref{eq:vincoloAuxY} il termine $y_{ji}-Y_{BASE}$ serve per calcolare la differenza di produzione energetica rispetto al caso di nessun incentivo.\\*
Per chiarezza ricordiamo che gli indici $j$ e $i$ selezionano rispettivamente metodo incentivante e punto di campionamente e quindi nei vincoli ~\ref{eq:vincoloAuxX}, ~\ref{eq:vincoloAuxY} e ~\ref{eq:vincoloAuxLambda} si ha $j \in [1..4]$.

\begin{equation} \label{eq:vincoloAuxX}
	b_j = \sum_{i=1}^k \lambda_{ji} \cdot x_{ji} 
\end{equation}
\begin{equation} \label{eq:vincoloAuxY}
	o_j = \sum_{i=1}^k \lambda_{ji} \cdot (y_{ji}-Y_{BASE}) 
\end{equation}
\begin{equation} \label{eq:vincoloAuxLambda}
	\sum_{i=1}^k \lambda_{ji} \leq 1
\end{equation}

Imponiamo poi che la somma dei costi dei vari incentivi sia minore del budget totale dedicato agli incentivi per l'energia fotovoltaica, indicato dalla costante $BUDGET_{PV}$.

\begin{equation} \label{eq:vincoloCosti}
	\sum_{j=1}^4 b_j \leq BUDGET_{PV}
\end{equation}

Infine la funzione obiettivo fornita al risolutore del problema a vincoli tenta di massimizzare la produzione energetica distribuendo il budget disponibile ai vari incentivi.

\begin{equation} \label{eq:vincoloCosti}
	\max ( \sum_{j=1}^4 o_j )
\end{equation}



\section{ASSEGNAZIONE FONDI}

Per valutare ora l'interazione tra il DSS e il simulatore analizzeremo ora in che modo vengono finanziati le diverse tipologie di incentivazione sulla base del problema a vincoli presentato in precedenza.\\*
Per come il sotto-problema dell'assegnazione dei fondi è stato definito, il risolutore procede distribuendo i fondi a partire dall'incentivo che risulta essere più efficace e prosegue poi finanziando gli incentivi restanti fino a esaurimento del budget previsto. Per determinare quale sia il tipo di incentivo più efficace vengono sfruttate le funzioni lineari a tratti che approssimano le relazioni tra budget e produzione energetica.\\*
Nelle tabelle seguenti sarà mostrato in che modo vengono assegnati i fondi ai vari incentivi a partire da un determinato budget, evidenziando anche l'aumento di produzione energetica ottenibile rispetto al caso in cui nessun incentivo venga finanziato; per avere un riferimento ricordiamo che nel nostro simulatore, quindi non adeguatamente scalato, il valore di produzione energetica ottenibile in assenza di meccanismi incentivanti è di circa 21.664MW.
\\* \\*
\paragraph{FONDO INCENTIVI – 2.5 MILIONI EURO}
\begin{center}
	\begin{tabular}{ | p{4.5cm}  | p{4.5cm} | p{4.5cm} | }
		\hline
		\nohyphens{\emph{Tipologia Incentivo}} & \nohyphens{\emph{Costo (Milioni di Euro)}} & \nohyphens{\emph{Produzione Energetica Differenziale(MW)}} \\ \hline
		Asta &  0.00 & 0.000 \\ \hline
		Conto Interessi & 2.50 & 3.563 \\ \hline
		Rotazione & 0.00 & 0.000 \\ \hline
		Garanzia & 0.00 & 0.000 \\ \hline \hline 
		Totale & 2.50 & 3.563 \\
		\hline
	\end{tabular}
\end{center}
Con questo fondo per gli incentivi viene finanziato totalmente il Conto Interessi consumando interamente il budget disponibile, impedendo quindi di stanziare fondi anche agli altri incentivi, e ottenendo un aumento di produzione energetica di 3.563MW.

\paragraph{FONDO INCENTIVI – 5 MILIONI EURO}
\begin{center}
	\begin{tabular}{ | p{4.5cm}  | p{4.5cm} | p{4.5cm} | }
		\hline
		\nohyphens{\emph{Tipologia Incentivo}} & \nohyphens{\emph{Costo (Milioni di Euro)}} & \nohyphens{\emph{Produzione Energetica Differenziale(MW)}} \\ \hline
		Asta &  0.00 & 0.000 \\ \hline
		Conto Interessi & 2.53 & 3.606 \\ \hline
		Rotazione & 1.00 & 0.286 \\ \hline
		Garanzia & 1.00 & 0.170 \\ \hline \hline 
		Totale & 5.00 & 4.121 \\
		\hline
	\end{tabular}
\end{center}
In questo caso è possibile finanziare interamente il Conto Interessi (il quale, come atteso, richiede decisamente meno risorse degli altri, ovvero potendo essere completamente soddisfatto con 2.53 milioni di euro) e distribuire il budget restante agli incentivi Fondo Rotazione e Garanzia, ottenendo così un miglioramento della produzione energetica di 4.121MW.

\paragraph{FONDO INCENTIVI – 10 MILIONI EURO}
\begin{center}
	\begin{tabular}{ | p{4.5cm}  | p{4.5cm} | p{4.5cm} | }
		\hline
		\nohyphens{\emph{Tipologia Incentivo}} & \nohyphens{\emph{Costo (Milioni di Euro)}} & \nohyphens{\emph{Produzione Energetica Differenziale(MW)}} \\ \hline
		Asta &  0.00 & 0.000 \\ \hline
		Conto Interessi & 2.53 & 3.606 \\ \hline
		Rotazione & 1.00 & 0.286 \\ \hline
		Garanzia & 6.47 & 0.816 \\ \hline \hline 
		Totale & 10.00 & 4.708 \\
		\hline
	\end{tabular}
\end{center}
Con un ulteriore aumento del budget l'assegnazione dei fondi ottimale prevede di distribuire prima al Conto Interessi seguito sempre dai Fondi Garanzia e Rotazione, ottenendo un incremento della produzione energetica pari a 4.708MW; in realtà il scendo miglior incentivo è il Fondo Garanzia e quindi esso riceve finanziamenti maggiori rispetto al Rotazione, a quest'ultimo viene comunque assegnato un budget di 1 milione di euro (anche nel caso precedente) in quanto, per come sono stati campionati i punti che definiscono l'approsimazione lineare a tratti, a tale budget è associata una produzione energetica migliore rispetto a quella del Fondo Garanzia.

\paragraph{FONDO INCENTIVI – 20 MILIONI EURO}
\begin{center}
	\begin{tabular}{ | p{4.5cm}  | p{4.5cm} | p{4.5cm} | }
		\hline
		\nohyphens{\emph{Tipologia Incentivo}} & \nohyphens{\emph{Costo (Milioni di Euro)}} & \nohyphens{\emph{Produzione Energetica Differenziale(MW)}} \\ \hline
		Asta &  1.00 & 0.125 \\ \hline
		Conto Interessi & 2.53 & 3.606 \\ \hline
		Rotazione & 2.73 & 0.458 \\ \hline
		Garanzia & 13.74 & 1.734 \\ \hline \hline 
		Totale & 20.00 & 5.923 \\
		\hline
	\end{tabular}
\end{center}
Con un budget di 20 milioni di euro notiamo due differenze. \\* La prima è che dopo aver assegnato circa 13 milioni e mezzo al Fondo Garanzia diventa più utile destinare i fondi rimanenti al Fondo Rotazione, ricavando un aumento della produzione di 5.923MW; questo accade perché in corrispondenza di quel budget la pendenza della curva che descrive l'incentivo Fondo Garanzia diminuisce sensibilmente (Fig. ~\ref{incentCompare_piecewise}), rendendo quindi l'incentivo meno efficace in confronto al Fondo Rotazione.\\* La seconda differenza è che conviene anche destinare una minima parte (1 milione di euro) del budget anche al Fondo Asta e questo è probabilmente dovuto nuovamente al modo in cui sono stati campionati i punti per l'approssimazione.

\paragraph{FONDO INCENTIVI – 40 MILIONI EURO}
\begin{center}
	\begin{tabular}{ | p{4.5cm}  | p{4.5cm} | p{4.5cm} | }
		\hline
		\nohyphens{\emph{Tipologia Incentivo}} & \nohyphens{\emph{Costo (Milioni di Euro)}} & \nohyphens{\emph{Produzione Energetica Differenziale(MW)}} \\ \hline
		Asta &  1.00 & 0.125 \\ \hline
		Conto Interessi & 2.53 & 3.606 \\ \hline
		Rotazione & 22.73 & 2.448 \\ \hline
		Garanzia & 13.74 & 1.734 \\ \hline \hline 
		Totale & 40.00 & 7.913 \\
		\hline
	\end{tabular}
\end{center}
Con un budget di 40 milioni di euro i risultati sono molto simili al caso precedente, con la differenza che i maggiori finanziamenti disponibili vengono interamente indirizzati al Fondo Rotazione, poichè il Conto Interessi è in grado di rendere al massimo anche con una spesa minima e il Fondo Garanzia diventa meno efficace con budget maggiori di 13 milioni e mezzo di euro (come anche il Fondo Asta con qualsiasi budget); l'incremento di produzione energetica rispetto alla mancanza di metodologie incentivanti è di 7.913MW.

\end{document}
