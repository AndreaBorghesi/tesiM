%Andrea Borghesi
%Università degli studi di Bologna

%tesi laurea magistrale

\documentclass[12pt,a4paper,openright,twoside]{book}
\usepackage[italian]{babel}
\usepackage{indentfirst}
\usepackage[utf8]{inputenc}
\usepackage[T1]{fontenc}
\usepackage{fancyhdr}
\usepackage{graphicx}
\usepackage{titlesec,blindtext, color}
\usepackage[font={small,it}]{caption}
\usepackage{subfig}
\usepackage{listings}
\usepackage{color}
\usepackage{url}
\usepackage{textcomp}
\usepackage{eurosym}
\usepackage{amsmath}
\usepackage{frontesp} 	% frontespizio unibo 
%\usepackage[colorlinks=false]{hyperref}%<--inserisce segnalibri e riferimenti indice

%impostazioni generali per visualizzare codice
\definecolor{dkgreen}{rgb}{0,0.6,0}
\definecolor{gray}{rgb}{0.5,0.5,0.5}
\definecolor{mauve}{rgb}{0.58,0,0.82}
 
\lstset{ %
  basicstyle=\footnotesize,           % the size of the fonts that are used for the code
  backgroundcolor=\color{white},      % choose the background color. You must add \usepackage{color}
  numbers=left,                   % where to put the line-numbers
  numberstyle=\tiny\color{gray},  % the style that is used for the line-numbers
  numbersep=5pt,  
  showspaces=false,               % show spaces adding particular underscores
  showstringspaces=false,         % underline spaces within strings
  showtabs=false,                 % show tabs within strings adding particular underscores
  rulecolor=\color{black}, 
  tabsize=2,                      % sets default tabsize to 2 spaces
  breaklines=true,                % sets automatic line breaking
  breakatwhitespace=false,        % sets if automatic breaks should only happen at whitespace
  title=\lstname,                   % show the filename of files included with \lstinputlisting;
  frame=single,                   % adds a frame around the code
                                  % also try caption instead of title
  keywordstyle=\color{blue},          % keyword style
  commentstyle=\color{dkgreen},       % comment style
  stringstyle=\color{mauve},         % string literal style
  escapeinside={\%*}{*)},            % if you want to add LaTeX within your code
  morekeywords={*,...},              % if you want to add more keywords to the set
  deletekeywords={...}              % if you want to delete keywords from the given language
}

%per avere un bordo intorno alle figure
\usepackage{float}
\floatstyle{boxed} 
\restylefloat{figure}

%per poter poi impedire che certe parole vadano a capo
\usepackage{hyphenat}

%ridefinisco font per fancyhdr, per ottenere un'intestazione pulita
\newcommand{\changefont}{ \fontsize{9}{11}\selectfont }
\fancyhf{}
\fancyhead[LE,RO]{\changefont \slshape \rightmark} 	%section
\fancyhead[RE,LO]{\changefont \slshape \leftmark}	%chapter
\fancyfoot[C]{\changefont \thepage}					%footer

%titolo capitolo con "numero | titolo"
\definecolor{gray75}{gray}{0.75}
\newcommand{\hsp}{\hspace{20pt}}
\titleformat{\chapter}[hang]{\Huge\bfseries}{\thechapter\hsp\textcolor{gray75}{|}\hsp}{0pt}{\Huge\bfseries}


%\oddsidemargin=30pt \evensidemargin=20pt

%sillabazioni non eseguite correttamente
\hyphenation{sil-la-ba-zio-ne pa-ren-te-si si-mu-la-to-re ge-ne-ra-re pia-no}

%interlinea
\linespread{1.15}  
\pagestyle{fancy}

%cartelle contenenti le immagini
\graphicspath{{/media/sda4/tesi/immagini/grafici/}{/media/sda4/tesi/immagini/grafici/incCompare/}{/media/sda4/tesi/immagini/grafici/rawData/}{/media/sda4/tesi/immagini/grafici/regressionAnalysis/}{/media/sda4/tesi/immagini/schemi/}{/media/sda4/tesi/immagini/simulazione/}{/media/sda4/tesi/immagini/epolicy/}{/media/sda4/tesi/immagini/ottimizzazione/}
{/media/sda4/tesi/immagini/interazione/}}

%in modo che dopo il titolo di un paragrafo il testo vada a capo
\newcommand{\myparagraph}[1]{\paragraph{#1}\mbox{}\\}

%per scrivere bene CLP(R) e CLP(FD)
\newcommand{\clpr}{CLP({\ensuremath{\cal R}})}
\newcommand{\clpfd}{CLP({\ensuremath{\cal FD}})}

\begin{document}
\frontmatter

\title{Integrazione di ottimizzazione e simulazioni per il piano energetico regionale dell'Emilia-Romagna}
\author{Andrea Borghesi}
\date{}
\titolocorso{Ingegneria Informatica M}
\nomemateria{Sistemi Intelligenti}
\degreeyear{2011/2012}					% Anno Accademico di Laurea
\session{III} 						% Sessione di Laurea
\principaladviser{Chiar.ma Prof. Michela Milano}	% Relatore principale

% non funziona modificare frontesp.sty al variare del numero di correlatori
\firstreader{Chiar.mo Prof. Marco Gavanelli}
\secondreader{}			% Correlatori
\thirdreader{}			% Correlatori

\maketitle % Frontespizio

\thispagestyle{empty}
\textcolor{white}{.}\\
\pagebreak 

%Se si vuole mettere l'indice nell'indice (comodo per la navigazione con pdf)
%\markboth{Indice}{Indice}
%\addcontentsline{toc}{chapter}{Indice}

\tableofcontents 		% Indice
\newpage
\listoffigures
\listoftables

\mainmatter
%Andrea Borghesi
%Università degli studi di Bologna

%introduzione

%\documentclass[12pt,a4paper,openright,twoside]{book}
%\usepackage[italian]{babel}
%\usepackage{indentfirst}
%\usepackage[utf8]{inputenc}
%\usepackage[T1]{fontenc}
%\usepackage{fancyhdr}
%\usepackage{graphicx}
%\usepackage{titlesec,blindtext, color}
%\usepackage[font={small,it}]{caption}

%\begin{document}

\clearpage{\pagestyle{empty}\cleardoublepage}
\chapter*{Introduzione} 
\markboth{Introduzione}{Introduzione}
\addcontentsline{toc}{chapter}{Introduzione}

La definizione delle politiche pubbliche a livello nazionale, regionale o locale è un compito complesso, in quanto occorre operare in ambiti caratterizzati da dinamicità e incertezza, tentando di risolvere diverse problematiche e conciliando interessi conflittuali. Fattori come la globalizzazione o la sostenibilità ambientale rendono ancora più difficili le scelte che i decisori politici sono tenuti ad effettuare per l'ideazione e l'implementazione di strategie in grado di affrontare le sfide reali della società odierna, senza sottovalutare il fatto che l'elevata complessità dei sistemi considerati non consente di determinare facilmente gli effetti relativi alle decisioni prese.

Da tutto ciò consegue che sia profondamente avvertita l'esigenza di sviluppare metodologie e strumenti di cui i decisori politici si possano avvalere per gestire le problematiche di questo settore. In questa direzione procede lo sviluppo di modelli matematici e computazionali alla base dei sistemi di supporto alle decisioni politiche; tali sistemi devono essere in grado di fornire una serie di scenari decisionali alternativi, con i quali è possibile aiutare il politico a svolgere il proprio compito, ma certamente senza sostituirvisi. Prendere le decisioni senza un supporto informatico è estremamente difficile poiché sia esse che le loro interconnessioni, ovvero impatti e conseguenze che ne derivano, sono moltissime e anche perché occorre prendere in considerazione diversi aspetti, da quelli economici a quelli ambientali e sociali che hanno un grado di complessità intrinseca molto elevato.

Come esempio, basti pensare alle valutazioni da fare per l'ottimizzazione di uno o più recettori ambientali, come la qualità dell'aria o delle acque. Invece, per quanto riguarda gli aspetti sociali, è necessario tenere conto di come la società reagirà alle politiche che si vogliono implementare: ad esempio se a fronte di determinati meccanismi incentivanti i cittadini o gli imprenditori investiranno in impianti di energia da fonti rinnovabili. Disporre quindi di un sistema che modelli dal punto di vista matematico le decisioni di un piano (locale, regionale, nazionale e così via) permette di prendere in considerazione tutti questi aspetti contemporaneamente, in modo da generare politiche che abbiano impatti economici, sociali e ambientali accettabili e controllati. 
\\*

Il lavoro da noi svolto è che ci accingiamo a illustrare rientra nell'ambito sopra esposto. In particolare rientra all'interno del progetto e-Policy, VII Programma quadro dell'Unione Europea, dedicato allo sviluppo di sistemi di supporto ai decisori per produrre politiche sostenibili dal punto di vista ambientale e socialmente accettate; la regione Emilia-Romagna è partner di questo progetto e lo sviluppo del piano regionale energetico ha fornito il caso di studio per e-Policy e il lavoro in seguito presentato. Da un punto di vista molto generale, il sistema per il supporto alle decisioni sviluppato è costituito da componenti che si avvalgono di metodi provenienti da settori diversi come l'intelligenza artificiale, la ricerca operativa, sociologia, economia, etc. Per quanto riguarda il lavoro qui descritto, l'ambito considerato è quello dell'intelligenza artificiale e i componenti studiati sono un simulatore ad agenti per la comprensione del comportamento dei cittadini in reazione alle politiche che si desidera implementare e un ottimizzatore che si occupa di modellare matematicamente e individuare un piano regionale energetico ottimo. 

Per questi motivi, nel primo capitolo di questa trattazione forniremo un quadro dettagliato del progetto e-Policy e delle problematiche relative alla pianificazione regionale. 
Successivamente nei capitoli secondo e terzo saranno mostrati rispettivamente il simulatore economico-sociale e l'analisi statistica dei risultati delle simulazioni. 
Nel quarto capitolo discuteremo la modellazione matematica e la fase di ottimizzazione, mentre nel quinto capitolo parleremo dell'interazione tra quest'ultima fase e quella di simulazione.

%\end{document}


% descrizione progetto ePolicy e introduzione agli strumenti incentivanti
%Andrea Borghesi
%Università degli studi di Bologna

%capitolo dedicato alla descrizione del progetto e-policy e ai meccanismi incentivanti 

\documentclass[12pt,a4paper,openright,twoside]{report}
\usepackage[italian]{babel}
\usepackage{indentfirst}
\usepackage[utf8]{inputenc}
\usepackage[T1]{fontenc}
\usepackage{fancyhdr}
\usepackage{graphicx}
\usepackage{titlesec,blindtext, color}
\usepackage[font={small,it}]{caption}
\usepackage{subfig}
\usepackage{listings}
\usepackage{color}
\usepackage{url}
\usepackage{textcomp}

%impostazioni generali per visualizzare codice
\definecolor{dkgreen}{rgb}{0,0.6,0}
\definecolor{gray}{rgb}{0.5,0.5,0.5}
\definecolor{mauve}{rgb}{0.58,0,0.82}
 
\lstset{ %
  basicstyle=\footnotesize,           % the size of the fonts that are used for the code
  backgroundcolor=\color{white},      % choose the background color. You must add \usepackage{color}
  numbers=left,                   % where to put the line-numbers
  numberstyle=\tiny\color{gray},  % the style that is used for the line-numbers
  numbersep=5pt,  
  showspaces=false,               % show spaces adding particular underscores
  showstringspaces=false,         % underline spaces within strings
  showtabs=false,                 % show tabs within strings adding particular underscores
  rulecolor=\color{black}, 
  tabsize=2,                      % sets default tabsize to 2 spaces
  breaklines=true,                % sets automatic line breaking
  breakatwhitespace=false,        % sets if automatic breaks should only happen at whitespace
  title=\lstname,                   % show the filename of files included with \lstinputlisting;
  frame=single,                   % adds a frame around the code
                                  % also try caption instead of title
  keywordstyle=\color{blue},          % keyword style
  commentstyle=\color{dkgreen},       % comment style
  stringstyle=\color{mauve},         % string literal style
  escapeinside={\%*}{*)},            % if you want to add LaTeX within your code
  morekeywords={*,...},              % if you want to add more keywords to the set
  deletekeywords={...}              % if you want to delete keywords from the given language
}

%per avere un bordo intorno alle figure
\usepackage{float}
\floatstyle{boxed} 
\restylefloat{figure}

%per poter poi impedire che certe parole vadano a capo
\usepackage{hyphenat}
\usepackage{listings}

%ridefinisco font per fancyhdr, per ottenere un'intestazione pulita
\newcommand{\changefont}{ \fontsize{9}{11}\selectfont }
\fancyhf{}
\fancyhead[LE,RO]{\changefont \slshape \rightmark} 	%section
\fancyhead[RE,LO]{\changefont \slshape \leftmark}	%chapter
\fancyfoot[C]{\changefont \thepage}					%footer

%titolo capitolo con "numero | titolo"
\definecolor{gray75}{gray}{0.75}
\newcommand{\hsp}{\hspace{20pt}}
\titleformat{\chapter}[hang]{\Huge\bfseries}{\thechapter\hsp\textcolor{gray75}{|}\hsp}{0pt}{\Huge\bfseries}


\oddsidemargin=30pt \evensidemargin=20pt

%sillabazioni non eseguite correttamente
\hyphenation{sil-la-ba-zio-ne pa-ren-te-si si-mu-la-to-re ge-ne-ra-re pia-no}

%interlinea
\linespread{1.15}  
\pagestyle{fancy}

%cartelle contenenti le immagini
\graphicspath{{/media/sda4/tesi/immagini/grafici/}{/media/sda4/tesi/immagini/grafici/incCompare/}{/media/sda4/tesi/immagini/grafici/rawData/}{/media/sda4/tesi/immagini/grafici/regressionAnalysis/}{/media/sda4/tesi/immagini/schemi/}{/media/sda4/tesi/immagini/simulazione/}{/media/sda4/tesi/immagini/epolicy/}}

%in modo che dopo il titolo di un paragrafo il testo vada a capo
\newcommand{\myparagraph}[1]{\paragraph{#1}\mbox{}\\}

\begin{document}
\chapter{\nohyphens{QUADRO GENERALE}}

I problemi delle politiche pubbliche sono estremamente complessi, avvengono in ambienti che cambiano rapidamente caratterizzati da incertezza e coinvolgono conflitti tra diversi interessi. La nostra società è sempre più complessa a causa della globalizzazione, ampliamento e cambiamento delle situazioni geopolitiche. Questo implica che l'attività politica e la sua area di intervento si siano estese, rendendo più difficili da determinare gli effetti di tali interventi, mentre al tempo stesso diventa sempre più importante assicurarsi che le azioni intraprese affrontino in maniera efficacie le sfide reali che la crescente complessità comporta.\\*
Da questo consegue che coloro responsabili di creare, implementare e far rispettare le politiche devono essere in grado di giungere a delle decisioni nel caso di problemi mal definiti e non pienamente compresi, senza una singola risposta corretta, che coinvolgono diversi interessi in competizione e interagiscono con altre politiche su multipli livelli.; è quindi necessario trattare con coerenza tali problematiche e ricercare tecniche, metodologie e strumenti per affrontare la complessità in questo settore.\\*\\*
Con questo scopo in mente è stato ideato il progetto ePolicy che verrà ora introdotto; nel resto del capitolo verranno quindi fornite una descrizione di questo progetto in termini generali, seguito dalla presentazione del caso di studio con cui si è deciso di testare le tecniche sviluppate - la regione Emilia Romagna - e per passare infine a descrivere le strategie implementative adottabili per la messa in atto delle politiche studiate.


\section[E-POLICY]{PROGETTO E-POLICY}

Il progetto europeo \emph{ePolicy} (dall'inglese Engineering the Policy Making Life Cycle, cioè ingegnerizzare il processo di creazione delle politiche) ha come obiettivo la creazione di un sistema di supporto alle decisioni per la pianificazione regionale e la valutazione degli impatti sociali, economici e ambientali. Con l'espressione \emph{sistema di supporto alle decisioni} (a cui in seguito ci riferiremo anche utilizzando l'acronimo \emph{DSS}, dall'inglese Decision Support System) si intende una classe molto ampia di sistemi software che hanno come scopo aiutare a prendere decisioni in caso di gestione di problemi complessi, facilitando l'analisi di grandi quantità di dati e suggerendo strategie e  politiche da adottare.\\*
Avviato nell'Ottobre del 2011, il progetto è coordinato dall'Università di Bologna e coinvolge nove partner tra mondo dell'accademia e della ricerca, governi regionali e settore privato, distribuiti in cinque paesi diversi dell'Unione Europea.\\*\\*
I decisori politici devono prendere decisioni complesse valutando un notevole numero di variabili e vincoli, tenendo conto quindi degli impatti che le loro scelte avranno su diversi aspetti ambientali, economici e sociali. Al tempo stesso, si è osservata negli anni un sempre crescente desiderio da parte dei cittadini di contribuire alla creazione delle politiche attraverso mezzi come i social network e i blog.\\*\\*
L'intenzione del progetto, una volta concluso, è quella di permettere a coloro che effettuano le decisioni di disporre di un sistema integrato e user-friendly, in grado di creare e valutare piani alternativi altamente ottimizzati tra i quali poter scegliere sulla base di una dettagliata analisi dei costi e benefici degli stessi.\\*\\*
Oltre a esaminare gli aspetti teorici, il progetto ePolicy mira a applicare i suoi risultati a un caso pratico: la pianificazione energetica nella regione Emilia Romagna. In particolare, il governo regionale si è posto l'obiettivo di incrementare la produzione di energia da fonti rinnovabili, concentrandosi soprattutto sulle tecnologie fotovoltaiche (PV) e a biomassa. Di conseguenza, ePolicy punta a sviluppare un modello che fornirà supporto ai decisori politici della regione che stanno cercando di mettere in pratica il miglior meccanismo incentivante per stimolare la crescita della produzione energetica da alcune tecnologie rinnovabili.

\begin{figure}[hbt]
	\centering
	\includegraphics[scale=0.55]{epolicyLifeCycle}
	\caption{Processo di decisione delle Politiche}
	\label{epolicyLifeCycle}
\end{figure}

In Figura ~\ref{epolicyLifeCycle} osserviamo in che modo sia strutturato il ciclo di vita del processo di creazione delle politiche all'interno del progetto ePolicy: \begin{itemize}
\item il livello di ottimizzazione globale, che prende in considerazione gli obiettivi, gli aspetti finanziari e gli impatti socio-ambientali su larga scala (produce dei piani e degli scenari per le politiche);
\item il livello individuale delle simulazioni ad agenti, con il quale si intendono simulare  comportamenti sociali riguardanti le nuove politiche sulla base delle opinioni e desideri personali (per ottenere le strategie implementative);
\item  l'integrazione tra la prospettiva globale e quella individuale, ad esempio con tecniche mutuate dalla teoria dei giochi;
\item l'individuazione degli impatti sociali e le reazioni delle persone attraverso l'uso di tecniche di opinion mining , cioè estrazione delle opinioni, con i dati raccolti in rete (servendosi di blog, forum, social network,...);
\item la visualizzazione dei risultati attraverso strumenti appositamente ideati per aiutare i decisori politici.
\end{itemize}


\begin{figure}[hbt]
	\centering
	\includegraphics[scale=0.55]{epolicyScheme}
	\caption{Schema generale del sistema}
	\label{epolicyScheme}
\end{figure}

In Figura ~\ref{epolicyScheme} è mostrato lo schema generale del progetto ePolicy. Si possono osservare le varie componenti del sistema come l'ottimizzatore che lavora a livello globale o il simulatore per il livello individuale e le interazioni tra loro e con gli utenti, ovvero i decisori politici che specificano vincoli, obiettivi e impatti e i cittadini dai quali ottenere informazioni per poter meglio pianificare (ex ante opinion mining) e osservarne le reazioni alle strategie implementative (ex post opinion mining).\\*\\*
Un aspetto importante da tenere in considerazione per fornire supporto ai decisori politici è la definizione formale dei modelli delle politiche. In letteratura la maggioranza dei modelli politici è basata su simulazioni ad agenti \cite{AgentBaseLandUseModel,Nigel,socialScienceMicrosim} dove gli agenti rappresentano le parti coinvolte nel processo decisionale e implementativo. L'idea è che modelli ad agenti e relative simulazioni siano adatti per sistemi complessi. In particolare, questi modelli permettono di effettuare esperimenti computazionali per garantire una migliore comprensione della complessità dei sistemi economici, sociali e ambientali, cambiamenti strutturali e adattamenti reattivi endogeni in riposta ai cambi di politiche.
\\*\\*
Per riassumere, i principali obiettivi del progetto ePolicy sono i seguenti:
\begin{itemize}
\item supportare i decisori politici nel loro lavoro, ovvero uno sforzo multidisciplinare mirato a ingegnerizzare il ciclo di vita del processo di creazione delle politiche;
\item integrare le prospettive globale e individuale all'interno del processo decisionale;
\item valutare gli impatti sociali, economici e ambientali durante lo sviluppo delle politiche (sia a livello globale che individuale);
\item stabilire i probabili effetti sociali attraverso opinion mining;
\item aiutare tutti coloro che sono coinvolti nei processi decisionali e i cittadini interessati con degli strumenti di visualizzazione efficaci.
\end{itemize}
Una volta realizzati questi obiettivi, è possibile aspettarsi alcuni benefici sociali ed economici, tra i quali una migliore previsione degli impatti delle politiche attuate in grado di condurre a una più efficiente implementazione delle politiche regionali e migliore identificazione degli effetti positivi per cittadini e imprese; o ancora, un aumentato impegno dei cittadini e un più ampio uso degli strumenti informatici e di telecomunicazione (ITC), che possono risultare in iterazioni innovative tra cittadini e governi. In secondo luogo si punta a ottenere una maggiore trasparenza delle informazioni sull'impatto delle decisioni economiche sulla società e una migliorate capacità di reagire alle principali sfide poste alla società e maggiore fiducia pubblica verso le attività governative e burocratiche.


\section{PIANIFICAZIONE REGIONALE}

Il caso di studio scelto per sperimentare le metodologie sviluppate con il progetto ePolicy è la creazione del Piano Regionale dell'Energia per la Regione Emilia Romagna (d'ora in poi abbreviata anche con l'acronimo RER).\\*\\*
La pianificazione regionale è lo studio della disposizione efficiente delle attività e delle infrastrutture territoriali per una crescita sostenibile della regione. I piani regionali sono classificati in base all'ambito che considerano, come ad esempio Agricolo, Forestale, Energia, Industria, Trasporti, Risorse Idriche, Urbano, Ambientale, etc.\\*
Nonostante i diversi piani differiscano per obiettivi e tipo di attività, essi condividono alcune caratteristiche comuni che consentono un trattamento uniforme in termini di requisiti per un sistema di supporto alle decisioni.\\*
A grandi linee, i piani regionali sono organizzati secondo quanto segue: \begin{itemize}
\item analisi della situazione e dei piani precedenti, nella quale vengono considerati aspetti sociali, economici e ambientali e i risultati degli strumenti implementati in passato sono identificati e valutati;
\item obiettivi e strategie, possono essere derivati dalle linee guida europee o nazionali, leggi e norme esistenti, opinioni dai cittadini, specifiche necessità regionali;
\item priorità e linee di intervento, la parte decisionale del piano durante la quale vengono allocate le risorse mirata a soddisfare gli obiettivi rispettando determinati vincoli;
\item implementazione e monitoraggio, definendo strumenti che possono essere economici, come tasse o sussidi , regolatori, cooperativi, ad esempio accordi volontari o tra produttori e consumatori, informativi, come campagne informative e pubblicitarie o trasferimenti tecnologici.   
\end{itemize}
L'approccio di ePolicy un piano consiste in un insieme di attività che dovrebbero essere effettuate per raggiungere certi obiettivi. Con il fine di aiutare durante la pianificazione, la modellazione delle politiche deve tenere conto di alcuni aspetti, descritti in modo più esteso nei prossimi paragrafi e capitoli. Innanzitutto ogni piano presenta un certo numero di differenti obiettivi (anche diversi a seconda dell'aspetto del funzionamento della regione che affrontano); durante la creazione di un piano, essi devono essere tenuti contemporaneamente in considerazione, compito non semplice poiché potrebbero essere in conflitto tra loro. In secondo luogo, l'implementazione di un piano è limitata da un insieme di vincoli finanziari ed economici, tipicamente espressi nei termini dei fondi disponibili e dei costi privati stimati. Ancora, gli effetti positivi o negativi in grado di influenzare aspetti sociali o ambientali devono essere considerati durante la pianificazione. Infine, un'altra attività fondamentale per la creazione di un piano è la definizione di strategie implementative, cioè i meccanismi usati per portare a compimento le attività previste, i quali hanno ovviamente un impatto sulle possibilità di conseguire gli scopi prefissati.

\subsection[VINCOLI FINANZIARI]{\nohyphens{VINCOLI FINANZIARI}}

\subsection[IMPATTI]{\nohyphens{IMPATTI ECONOMICI, SOCIALI E AMBIENTALI}}

\subsection[OBIETTIVI]{\nohyphens{OBIETTIVI DEL PIANO}}



\section{\nohyphens{STRATEGIE IMPLEMENTATIVE}}

\subsection[INCENTIVI]{\nohyphens{TIPOLOGIE DI INCENTIVI}}

\myparagraph{MECCANISMI INCENTIVANTI}

\myparagraph{CONFRONTO DEI MECCANISMI D'INCENTIVAZIONE}

\subsection[INCENTIVI EUROPEI]{\nohyphens{INCENTIVI IN EUROPA}}

\myparagraph{TARIFFE DI INCENTIVAZIONE}

\myparagraph{QUOTE OBBLIGATORIE DA RINNOVABILI}

\myparagraph{SUSSIDI AGLI INVESTIMENTI}

\myparagraph{INCENTIVI O ESENZIONI PER LE TASSE}

\myparagraph{INCENTIVI FISCALI}


\subsection[INCENTIVI ITALIANI]{\nohyphens{INCENTIVI IN ITALIA}}

\myparagraph{TARIFFA INCENTIVANTE}

\myparagraph{TARIFFA INCENTIVANTE ONNICOMPRENSIVA}

\subsection[INCENTIVI REGIONALI]{\nohyphens{INCENTIVI IN EMILIA ROMAGNA}}

\myparagraph{MECCANISMI INCENTIVANTI REGIONALI}

\myparagraph{INCENTIVI FISCALI}


\nocite{*}
\bibliographystyle{plain}
\bibliography{bibliography}

\end{document}


% descrizione simulatore
%Andrea Borghesi
%Università degli studi di Bologna

%capitolo dedicato alla simulazione 

%\documentclass[12pt,a4paper,openright,twoside]{report}
%\usepackage[italian]{babel}
%\usepackage{indentfirst}
%\usepackage[utf8]{inputenc}
%\usepackage[T1]{fontenc}
%\usepackage{fancyhdr}
%\usepackage{graphicx}
%\usepackage{titlesec,blindtext, color}
%\usepackage[font={small,it}]{caption}
%\usepackage{subfig}
%\usepackage{listings}
%\usepackage{color}
%\usepackage{url}
%\usepackage{textcomp}

%%impostazioni generali per visualizzare codice
%\definecolor{dkgreen}{rgb}{0,0.6,0}
%\definecolor{gray}{rgb}{0.5,0.5,0.5}
%\definecolor{mauve}{rgb}{0.58,0,0.82}
% 
%\lstset{ %
%  basicstyle=\footnotesize,           % the size of the fonts that are used for the code
%  backgroundcolor=\color{white},      % choose the background color. You must add \usepackage{color}
%  showspaces=false,               % show spaces adding particular underscores
%  showstringspaces=false,         % underline spaces within strings
%  showtabs=false,                 % show tabs within strings adding particular underscores
%  tabsize=2,                      % sets default tabsize to 2 spaces
%  breaklines=true,                % sets automatic line breaking
%  breakatwhitespace=false,        % sets if automatic breaks should only happen at whitespace
%  title=\lstname,                   % show the filename of files included with \lstinputlisting;
%                                  % also try caption instead of title
%  keywordstyle=\color{blue},          % keyword style
%  commentstyle=\color{dkgreen},       % comment style
%  stringstyle=\color{mauve},         % string literal style
%  escapeinside={\%*}{*)},            % if you want to add LaTeX within your code
%  morekeywords={*,...},              % if you want to add more keywords to the set
%  deletekeywords={...}              % if you want to delete keywords from the given language
%}

%%per avere un bordo intorno alle figure
%\usepackage{float}
%\floatstyle{boxed} 
%\restylefloat{figure}

%%per poter poi impedire che certe parole vadano a capo
%\usepackage{hyphenat}
%\usepackage{listings}

%%ridefinisco font per fancyhdr, per ottenere un'intestazione pulita
%\newcommand{\changefont}{ \fontsize{9}{11}\selectfont }
%\fancyhf{}
%\fancyhead[LE,RO]{\changefont \slshape \rightmark} 	%section
%\fancyhead[RE,LO]{\changefont \slshape \leftmark}	%chapter
%\fancyfoot[C]{\changefont \thepage}					%footer

%%titolo capitolo con "numero | titolo"
%\definecolor{gray75}{gray}{0.75}
%\newcommand{\hsp}{\hspace{20pt}}
%\titleformat{\chapter}[hang]{\Huge\bfseries}{\thechapter\hsp\textcolor{gray75}{|}\hsp}{0pt}{\Huge\bfseries}


%\oddsidemargin=30pt \evensidemargin=20pt

%%sillabazioni non eseguite correttamente
%\hyphenation{sil-la-ba-zio-ne pa-ren-te-si si-mu-la-to-re ge-ne-ra-re pia-no}

%%interlinea
%\linespread{1.15}  
%\pagestyle{fancy}

%%cartelle contenenti le immagini
%\graphicspath{{/media/sda4/tesi/immagini/grafici/}{/media/sda4/tesi/immagini/grafici/incCompare/}{/media/sda4/tesi/immagini/grafici/rawData/}{/media/sda4/tesi/immagini/grafici/regressionAnalysis/}{/media/sda4/tesi/immagini/schemi/}{/media/sda4/tesi/immagini/simulazione/}}

%%in modo che dopo il titolo di un paragrafo il testo vada a capo
%\newcommand{\myparagraph}[1]{\paragraph{#1}\mbox{}\\}

%\begin{document}
\clearpage{\pagestyle{empty}\cleardoublepage}
\chapter{Simulazione}

In questo capitolo verrà illustrato in che modo abbiamo studiato le relazioni che legano i diversi aspetti del piano energetico regionale, concentrandoci particolarmente su come la creazione di meccanismi di incentivazione da parte della regione Emilia-Romagna possa influenzare la produzione di energia elettrica proveniente da impianti fotovoltaici.
Per capire queste relazioni abbiamo scelto un approccio basato sulle simulazioni. Abbiamo quindi creato un modello della realtà (o almeno l'aspetto da noi preso in esame) e attraverso esso esaminato le dinamiche del sistema di nostro interesse.

Ora saranno presentati gli strumenti di cui ci siamo serviti implementare il modello sopra citato e in seguito il simulatore vero e proprio, spiegandone caratteristiche, funzionalità e limitazioni.

\section{Strumenti}

Lo strumento che abbiamo utilizzato per realizzare il simulatore  è \emph{Netlogo}, un software che offre un ambiente di sviluppo ideale per la realizzazione di modelli di simulazioni ad agenti, di networks e di sistemi dinamici (sviluppato nel presso il Center for Connected Learning and Computer-Based Modeling della Northwestern University).


\subsection{Netlogo}

Netlogo è un linguaggio di programmazione e ambiente di sviluppo open source che permette la modellazione di sistemi complessi formati da molteplici agenti che interagiscono tra loro, studiandone l'evoluzione  e visualizzandola in tempo reale.

L'ambiente di sviluppo è scritto in Java (con il vantaggio quindi di ottenere una grande portabilità del software stesso) e il linguaggio eredita ed estende le caratteristiche del linguaggio di programmazione multiparadigma Logo, realizzato negli anni '60 presso il Massachusetts Institute of Technology e caratterizzato dalla sua derivazione dal Lisp e le numerose applicazioni in ambito educativo. Il codice che definisce il comportamento degli agenti è interpretato senza necessità di essere compilato e questa caratteristica permette un'interazione a run time con il modello stesso (modificare parametri di controllo attraverso pulsanti e sliders, visualizzare dinamicamente variabili o grafici relativi alla simulazione in corso, etc..).

\myparagraph{Ambiente di sviluppo}

All'interno di Netlogo un elemento fondamentale è il "mondo virtuale", ovvero l'ambiente della simulazione all'interno del quale i diversi agenti agiscono. Gli agenti, le entità che possono eseguire istruzioni, possono essere di quattro tipi (Fig.~\ref{netlogoWorld}): \begin{itemize}
\item \emph{turtles} (tartarughe), gli agenti che possono muoversi all'interno del mondo;
\item \emph{patches}, le aeree quadrate che costituisco il mondo bidimensionale di Netlogo e sopra le quali si possono spostare le turtles;
\item \emph{links}, i collegamenti tra due diverse turtles, non hanno una posizione né risiedono su patches e possono orientati o non orientati;
\item \emph{observer}, il quale rappresenta concettualmente la vista complessiva del modello visto da fuori e contiene tutte le informazioni macro del modello e, quindi, tutte le variabili globali che lo caratterizzano.
\end{itemize}
Le tartarughe popolano il modello ed operano in parallelo, interagendo tra loro e con le patches su cui si muovono; possono essere specificate diverse tipologie di turtles che, nel linguaggio Logo, prendono il nome di breed, ovvero razza, caratteristica che in parte richiama il concetto di classe nel paradigma della programmazione ad oggetti, in quanto ogni razza possiede una lista di attributi e variabili proprietarie comuni solo agli agenti che vi appartengono.

\begin{figure}[H]
	\begin{center}
	\includegraphics[scale=0.3]{netlogoWorld}
	\end{center}
	\caption{Il mondo virtuale di Netlogo}
  	\label{netlogoWorld}
\end{figure}

L'ambiente di sviluppo è costituito da un'interfaccia grafica che consente di interagire intuitivamente con i parametri che regolano il modello o eseguire specifiche azioni, attraverso l'uso di determinati pulsanti, sliders o altri elementi inseriti durante lo sviluppo del modello; questa interfaccia ha anche l'importante funzione di mostrare in tempo reale i movimenti degli agenti all'interno del mondo virtuale e presentare durante e dopo la simulazione le informazioni relative alla stessa, sotto forma di grafici, tabelle, etc...

Accanto all'interfaccia grafica è ovviamente presente la sezione che riguarda il codice, il quale definisce i comportamenti delle entità  che agiscono dentro il mondo virtuale. Il codice della simulazione risiede tutto all'interno di un unico listato, suddiviso in diverse procedure che sono destinate all'esecuzione da parte degli agenti o, in maniera del tutto generale, di tutte le istanze che costituiscono il modello. Le procedure, in NetLogo, vengono suddivise in due diverse tipologie, ovvero \emph{commands}, azioni che devono essere portate a termine da un agente producendo un qualche risultato, e \emph{reporters}, istruzioni per calcolare un determinato valore che verrà riportato dall'agente a chiunque lo richieda. 

In Figura~\ref{netlogoUI_code} sono mostrati un esempio di una parte di iterfaccia grafica con relativi selettori e rappresentazione del mondo virtuale e il codice relativo al pulsante \emph{setup} di tale interfaccia.

\begin{figure}[H]
	\centering
	\subfloat[Interfaccia Grafica]{\includegraphics[scale=0.4]{netlogoUI}\label{netlogoUI}}
	\quad 
	\subfloat[Codice]{\includegraphics[scale=0.45]{netlogoCode}\label{netlogoCode}}
	\caption{Esempio dell'ambiente di sviluppo di NetLogo}
	\label{netlogoUI_code}
\end{figure}

Un altro elemento fondamentale dell'ambiente di sviluppo è rappresentato dalla gestione delle variabili aggregate e le relative statistiche. Poiché lo scopo di un modello di simulazione ad agenti è spesso rappresentato dalla necessità di comprendere a livello globale il comportamento di un sistema descritto nelle sue singole componenti ne deriva la necessità di calcolare e rappresentare l'andamento nel tempo di variabili aggregate per valutare qualitativamente il modello. In NetLogo ad ognuna delle variabili del modello è possibile associare un oggetto monitor, che ne mostra dinamicamente le variazioni, oppure un grafico, che ne raffigura l'andamento nel tempo. Gli  oggetti che permettono la gestione dei grafici sono \emph{plot} e \emph{histogram}, entrambi da definire sia a livello d'interfaccia grafica come avviene anche per le procedure, sia nel codice della simulazione.


\section{Modello ad agenti}

Dopo aver mostrato di quali strumenti software ci siamo serviti per sviluppare un modello che permetta di esaminare la produzione di energia da fonti rinnovabili nella regione Emilia-Romagna, introdurremo ora la metodologia scelta per realizzare la nostra modellazione di realtà.

In generale esistono numerose tecniche di simulazione che si differenziano per i diversi metodi di formalizzazione dei modelli, i linguaggi simbolici usati e i gli ambiti di applicazione più indicati (tecniche matematiche, statistiche, sperimentali,..). In questo lavoro abbiamo scelto di sviluppare un modello basato sul paradigma della simulazione ad agenti, il quale trae origine dall'unione della teoria della complessità con l'intelligenza artificiale distribuita.\\*\\*
Riprendendo la definizione di Russel e Norvig (\cite{Russell}) un \emph{agente} è qualsiasi cosa possa essere vista come un sistema che percepisce il suo ambiente attraverso sensori e agisce su di esso mediante attuatori, in particolare un agente software è un'istanza di una classe del paradigma della programmazione ad oggetti, ovvero un oggetto indipendente  che incapsula proprietà e funzioni autonome ed interagisce, secondo protocolli di comunicazione, con altri oggetti/agenti altrettanto autonomi. Sempre secondo Russel e Norving, ogni agente è in grado di percepire le sue stesse azioni ma non sempre gli effetti che ne derivano.
\\*
Gli autori descrivono accuratamente quattro tipi di agenti differenti:
\begin{itemize}
\item agenti reattivi semplici;
\item agenti reattivi basati sul modello;
\item agenti basati su obiettivi;
\item agenti basati sull'utilità.
\end{itemize}
Gli agenti reattivi semplici basano le loro azioni e decisioni solo sulla base dello stato attuale del modello, ovvero sulla percezione corrente, ignorando tutta la storia precedente. Agiscono quindi sulla base di una determinata regola la cui condizione corrisponde allo stato corrente.
\\*
Gli agenti basati sul modello invece devono memorizzare un stato interno che dipende dalla storia delle percezioni: questo tipo di agente deve, prima di compiere un'azione, aggiornare lo stato interno in funzione di quello presente, del tipo di azione che deve compiere e delle percezioni che rileva dall'ambiente, dopodiché si comporterà come un normale agente reattivo.\\* \\*
Gli agenti basati su obiettivi sono definiti così poiché oltre alle informazioni sullo stato corrente hanno bisogno di informazioni riguardanti i propri obiettivi. Infatti quest'ultimi, oltre a tenere traccia dell'ambiente nel quale agiscono, devono memorizzare un insieme di obiettivi e scegliere l'azione opportuna che li porterà a soddisfarli. Tale scelta è piuttosto semplice quando l'obiettivo può essere raggiunto con un solo passo esecutivo; diversamente, quando l'agente deve considerare lunghe sequenze di azioni alternative per poter scegliere il "cammino" che porta al risultato prefissato, può dotarsi di tecniche dell'intelligenza artificiale (IA) come la ricerca e la pianificazione.

Nella maggioranza dei casi reali non è sufficiente specificare degli obiettivi da raggiungere, ma occorre anche specificare quali stati siano maggiormente utili e quindi preferibili ad altri; è quindi opportuno considerare una funzione di utilità (detta funzione di fitness) che permetta di assegnare ad un determinato stato il relativo grado di utilità di quest'ultimo al fine del raggiungimento dell'obiettivo finale. Una specifica completa della funzione di utilità permette di effettuare scelte razionali nel caso in cui i soli obiettivi non bastano (es. più obiettivi in conflitto fra loro, confronto probabilità di successo e importanza degli obiettivi). Gli agenti che si servono della funzione di utilità per valutare il giusto cammino sono chiamati agenti basati sull'utilità.
\\* \\*
Dopo la breve panoramica sulle diverse tipologie di agenti che possono popolare un ambiente simulato, definiamo il concetto di sistema multi agente (\emph{MAS} Multi Agent System) come un insieme di agenti collocati in un determinato ambiente ed interagenti tra loro mediante una specifica organizzazione. Questi sistemi sono oggetto di ricerche da lunga data nell'ambito dell'intelligenza artificiale e costituiscono un'interessante tipologia di modellazione di società e proprio per questo troviamo diversi campi dove vengono utilizzati.

Da un punto di vista meno tecnico ma più sociologico, un modello ad agenti consente di ``indagare un dato fenomeno sociale macro attraverso la rappresentazione di regole di comportamento micro seguite da agenti che interagiscono all'interno di vincoli ambientali macro, siano essi di tipo geografico, spaziale, strutturale e/o istituzionale'' (Squazzoni, \cite{Squazzoni})

Una delle principali caratteristiche degli agenti è rappresentata dalla eterogeneità di questi ultimi che permette ai MAS di fornire nell'ambiente simulato una rappresentazione più inerente possibile della realtà, tenendo ovviamente in considerazione le inevitabili semplificazioni da apportare al modello. Inoltre nei sistemi MAS un altro vantaggio è la possibilità di rappresentare l'ambiente macro in maniera esplicita attraverso la definizione di vincoli e regole che ben si adattano ad essere modellate in un sistema software; a questo si possono aggiungere le tecniche di IA attraverso le quali gli agenti, all'interno della simulazione, possono apprendere (ad esempio sistemi a classificazione e alberi decisionali) ed evolvere nel medio/lungo periodo (algoritmi genetici).\\*\\*
Per concludere possiamo affermare che i sistemi multi agente si adattano molto bene a modellare fenomeni macro (prospettiva globale) attraverso l'utilizzo di condizioni micro che definiscono il comportamento dei singoli agenti. Scegliendo di utilizzare le simulazioni ad agenti tentiamo quindi di riprodurre un contesto di interazione da cui i fenomeni sociali emergono secondo un approccio bottom-up.


\section{Simulatore base}


Dopo aver introdotto nei paragrafi precedenti gli strumenti software che abbiamo utilizzato e il tipo di metodologia scelta, è giunto ora il momento di illustrare come il modello da simulare sia stato realizzato. Ci concentreremo prima sulla versione iniziale della simulazione ad agenti che ha lo scopo di modellare, individuare ed analizzare le principali caratteristiche che influenzano la scelta di un investimento nel settore dell'energia da fonti rinnovabili sul territorio della regione Emilia-Romagna, ed in particolare nel fotovoltaico.

Dalla prospettiva degli agenti (privati, aziende,...) prima di realizzare un investimento in questo settore è necessario effettuare alcuni studi relativi al luogo di installazione, all'irraggiamento solare, alla potenza dell'impianto ed al suo rendimento e contestualmente è di fondamentale importanza effettuare un'analisi approfondita che consenta di verificare con esattezza la convenienza ed il ritorno economico dell'investimento.

La simulazione è stata pensata con lo scopo di analizzare le esigenze connesse allo sviluppo di un nuovo progetto  fotovoltaico in fase di pianificazione e verificare la fattibilità dell'idea, fornendo quindi un utile strumento di valutazione per i singoli investitori, ma allo stesso tempo consente di ricavare informazioni di natura più globale, come la quantità totale di energia prodotta con tecnologie fotovoltaiche o le spese sostenute dalla regione, grazie alle quali è possibile integrare i risultati delle simulazioni all'interno del problema di ottimizzazione che ha come obiettivo la creazione di un piano energetico regionale.\\* \\*
La descrizione che segue non entrerà troppo nel dettaglio poiché l'implementazione vera e propria del simulatore è stata oggetto di un lavoro precedente \cite{tesiCroce}, ma una panoramica generale è necessaria per comprendere le dinamiche dell'ambiente simulato e comprendere quindi i risultati ottenuti.

\subsection{Descrizione simulatore}
Come molte simulazioni ad agenti la gestione del tempo simulato rappresenta un elemento fondamentale nella costruzione di un buon modello e nell'applicazione sviluppata si è deciso di riprodurre lo scorrere del tempo e quindi, di conseguenza, degli eventi ad esso legati con cadenza semestrale (questa scelta è dovuta alla natura della normativa Italiana sugli incentivi nel settore fotovoltaico). Secondariamente si è dovuto determinare l'arco temporale in cui sviluppare le simulazioni: il modello simula la creazione di nuovi impianti fotovoltaici a partire dal primo semestre del 2012 sino al secondo semestre del 2016 e, poiché la tariffa incentivante applicata all'energia prodotta dagli impianti, a partire dalla data di entrata in esercizio, è garantita per un periodo di 20 anni, ne consegue che la durata delle simulazioni si estende dal primo semestre del 2012 fino al secondo semestre del 2036.
\\* \\*
Per modellare correttamente le dinamiche del sistema complesso studiato è stato necessario introdurre diversi parametri che regolano diversi aspetti della simulazione. Ad esempio, l'energia elettrica che ogni impianto è capace di generare è strettamente legata alla posizione geografica e all'orientamento dei pannelli fotovoltaici che lo compongono(per semplificare l'orientamento e l'angolo d'inclinazione dei pannelli sono state considerate ottimali, ovvero verso Sud e inclinazione 30\textdegree). Per poter realizzare un modello che simuli la nascita di nuovi impianti in qualsiasi zona del territorio italiano l'irradiazione media annuale è un parametro globale del modello il cui valore è possibile variare l'interfaccia grafica. Ancora, il parametro che controlla il costo medio al kWp in funzione di quelli che sono i relativi costi degli impianti è regolabile tramite l'interfaccia, anche dinamicamente durante la simulazione in modo tale che il costo degli impianti vari nel tempo (generalmente tenderà a diminuire nel corso degli anni). Altre variabili globali legate agli impianti sono la perdita di efficienza annuale dei pannelli fotovoltaici e il costo di manutenzione.
\\* \\*
Una volta stabiliti parametri fondamentali e costi di un impianto avviene, ogni anno, la valutazione economica legata al rendimento di quest'ultimo e quindi relativa all'investimento sostenuto per realizzarlo. Questa fase prende in considerazioni fattori strettamente legati all'andamento del costo dell'elettricità oltre che alla tariffa incentivante riconosciuta per l'energia prodotta dall'impianto. Infatti i ricavi derivanti da un impianto fotovoltaico sono sia diretti, ovvero derivanti da incentivi e dall'eventuale vendita dell'energia in eccesso, ma anche indiretti come l'autoconsumo; per semplificare il modello è stata prevista una sola modalità di valorizzazione dell'energia prodotta e non direttamente consumata, ovvero il ritiro dedicato da parte del Gestore della Servizi Energetici (GSE).

\subsection{Agenti del modello}
Per ogni step esecutivo, compreso tra il 2012 e il 2016 il sistema genera un determinato numero di agenti, numero che può essere anche fatto variare dinamicamente tramite un opportuno selettore dell'interfaccia di Netlogo; questi agenti rappresentano gli attori del modello interessati a investire in un impianto fotovoltaico. Ognuno di essi è caratterizzato da diversi parametri, tra i quali è opportuno ricordare:
\begin{itemize}
\item \emph{Id}, identificativo univoco agente;
\item \emph{Superficie a disposizione}, metri quadri disponibili per installare l'impianto fotovoltaico;
\item \emph{Budget}, importo dedicabile all'investimento;
\item \emph{Consumo medio annuale di elettricità}, espresso in kWh annui;
\item \emph{Percentuale di copertura consumi richiesta}, obiettivo di copertura in percentuale dei consumi di elettricità tramite la produzione di energia dall'impianto fotovoltaico;
\item \emph{Aumento percentuale dei consumi annuali}, poiché tendenzialmente ogni anno i consumi tendono a variare, di solito aumentando;
\item \emph{Ostinazione}, nell'effettuare l'investimento, intervallo [1..100].
\end{itemize}
Questi parametri sono fondamentali per far si che l'agente posso valutare la fattibilità di realizzazione dell'impianto. Ad esempio nel caso di un agente che abbia come obiettivo la copertura del 100\% dei consumi annuali di elettricità ma abbia a disposizione una superficie che può ospitare un impianto che al massimo garantirebbe una produzione di energia pari al 50\% del fabbisogno, questo agente sarebbe costretto ad abbandonare l'idea di un investimento a meno di non ridurre i propri obiettivi di produzione energetica e quindi di copertura dei consumi risultando particolarmente ostinato ad effettuare l'investimento. Oppure potrebbero comparire considerazioni legate a questioni economiche: nel caso in cui il budget a disposizione dell'agente non sia sufficientemente a garantire la copertura dei costi di realizzazione e installazione dell'impianto potrebbe risultare obbligatorio abbandonare il progetto a meno di non ricorrere a un prestito. 

Gli agenti che optano per abbandonare l'idea di un investimento vengono eliminati dal modello mentre quelli che, a margine dello studio di fattibilità, decidono di effettuare l'investimento daranno vita alla generazione di un nuovo impianto. Come per gli agenti, gli impianti possiedono attributi che li caratterizzano: data di entrata in funzione, tipologia, tecnologia di realizzazione, costo, fascia potenza, tariffa incentivante riconosciuta e dimensione.\\* \\*
Oltre agli agenti che modellano i possibili investitori appena descritti, abbiamo aggiunto al simulatore anche un altro tipo di entità, cioè un unico agente che rappresenta la regione Emilia-Romagna e ci ha permesso di implementare alcune politiche centralizzate anche nel mondo totalmente distribuito di Netlogo (ad esempio è stato così possibile fare in modo che il budget che la regione mette a disposizione per gli incentivi non dovesse essere fissato all'inizio del periodo compreso tra 2012 e 2016, ma fosse frazionato in diversi budget annuali).

\subsection{Valutazione fattibilità}
Sulla base dei parametri che definiscono il modello ogni agente è in grado di valutare la fattibilità economica dell'investimento calcolando il VAN (il Valore attuale netto dell'investimento) per l'impianto da realizzare; da questo valore è possibile individuare il PBT (Pay Back Time), il periodo necessario per il ``ritorno'' dell'investimento iniziale, e il ROE (Return On Equity), l'indice di redditività del capitale investito.
\\*
Alcune variabili globali usate per la valutazione di fattibilità sono:
\begin{itemize}
\item i prezzi dell'energia elettrica praticati all'utente finale (distinti in cinque fasce di consumo);
\item la variazione annuale prezzi elettricità, parametro modificabile dinamicamente, incide sia sui prezzi dell'energia elettrica sia sui prezzi minimi garantiti dal GSE per il ritiro dedicato;
\item costo medio al kWp, cioè il costo medio per ogni kWp installato;
\item incentivi installazione e percentuale incentivi installazione, rappresentano il meccanismo principale con cui la regione può tentare di influenzare le scelte degli agenti sottraendo dal costo sostenuto per l'impianto una percentuale del costo stesso;
\item tasso lordo rendimento BOT, utilizzato per il calcolo dei flussi di cassa ed avere un parametro di confronto per quanto concerne la redditività del capitale investito;
\item prezzi minimi garantiti dal GSE per il ritiro dedicato (sulla base della potenza minima dell'impianto).
\end{itemize}

Una volta assegnati i parametri globali e i parametri individuali (con valori casuali o a partire da serie storiche) per ogni agente vengono individuate le caratteristiche di potenza che l'impianto fotovoltaico dovrà soddisfare, in particolare per determinare la quantità di energia elettrica che l'impianto dovrà produrre al fine di soddisfare l'esigenze dell'agente. Sulla base di tali esigenze si può individuare quale sarà la dimensione dell'impianto, dalla quale è poi facile dedurre i costi di realizzazione dello stesso.

Una volta che tutte queste informazioni sono a disposizione degli agenti essi possono effettuare la valutazione della fattibilità dell'investimento, procedimento che può essere riassunto nel flow chart di Figura~\ref{flowChartValFatt}.
\\*
Nel  momento in cui un agente valuta la fattibilità dell'investimento si possono verificare i seguenti scenari:
\begin{itemize}
\item le dimensione e il costo dell'impianto sono inferiori rispettivamente  alla superficie e al budget a disposizione, l'investimento può essere fatto e l'agente valuta se aumentare eventualmente le dimensioni dell'impianto;
\item le dimensioni dell'impianto e il costo sono superiori alla superficie e al budget a disposizione quindi l'agente viene eliminato dal modello in quanto non ci sono le condizioni per effettuare l'investimento;
\item le dimensioni dell'impianto sono superiori alla superficie disponibile ma il budget a disposizione è sufficiente a coprire il costo di realizzazione, l'agente decide se accettare una riduzione di potenza dell'impianto e quindi della sua dimensione;
\item le dimensioni dell'impianto sono inferiori alla superficie disponibile ma il budget non è sufficiente a coprire i costi dell'investimento, l'agente considera l'ipotesi di prendere in prestito la somma residua.
\end{itemize}

\begin{figure}[hbt]
	\centering
	\includegraphics[scale=0.5]{valutaFatt}
	\caption{Flow chart della valutazione degli investimenti}
	\label{flowChartValFatt}
\end{figure}

In tutti casi, a esclusione dell'eliminazione dal modello, il comportamento successivo dell'agente dipende, oltre che da fattori economici, anche da un parametro, l'\emph{ostinazione}, che simula la determinazione a istallare un impianto fotovoltaico (ad esempio per motivazioni legate a considerazioni ecologiche), ed entra in gioco per le decisioni inerenti l'aumento delle dimensioni dell'impianto, il suo ridimensionamento o l'accettazione di un prestito.

In Figura~\ref{mondoVirtualeSim} possiamo osservare il mondo virtuale di Netlogo popolato di agenti che hanno deciso di effettuare l'investimento e realizzare un impianto fotovoltaico; in verde troviamo gli agenti che non hanno avuto problemi di budget o spazio a disposizione, in blu quelli che hanno accettato il ridimensionamento e in rosso quelli che anno accettato un prestito.

\begin{figure}[hbt]
	\centering
	\includegraphics[scale=0.5]{mondoVirtualeSim}
	\caption{Mondo virtuale di Netlogo}
	\label{mondoVirtualeSim}
\end{figure}

\subsection{Esecuzione del modello}

Una volta popolato il modello con un certo numero di agenti (nell'intervallo [0..100]) viene avviata l'esecuzione della simulazione, per tutta la durata della quale continueranno ad essere svolte azioni come l'aggiornamento degli impianti (anni di vita, rendimento, calcolo energia prodotta), aggiornamento dei consumi di elettricità degli agenti, calcolo dei ricavi connessi alla produzione energetica, aggiornamento dei parametri globali del modello che cambiano insieme allo scorrere del tempo simulato. 
\\* \\*
Una volta terminata l'esecuzione della simulazione vengono prodotte e comunicate tramite l'apposita interfaccia grafica una serie di informazioni, alcune di maggior interesse per i singoli investitori, come ad esempio il PBT ed il ROE medio per i diversi semestri in cui è prevista la creazione di un impianto, altre utili per  per determinare le grandezze caratteristiche dell'ambiente simulato, come ad esempio la potenza installata nel complesso e quella nei relativi anni(espressa in kWp), spesa complessiva per gli incentivi all'installazione ed alla produzione, spesa totale e percentuale di impianti realizzati (cioè quanti agenti tra quelli generati all'inizio della simulazione hanno effettivamente portato a termine un impianto.)

Questo tipo di informazioni (mostrate in Figura~\ref{infoTotSim}) sono molto importanti per politici o aziende del settore, per valutare nel complesso quale potrebbe essere la risposta degli investitori al variare di determinate grandezze quali, ad esempio, le tariffe incentivanti o i costi di realizzazione. 

\begin{figure}[hbt]
	\centering
	\includegraphics[scale=0.8]{infoTotSim}
	\caption{Grandezze caratteristiche della simulazione}
	\label{infoTotSim}
\end{figure}

Ricordiamo che in questa prima versione del simulatore lo strumento unico attraverso cui la regione può intervenire direttamente (senza considerare, ad esempio, miglioramenti nelle tecnologie utilizzate che possano diminuire i costi di un impianto a pannelli fotovoltaici) al fine di incrementare la produzione energetica da tecnologia fotovoltaica, consiste nel coprire una percentuale della spesa sostenuta dagli agenti all'atto dell'installazione di un impianto.\\*\\* 
La cosa principale da capire riguardante il comportamento degli agenti è il fatto che essi prendano le loro decisioni in maniera individuale sulla base delle proprie considerazioni di fattibilità economica, guadagno o ostinazione personale e alla fine le loro scelte singole concorrono a generare una certa produzione totale di energia da impianti fotovoltaici. Dal punto di vista della regione Emilia-Romagna il valore della produzione di energia fotovoltaica totale è fondamentale in quanto è il parametro osservabile in uscita dal simulatore che rivela se il piano energetico regionale, il quale prevede che certe quote di energia debbano essere prodotte da fonti rinnovabili, sia fattibile, o se invece sia necessario un investimento maggiore fornendo ad esempio incentivi di entità maggiore, grazie ai quali un maggior numero di agenti possa effettuare la scelta di installare un impianto fotovoltaico.


\section{Simulatore esteso}

Dopo aver realizzato il primo modello descritto nel paragrafo precedente \cite{tesiCroce}, ci è sembrato utile estenderlo con nuove funzionalità nella direzione di rendere le simulazioni maggiormente realistiche rispetto alle dinamiche del sistema complesso reale e fornire modalità di incentivazione più avanzate con le quali i decisori politici possano intervenire per ottenere la produzione energetica da energia fotovoltaica desiderata \cite{lavoroCerri}.

Le estensioni introdotte e di cui andremo ora a discutere sono quindi due: quattro distinte modalità di incentivazione e l'inclusione all'interno del modello dell'aspetto legato all'iterazione sociale.

\subsection{Modalità incentivanti}

La regione deve destinare un budget per gli incentivi, esaurito il quale nessuno può usufruire di tali facilitazioni economiche. Le varie tipologie d’incentivi sono tra loro alternative e prima della partenza della simulazione è necessario scegliere quale applicare; non è quindi possibile studiare attraverso il simulatore quali possano essere le interazioni tra diverse metodologie incentivanti. Un agente non è tenuto ad usufruire degli incentivi, ad esempio può non essere a conoscenza di tali iniziative regionali, o può non avere intenzione di accendere un mutuo (per questo un parametro della simulazione è la probabilità che un agente voglia ricorrere all’incentivo).

\myparagraph{Assegnazione di fondi}

In questo caso ogni agente chiede alla regione di finanziargli una percentuale dell'investimento da effettuare per istallare l'impianto (le percentuali massima e minima che è possibile richiedere sono parametri della simulazione). Nella nostra implementazione la regione considera le richieste nel semplice ordine di arrivo e assegna i fondi fino ad esaurimento del budget dedicato agli incentivi (nella realtà sicuramente potrebbero essere implementate politiche più complesse ed elaborate per scegliere quali richieste soddisfare e in che ordine). Questo tipo di incentivo è definito a fondo perduto poiché non prevede nessun ritorno economico per la regione.

\myparagraph{Conto interessi}

Gli agenti possono decidere ora di accendere un mutuo presso una banca e gli interessi relativi saranno pagati dalla regione, attingendo dal budget dedicato agli incentivi (perciò anche in questo caso a fondo perduto); da questo consegue che gli agenti hanno la possibilità di rateizzare l'investimento iniziale ed essere quindi influenzati positivamente verso la scelta di procedere con l'istallazione dell'impianto, anche sulla base di un parametro della simulazione che modella quanto sia importante poter pagare a rate.

Anche in questo caso nella nostra implementazione di questa metodologia prevede che i primi agenti a presentare richiesta siano i primi ad essere soddisfatti fino a esaurimento dei fondi; ulteriori semplificazioni sono dati dal fatto di aver inserito solo due parametri che rappresentino gli interessi di una generica banca e la probabilità che la stessa non permetta sempre di stipulare un mutuo.

\myparagraph{Fondo Rotazione}

La regione permette agli agenti di realizzare mutui presso di sé, offrendo tassi ad interesse agevolato; questo è l'unico metodo di incentivazione grazie al quale la regione può ricavare nuovi fondi, dal momento che gli incentivi, seppur bassi, che dovranno essere pagati garantiscono una fonte di guadagno.

Anche in questo caso l'eventuale numero maggiore di impianti realizzati è correlato all'ostinazione degli agenti, oltre che a parametri strettamente di natura economica come gli interessi applicati dalla regione e il numero di anni concessi per la restituzione del mutuo.

\myparagraph{Fondo Garanzia}

Con quest'ultimo metodo, ancora a fondo perduto, abbiamo considerato il caso in cui un'ipotetica banca non conceda ad un cliente di accendere un mutuo per mancanza di garanzie economiche: l'incentivo \emph{Fondo Garanzia} prevede che la regione fornisca le garanzie richieste dalla banca, ovvero è sempre possibile per un agente accendere un mutuo presso la banca, ma nel caso in cui tale agente si trovi nell'impossibilità di continuare a pagare le rate (la probabilità che questo accada è un parametro del modello), sarà la regione a intervenire coprendo le spese rimanenti, attingendo al fondo stanziato per gli incentivi fino al suo esaurimento.\\*\\*

Lo studio della fattibilità di un investimento nel caso del simulatore esteso con i tipi di incentivi sopra citati possiede lo stesso schema di quello presente nel primo modello implementato (Fig.~\ref{flowChartValFatt}). Le differenze si possono riscontrare, ad esempio, nel diverso valore della variabile \emph{Costo impianti}, che ora deve tenere in considerazione il costo degli interessi di eventuali mutui e le riduzioni derivanti dalla presenza di tecniche incentivanti, o ancora nel fatto che il parametro che \emph{Ostinazione} (carateristico di ogni agente e che influisce sulla determinazione a realizzare un impianto nei casi dove sia necessaria la richiesta di un prestito o il ridimensionamento) sia ora costituito da tre diverse componenti, cioè il valore originario di ostinazione, il livello di influenza delle rate (o meglio la possibilità di rateizzare le spese) e il livello di influenza dell'iterazione sociale (descritto nel paragrafo successivo).

\subsection{Interazione sociale}

Poiché il simulatore iniziale aveva un comportamento esclusivamente deterministico col quale la produzione di energia fotovoltaica degli agenti era condizionata unicamente da fattori di tipo economico, abbiamo deciso di estenderlo verso una direzione più realistica, dove considerare le interazioni sociali tra i diversi agenti. Il tentativo è stato fatto col fine di approssimare una rete small word, un sistema cioè in cui se un agente prende una decisione, anche i suoi vicini ne sono influenzati.

In sintesi è stato assegnato ad ogni agente un valore (una componente dell'ostinazione vista in precedenza) che rappresenta quanto sia significativa l'influenza del comportamento dei vicini, cercando di riflettere la tendenza a seguire il comportamento del gruppo in cui ci si trova tipica degli esseri umani del mondo reale. In particolare le decisioni di ogni agente sono modificate dalla sua personale sensibilità ai comportamenti dei vicini e dalle dimensioni dell'area di influenza, ovvero dal raggio che determina la zona circolare all'interno della quale le scelte fatte da un agente possono influenzare il comportamento degli altri.

In Figura~\ref{iterSocMondoVirt} possiamo osservare un esempio di come si formino le aree di influenza tra agenti all'interno del mondo virtuale di Netlogo - le aree di influenza sono le zone chiare di forma quasi circolare centrate sulle case che rappresentano gli agenti.

\begin{figure}[hbt]
	\centering
	\includegraphics[scale=0.5]{iterSocMondoVirt}
	\caption{Mondo virtuale, area interazione sociale}
	\label{iterSocMondoVirt}
\end{figure}

\section{Limiti simulatore}

Il modello implementato presenta al suo interno diverse assunzioni e approssimazioni effettuate per semplificarne l'implementazione, ma al tempo stesso consente di effettuare uno studio accurato delle proprietà interessanti nell'ambito di questo lavoro, come ad esempio l'analisi delle risposte degli agenti di fronte alla presenza di diverse tipologie di incentivi e al variare del fondo destinato all'incentivazione delle tecnologie fotovoltaiche stanziato dalla regione Emilia-Romagna. Vogliamo quindi precisare che per il caso di studio considerato, ovvero le interazioni tra simulazione e ottimizzazione per la realizzazione di un piano energetico regionale, il modello utilizzato soddisfa i requisiti necessari ad una corretta trattazione.

Dopo questa debita premessa è comunque utile presentare brevemente quali possono essere alcuni limiti del simulatore realizzato e in che direzione possano procedere eventuali futuri lavori.\\* \\*
Innanzitutto occorre precisare che le grandezze prodotte dal simulatore, come la produzione energetica totale o la spesa, non siano quantitativamente confrontabili con i corrispondenti valori nel mondo reale, non fosse altro per il fatto che il nostro simulatore prevede un numero di agenti compreso tra nell'intervallo [1..100] e quindi sicuramente un valore minore rispetto ai possibili privati e aziende interessati a realizzare impianti fotovoltaici in Emilia-Romagna; questo limite può essere facilmente affrontato applicando un fattore di scala, opportunamente calcolato, ai risultati generati durante le simulazioni.

Per quanto riguarda la gestione degli incentivi, possiamo far notare che un meccanismo di assegnazione di fondi ad asta nella realtà prevederebbe sicuramente modalità più complesse (ad esempio la presenza di intervalli temporali in cui i potenziali investitori comunichino alla regione la percentuale loro necessaria, un ordinamento della richieste secondo qualche criterio, una negoziazione sulla percentuale da finanziare) e analogamente anche per le restanti modalità d'incentivazione.

Sempre considerando gli incentivi, un'estensione molto utile sarebbe la possibilità di effettuare simulazioni con diversi tipi di incentivazione contemporaneamente, poiché invece al momento è selezionabile un solo tipo per volta; per ora le metodologie incentivanti sono considerate come indipendenti, mentre nel mondo reale la situazione è certamente più complessa per via di eventuali interazioni tra tali metodologie.

Un altro punto importante è il meccanismo con cui viene effettuata la valutazione di fattibilità dell'investimento, cercando di renderlo più realistico implementando una strategia decisionale più elaborata che tenga conto del budget di partenza dei potenziali investitori, la possibilità di pagare a rate e la presenza di interazione sociale.
\\* \\*
Per quanto riguarda l'interazione sociale qui implementata, essa considera semplicemente l’influenza dei vicini, cioè
all’aumentare dei vicini che hanno realizzato l’investimento aumenta anche la probabilità dell’agente stesso di realizzarlo, mentre esistono ovviamente modelli sociologici molto più complessi, completi e realistici che possono essere presi in considerazione. Tra le altre cose potrebbe essere introdotto un altro fattore che agisca parallelamente all'ostinazione, all'influenza delle rate e quella sociale, in particolare uno che tenga in conto il fatto che all’aumentare di risorse che fornite dalla regione vi possa essere un maggior desiderio di intraprendere l’investimento; o ancora, nel nostro caso abbiamo solamente considerato l'effetto positivo dell'iterazione sociale, mentre un modello più accurato potrebbe anche tenere conto di chi non effettua l'investimento e scoraggia così i restanti investitori del vicinato.

%\nocite{*}
%\bibliographystyle{plain}
%\bibliography{bibliography}

%\end{document}


% analisi dei risultati delle simulazioni
%Andrea Borghesi
%Università degli studi di Bologna

%capitolo dedicato all'analisi dei risultati delle simulazioni 

\documentclass[12pt,a4paper,openright,twoside]{report}
\usepackage[italian]{babel}
\usepackage{indentfirst}
\usepackage[utf8]{inputenc}
\usepackage[T1]{fontenc}
\usepackage{fancyhdr}
\usepackage{graphicx}
\usepackage{titlesec,blindtext, color}
\usepackage[font={small,it}]{caption}
\usepackage{subfig}
\usepackage{listings}
\usepackage{color}
\usepackage{url}
\usepackage{textcomp}

%impostazioni generali per visualizzare codice
\definecolor{dkgreen}{rgb}{0,0.6,0}
\definecolor{gray}{rgb}{0.5,0.5,0.5}
\definecolor{mauve}{rgb}{0.58,0,0.82}
 
\lstset{ %
  basicstyle=\footnotesize,           % the size of the fonts that are used for the code
  backgroundcolor=\color{white},      % choose the background color. You must add \usepackage{color}
  showspaces=false,               % show spaces adding particular underscores
  showstringspaces=false,         % underline spaces within strings
  showtabs=false,                 % show tabs within strings adding particular underscores
  tabsize=2,                      % sets default tabsize to 2 spaces
  breaklines=true,                % sets automatic line breaking
  breakatwhitespace=false,        % sets if automatic breaks should only happen at whitespace
  title=\lstname,                   % show the filename of files included with \lstinputlisting;
                                  % also try caption instead of title
  keywordstyle=\color{blue},          % keyword style
  commentstyle=\color{dkgreen},       % comment style
  stringstyle=\color{mauve},         % string literal style
  escapeinside={\%*}{*)},            % if you want to add LaTeX within your code
  morekeywords={*,...},              % if you want to add more keywords to the set
  deletekeywords={...}              % if you want to delete keywords from the given language
}

%per avere un bordo intorno alle figure
\usepackage{float}
\floatstyle{boxed} 
\restylefloat{figure}

%per poter poi impedire che certe parole vadano a capo
\usepackage{hyphenat}
\usepackage{listings}

%ridefinisco font per fancyhdr, per ottenere un'intestazione pulita
\newcommand{\changefont}{ \fontsize{9}{11}\selectfont }
\fancyhf{}
\fancyhead[LE,RO]{\changefont \slshape \rightmark} 	%section
\fancyhead[RE,LO]{\changefont \slshape \leftmark}	%chapter
\fancyfoot[C]{\changefont \thepage}					%footer

%titolo capitolo con "numero | titolo"
\definecolor{gray75}{gray}{0.75}
\newcommand{\hsp}{\hspace{20pt}}
\titleformat{\chapter}[hang]{\Huge\bfseries}{\thechapter\hsp\textcolor{gray75}{|}\hsp}{0pt}{\Huge\bfseries}


\oddsidemargin=30pt \evensidemargin=20pt

%sillabazioni non eseguite correttamente
\hyphenation{sil-la-ba-zio-ne pa-ren-te-si si-mu-la-to-re ge-ne-ra-re pia-no}

%interlinea
\linespread{1.15}  
\pagestyle{fancy}

%cartelle contenenti le immagini
\graphicspath{{/media/sda4/tesi/immagini/grafici/}{/media/sda4/tesi/immagini/grafici/incCompare/}{/media/sda4/tesi/immagini/grafici/rawData/}{/media/sda4/tesi/immagini/grafici/regressionAnalysis/}{/media/sda4/tesi/immagini/schemi/}{/media/sda4/tesi/immagini/simulazione/}}

%in modo che dopo il titolo di un paragrafo il testo vada a capo
\newcommand{\myparagraph}[1]{\paragraph{#1}\mbox{}\\}

\begin{document}
\chapter{RISULTATI SIMULAZIONI}

Nel capitolo precedente abbiamo descritto il modello ad agenti implementato, evidenziandone le finalità e le caratteristiche fondamentali, accennando brevemente alle informazioni ricavabili dal simulatore.\\*
Lo scopo di questo capitolo sarà quindi la dettagliata analisi dei dati prodotti dalle simulazioni, l'individuazione delle relazioni che legano le grandezze in gioco all'interno dell'ambiente simulato, la presentazione e discussione dei risultati ottenuti.\\*
Inizialmente presenteremo gli strumenti utilizzati per effettuare l'analisi sopra descritta, per poi passare alla discussione vera e propria nei paragrafi successivi.

\section{STRUMENTI}
Per esaminare i dati prodotti dalle simulazioni effettuate e visualizzare i risultati ottenuti abbiamo utilizzato \emph{R}, un ambiente di sviluppo specifico per l'analisi statistica dei dati, basato sull'omonimo linguaggio di programmazione.

\subsection{R}

R è un linguaggio di programmazione open source e un ambiente software usato per la manipolazione di dati, calcolo e analisi statistica e presentazione grafica dei risultati. Il design R è stato ampiamente influenzato da due linguaggi preesistenti, S sviluppato da J.Chambers e colleghi presso i Bell Laboratories negli anni '70 e Scheme creato presso il MIT AI Lab sempre negli anni settanta da G.L.Steele e G.J.Sussman. \\*
Il nucleo di R consiste di un linguaggio interpretato a cui sono state aggiunte numerose funzionalità per un grande numero di procedure statistiche;  tra queste è possibile ricordarne alcune: modelli di regressione lineare, lineare generalizzata e non lineare, analisi di serie temporali, classici test parametrici e non, clustering, classificazione e altre. R è facilmente estendibile grazie alla presenza di numerosi pacchetti software creati dagli utenti e dedicati a specifiche aree di studio e possiede inoltre un grande insieme di funzioni indicate per una presentazione flessibile ed efficiente dei dati e la produzione di grafici di qualità.\\*
Per interagire con l'interprete del linguaggio R è possibile fornire le istruzioni direttamente da riga di comando oppure appoggiarsi a interfacce grafiche, ma per le nostre necessità è stato sufficiente utilizzare la riga di comando\\* \\* 
Per via della sua derivazione da S, R presenta alcune caratteristiche che lo fanno rientrare all'interno del paradigma dei linguaggi Object Oriented, almeno parzialmente, e al tempo stesso possiede alcuni aspetti che lo avvicinano alla natura dei linguaggi funzionali(come Scheme), come ad esempio la possibilità di trattare le funzioni stesse come oggetti.\\* Le principali strutture dati sono le seguenti: \begin{itemize}
\item \emph{vettori}, singole entità costituite da una collezione di valori di un certo tipo come ad esempio numerici,logici o caratteri;
\item \emph{matrici (arrays)}, generalizzazioni multi-dimensionali di vettori;
\item \emph{liste}, forme di vettori più generali nelle quali gli elementi non devono necessariamente essere dello stesso tipo;
\item \emph{fattori}, oggetti simili ai vettori usati per specificare una classificazione (raggruppamento) delle componenti di altri vettori con la stessa lunghezza;
\item \emph{data frames}, strutture simili alle matrici in cui le colonne possono essere di tipi diversi;
\item \emph{funzioni}, le quali sono esse stesse oggetti e forniscono così un modo semplice e flessibile di estendere R.
\end{itemize}
Come in ogni linguaggio di programmazione è poi ovviamente possibile manipolare queste strutture dati attraverso operatori, strutture di controllo, funzioni, etc...\\* \\*
Illustreremo ora un brevissimo esempio per far capire un possibile utilizzo di R per effettuare una semplice analisi statistica. Supponiamo di voler studiare la relazione che lega due variabili, \emph{a} e \emph{b}, i cui valori si trovano in un file di tipo \emph{Comma Separated Values}. Il primo passo è importare tali valori dal file e inserirli in una struttura dati, in questo caso una matrice con due colonne (una per ogni variabile) e ordinarli in base ai valori della prima variabile.

\lstset{language=R}
\begin{lstlisting}
> matrice.dati <- read.csv("file.csv")
> matrice.ordinata <- matrice_dati[order(matrice.dati$a),]
\end{lstlisting}

A questo punto sarebbe possibile svolgere diverse operazioni sui dati (ad esempio calcolare per ogni valore di ogni variabile i valori medi,...) ma ci limiteremo a effettuare una semplice regressione lineare.
\begin{lstlisting}
> modello.lineare <- lm(matrice$b ~ matrice$a)
\end{lstlisting}

R ci consente ora di effettuare analisi statistiche sul modello di regressione applicato per stabilirne validità e significatività in rapporto ai dati in nostro possesso e successivamente di presentare graficamente i risultati ottenuti.
\begin{lstlisting}
> #analisi statistica minima
> summary(modello.lineare)       
> #disegna i punti corrispondenti ai valori nella matrice
> plot(matrice$b ~ matrice$a,type="p",lwd=3,ylab="b",xlab="a")    
> #disegna la curva di regressione
> lines(matrice$a,predict(modello.lineare), lty="solid", col="darkred", lwd=2)    
\end{lstlisting}


In Figura ~\ref{example_r} sono stati riportati il grafico prodotto da questo esempio e i risultati ottenuti dalla semplicissima analisi statistica, tra i quali notiamo il coefficiente di determinazione (R-squared) e l'errore residuo ( 
Residual standard error).

\begin{figure}[H]
	\centering
	\subfloat[Grafico]{\includegraphics[scale=0.4]{example_r_graph}\label{example_r_graph}} 
	\quad
	\subfloat[Analisi statistica]{\includegraphics[scale=0.6]{example_r_result}\label{example_r_result}}
	\caption{Esempio di utilizzo di R}
	\label{example_r}
\end{figure}


\section{METODOLOGIA ANALITICA}

Illustreremo ora con quali metodi sono stati analizzati i dati ricavati dalle simulazioni; questo comporta anche una rapida descrizione delle tecniche statistiche impiegate e della loro applicazione nel nostro contesto.

\subsection{ANALISI DI REGRESSIONE}
Una delle tecniche statistiche maggiormente utilizzate per stimare le relazioni tra variabili è l'\emph{analisi della regressione}; in questa categoria rientrano diversi metodi che hanno come obiettivo quello di trovare un modello che leghi una variabile dipendente ed una o più variabili indipendenti (in particolare l'analisi della regressione consente di capire come cambia il valore di una variabile dipendente al variare del valore di una variabile indipendente, mantenendo fisse le restanti); seguendo la terminologia di uso comune in seguito le variabili indipendenti saranno chiamate anche \emph{predittori}.  
I modelli con i quali si tenta di approssimare le relazioni studiate possono essere di numerosi tipi, tra i quali è possibile ricordare quelli parametrici (come la regressione lineare e più in generale tutte le forme di regressione polinomiale), nei quali la funzione di regressione è definita attraverso un certo numero di parametri stimati a partire dai dati, e quelli non parametrici, poi ancora modelli locali (regressione LOESS), Bayesiani, segmentati, etc..\\*
In genere la scelta del giusto modello da applicare ai propri dati è un procedimento empirico che prevede di tentare differenti tecniche di regressione sulla stessa serie di dati per poi valutare quale fosse la scelta migliore, ovvero quale sia il modello di regressione che presenta la maggiore bontà di adattamento (in inglese ''goodness of fit''), cioè una misura che riassume la discrepanza tra i valori osservati e i valori attesi sotto il modello in questione. 
Una volta scelto il modello migliore, questo può essere usato per fare predizioni, comprendere in che modo in che modo certe variabili o aspetti di un problema ne influenzino altri, essere integrato all'interno di un sistema informatico attraverso tecniche di apprendimento automatico (come sarà mostrato nel prosieguo d questa trattazione). 
\\*\\*
Un aspetto di grande importanza è quindi la validazione del modello, valutare cioè se è in accordo con i dati presi in esame. Tra i diversi metodi di validazione possibili alcuni prevedono metodi numerici, come ad esempio il calcolo del coefficiente di determinazione, altri richiedono l'uso di tecniche più qualitative, ad esempio l'analisi grafica dei valori residuali; in genere per effettuare una validazione completa e affidabile vengono impiegate tecniche appartenenti ad entrambe le categorie ed anche in questo lavoro abbiamo agito in questo modo.\\*\\*
Uno dei principali indicatori numerici usati per valutare la bontà di un modello di regressione è il calcolo del \emph{coefficiente di determinazione}, o $R^2$, un numero reale compreso tra 0 e 1 che misura proporzione di variabilità della risposta dovuta al modello statistico; un valore vicino a 0 indica che la regressione scelta non si adatta ai dati, viceversa valori vicini a 1 indicano che il modello è buono.

\subsection{IMPLEMENTAZIONE}

Per effettuare l'analisi della regressione nel nostro caso, il primo passo è consistito nello scegliere le variabili di cui studiare la relazione; per ogni tipologia di incentivazione sono stati considerati tre casi (riportati in seguito in maniera estesa):
\begin{itemize}
\item relazione tra il budget che la regione dedica agli incentivi e produzione energetica da impianti fotovoltaici, con il budget variabile indipendente e la produzione energetica variabile dipendente;
\item relazione tra la sensibilità degli agenti simulati all'influenza dell'interazione sociale e la produzione energetica, essendo ancora ques'ultima la variabile dipendente;
\item relazione tra il raggio dell'interazione sociale, predittore, e produzione energetica da fotovoltaico.
\end{itemize}

\section[ANALISI RISULTATI]{ANALISI RISULTATI SIMULAZIONI}

Dopo aver implementato il simulatore descritto nel precedente capitolo, considerando in particolar modo la versione estesa, siamo passati ad analizzare le relazioni che legano la produzione di energia elettrica alle diverse metodologie di incentivi e relativi fondi stanziati dalla regione. Possiamo subito anticipare che, come era lecito attendersi, la presenza di un qualsiasi tipo di incentivo permette di ottenere una produzione energetica maggiore rispetto al caso di assenza di incentivi e inoltre all'aumentare dei fondi stanziati per finanziare un tipo di incentivo la produzione di energia da impianti fotovoltaici aumenta di conseguenza.\\*
In un secondo momento siamo passati a studiare la relazione tra produzione energetica e interazione sociale (considerando sia variazioni del raggio che della sensibilità); in modo conforme alle nostre aspettative, anche in questo caso i risultati ottenuti indicano che una maggiore produzione energetica è associata ad un'interazione sociale più intensa.\\* \\*
Per studiare le relazioni di nostro interesse la metodologia scelta consiste nell'aver effettuato un grande numero di simulazioni controllate (ovvero fissando tutti i parametri non rilevanti e variando quelli di cui osservare il comportamento, come il budget regionale o il tipo di incentivo), dopodiché abbiamo effettuato una semplice analisi statistica dei dati e tentato di risalire alle curve relative all'andamento delle relazioni attraverso l'uso di tecniche di regressione lineare e non.


\subsection{COMPORTAMENTO DEGLI INCENTIVI}

Il comportamento degli incentivi è stato studiato effettuando numerose simulazioni per ogni tipo di incentivo, variando la dimensione del fondo dedicato agli incentivi per il fotovoltaico. Il fondo è stato aumentato con incrementi di un milione di euro a partire da zero  fino a un massimo di 40 milioni ( considerando un arco temporale di cinque anni ); per ogni valore sono state effettuate 300 simulazioni, per un totale di 48000 simulazioni considerando tutti i diversi incentivi.\\*
Proseguiremo ora esaminando i singoli incentivi per poi concludere confrontandoli tra loro.

\myparagraph{Fondo Asta}

\begin{figure}[H]
	\centering
	\subfloat[Simulazioni]{\includegraphics[scale=0.55]{graphSimA_R}\label{graphSimA_R}}
	\qquad
	\subfloat[Relazione]{\includegraphics[scale=0.55]{regr_graphSimA_R}\label{regr_graphSimA_R}}
	\caption{Fondo Asta}
	\label{graphSimA}
\end{figure}


\myparagraph{Conto Interessi}

\myparagraph{Fondo Rotazione}

\myparagraph{Fondo Garanzia}

\myparagraph{Confronto Incentivi}

\subsection{EFFETTI DELL'INTERAZIONE SOCIALE}

Gli effetti dell'iterazione sociale sulla produzione energetica sono stati studiati agendo sui due parametri che possono influenzarla, il raggio dell'interazione e la sensibilità all'influenza derivante dal comportamento dei vicini, ed effettuando numerose simulazioni controllate: per ricavare la relazione tra produzione energetica e raggio questo è stato fatto variare da 1 fino a 40 ( valori espressi con un'unità di misura interna al simulatore ), con incrementi di una un'unità e 200 prove per valore, per un totale di 32000 simulazioni; per la relazione tra produzione e sensibilità questa è stata fatta crescere da 0 fino a 20 a intervalli di 0.5, ancora con un totale di 32000 simulazioni.

\myparagraph{Sensibilità a interazione}

\myparagraph{Raggio dell'interazione}


\section[APPROSSIMAZIONE LINEARE]{APPROSSIMAZIONE LINEARE}

\subsection{REGRESSIONE LINEARE A TRATTI}


\nocite{*}
\bibliographystyle{plain}
\bibliography{bibliography}

\end{document}


% descrizione dell'ottimizzazione
%Andrea Borghesi
%Università degli studi di Bologna

%capitolo dedicato alla descrizione (breve) dell'ottimizzatore

\documentclass[12pt,a4paper,openright,twoside]{report}
\usepackage[italian]{babel}
\usepackage{indentfirst}
\usepackage[utf8]{inputenc}
\usepackage[T1]{fontenc}
\usepackage{fancyhdr}
\usepackage{graphicx}
\usepackage{titlesec,blindtext, color}
\usepackage[font={small,it}]{caption}
\usepackage{subfig}
\usepackage{listings}
\usepackage{color}
\usepackage{url}
\usepackage{textcomp}
\usepackage{eurosym}

%impostazioni generali per visualizzare codice
\definecolor{dkgreen}{rgb}{0,0.6,0}
\definecolor{gray}{rgb}{0.5,0.5,0.5}
\definecolor{mauve}{rgb}{0.58,0,0.82}
 
\lstset{ %
  basicstyle=\footnotesize,           % the size of the fonts that are used for the code
  backgroundcolor=\color{white},      % choose the background color. You must add \usepackage{color}
  numbers=left,                   % where to put the line-numbers
  numberstyle=\tiny\color{gray},  % the style that is used for the line-numbers
  numbersep=5pt,  
  showspaces=false,               % show spaces adding particular underscores
  showstringspaces=false,         % underline spaces within strings
  showtabs=false,                 % show tabs within strings adding particular underscores
  rulecolor=\color{black}, 
  tabsize=2,                      % sets default tabsize to 2 spaces
  breaklines=true,                % sets automatic line breaking
  breakatwhitespace=false,        % sets if automatic breaks should only happen at whitespace
  title=\lstname,                   % show the filename of files included with \lstinputlisting;
  frame=single,                   % adds a frame around the code
                                  % also try caption instead of title
  keywordstyle=\color{blue},          % keyword style
  commentstyle=\color{dkgreen},       % comment style
  stringstyle=\color{mauve},         % string literal style
  escapeinside={\%*}{*)},            % if you want to add LaTeX within your code
  morekeywords={*,...},              % if you want to add more keywords to the set
  deletekeywords={...}              % if you want to delete keywords from the given language
}

%per avere un bordo intorno alle figure
\usepackage{float}
\floatstyle{boxed} 
\restylefloat{figure}

%per poter poi impedire che certe parole vadano a capo
\usepackage{hyphenat}
\usepackage{listings}

%ridefinisco font per fancyhdr, per ottenere un'intestazione pulita
\newcommand{\changefont}{ \fontsize{9}{11}\selectfont }
\fancyhf{}
\fancyhead[LE,RO]{\changefont \slshape \rightmark} 	%section
\fancyhead[RE,LO]{\changefont \slshape \leftmark}	%chapter
\fancyfoot[C]{\changefont \thepage}					%footer

%titolo capitolo con "numero | titolo"
\definecolor{gray75}{gray}{0.75}
\newcommand{\hsp}{\hspace{20pt}}
\titleformat{\chapter}[hang]{\Huge\bfseries}{\thechapter\hsp\textcolor{gray75}{|}\hsp}{0pt}{\Huge\bfseries}


\oddsidemargin=30pt \evensidemargin=20pt

%sillabazioni non eseguite correttamente
\hyphenation{sil-la-ba-zio-ne pa-ren-te-si si-mu-la-to-re ge-ne-ra-re pia-no}

%interlinea
\linespread{1.15}  
\pagestyle{fancy}

%cartelle contenenti le immagini
\graphicspath{{/media/sda4/tesi/immagini/grafici/}{/media/sda4/tesi/immagini/grafici/incCompare/}{/media/sda4/tesi/immagini/grafici/rawData/}{/media/sda4/tesi/immagini/grafici/regressionAnalysis/}{/media/sda4/tesi/immagini/schemi/}{/media/sda4/tesi/immagini/simulazione/}{/media/sda4/tesi/immagini/epolicy/}{/media/sda4/tesi/immagini/ottimizzazione/}}

%in modo che dopo il titolo di un paragrafo il testo vada a capo
\newcommand{\myparagraph}[1]{\paragraph{#1}\mbox{}\\}

\begin{document}
\chapter{\nohyphens{OTTIMIZZAZIONE}}
In aggiunta al modello di simulazione ad agenti, con la relativa prospettiva individuale, il progetto ePolicy ritiene necessario considerare anche una prospettiva globale (regionale nel caso della Regione), in grado di affrontare il problema della pianificazione e dalla valutazione degli impatti dei piani da un punto di vista più ampio, mantenendo al tempo stesso una stretta integrazione con il livello individuale.\\*
La pianificazione delle attività regionale può essere vista come un complesso problema di ottimizzazione combinatoria; i decisori politici devono prendere decisioni e soddisfare un insieme di vincoli, tentando al tempo stesso di realizzare un certo numero di obiettivi, come ad esempio ridurre gli effetti negativi e incrementare i positivi su ambiente, società ed economia. La fase di valutazione - valutare quali siano gli impatti delle politiche scelte sull'ambiente ed in misura minora in ambito economico e sociale - è ora in genere effettuata in sequenza dopo la creazione di un piano, con lo svantaggio che se questo contenesse impatti negativi sull'ambiente potrebbero venire applicate solo delle contromisure correttive; per evitare ciò, nell'approccio di ePolicy la valutazione e la pianificazione sono condotte allo stesso tempo.\\*\\*
Per la valutazione ambientale sono stati proposti diversi metodi: un modello probabilistico \cite{logicDSSstrategicAss}, un modello fuzzy (la logica fuzzy prevede che si possa attribuire a una proposizione un grado di verità compreso tra 0 e 1) \cite{fuzzyLogicstrategicAss} e un modello lineare a vincoli (\emph{Constraint Logic Programming}, programmazione logica a vincoli, chiamata in seguito anche CLP) \cite{GavanelliEtAl}. Il motivo per sperimentare diversi tipi di modello è che le matrici usate dagli esperti ambientali si prestano a differenti interpretazioni, quindi era importante capire quale fosse la migliore scelta possibile. Il modello CLP è risultato essere il più veloce - a livello computazionale - in quanto per la programmazione lineare esistono tecniche di risoluzione molto efficienti. In secondo luogo, questo modello può essere facilmente esteso aggiungendo nuovi vincoli, per risolvere nuovi tipi di problemi; ad esempio se le attività da pianificare fossero variabili decisionali (invece che valori fissi) potremmo svolgere la pianificazione contemporaneamente alla valutazione ambientale. Dal momento che questo era uno degli scopi del progetto si è scelto di utilizzare l'approccio CLP.\\*\\*
In questo capitolo introdurremo molto brevemente la programmazione logica a vincoli, con una breve panoramica e citando gli strumenti software di cui ci siamo serviti, passeremo poi a presentare il modello sviluppato che incorpora al suo interno le attività di pianificazione e valutazione e infine mostreremo i risultati ottenuti applicando il modello al caso di studio scelto, ovvero il piano energetico 2011-2013 per la regione Emilia-Romagna.

\section[CLP]{\nohyphens{PROGRAMMAZIONE LOGICA A VINCOLI}}
Come abbiamo già ampiamente spiegato in precedenza il problema della creazione di un piano regionale può essere considerato come un problema caratterizzato da un insieme di vincoli e una funzione obiettivo. Nell'ambito dell'Intelligenza Artificiale i problemi per cui è richiesto soddisfare un insieme di vincoli sono definiti come \emph{Problemi di Soddisfacimento di Vincoli} (in inglese Constraint Satisfaction Problem, da cui l'acronimo \emph{CSP}). Un CSP è definito da una terna $<X,D,C>$, dove $X$ è un insieme di variabili $X=\{X_1,X_2...X_n\}$, $D$ è un dominio discreto per ogni variabile $D=\{D_1,D_2...D_n\}$ e $C$ è l'insieme di vincoli - un vincolo è una relazione tra variabili che definisce un sottoinsieme del prodotto cartesiano dei domini $D_1 \times D_2 \times ... \times D_n$; con queste premesse un la soluzione di un problema di soddisfacimento di vincoli è data da un assegnamento di valori alle variabili consistente con i vincoli \cite{cspFoundations}. Analogamente, un problema di ottimizzazione con vincoli (Constraint Optimization Problem, \emph{COP}) è definito da $<X,D,C,f>$, cioè un CSP più una \emph{funzione obiettivo} $f(X_1,X_2...X_n)$, la cui soluzione è un assegnamento di valori alle variabili compatibile con i vincoli del problema che ottimizza la funzione obiettivo. Le metodologie risolutive impiegate per risolvere problemi a vincoli attingono in buona parte alle tecniche di Ricerca Operativa e Intelligenza Artificiale, in particolare noi considereremo la \emph{Programmazione Logica a Vincoli}.\\*\\*
La programmazione logica a vincoli \cite{clpSurvey} (in inglese \emph{Constraint Logic Programming} da cui CLP) è una classe di linguaggi di programmazione che estendono la classica Programmazione Logica - il paradigma di programmazione basato sulla logica del primo ordine implementato da linguaggi come ad esempio Prolog \cite{Colmerauer,Kowalski,clocksin2003programming}, sviluppato nei primi anni settanta e ampiamente diffuso ancora oggi. Alle variabili possono essere assegnati sia termini (i tipi di dato e le strutture riconosciute in Prolog) sia valori interpretati, appartenenti a una determinata \emph{classe}, un parametro caratteristico dello specifico linguaggio CLP; per esempio è possibile avere CLP(\emph{R}) \cite{clpR}, in grado di operare sui valori reali, oppure CLP(FD), in cui le variabili appartengono a domini finiti. All'interno di una classe sono definite le funzioni interpretate (che possono essere nei domini numerici i soliti operatori $+$, $-$, $\times$, etc.) e i predicati (ad esempio, $<$, $\neq$, $\geq$, etc.), che sono chiamati \emph{vincoli}. La semantica dichiarativa consente l'interpretazione intuitiva per i vincoli e i termini, relativamente al dominio considerato: ad esempio, $1.3+2<5$ è \emph{vero} in CLP(\emph{R}); ciò è un'estensione molto significativa rispetto alla programmazione logica standard in quanto i linguaggi logici operano in domini non interpretati (''Universo di Herbrand'') e quindi le relazioni tra variabili possono essere solamente verificate a posteriori e non trattate come vincoli veri propri. La semantica operativa somiglia a quella di Prolog per atomi costruiti sui predicati usuali - quelli definiti da un insieme di clausole - ma conserva quelli da interpretare, i vincoli, in una struttura dati speciale, chiamata \emph{constraint store}; essa è in seguito interpretato e modificato da un meccanismo esterno, il \emph{constraint solver}, il risolutore dei vincoli. Il risolutore è in grado di controllare se una combinazione di vincoli è soddisfacibile o meno, e può modificare lo store, sperabilmente per semplificarne lo stato. Generalmente il solver non effettua una propagazione \emph{completa}: se una valutazione ha come risultati \emph{falso}, allora sicuramente la soluzione è impossibile, anche se in certi casi può capitare che il risolutore non rilevi l'impossibilità di un problema anche se non esistono soluzioni.\\*\\*  
CLP(\emph{R}) è una classe di programmazione logica a vincoli in cui le variabili appartengo all'insieme dei numeri reali, i vincoli disponibili sono uguaglianze e disuguaglianze lineari e generalmente il risolutore è implementato tramite l'algoritmo del simplesso, molto veloce e completo per (dis)equazioni lineari (cioè è sempre in grado di restituire \emph{vero} o \emph{falso}). In alcuni sistemi, grazie alla disponibilità di risolutori efficienti per la programmazione lineare intera, certi vincoli non lineari sono accettati nel linguaggio, in particolare è possibile imporre che certe variabili assumano esclusivamente valori interi; in casi come questo la complessità del problema passa da P a NP-hard e il risolutore deve spesso ricorrere a tecniche di branch-and-bound. Ad ogni modo, l'utente può specificare anche una funzione obiettivo, un termine lineare che dovrebbe essere massimizzato o minimizzato garantendo al tempo stesso che tutti vincoli siano soddisfatti.\\*\\*


\section{STRUMENTI}
Al giorno d'oggi esistono numerose implementazioni di CLP(\emph{R}) \cite{inclpR} e diverse versioni di Prolog dispongono di una propria libreria per CLP(\emph{R}). Per il progetto ePolicy si è scelto di adottare il software open source  ECL$^i$PS$^e$ \cite{clpEclipse,fromLPtoCLPeclipse}.

\subsection{ECL$^i$PS$^e$}
ECL$^i$PS$^e$ è un sistema software per lo sviluppo di applicazioni di programmazione a vincoli e indicata per lo studio di aspetti relativi alla risoluzione di problemi combinatori, come appunto la programmazione vincolata, modellazione di problemi, programmazione matematica, tecniche di ricerca di soluzioni,... Al suo interno sono contenuti librerie per risolutori a vincoli, un linguaggio di alto livello (derivato da Prolog), interfacce per risolutori esterni e altre funzionalità. In Figura~\ref{eclipseUI} mostriamo come si presenta l'interfaccia utente di ECL$^i$PS$^e$.

\begin{figure}[h]
	\centering
	\includegraphics[scale=0.28]{eclipseUI}
	\caption{ECL$^i$PS$^e$, Interfaccia Utente}
	\label{eclipseUI}
\end{figure}

Tra le diverse librerie disponibili, ne esiste una denominata \emph{Eplex} \cite{eplex} che interfaccia ECL$^i$PS$^e$ a un risolutore lineare intero, il quale può essere sia uno strumento commerciale, come CPLEX o Xpress-MP, che open source. A default Eplex nasconde molti dei dettagli del risolutore, ma nondimeno, quando richiesto, l'utente può regolare diversi parametri per migliorare le prestazioni ed esaminare lo stato interno del solver. Nell'ambito del progetto ePolicy ci siamo serviti di questa libreria, insieme alle altre funzionalità offerte da ECL$^i$PS$^e$, per modellare e risolvere i problemi relativi alla pianificazione energetica regionale.
\\*\\*
Illustreremo ora due esempi di modellazione di problemi a vincoli sfruttando il linguaggio ECL$^i$PS$^e$ (nel primo caso considerando domini finiti e nel secondo valori reali, avvalendoci anche della libreria Eplex), anche per mostrare come possono essere strutturati i problemi di programmazione logica a vincoli; in questa trattazione supporremo noti i concetti elementari della programmazione logica (procedimenti risolutivi, definizioni di un termine, etc.), la cui discussione esula da questo lavoro.

\myparagraph{ESEMPIO CLP(FD)}
Il cosiddetto \emph{Send More Money} puzzle è un esempio classico di programmazione a vincoli; le variabili $[S,E,N,D,M,O,R,Y]$ rappresentano cifre da 0 a 9 e lo scopo è assegnare alle variabili valori diversi in modo che l'operazione aritmetica di Figura ~\ref{SendMoreMoney} risulti corretta - inoltre i numeri devono essere ben formati, da cui $S>0$ e $M>0$. 

\begin{figure}[h]
	\centering
	\includegraphics[width=0.215\textwidth]{sendMoreMoney}
	\caption{Send More Money puzzle}
	\label{SendMoreMoney}
\end{figure}

Con la programmazione convenzionale si avrebbe necessità di esprimere una strategia di ricerca in modo esplicito (senza contare possibili ottimizzazioni come cicli innestati), mentre con linguaggi logici come Prolog verrebbe sfruttata la ricerca fornita dal risolutore interno (il motore inferenziale), con il vantaggio di una programmazione estremamente facilitata ma col rischio di un'efficienza non elevata - a meno di programmi ottimizzati, i quali richiederebbero comunque maggiori tempo e abilità. 

Questo è in effetti il campo di applicazione ideale della programmazione logica a vincoli, in particolare nell'ambito  dei domini finiti (CLP(FD)): le variabili possono assumere valori appartenenti ad un insieme finito di numeri interi, i vincoli sono facilmente esprimibili formalmente e occorre effettuare una certa quantità di ricerca nello spazio delle soluzioni. In questo problema sarebbe naturale usare le variabili del programma per rappresentare le diverse cifre e la soluzione finale dovrà essere un assegnamento di un valore unico per ogni variabile. 

Risolvere questo problema con Prolog comporta l'utilizzo della strategia di ricerca chiamata \emph{Generate and Test}, che prevede che prima la generazione di una soluzione e poi la verifica della consistenza dei vincoli e, nel caso che questa dia esito negativo, l'assegnamento di nuovi valori alle variabili seguita da nuova verifica e così via. In questo modo l'esplorazione dello spazio delle soluzioni è chiaramente inefficiente - per esempio la possibile implementazione in Prolog mostrata qui sotto, per quanto suscettibile a miglioramenti, deve gestire $\frac{10!}{2}$ possibili assegnamenti di valori alle variabili. 

\lstset{language=Prolog}
\begin{lstlisting}
% Send More Money puzzle in Prolog
smm :-
        X = [S,E,N,D,M,O,R,Y],           % variabili
        Digits = [0,1,2,3,4,5,6,7,8,9],	 % domini
        
        % predicato che assegna una soluzione
        assign_digits(X, Digits),
       	
       	%  verifica dei vincoli vincoli
        M > 0, 
        S > 0,
                  1000*S + 100*E + 10*N + D +
                  1000*M + 100*O + 10*R + E =:=
        10000*M + 1000*O + 100*N + 10*E + Y,
        write(X).

select(X, [X|R], R).
select(X, [Y|Xs], [Y|Ys]):- select(X, Xs, Ys).

assign_digits([], _List).
assign_digits([D|Ds], List):-
        select(D, List, NewList),
        assign_digits(Ds, NewList).
\end{lstlisting}

L'implementazione realizzata con ECL$^i$PS$^e$ presenta i vantaggi di semplificare ulteriormente la modellazione del problema e di appoggiarsi all'efficiente risolutore interno per l'esplorazione dello spazio delle soluzioni, in modo particolare il fatto che ogni volta che una variabile viene istanziata i vincoli vengono propagati per eliminare a priori strade inconsistenti, riducendo gli spazi delle soluzioni e prevenendo fallimenti sicuri. 

\begin{lstlisting}
% Send More Money puzzle in ECLiPSe
smm :-
     X = [S,E,N,D,M,O,R,Y],		% variabili
     X :: [0 .. 9],				% domini finiti
     
     % vincoli
     M #> 0,
     S #> 0,
               1000*S + 100*E + 10*N + D +
               1000*M + 100*O + 10*R + E #=
     10000*M + 1000*O + 100*N + 10*E + Y,
     alldistinct(X),
     
     % ricerca della soluzione
     labeling(X),
     write(X).

\end{lstlisting}

\myparagraph{ESEMPIO CLP(R) - EPLEX}

Presentiamo ora un esempio di un problema (Fig.~\ref{clpR_example}) che rientra nell'ambito dei CLP(R) e che fa uso della libreria Eplex, tratto dal manuale di ECL$^i$PS$^e$ \cite{eclipseTut}. Ci sono tre impianti, o fabbriche, (1-3) in grado di produrre un certo prodotto con capacità diverse e i cui prodotti devono essere trasportati a quattro clienti (A-D) con quantità richieste diverse; anche il costo unitario di trasporto ai clienti è variabile. L'obiettivo del problema è minimizzare i costi di trasporto soddisfacendo le esigenze dei clienti. 

\begin{figure}[h]
	\centering
	\includegraphics[scale=0.7]{clpR_example}
	\caption{Esempio di un problema CLP(R). Fonte \cite{eclipseTut}}
	\label{clpR_example}
\end{figure}

Per formulare il problema definiamo la quantità di prodotto trasportata dall'impianto $N$ al cliente $p$ come variabile $N_p$ - ad esempio $A_1$ rappresenta il costo di trasporto dalla fabbrica $A$ al cliente $1$. I vincoli da considerare sono di due tipi(sempre facendo riferimento alla Figura~\ref{clpR_example}):
\begin{itemize}
\item La quantità di prodotto consegnata da tutti gli impianti a un cliente deve deve essere uguale alla domanda del cliente, ad esempio per il cliente $A$ che può essere rifornito dagli impianti 1-3, abbiamo che $A_1+A_2+A_3=21$
\item La quantità di prodotto in uscita da una fabbrica non può essere superiore alla sua capacità produttiva, ad esempio per l'impianto $1$ che invia prodotti ai clienti A-D si ha che $A_1+B_1+C_1+D_1 \leq 50$
\end{itemize}   
Poiché lo scopo è minimizzare i costi di trasporto, la funzione obiettivo è di minimizzare i costi combinati del trasporto dei prodotti dai tre impianti a tutti e quattro i clienti.\\*
La formulazione del problema è quindi la seguente.\\*
Funzione obiettivo: 
\begin{equation}  \label{exampleClpRObiett}
	\min(10A_1+7A_2+200A_3+8B_1+5B_2+10B_3+5C_2+5C_2+8C_3+9D_1+3D_2+7D_3)
\end{equation}
Vincoli:
\begin{equation}  \label{exampleCons1}
	A_1+A_2+A_3=21
\end{equation}
\begin{equation}  \label{exampleCons2}
	B_1+B_2+B_3=40
\end{equation}
\begin{equation}  \label{exampleCons3}
	C_1+C_2+C_3=34
\end{equation}
\begin{equation}  \label{exampleCons4}
	D_1+D_2+D_3=10
\end{equation}
\begin{equation}  \label{exampleCons5}
	A_1+B_1+C_1+D_1 \leq 50
\end{equation}
\begin{equation}  \label{exampleCons6}
	A_2+B_2+C_2+D_2 \leq 30
\end{equation}
\begin{equation}  \label{exampleCons7}
	A_3+B_3+C_3+D_3 \leq 40
\end{equation}

Mostriamo ora come questo problema venga modellato sfruttando la libreria Eplex. In primo luogo occorre caricare la libreria Eplex di cui si dispone (in questo caso abbiamo sfruttato un risolutore esterno open source) e ottenerne un'\emph{istanza}, la quale rappresenta un singolo problema sotto forma di modulo, a cui possono essere riferiti vincoli e funzione obiettivo consentendo quindi al solver esterno di risolvere il problema. Il codice che segue mostra come il problema di Figura ~\ref{clpR_example} sia stato trasposto all'interno di ECL$^i$PS$^e$.

\begin{lstlisting}
:- lib(eplex).		% caricamento della libreria Eplex
:- eplex_instance(prob).		% definizione dell'istanza - chiamata 'prob'

main(Cost, Vars) :-
		% dichiarazione delle variabili e definizione del loro dominio		
		Vars = [A1,A2,A3,B1,B2,B3,C1,C2,C3,D1,D2,D3],
		prob: (Vars $:: 0.0..1.0Inf),  % valori maggiori o uguali a 0
		
		% definizione dei vincoli applicati all'istanza eplex
		prob: (A1 + A2 + A3 $= 21), 
		prob: (B1 + B2 + B3 $= 40),
		prob: (C1 + C2 + C3 $= 34),
		prob: (D1 + D2 + D3 $= 10),

		prob: (A1 + B1 + C1 + D1 $=< 50),
		prob: (A2 + B2 + C2 + D2 $=< 30),
		prob: (A3 + B3 + C3 + D3 $=< 40),

		% inizializza il solver esterno con la funzione obiettivo
		prob: eplex_solver_setup(min(10*A1 + 7*A2 + 200*A3 + 
			8*B1 + 5*B2 + 10*B3 +
		 	5*C1 + 5*C2 + 8*C3 +
		 	9*D1 + 3*D2 + 7*D3)),

		% ---------- Fine Modellazione ----------

		% risoluzione del problema
		prob: eplex_solve(Cost).
\end{lstlisting}

Per usare un'istanza Eplex occorre prima dichiararla con \emph{eplex\_instance/1}; una volta dichiarata, l'istanza viene riferita tramite il nome specificato. 

Come primo passo creiamo le variabili del problema e imponiamo che possano assumere solamente valori non negativi e rendiamo noti all'istanza il loro dominio (\emph{\$::/2}). Successivamente imponiamo i vincoli che modellano il problema sotto forma di uguaglianze e disuguaglianze aritmetiche; per via del solver esterno scelto, gli unici tipi di vincoli accettati sono quelli lineari - che ovviamente consentono una maggiore efficienza nella risoluzione.

Occorre poi inizializzare il risolutore esterno con l'istanza eplex creata, in modo che questa possa essere risolta. Questo è fatto dal \emph{eplex\_solver\_setup/1}, che prende come argomento la funzione obiettivo, la quale può essere di minimizzazione o massimizzazione. Infine è possibile risolvere il problema modellato attraverso \emph{eplex\_solve/1}.

Quando un'istanza viene risolta, il solver prende in considerazione tutti i vincoli ad essa relativi, i valori che le variabili del problema possono assumere e la funzione obiettivo specificata. In questo caso è possibile ottenere una soluzione ottimale pari a 710.0: 
\begin{lstlisting}
?-	main(Cost, Vars).

Cost = 710.0
Vars = [A1{0.0 .. 1e+20 @ 0.0}, A2{0.0 .. 1e+20 @ 21.0}, ....]
\end{lstlisting}

\section{MODELLAZIONE PROBLEMA}

\subsection[APPROCCIO A VINCOLI]{\nohyphens{PERCHÉ UN APPROCCIO BASATO SUI VINCOLI}}
L'attività di pianificazione regionale è al momento svolta da esperti umani che costruiscono un singolo piano, considerando gli obbiettivi strategici regionali che seguono le direttive nazionali ed europee. Dopo che il piano è stato ideato l'ente per la protezione ambientale è chiamata a valutarne la sostenibilità dal punto di vista ambientale. In genere non c'è nessuna retroazione, la valutazione può solamente stabilire se il piano sia ecocompatibile o meno ma senza poterlo per modificare; in rari casi può proporre alcune misure correttive, le quali possono però solamente mitigare gli effetti negativi di decisioni di pianificazione sbagliate.\\* 
Oltre a ciò, sebbene le normative prevedano che una valutazione ambientale significativa debba confrontare due o più opzioni (piani differenti), questo è fatto raramente in Europa poiché la valutazione è tipicamente fatta a mano e richiede un lungo lavoro; anche nei pochi casi in cui due opzioni vengano considerate, solitamente una è il piano e l'altra è l'assenza di pianificazione.\\*\\*
La modellazione a vincoli supera le limitazioni dei processi manuali per diversi motivi. In primo luogo, essa fornisce uno strumento che automaticamente prende decisioni di pianificazione, tenendo in considerazione il budget allocato sulla base sia del piano operativo regionale che delle linee guida nazionali/europee.\\*
Secondo, gli aspetti ambientali sono considerati durante la costruzione del piano, evitando di procedere per tentativi ed errori.\\*
Come terza ragione, il ragionamento con i vincoli è uno strumento potente nelle mani di un decisore politico in quanto la generazione di scenari alternativi è estremamente semplificata ed il confronto e valutazione seguono naturalmente. Nel caso in cui i risultati non soddisfino coloro che stabiliscono le politiche o gli esperti ambientali gli aggiustamenti possono essere introdotti molto facilmente all'interno del modello; ad esempio, nel settore della pianificazione energetica regionale, cambiando i limiti della quantità di energia che ogni fonte può fornire, possiamo correggere il piano considerando l'andamento del mercato e anche la potenziale ricettività della regione.

\subsection{MODELLO CLP}
%D.3 2

\myparagraph{VALUTAZIONE IMPATTI}
%D.3 6.1 6.2 6.3

\myparagraph{PIANIFICAZIONE}
%gavanelliEtAl 1.3 

\subsection{IL PIANO REGIONALE 2011-2013}
%gavanelliEtAl 1.4

\subsection{VALORE AGGIUNTO DEL CLP}
%gavanelliEtAl 1.5


%  ?gavanelliEtAl 1.6  D.3 10

\nocite{*}
\bibliographystyle{plain}
\bibliography{bibliography}

\end{document}


% interazione tra ottimizzatore e simulatore
%Andrea Borghesi
%Università degli studi di Bologna

%capitolo dedicato all'interazione tra ottimizzatore e simulatore

\documentclass[12pt,a4paper,openright,twoside]{report}
\usepackage[italian]{babel}
\usepackage{indentfirst}
\usepackage[utf8]{inputenc}
\usepackage[T1]{fontenc}
\usepackage{fancyhdr}
\usepackage{graphicx}
\usepackage{titlesec,blindtext, color}
\usepackage[font={small,it}]{caption}
\usepackage{subfig}
\usepackage{listings}
\usepackage{color}
\usepackage{url}
\usepackage{textcomp}
\usepackage{eurosym}
\usepackage{amsmath}
\usepackage{url}

%impostazioni generali per visualizzare codice
\definecolor{dkgreen}{rgb}{0,0.6,0}
\definecolor{gray}{rgb}{0.5,0.5,0.5}
\definecolor{mauve}{rgb}{0.58,0,0.82}
 
\lstset{ %
  basicstyle=\footnotesize,           % the size of the fonts that are used for the code
  backgroundcolor=\color{white},      % choose the background color. You must add \usepackage{color}
  numbers=left,                   % where to put the line-numbers
  numberstyle=\tiny\color{gray},  % the style that is used for the line-numbers
  numbersep=5pt,  
  showspaces=false,               % show spaces adding particular underscores
  showstringspaces=false,         % underline spaces within strings
  showtabs=false,                 % show tabs within strings adding particular underscores
  rulecolor=\color{black}, 
  tabsize=2,                      % sets default tabsize to 2 spaces
  breaklines=true,                % sets automatic line breaking
  breakatwhitespace=false,        % sets if automatic breaks should only happen at whitespace
  title=\lstname,                   % show the filename of files included with \lstinputlisting;
  frame=single,                   % adds a frame around the code
                                  % also try caption instead of title
  keywordstyle=\color{blue},          % keyword style
  commentstyle=\color{dkgreen},       % comment style
  stringstyle=\color{mauve},         % string literal style
  escapeinside={\%*}{*)},            % if you want to add LaTeX within your code
  morekeywords={*,...},              % if you want to add more keywords to the set
  deletekeywords={...}              % if you want to delete keywords from the given language
}

%per avere un bordo intorno alle figure
\usepackage{float}
\floatstyle{boxed} 
\restylefloat{figure}

%per poter poi impedire che certe parole vadano a capo
\usepackage{hyphenat}

%ridefinisco font per fancyhdr, per ottenere un'intestazione pulita
\newcommand{\changefont}{ \fontsize{9}{11}\selectfont }
\fancyhf{}
\fancyhead[LE,RO]{\changefont \slshape \rightmark} 	%section
\fancyhead[RE,LO]{\changefont \slshape \leftmark}	%chapter
\fancyfoot[C]{\changefont \thepage}					%footer

%titolo capitolo con "numero | titolo"
\definecolor{gray75}{gray}{0.75}
\newcommand{\hsp}{\hspace{20pt}}
\titleformat{\chapter}[hang]{\Huge\bfseries}{\thechapter\hsp\textcolor{gray75}{|}\hsp}{0pt}{\Huge\bfseries}


\oddsidemargin=30pt \evensidemargin=20pt

%sillabazioni non eseguite correttamente
\hyphenation{sil-la-ba-zio-ne pa-ren-te-si si-mu-la-to-re ge-ne-ra-re pia-no}

%interlinea
\linespread{1.15}  
\pagestyle{fancy}

%cartelle contenenti le immagini
\graphicspath{{/media/sda4/tesi/immagini/grafici/}{/media/sda4/tesi/immagini/grafici/incCompare/}{/media/sda4/tesi/immagini/grafici/rawData/}{/media/sda4/tesi/immagini/grafici/regressionAnalysis/}{/media/sda4/tesi/immagini/schemi/}{/media/sda4/tesi/immagini/simulazione/}{/media/sda4/tesi/immagini/epolicy/}{/media/sda4/tesi/immagini/ottimizzazione/}
{/media/sda4/tesi/immagini/interazione/}}

%in modo che dopo il titolo di un paragrafo il testo vada a capo
\newcommand{\myparagraph}[1]{\paragraph{#1}\mbox{}\\}

%per scrivere bene CLP(R) e CLP(FD)
\newcommand{\clpr}{CLP({\ensuremath{\cal R}})}
\newcommand{\clpfd}{CLP({\ensuremath{\cal FD}})}

\begin{document}

\chapter{\nohyphens{INTERAZIONE COMPONENTI}}
Nei capitoli precedenti abbiamo descritto gli elementi principali che concorrono a definire l'architettura del sistema ePolicy (certamente all'interno del progetto sono presente ulteriori componenti, come quello dedicato all'opinion mining, ma in questo caso ci riferiamo a quelli considerati in questo lavoro). Da un punto di vista generale essi sono il modello a vincoli che garantisce una ottimizzazione a livello globale e il simulatore che studia il comportamento delle modalità di incentivazione a livello locale/regionale. Un aspetto molto importante è quindi capire come gestire l'interazione tra ottimizzatore e simulatore in modo ottimale, in modo da integrare le prospettive globale e locale.

Possiamo illustrare la necessità di comprendere a fondo questa interazione con un esempio. Supponiamo che la fase di ottimizzazione abbia prodotto due scenari alternativi, il primo concentrandosi sulla creazione di impianti a biomassa e il secondo sostenendo la costruzione di centrali idroelettriche; entrambi i piani avrebbero un impatto non indifferente sui cittadini. La produzione energetica con la biomassa comporta un impatto sostanziale sulle aree boschive, potenziale inquinamento del suolo e delle coltivazioni, inquinamento dell'aria nelle aree urbane vicine alla centrale; d'altra parte, le centrali idroelettriche prevedono l'allagamento vaste porzioni di territorio. In ogni caso le strategie per implementare il piano, studiate tramite il simulatore, dovrebbero tenere in considerazione questi effetti sugli individui; le attività di implementazione implicherebbero quindi costi aggiuntivi che dovrebbero essere inseriti come nuovi vincoli all'interno dell'ottimizzatore, il quale potrebbe poi effettuare nuovamente la fase di pianificazione, con la possibilità di ottenere risultati diversi. 

Un approccio molto basilare sarebbe il semplice scambio di risultati tra i due livelli di pianificazione delle politiche, svolgendo anche diverse iterazioni, ma questo metodo rischierebbe di non garantire la convergenza. A un certo punto le iterazioni possono essere fermate quando un equilibrio è stato raggiunto o quando il decisore politico valuta che ulteriori aggiustamenti non siano più necessari o richiesti. Citando Clement Attlee\footnote{Fonte: A. Sampson, \emph{Anatomy of Britain}, Hodder \& Stoughton, 1962} \emph{``Democracy means government by discussion but it is only effective if you can stop people talking.''} - democrazia significa governo fondato sulla discussione, ma funziona solamente se si riesce a far smettere la gente di discutere.
\\*

Il tema dell'integrazione efficacie tra pianificazione regionale e simulazione è oggetto di intensa ricerca per ottenere una soluzione ottimale all'interno del progetto ePolicy (in modo particolare servendosi di metodologie appartenenti alla \emph{teoria dei giochi}); nel resto del capitolo verrà mostrato un possibile approccio, da noi implementato e messo alla prova con il problema dell'assegnazione dei fondi regionali per l'incentivazione della tecnologia fotovoltaica in Emilia-Romagna.

\section{INTEGRARE DSS E SIMULAZIONI}
Un primo approccio a cui è possibile pensare, consiste nello sfruttare i metodi e le tecniche dell'\emph{Apprendimento Automatico} (noto in letteratura anche come \emph{Machine Learning}), il quale rappresenta un'area dell'Intelligenza Artificiale che si occupa della realizzazione di sistemi e algoritmi che si basano su serie di osservazioni e dati per la sintesi di nuova conoscenza. Senza voler entrare nel dettaglio, possiamo comunque citare una definizione comunemente accettata di Apprendimento Automatico: \emph{``Un programma apprende da una certa esperienza E se, con rispetto a una classe di compiti T e una misura delle prestazioni P, le prestazioni P nello svolgere un compito dell'insieme T sono migliorate dall'esperienza E''} \cite{Mitchell}.
 
Nel nostro caso abbiamo sfruttato le tecniche di regressione viste nei capitoli precedenti per ricavare le relazioni che legano i fondi destinati agli incentivi regionali con la produzione elettrica di energia elettrica fotovoltaica, partendo dalle serie di dati forniti dal simulatore, affinché fosse poi possibile inserirle all'interno dell'ottimizzatore sotto forma di ulteriori vincoli, da tenere in considerazione per la fase di pianificazione. Il nostro fine è stato quello di estrarre dalla grande quantità di dati generata dal simulatore delle informazioni utili per migliorare il modello del problema da ottimizzare.

La Figura~\ref{schemaIterLearn} mostra lo schema dell'interazione tra il livello globale e il livello locale realizzata tramite l'approccio dell'Apprendimento Automatico. Nella parte alta osserviamo il sistema di supporto alle decisioni, l'ottimizzatore, che, a fronte delle possibili decisioni (l'allocazione di risorse per lo svolgimento di attività per la produzione di energia energetica nel rispetto dei diversi vincoli), produce un piano (oppure un insieme di piani o scenari). Il modello del DSS è arricchito con i vincoli che vengono appresi nella fase di \emph{Learning} a partire dai risultati prodotti dal simulatore; in ingresso al simulatore troviamo un insieme di piani interessanti per la relazione che stiamo tentando di apprendere - ad esempio per studiare il rapporto tra i fondi investiti nel metodo di incentivazione Conto Interessi è stato necessario effettuare simulazioni per un ampio numero di valori di budget. 
\begin{figure}[htb]
	\begin{center}
	\includegraphics[scale=0.65]{schemaIterLearn}
	\end{center}
	\caption{Modello di interazione basato su Apprendimento Automatico}
  	\label{schemaIterLearn}
\end{figure}

La fase di apprendimento, e quindi le simulazioni, devono essere effettuate prima della fase di pianificazione (\emph{offline}), in quanto occorre inserire all'interno del modello i nuovi vincoli appresi, i quali non saranno più modificati (se non nel caso in cui vengano sostituiti da altri ricavati da un nuovo processo di apprendimento). \`E necessario effettuare un grande numero di simulazioni per garantire un valore statistico alle relazioni apprese e fornire un buon insieme di dati tramite cui effettuare l'apprendimento; questo rappresenta sicuramente il maggiore limite di questo tipo di approccio, in quanto comporta che i vincoli appresi non possano essere modificati facilmente - effettuare un gran numero di simulazioni richiede molto tempo - e l'interazione avvenga sostanzialmente in una sola direzione, dall'simulatore verso il DSS. 
\\*\\*
Una seconda metodologia per integrare ottimizzazione e simulazione da noi considerata, nonostante non sia stata concretamente implementata a differenza della prima, è una tecnica classica di decomposizione dei problemi presa in prestito dall'ambito della Ricerca Operativa, la cosiddetta \emph{decomposizione di Benders} \cite{bendersDec}. Essa consiste in un metodo per risolvere problemi di ottimizzazione combinatoria che possono essere scomposti in due componenti, un problema master e un sotto-problema. Originariamente era stata concepita per il campo della Programmazione Lineare Intera ma è stata in seguito estesa per trattare risolutori più generali, \emph{Logic-Based Benders Decomposition} (decomposizione di Benders basata sulla Logica) \cite{Hooker95logic-basedbenders}. 

Nel caso da noi preso in esame il problema master è la definizione del piano energetico regionale che partizioni l'energia necessaria tra le diverse fonti energetiche rinnovabili e viene risolto tramite il modello a vincoli descritto nel capitolo precedente. Il sotto-problema consiste nella definizione della strategia di incentivazione per raggiungere la produzione energetica desiderata, in modo consistente con i vincoli regionali sul budget. Partendo dalle soluzioni ottenute con l'ottimizzazione, ovvero la produzione energetica attesa, per comprendere quale sia il budget corretto da allocare per gli incentivi vengono portate a termine diverse simulazioni - in numero comunque molto inferiore rispetto all'interazione basata sull'Apprendimento Automatico. Nel caso in cui gli incentivi non siano compatibili con il budget regionale viene generato un cosiddetto \emph{taglio di Benders} (chiamati anche \emph{no-good}), cioè un vincolo che va ad aggiungersi al modello del problema master, e successivamente una nuova soluzione viene generata dal DSS. 

A differenza del primo metodo, con questo approccio la comunicazione tra i due componenti qui considerati viene estesa ad un ciclo, come si può facilmente notare in Figura~\ref{schemaIterBend}.

\begin{figure}[htb]
	\begin{center}
	\includegraphics[scale=0.65]{schemaIterBend}
	\end{center}
	\caption{Modello di interazione basato su Decomposizione di Benders}
  	\label{schemaIterBend}
\end{figure}

L'interazione inizia dall'ottimizzatore che fornisce una soluzione per il problema master, soluzione che contiene la produzione energetica attesa da fotovoltaico e dei valori ipotetici della dimensione dei fondi da destinare agli incentivi regionali. Questi valori ipotetici sono passati al simulatore, il quale esegue delle simulazioni esclusivamente con tali parametri forniti dal DSS e produce le corrispondenti statistiche (il tempo di calcolo è di qualche ordine di grandezza minore rispetto all'approccio basato sull'apprendimento); queste ultime possono confermare o meno i valori ipotizzati in fase di ottimizzazione: se il valore (medio) di produzione energetica ottenuto dalle simulazioni è maggiore o uguale di quello atteso, l'iterazione può concludersi è il risultato è probabilmente ottimale \cite{bendersDec}. Viceversa se invece il valore atteso è maggiore di quello simulato un'altra iterazione è necessaria, quindi  all'ottimizzatore è comunicato un nuovo vincolo, il quale può essere visto come spiegazione del fatto che non è possibile ottenere la produzione energetica richiesta con i fondi agli incentivi ipotizzati. A questo punto il DSS inserisce il vincolo all'interno del modello del problema, ricerca nuovamente una soluzione ottimale e ipotizza nuovi valori da fornire in input al simulatore.

La sfida principale consiste nel determinare l'insieme dei vincoli che vengono trasferiti tra le due componenti: se venissero esclusi dall'insieme dei valori ammissibili solamente quelli ipotizzati - e trovati non adatti grazie alle simulazioni - si correrebbe il rischio di effettuare troppe iterazioni, arrivando al caso limite di effettuare una simulazione esaustiva per tutti i parametri (in pratica verrebbero nuovamente fatte delle simulazioni per ogni valore del budget per gli incentivi regionali); se invece dall'insieme dei valori  venissero esclusi (troppi) valori ulteriori il pericolo sarebbe quello di scartare delle soluzioni promettenti. Questo tema e l'implementazione effettiva di questo secondo approccio sono attualmente oggetto di ricerca.

\section{REGRESSIONE LINEARE A TRATTI}
Passiamo ora a discutere del modo in cui l'approccio basato sull'Apprendimento Automatico sia stato implementato nel nostro modello a vincoli.
\\*\\*
Come è stato descritto nel terzo capitolo, tramite un grande numero di simulazioni è stato possibile ottenere una grande quantità di dati dalla quale abbiamo successivamente ricavato le relazioni che legano i fondi per gli incentivi regionale alla produzione energetica da fotovoltaico e quest'ultima alla forza dell'interazione sociale tra gli agenti. Tali relazioni sono state espresse sotto forma di funzioni e corrispondenti curve, ottenute attraverso l'applicazione di tecniche di regressione. A questo punto la nostra intenzione è stata quella di integrare queste funzioni all'interno modello a vincoli del problema di ottimizzazione, aggiungendo cioè i nuovi vincoli appresi grazie alle simulazioni svolte; è sorto quindi un problema, poiché, come descritto nel capitolo precedente, il risolutore dai noi utilizzato l'ottimizzazione gestisce esclusivamente equazioni lineari - per motivi di efficienza. Dal momento che modificare questa caratteristica, ovvero impiegare un risolutore in grado di trattare le funzioni quadratiche e di grado anche superiore ricavate dalla regressione, avrebbe richiesto cambiamenti radicali nella struttura generale e nel codice dell'ottimizzatore, abbiamo ritenuto che fosse meglio procedere in un altro modo, che ci consentisse di preservare la linearità del modello a vincoli sviluppato. Per questo motivo abbiamo deciso di tentare di rendere lineari le relazioni  ottenute con la regressione sfruttando una tecnica matematica definita \emph{approssimazione lineare a tratti} \cite{piecewiseApprox,cattafi} (dall'inglese, \emph{piece-wise linear approximation}), che consiste nell'approssimare un'arbitraria funzione con un insieme di equazioni lineari con la massima accuratezza possibile. Possiamo ad esempio osservare in Figura~\ref{piecewiseApprox_example} l'approssimazione di una semplice funzione quadratica (in blu) attraverso cinque funzioni lineari (in rosso). 

\begin{figure}[htb]
	\begin{center}
	\includegraphics[scale=0.8]{piecewiseApprox_example}
	\end{center}
	\caption{Una funzione (in blu) e la sua approssimazione lineare a tratti (in rosso). Fonte {\tt http://commons.wikimedia.org/wiki/File:Finite\_element\_method\_1D\_illustration1.svg}}
  	\label{piecewiseApprox_example}
\end{figure}

Illustreremo adesso il funzionamento di questo metodo. Data una funzione (anche non lineare) $y=f(x)$, campioniamo la curva $g$ in $k$ punti $x_1,...,x_k$ e l'approssimazione lineare a tratti $y'=g'(x)\simeq g(x)$ è definita come
\begin{equation} 
\label{eq:aprroxEqX}
	x = \sum_{i=1}^k \lambda_i \cdot x_i,
\end{equation}
\begin{equation} 
\label{eq:aprroxEqY}
	y = \sum_{i=1}^k \lambda_i \cdot y_i,
\end{equation}
dove $\lambda_i \in [0..1]$ sono variabili continue soggette ai vincoli:
\begin{equation} 
\label{eq:aprroxEqLambda}
		\sum_{i=1}^k \lambda_i = 1
\end{equation}
Al massimo due $\lambda_i$ possono essere diverse da zero e in tal caso queste devono essere adiacenti.

Chiaramente queste ultime due condizioni non sono lineari, ma potrebbero essere modellate in un problema di Programmazione Logica Intera introducendo nuove 0-1 variabili intere, ma esiste una opzione più efficiente. In molti risolutori - compreso quello da noi impiegato - è possibile dichiarare $(\lambda_1,...,\lambda_k)$ come Special Order Set del secondo tipo (SOS2) \cite{bealeTomlin}, cioè un insieme ordinato di variabili utilizzato per specificare  determinate condizioni in problemi di ottimizzazione, e il risolutore sfrutterà questa informazione per ricercare una soluzione ottimale in modo più efficiente (in pratica, sapere che una variabile appartiene ad un certo insieme ordinato consente di usare in modo più intelligente gli algoritmi di branch-and-bound del solver).

\subsection{IMPLEMENTAZIONE IN R}

Le informazioni necessarie per poter inserire le equazioni (\ref{eq:aprroxEqX}), (\ref{eq:aprroxEqY}) e (\ref{eq:aprroxEqLambda}) all'interno del modello a vincoli sono le coordinate dei punti di campionamento. Per trovarle ci siamo serviti del precedentemente introdotto R e in particolare del pacchetto software \emph{Segmented} \cite{segmentedPackage}. Grazie ad esso è stato molto semplice trovare un'ottima approssimazione lineare per le funzioni che legavano il budget alla produzione energetica (una per ogni tipo di incentivo), come si può facilmente osservare nel codice qui presentato.

\lstset{language=R}
\begin{lstlisting}
> library(segmented)
> #  ...
> # inserisci i dati delle simulazioni in apposite strutture 
> #  ...
> # operazioni varie (ordina dati, etc.)
> #  ...
> # estrai un modello lineare a tratti per l'incentivo Conto Interessi
> modelloLineareATratti_CI <- segmented(modelloGrezzo_CI,seg.Z=~ Budget,psi=c(3))
> # estrai un modello lineare a tratti per l'incentivo Fondo Garanzia
> modelloLineareATratti_FG <- segmented(modelloGrezzo_FG,seg.Z=~ Budget,psi=c(12,30))
> #  ...
> # incentivi restanti
> #  ...
\end{lstlisting}

Una volta ricavate le approssimazioni lineari a tratti delle funzioni, è possibile visualizzare il risultato ottenuto, come riportato in Figura~\ref{incentCompare_piecewise}.

\begin{figure}[hbt]
	\centering
	\includegraphics[scale=0.6]{incentComparePiecewise}
	\caption{Confronto tra i diversi incentivi - Approssimazione lineare a tratti}
	\label{incentCompare_piecewise}
\end{figure}

\section{INTEGRAZIONE MODELLO}

Possiamo ora descrivere in che modo abbiamo inserito le relazioni approssimate all'interno del modello a vincoli dell'ottimizzatore. In particolare mostreremo quali estensioni siano state aggiunte al modello per permettere al risolutore di calcolare l'assegnazione ottima dei fondi disponibili ai vari tipi di incentivo, con lo scopo di massimizzare la produzione di energia.

\subsection{VARIABILI}

Innanzitutto per ogni tipologia di incentivo regionale sono state introdotte due variabili che rappresentano il costo associato e la relativa produzione energetica. Per i costi le variabili sono chiamate $costo_A$, $costo_{CI}$, $costo_R$, $costo_G$, con $costo_j \in[0..50]$ (dominio espresso in milioni di euro) - rispettivamente incentivo in conto capitale (denominato in precedenza anche Fondo Asta), Conto Interessi, Fondo Rotazione e Fondo Garanzia. In modo simile, alle produzioni energetiche garantite dagli incentivi abbiamo associato delle variabili chiamate $prod_A$, $prod_{CI}$, $prod_R$, $prod_G$, con $prod_j \in [0..10]$ (dominio espresso in MWatt); il codice relativo alla creazione di tali variabili è qui riportato.

\lstset{language=Prolog}
\begin{lstlisting}
%creazione istanza eplex
:- eplex_instance(eplex_instance).

%predicato che modella il problema dell'assegnazione ottimale dei fondi agli incentivi regionali
assegna_fondi(...):-
	%crea le variabili per i costi  (una per ogni tipo di incentivo specificato nella lista TipiInc)
	crea_var_names(TipiInc,Costi,0,50),
	%crea le variabili per le produzioni
	crea_var_names(TipiInc,Prods,0,60),
	...
\end{lstlisting}

Occorre fare subito un'importante precisazione, ovvero spiegare che per produzione energetica associata ad ogni metodologia incentivante abbiamo in questo caso inteso la produzione di energia non in termini assoluti, bensì la differenza (aumento) di produzione energetica che si ottiene finanziando un tipo di incentivo rispetto al caso in cui nessun meccanismo incentivante sia supportato (il valore di produzione energetica associato a nessuna incentivazione è definito dalla costante $PR_{base}$). Questa scelta è stata dettata dal fatto che in questa maniera fosse molto facile valutare il guadagno in termini di produzione offerto dai diversi incentivi.

Abbiamo poi introdotto le variabili ausiliarie necessarie all'approssimazione lineare chiamate $\lambda_i$ nell'equazione (\ref{eq:aprroxEqLambda}) e ugualmente nominate all'interno del problema e sempre con $\lambda_i \in [0..1]$; per ciascun tipo di incentivo e per ogni punto scelto per la campionatura sulle curve budget-produzione è stata introdotta una variabile ausiliaria $\lambda_{ji}$, con $j$ indice per il tipo di incentivo e $i$ per il punto campionato. Come scelta iniziale i punti utilizzati per la campionatura sono il minor numero possibile tramite cui rappresentare correttamente la funzione approssimata, cioè i punti in cui si osserva un cambiamento del gradiente della curva; in particolare abbiamo usato tre punti per gli incentivi Fondo Asta, Conto Interessi e Fondo Rotazione e quattro per il Fondo Garanzia. Per maggiore precisione, osserviamo che le coordinate dei punti per la campionatura sono quelle ricavate tramite le semplici righe di codice scritte in \emph{R} e descritte nella sezione precedente. 

\subsection{VINCOLI}

Dopo le variabili occorre esprimere i vincoli che consentono di modellare il problema. Come prima cosa è necessario trasporre sotto forma di vincoli le equazioni descritte in precedenza (\ref{eq:aprroxEqX}), (\ref{eq:aprroxEqY}) e (\ref{eq:aprroxEqLambda}), utilizzate per approssimare le curve delle relazioni con delle funzioni lineari a tratti; nell'equazione \ref{eq:vincoloAuxY} il termine $y_{ji}-PR_{base}$ serve per calcolare la differenza di produzione energetica rispetto al caso di nessun incentivo.

Per chiarezza ricordiamo che gli indici $j$ e $i$ selezionano rispettivamente metodo incentivante e punto di campionamento e quindi nei vincoli ~\ref{eq:vincoloAuxX}, ~\ref{eq:vincoloAuxY} e ~\ref{eq:vincoloAuxLambda} si ha $j \in [1..4]$.

\begin{equation} \label{eq:vincoloAuxX}
	costo_j = \sum_{i=1}^k \lambda_{ji} \cdot x_{ji} 
\end{equation}
\begin{equation} \label{eq:vincoloAuxY}
	prod_j = \sum_{i=1}^k \lambda_{ji} \cdot (y_{ji}-PR_{base}) 
\end{equation}
\begin{equation} \label{eq:vincoloAuxLambda}
	\sum_{i=1}^k \lambda_{ji} \leq 1
\end{equation}

Nel codice seguente è mostrata la creazione delle variabili ausiliarie e l'imposizione di questi vincoli.
\begin{lstlisting}
%predicato che modella l'approssimazione lineare a tratti
%viene invocato da assegna_fondi tante volte quanti sono i tipi di incentivo
piecewise_linear_model(TipoIncentivo,Costo_inc,Prod_inc):-
	%creo tante variabili ausiliarie quanti sono i punti che caratterizzano l'incentivo
	crea_var_sub(Punti,Auxs,0,1),
	...
	%vincoli per le variabili ausiliarie
	eplex_instance:(sum(Auxs) $=< 1),
	eplex_instance:(Costo_inc $= (Auxs*Xs) ),
	eplex_instance:(Prod_inc $= (Auxs*YsSub) ),
	...
\end{lstlisting}

A questo punto le funzioni approssimate (quelle rappresentate in Fig.~\ref{incentCompare_piecewise}) sono state inserite all'interno del modello e il passo successivo consiste nello sfruttamento di questa conoscenza per la ricerca della distribuzione ottimale dei fondi da parte del risolutore del problema.

Per questo motivo, imponiamo che la somma dei costi dei vari incentivi sia minore del budget totale dedicato agli incentivi regionali per l'energia fotovoltaica, indicato dalla costante $BUDGET_{PV}$.

\begin{equation} \label{eq:vincoloCosti}
	\sum_{j=1}^4 costo_j \leq BUDGET_{PV}
\end{equation}

In un secondo momento, specifichiamo la funzione obiettivo fornita al risolutore del problema a vincoli, il quale dovrà di massimizzare la produzione energetica assegnando il budget disponibile ai vari incentivi nel modo più efficacie possibile.

\begin{equation} \label{eq:vincoloCosti}
	\max ( \sum_{j=1}^4 prod_j )
\end{equation}

Riportiamo quindi l'implementazione.

\begin{lstlisting}
	(assegna_fondi)
	...
	%la somma dei fondi assegnati ad ogni incentivo deve essere minore del budget per il PV
	eplex_instance:(sum(Costi) $=< BudgetPV ),
	%inserisco in liste apposite le variabili ausiliarie che devono formare SOS2 (un SOS2 per tipo di incentivo)
	get_aux_sos(Auxs,AuxA,AuxCI,AuxR,AuxG,"A"),
	%la funzione obiettivo cerca di massimizzare la produzione energetica
	eplex_instance:(VarObiettivo $= (sum(Prods))),
	%inizializzo il solver con la funzione obiettivo specificata indicando anche di sfruttare SOS2 per la risoluzione ottimizzata 
	eplex_instance:eplex_solver_setup(max(sum(Outs)),VarObiettivo,[...],[...]),
	...
\end{lstlisting}

\section{ASSEGNAZIONE FONDI}

Per valutare ora l'interazione tra il DSS e il simulatore analizzeremo ora in che modo vengono finanziati le diverse tipologie di incentivazione sulla base del problema a vincoli presentato in precedenza.

Per come il sotto-problema dell'assegnazione dei fondi è stato definito, il risolutore procede distribuendo i fondi a partire dall'incentivo che risulta essere più efficace e prosegue poi finanziando gli incentivi restanti fino a esaurimento del budget previsto. Per determinare quale sia il tipo di incentivo più efficace vengono sfruttate le funzioni lineari a tratti che approssimano le relazioni tra budget e produzione energetica. In Figura~\ref{assegnFondi} osserviamo un esempio di distribuzione dei fondi, con un budget pari a dodici milioni di euro: le quattro funzioni lineari a tratti sono le relazioni inserite all'interno del modello e i punti individuati su di esse rappresentano le combinazioni di spesa (la spesa viene mostrata sull'asse delle ascisse) migliori dal punto di vista della produzione (asse delle ordinate); ad esempio, notiamo come la massima produzione energetica possibile per il Conto Interessi (in rosso) sia raggiunta con una spesa relativamente contenuta e minore del budget totale, consentendo di distribuire i fondi rimanenti ad altri tipi di incentivo.

\begin{figure}[hbt]
	\centering
	\includegraphics[scale=0.7]{assegnFondi}
	\caption{Assegnazione Fondi - Budget \euro12M }
	\label{assegnFondi}
\end{figure}

Nelle tabelle seguenti sarà mostrato in che modo vengono assegnati i fondi ai vari incentivi a partire da un determinato budget, evidenziando anche l'aumento di produzione energetica ottenibile rispetto al caso in cui nessun incentivo venga finanziato; per avere un riferimento ricordiamo che nel nostro simulatore, quindi non adeguatamente scalato, il valore di produzione energetica ottenibile in assenza di meccanismi incentivanti è di circa 21.664MW. Ricordiamo inoltre che, per come è stato progettato il simulatore, il budget destinato agli incentivi è distribuito lungo un arco temporale quinquennale. 

\myparagraph{FONDO INCENTIVI – \euro2.5M}

Con questo fondo per gli incentivi viene finanziato totalmente il Conto Interessi consumando interamente il budget disponibile, impedendo quindi di stanziare fondi anche agli altri incentivi, e ottenendo un aumento di produzione energetica di 3.563MW.

\begin{table}[h]
\centering
	\begin{tabular}{ p{0.3\textwidth}  | p{0.3\textwidth} | p{0.3\textwidth}  }
		\hline \hline 
		\nohyphens{\emph{Tipologia Incentivo}} & \nohyphens{\emph{Costo (M\euro)}} & \nohyphens{\emph{Produzione Energetica Differenziale(MW)}} \\ \hline
		Asta &  0.00 & 0.000 \\ 
		Conto Interessi & 2.50 & 3.563 \\ 
		Rotazione & 0.00 & 0.000 \\ 
		Garanzia & 0.00 & 0.000 \\ \hline 
		Totale & 2.50 & 3.563 \\
		\hline \hline 
	\end{tabular}
	\caption{Assegnazione Fondi - \euro2.5M}
	\label{tab:assegnFondi2M}	
\end{table}

\myparagraph{FONDO INCENTIVI – \euro5M}

In questo caso è possibile finanziare interamente il Conto Interessi (il quale, come atteso, richiede decisamente meno risorse degli altri, ovvero potendo essere completamente soddisfatto con 2.53 milioni di euro) e distribuire il budget restante agli incentivi Fondo Rotazione e Garanzia, ottenendo così un miglioramento della produzione energetica di 4.121MW.

\begin{table}[h]
\centering
	\begin{tabular}{ p{0.3\textwidth}  | p{0.3\textwidth} | p{0.3\textwidth}  }
		\hline \hline 
		\nohyphens{\emph{Tipologia Incentivo}} & \nohyphens{\emph{Costo (M\euro)}} & \nohyphens{\emph{Produzione Energetica Differenziale(MW)}} \\ \hline
		Asta &  0.00 & 0.000 \\ 
		Conto Interessi & 2.53 & 3.606 \\ 
		Rotazione & 1.00 & 0.286 \\ 
		Garanzia & 1.00 & 0.170 \\ \hline 
		Totale & 5.00 & 4.121 \\
		\hline \hline 
	\end{tabular}
	\caption{Assegnazione Fondi - \euro5M}
	\label{tab:assegnFondi5M}	
\end{table}

\myparagraph{FONDO INCENTIVI – \euro10M}

Con un ulteriore aumento del budget l'assegnazione dei fondi ottimale prevede di distribuire prima al Conto Interessi seguito sempre dai Fondi Garanzia e Rotazione, ottenendo un incremento della produzione energetica pari a 4.708MW; in realtà il scendo miglior incentivo è il Fondo Garanzia e quindi esso riceve finanziamenti maggiori rispetto al Rotazione, a quest'ultimo viene comunque assegnato un budget di un milione di euro (anche nel caso precedente) in quanto, per come sono stati campionati i punti che definiscono l'approssimazione lineare a tratti, a tale budget è associata una produzione energetica migliore rispetto a quella del Fondo Garanzia.

\begin{table}[h]
\centering
	\begin{tabular}{ p{0.3\textwidth}  | p{0.3\textwidth} | p{0.3\textwidth}  }
		\hline \hline 
		\nohyphens{\emph{Tipologia Incentivo}} & \nohyphens{\emph{Costo (M\euro)}} & \nohyphens{\emph{Produzione Energetica Differenziale(MW)}} \\ \hline
		Asta &  0.00 & 0.000 \\ 
		Conto Interessi & 2.53 & 3.606 \\ 
		Rotazione & 1.00 & 0.286 \\ 
		Garanzia & 6.47 & 0.816 \\ \hline 
		Totale & 10.00 & 4.708 \\
		\hline \hline 
	\end{tabular}
	\caption{Assegnazione Fondi - \euro10M}
	\label{tab:assegnFondi10M}	
\end{table}

\myparagraph{FONDO INCENTIVI – \euro15M}

Aumentando i fondi stanziati dalla regione di altri cinque milioni, si ottiene come risultato la distribuzione dei finanziamenti aggiuntivi esclusivamente al Fondo Garanzia, mentre per le restanti tipologie di incentivo sono previste le stesse spese del caso di budget pari a dieci milioni; ad ogni modo, con un fondo per l'incentivazione di quindici milioni di euro l'incremento della produzione energetica è di 5.34MW.

\begin{table}[h]
\centering
	\begin{tabular}{ p{0.3\textwidth}  | p{0.3\textwidth} | p{0.3\textwidth}  }
		\hline \hline 
		\nohyphens{\emph{Tipologia Incentivo}} & \nohyphens{\emph{Costo (M\euro)}} & \nohyphens{\emph{Produzione Energetica Differenziale(MW)}} \\ \hline
		Asta &  0.00 & 0.000 \\ 
		Conto Interessi & 2.53 & 3.606 \\ 
		Rotazione & 1.00 & 0.286 \\ 
		Garanzia & 11.47 & 1.447 \\ \hline 
		Totale & 10.00 & 5.340 \\
		\hline \hline 
	\end{tabular}
	\caption{Assegnazione Fondi - \euro15M}
	\label{tab:assegnFondi15M}	
\end{table}

\myparagraph{FONDO INCENTIVI – \euro20M}

Con un budget di venti milioni di euro notiamo due differenze. 

La prima è che dopo aver assegnato circa 13 milioni e mezzo al Fondo Garanzia diventa più utile destinare i fondi rimanenti al Fondo Rotazione, ricavando un aumento della produzione di 5.923MW; questo accade perché in corrispondenza di quel budget la pendenza della curva che descrive l'incentivo Fondo Garanzia diminuisce sensibilmente (Fig.~\ref{incentCompare_piecewise}), rendendo quindi l'incentivo meno efficace in confronto al Fondo Rotazione.

La seconda differenza è che conviene anche destinare una minima parte (un milione di euro) del budget anche al Fondo Asta e questo è probabilmente dovuto nuovamente al modo in cui sono stati campionati i punti per l'approssimazione.

\begin{table}[h]
\centering
	\begin{tabular}{ p{0.3\textwidth}  | p{0.3\textwidth} | p{0.3\textwidth}  }
		\hline \hline 
		\nohyphens{\emph{Tipologia Incentivo}} & \nohyphens{\emph{Costo (M\euro)}} & \nohyphens{\emph{Produzione Energetica Differenziale(MW)}} \\ \hline
		Asta &  1.00 & 0.125 \\ 
		Conto Interessi & 2.53 & 3.606 \\ 
		Rotazione & 2.73 & 0.458 \\ 
		Garanzia & 13.74 & 1.734 \\ \hline 
		Totale & 20.00 & 5.923 \\
		\hline 
	\end{tabular}
	\caption{Assegnazione Fondi - \euro20M}
	\label{tab:assegnFondi20M}	
\end{table}

\myparagraph{FONDO INCENTIVI – \euro40M}

Con un budget di 40 milioni di euro i risultati sono molto simili al caso precedente, con la differenza che i maggiori finanziamenti disponibili vengono interamente indirizzati al Fondo Rotazione, poichè il Conto Interessi è in grado di rendere al massimo anche con una spesa minima e il Fondo Garanzia diventa meno efficace con budget maggiori di 13 milioni e mezzo di euro (come anche il Fondo Asta con qualsiasi budget); l'incremento di produzione energetica rispetto alla mancanza di metodologie incentivanti è di 7.913MW.

\begin{table}[h]
\centering
	\begin{tabular}{ p{0.3\textwidth}  | p{0.3\textwidth} | p{0.3\textwidth}  }
		\hline \hline 
		\nohyphens{\emph{Tipologia Incentivo}} & \nohyphens{\emph{Costo (M\euro)}} & \nohyphens{\emph{Produzione Energetica Differenziale(MW)}} \\ \hline
		Asta &  1.00 & 0.125 \\ 
		Conto Interessi & 2.53 & 3.606 \\ 
		Rotazione & 22.73 & 2.448 \\ 
		Garanzia & 13.74 & 1.734 \\ \hline 
		Totale & 40.00 & 7.913 \\
		\hline 
	\end{tabular}
	\caption{Assegnazione Fondi - \euro40M}
	\label{tab:assegnFondi40M}	
\end{table}

\nocite{*}
\bibliographystyle{plain}
\bibliography{bibliography}

\end{document}


%Andrea Borghesi
%Università degli studi di Bologna

%conclusioni

%\documentclass[12pt,a4paper,openright,twoside]{book}
%\usepackage[italian]{babel}
%\usepackage{indentfirst}
%\usepackage[utf8]{inputenc}
%\usepackage[T1]{fontenc}
%\usepackage{fancyhdr}
%\usepackage{graphicx}
%\usepackage{titlesec,blindtext, color}
%\usepackage[font={small,it}]{caption}

%\begin{document}

\clearpage{\pagestyle{empty}\cleardoublepage}
\chapter*{Conclusioni}

\markboth{Conclusioni}{Conclusioni}
\addcontentsline{toc}{chapter}{Conclusioni}

In questo lavoro abbiamo considerato le problematiche relative alla realizzazione di un sistema per il supporto alle decisioni in grado di fornire ausilio ai decisori politici nel loro compito di effettuare scelte e prendere decisioni per conseguire determinati obiettivi, nel rispetto dei vincoli economici, ambientali e sociali. Il nostro scopo principale è stato quindi quello di ideare tecniche e metodologie (provenienti anche da diversi ambiti di ricerca) con le quali fosse possibile affrontare in modo efficacie le sfide poste durante la pianificazione e implementazione delle politiche, fornendo al tempo stesso uno strumento informatico che potesse essere utilizzato dai decisori politici stessi. 

Muovendoci all'interno dell'ambito del progetto europeo e-Policy, ci siamo occupati in particolare di studiare il comportamento di cittadini e imprenditori ai quali fossero offerti diversi strumenti incentivanti per la produzione di energia elettrica attraverso l'impiego di impianti fotovoltaici; questo studio è stato effettuato implementando un simulatore ad agenti in grado di ricreare la prospettiva economica e sociale dei singoli investitori e osservandone poi l'evoluzione nell'arco di un periodo temporale significativo. Accanto a questo primo elemento, un altro obiettivo raggiunto è stata la realizzazione di un modello matematico, sulla base del paradigma della programmazione a vicoli, tramite il quale fosse possibile ideare un piano energetico per la regione Emilia-Romagna. Il terzo aspetto su cui ci siamo concentrati è costituito dall'interazione tra le due componenti appena citate, ovvero abbiamo fatto in modo che dai risultati ottenuti tramite il simulatore potessero essere ricavate delle informazioni con le quali fosse possibile estendere e arricchire il modello a vincoli iniziale, garantendo quindi un'integrazione tra il livello globale considerato dalla fase di ottimizzazione e quello locale (cioè basato sul comportamento dei singoli individui/agenti) delle simulazioni.

I risultati principali che abbiamo ottenuto con questo lavoro sono stati: 
\begin{itemize}
\item l'analisi dettagliata e rigorosa delle relazioni tra le variabili in gioco all'interno del simulatore, che ci ha permesso di comprendere in che modo i cambiamenti di parametri come la disponibilità di fondi per i meccanismi incentivanti abbiano ripercussioni sul comportamento degli agenti;
\item l'apprendimento di vincoli in grado di esprimere tali tali relazioni e l'inserimento di tali vincoli all'interno del modello matematico che si occupa della pianificazione regionale, consentendo così alle varie componenti del sistema di supporto alle decisioni di interagire in modo efficacie.
\end{itemize}

Nonostante il fatto che gli scopi che ci fossimo prefissati siano stati raggiunti, il sistema sviluppato presenta certamente ancora qualche limite ed è suscettibile a diversi tipi di miglioramento prima di diventare uno strumento completo e pienamente sfruttabile dai decisori politici per la loro attività, infatti la ricerca prosegue in tutti gli ambiti coinvolti nel progetto e-Policy. Passiamo ora a illustrare possibili limiti e sviluppi futuri per le parti pertinenti a questo lavoro.
\\*

In primo luogo è utile sottolineare nuovamente che il simulatore implementato presenta al suo interno diverse assunzioni e approssimazioni effettuate per semplificarne l'implementazione e, pur consentendo al tempo stesso di effettuare uno studio accurato delle proprietà interessanti nell'ambito di questo lavoro, sarà quindi necessario migliorarlo. La direzione da seguire è quella di renderlo più realistico (potenziare la fase di valutazione di fattibilità degli investimenti compiuta dagli agenti), estenderlo affinché rifletta in maniera più accurata le dinamiche della società modellata (interazione sociale più realistica) e consenta di valutare il comportamento di ulteriori metodologie di incentivazione (implementare un meccanismo ad asta), modificarne i parametri in modo che produca in uscita valori ``reali'' (ad esempio la produzione energetica totale ottenuta da energia fotovoltaica è ora nell'ordine di grandezza di poche decine di MW, mentre nella realtà per la regione dell'Emilia-Romagna le grandezze in gioco siano più verso le centinaia di MW). Sempre per quanto riguarda il simulatore, una modifica molto importante sarà quella di consentire di simulare la presenza contemporanea di diversi meccanismi incentivanti, per osservare come le reciproche interazioni possano influenzare il risultato finale.

Un altro aspetto molto importante su cui sarà utile intervenire è quello riguardante le interazioni tra le componenti del sistema e-Policy, con un riferimento particolare al rapporto tra la fase di ottimizzazione e il simulatore. Come già accennato nel quinto capitolo, l'integrazione di queste componenti può essere ottenuta attraverso diverse tecniche, delle quali solamente una è stata concretamente implementata in questo lavoro. Da ciò segue che possibili sviluppi futuri dovrebbero andare nella direzione di sperimentare metodologie diverse per conseguire un'interazione efficiente e suggeriamo che l'impiego di metodi provenienti da molteplici aeree di ricerca potrebbe portare quasi sicuramente vantaggi per affrontare questa sfida.
\\*

Questo approccio multidisciplinare appena citato è forse una delle aspetti più importanti a caratterizzare il progetto e-Policy (sicuramente un elemento che, parlando a titolo personale, ha reso più interessante e affascinante affrontare le sfide presentatecisi), poiché, come abbiamo visto, realizzare un sistema in grado di fornire supporto alle decisioni in un settore altamente complesso come l'ideazione e l'implementazione delle politiche, è un compito che richiede l'utilizzo di molteplici competenze, tecniche e strumenti proprio a causa della natura intrinsecamente complessa della materia trattata. 

%\end{document}


\appendix
%Andrea Borghesi
%Università degli studi di Bologna

% appendice sulla programmazione logica a vincoli

\clearpage{\pagestyle{empty}\cleardoublepage}
\chapter{Esempi di CLP} 
\label{appendiceCLP} 

Illustreremo ora due esempi di modellazione di problemi a vincoli sfruttando il linguaggio ECL$^i$PS$^e$ (nel primo caso considerando domini finiti e nel secondo valori reali, avvalendoci anche della libreria Eplex), per mostrare come possono essere strutturati i problemi di programmazione logica a vincoli; in questa trattazione supporremo noti i concetti elementari della programmazione logica (procedimenti risolutivi, definizioni di un termine, etc.), la cui discussione esula da questo lavoro.

\myparagraph{Esempio \clpfd}
Il cosiddetto \emph{Send More Money} puzzle è un esempio classico di programmazione a vincoli; le variabili $[S,E,N,D,M,O,R,Y]$ rappresentano cifre da 0 a 9 e lo scopo è assegnare alle variabili valori diversi in modo che l'operazione aritmetica di Figura~\ref{SendMoreMoney} risulti corretta - inoltre i numeri devono essere ben formati, da cui $S>0$ e $M>0$. 

\begin{figure}[h]
	\centering
	\includegraphics[width=0.215\textwidth]{sendMoreMoney}
	\caption{Send More Money puzzle}
	\label{SendMoreMoney}
\end{figure}

Con la programmazione convenzionale si avrebbe necessità di esprimere una strategia di ricerca in modo esplicito (senza contare possibili ottimizzazioni come cicli innestati), mentre con linguaggi logici come Prolog verrebbe sfruttata la ricerca fornita dal risolutore interno (il motore inferenziale), con il vantaggio di una programmazione estremamente facilitata ma col rischio di un'efficienza non elevata - a meno di programmi ottimizzati, i quali richiederebbero comunque maggiori tempo e abilità. 

Questo è in effetti il campo di applicazione ideale della programmazione logica a vincoli, in particolare nell'ambito  dei domini finiti \clpfd: le variabili possono assumere valori appartenenti ad un insieme finito di numeri interi, i vincoli sono facilmente esprimibili formalmente e occorre effettuare una certa quantità di ricerca nello spazio delle soluzioni. In questo problema sarebbe naturale usare le variabili del programma per rappresentare le diverse cifre e la soluzione finale dovrà essere un assegnamento di un valore unico per ogni variabile. 

Risolvere questo problema con Prolog comporta l'utilizzo della strategia di ricerca chiamata \emph{Generate and Test}, che prevede che prima la generazione di una soluzione e poi la verifica della consistenza dei vincoli e, nel caso che questa dia esito negativo, l'assegnamento di nuovi valori alle variabili seguita da nuova verifica e così via. In questo modo l'esplorazione dello spazio delle soluzioni è chiaramente inefficiente - per esempio la possibile implementazione in Prolog mostrata qui sotto, per quanto suscettibile a miglioramenti, deve gestire $\frac{10!}{2}$ possibili assegnamenti di valori alle variabili. 

\lstset{language=Prolog}
\begin{lstlisting}
% Send More Money puzzle in Prolog
smm :-
        X = [S,E,N,D,M,O,R,Y],           % variabili
        Digits = [0,1,2,3,4,5,6,7,8,9],	 % domini
        
        % predicato che assegna una soluzione
        assign_digits(X, Digits),
       	
       	%  verifica dei vincoli vincoli
        M > 0, 
        S > 0,
                  1000*S + 100*E + 10*N + D +
                  1000*M + 100*O + 10*R + E =:=
        10000*M + 1000*O + 100*N + 10*E + Y,
        write(X).

select(X, [X|R], R).
select(X, [Y|Xs], [Y|Ys]):- select(X, Xs, Ys).

assign_digits([], _List).
assign_digits([D|Ds], List):-
        select(D, List, NewList),
        assign_digits(Ds, NewList).
\end{lstlisting}

L'implementazione realizzata con ECL$^i$PS$^e$ presenta i vantaggi di semplificare ulteriormente la modellazione del problema e di appoggiarsi all'efficiente risolutore interno per l'esplorazione dello spazio delle soluzioni, in modo particolare il fatto che ogni volta che una variabile viene istanziata i vincoli vengono propagati per eliminare a priori strade inconsistenti, riducendo gli spazi delle soluzioni e prevenendo fallimenti sicuri. 

\begin{lstlisting}
% Send More Money puzzle in ECLiPSe
smm :-
     X = [S,E,N,D,M,O,R,Y],		% variabili
     X :: [0 .. 9],				% domini finiti
     
     % vincoli
     M #> 0,
     S #> 0,
               1000*S + 100*E + 10*N + D +
               1000*M + 100*O + 10*R + E #=
     10000*M + 1000*O + 100*N + 10*E + Y,
     alldistinct(X),
     
     % ricerca della soluzione
     labeling(X),
     write(X).

\end{lstlisting}

\myparagraph{Esempio \clpr - Eplex}

Presentiamo ora un esempio di un problema (Fig.~\ref{clpR_example}) che rientra nell'ambito dei \clpr e che fa uso della libreria Eplex, tratto dal manuale di ECL$^i$PS$^e$ \cite{eclipseTut}. Ci sono tre impianti, o fabbriche, (1-3) in grado di produrre un certo prodotto con capacità diverse e i cui prodotti devono essere trasportati a quattro clienti (A-D) con quantità richieste diverse; anche il costo unitario di trasporto ai clienti è variabile. L'obiettivo del problema è minimizzare i costi di trasporto soddisfacendo le esigenze dei clienti. 

\begin{figure}[h]
	\centering
	\includegraphics[scale=0.6]{clpR_example}
	\caption{Esempio di un problema \clpr. Fonte \cite{eclipseTut}}
	\label{clpR_example}
\end{figure}

Per formulare il problema definiamo la quantità di prodotto trasportata dall'impianto $N$ al cliente $p$ come variabile $N_p$ - ad esempio $A_1$ rappresenta il costo di trasporto dalla fabbrica $A$ al cliente $1$. I vincoli da considerare sono di due tipi (sempre facendo riferimento alla Figura~\ref{clpR_example}):
\begin{itemize}
\item La quantità di prodotto consegnata da tutti gli impianti a un cliente deve essere uguale alla domanda del cliente, ad esempio per il cliente $A$ che può essere rifornito dagli impianti 1-3, abbiamo che $A_1+A_2+A_3=21$
\item La quantità di prodotto in uscita da una fabbrica non può essere superiore alla sua capacità produttiva, ad esempio per l'impianto $1$ che invia prodotti ai clienti A-D si ha che $A_1+B_1+C_1+D_1 \leq 50$
\end{itemize}   
Poiché lo scopo è minimizzare i costi di trasporto, la funzione obiettivo è di minimizzare i costi combinati del trasporto dei prodotti dai tre impianti a tutti e quattro i clienti.

La formulazione del problema è quindi la seguente.\\*
Funzione obiettivo: 
\begin{equation}  \label{exampleClpRObiett}
	\min(10A_1+7A_2+200A_3+8B_1+5B_2+10B_3+5C_2+5C_2+8C_3+9D_1+3D_2+7D_3)
\end{equation}
Vincoli:
\begin{equation}  \label{exampleCons1}
	A_1+A_2+A_3=21
\end{equation}
\begin{equation}  \label{exampleCons2}
	B_1+B_2+B_3=40
\end{equation}
\begin{equation}  \label{exampleCons3}
	C_1+C_2+C_3=34
\end{equation}
\begin{equation}  \label{exampleCons4}
	D_1+D_2+D_3=10
\end{equation}
\begin{equation}  \label{exampleCons5}
	A_1+B_1+C_1+D_1 \leq 50
\end{equation}
\begin{equation}  \label{exampleCons6}
	A_2+B_2+C_2+D_2 \leq 30
\end{equation}
\begin{equation}  \label{exampleCons7}
	A_3+B_3+C_3+D_3 \leq 40
\end{equation}

Mostriamo ora come questo problema venga modellato sfruttando la libreria Eplex. In primo luogo occorre caricare la libreria Eplex di cui si dispone (in questo caso abbiamo sfruttato un risolutore esterno open source) e ottenerne un'\emph{istanza}, la quale rappresenta un singolo problema sotto forma di modulo, a cui possono essere riferiti vincoli e funzione obiettivo consentendo quindi al solver esterno di risolvere il problema. Il codice che segue mostra come il problema di Figura~\ref{clpR_example} sia stato trasposto all'interno di ECL$^i$PS$^e$.

\begin{lstlisting}
:- lib(eplex).		% caricamento della libreria Eplex
:- eplex_instance(prob).		% definizione dell'istanza - chiamata 'prob'

main(Cost, Vars) :-
		% dichiarazione delle variabili e definizione del loro dominio		
		Vars = [A1,A2,A3,B1,B2,B3,C1,C2,C3,D1,D2,D3],
		prob: (Vars $:: 0.0..1.0Inf),  % valori maggiori o uguali a 0
		
		% definizione dei vincoli applicati all'istanza eplex
		prob: (A1 + A2 + A3 $= 21), 
		prob: (B1 + B2 + B3 $= 40),
		prob: (C1 + C2 + C3 $= 34),
		prob: (D1 + D2 + D3 $= 10),

		prob: (A1 + B1 + C1 + D1 $=< 50),
		prob: (A2 + B2 + C2 + D2 $=< 30),
		prob: (A3 + B3 + C3 + D3 $=< 40),

		% inizializza il solver esterno con la funzione obiettivo
		prob: eplex_solver_setup(min(10*A1 + 7*A2 + 200*A3 + 
			8*B1 + 5*B2 + 10*B3 +
		 	5*C1 + 5*C2 + 8*C3 +
		 	9*D1 + 3*D2 + 7*D3)),

		% ---------- Fine Modellazione ----------

		% risoluzione del problema
		prob: eplex_solve(Cost).
\end{lstlisting}

Per usare un'istanza Eplex occorre prima dichiararla con \emph{eplex\_instance/1}; una volta dichiarata, l'istanza viene riferita tramite il nome specificato. 

Come primo passo creiamo le variabili del problema e imponiamo che possano assumere solamente valori non negativi e rendiamo noti all'istanza il loro dominio (\emph{\$::/2}). Successivamente imponiamo i vincoli che modellano il problema sotto forma di uguaglianze e disuguaglianze aritmetiche; per via del solver esterno scelto, gli unici tipi di vincoli accettati sono quelli lineari - che ovviamente consentono una maggiore efficienza nella risoluzione.

Occorre poi inizializzare il risolutore esterno con l'istanza eplex creata, in modo che questa possa essere risolta. Questo è fatto dal \emph{eplex\_solver\_setup/1}, che prende come argomento la funzione obiettivo, la quale può essere di minimizzazione o massimizzazione. Infine è possibile risolvere il problema modellato attraverso \emph{eplex\_solve/1}.

Quando un'istanza viene risolta, il solver prende in considerazione tutti i vincoli ad essa relativi, i valori che le variabili del problema possono assumere e la funzione obiettivo specificata. In questo caso è possibile ottenere una soluzione ottimale pari a 710.0: 
\begin{lstlisting}
?-	main(Cost, Vars).

Cost = 710.0
Vars = [A1{0.0 .. 1e+20 @ 0.0}, A2{0.0 .. 1e+20 @ 21.0}, ....]
\end{lstlisting}


\backmatter
\nocite{*}
\bibliographystyle{plain}
\bibliography{bibliography}
\addcontentsline{toc}{chapter}{Bibliografia}

\end{document}
