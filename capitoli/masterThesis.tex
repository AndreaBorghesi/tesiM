%Andrea Borghesi
%Università degli studi di Bologna

%tesi laurea magistrale

\documentclass[12pt,a4paper,openright,twoside]{book}
\usepackage[italian]{babel}
\usepackage{indentfirst}
\usepackage[utf8]{inputenc}
\usepackage[T1]{fontenc}
\usepackage{fancyhdr}
\usepackage{graphicx}
\usepackage{titlesec,blindtext, color}
\usepackage[font={small,it}]{caption}
\usepackage{subfig}
\usepackage{listings}
\usepackage{color}
\usepackage{url}
\usepackage{textcomp}
\usepackage{eurosym}
\usepackage{amsmath}
\usepackage{frontesp} 	% frontespizio unibo 
%\usepackage[colorlinks=false]{hyperref}%<--inserisce segnalibri e riferimenti indice

%impostazioni generali per visualizzare codice
\definecolor{dkgreen}{rgb}{0,0.6,0}
\definecolor{gray}{rgb}{0.5,0.5,0.5}
\definecolor{mauve}{rgb}{0.58,0,0.82}
 
\lstset{ %
  basicstyle=\footnotesize,           % the size of the fonts that are used for the code
  backgroundcolor=\color{white},      % choose the background color. You must add \usepackage{color}
  numbers=left,                   % where to put the line-numbers
  numberstyle=\tiny\color{gray},  % the style that is used for the line-numbers
  numbersep=5pt,  
  showspaces=false,               % show spaces adding particular underscores
  showstringspaces=false,         % underline spaces within strings
  showtabs=false,                 % show tabs within strings adding particular underscores
  rulecolor=\color{black}, 
  tabsize=2,                      % sets default tabsize to 2 spaces
  breaklines=true,                % sets automatic line breaking
  breakatwhitespace=false,        % sets if automatic breaks should only happen at whitespace
  title=\lstname,                   % show the filename of files included with \lstinputlisting;
  frame=single,                   % adds a frame around the code
                                  % also try caption instead of title
  keywordstyle=\color{blue},          % keyword style
  commentstyle=\color{dkgreen},       % comment style
  stringstyle=\color{mauve},         % string literal style
  escapeinside={\%*}{*)},            % if you want to add LaTeX within your code
  morekeywords={*,...},              % if you want to add more keywords to the set
  deletekeywords={...}              % if you want to delete keywords from the given language
}

%per avere un bordo intorno alle figure
\usepackage{float}
\floatstyle{boxed} 
\restylefloat{figure}

%per poter poi impedire che certe parole vadano a capo
\usepackage{hyphenat}

%ridefinisco font per fancyhdr, per ottenere un'intestazione pulita
\newcommand{\changefont}{ \fontsize{9}{11}\selectfont }
\fancyhf{}
\fancyhead[LE,RO]{\changefont \slshape \rightmark} 	%section
\fancyhead[RE,LO]{\changefont \slshape \leftmark}	%chapter
\fancyfoot[C]{\changefont \thepage}					%footer

%titolo capitolo con "numero | titolo"
\definecolor{gray75}{gray}{0.75}
\newcommand{\hsp}{\hspace{20pt}}
\titleformat{\chapter}[hang]{\Huge\bfseries}{\thechapter\hsp\textcolor{gray75}{|}\hsp}{0pt}{\Huge\bfseries}


%\oddsidemargin=30pt \evensidemargin=20pt

%sillabazioni non eseguite correttamente
\hyphenation{sil-la-ba-zio-ne pa-ren-te-si si-mu-la-to-re ge-ne-ra-re pia-no}

%interlinea
\linespread{1.15}  
\pagestyle{fancy}

%cartelle contenenti le immagini
\graphicspath{{/media/sda4/tesi/immagini/grafici/}{/media/sda4/tesi/immagini/grafici/incCompare/}{/media/sda4/tesi/immagini/grafici/rawData/}{/media/sda4/tesi/immagini/grafici/regressionAnalysis/}{/media/sda4/tesi/immagini/schemi/}{/media/sda4/tesi/immagini/simulazione/}{/media/sda4/tesi/immagini/epolicy/}{/media/sda4/tesi/immagini/ottimizzazione/}
{/media/sda4/tesi/immagini/interazione/}}

%in modo che dopo il titolo di un paragrafo il testo vada a capo
\newcommand{\myparagraph}[1]{\paragraph{#1}\mbox{}\\}

%per scrivere bene CLP(R) e CLP(FD)
\newcommand{\clpr}{CLP({\ensuremath{\cal R}})}
\newcommand{\clpfd}{CLP({\ensuremath{\cal FD}})}

\begin{document}
\frontmatter

\title{Integrazione di ottimizzazione e simulazioni per il piano energetico regionale dell'Emilia-Romagna}
\author{Andrea Borghesi}
\date{}
\titolocorso{Ingegneria Informatica M}
\nomemateria{Sistemi Intelligenti}
\degreeyear{2011/2012}					% Anno Accademico di Laurea
\session{III} 						% Sessione di Laurea
\principaladviser{Chiar.ma Prof. Michela Milano}	% Relatore principale

% non funziona modificare frontesp.sty al variare del numero di correlatori
\firstreader{Chiar.mo Prof. Marco Gavanelli}
\secondreader{}			% Correlatori
\thirdreader{}			% Correlatori

\maketitle % Frontespizio

\thispagestyle{empty}
\textcolor{white}{.}\\
\pagebreak 

%Se si vuole mettere l'indice nell'indice (comodo per la navigazione con pdf)
%\markboth{Indice}{Indice}
%\addcontentsline{toc}{chapter}{Indice}

\tableofcontents 		% Indice
\newpage
\listoffigures
\listoftables

\mainmatter
%Andrea Borghesi
%Università degli studi di Bologna

%introduzione

%\documentclass[12pt,a4paper,openright,twoside]{book}
%\usepackage[italian]{babel}
%\usepackage{indentfirst}
%\usepackage[utf8]{inputenc}
%\usepackage[T1]{fontenc}
%\usepackage{fancyhdr}
%\usepackage{graphicx}
%\usepackage{titlesec,blindtext, color}
%\usepackage[font={small,it}]{caption}

%\begin{document}

\clearpage{\pagestyle{empty}\cleardoublepage}
\chapter*{Introduzione} 
\markboth{Introduzione}{Introduzione}
\addcontentsline{toc}{chapter}{Introduzione}

La definizione delle politiche pubbliche a livello nazionale, regionale o locale è un compito complesso, in quanto occorre operare in ambiti caratterizzati da dinamicità e incertezza, tentando di risolvere diverse problematiche e conciliando interessi conflittuali. Fattori come la globalizzazione o la sostenibilità ambientale rendono ancora più difficili le scelte che i decisori politici sono tenuti ad effettuare per l'ideazione e l'implementazione di strategie in grado di affrontare le sfide reali della società odierna, senza sottovalutare il fatto che l'elevata complessità dei sistemi considerati non consente di determinare facilmente gli effetti relativi alle decisioni prese.

Da tutto ciò consegue che sia profondamente avvertita l'esigenza di sviluppare metodologie e strumenti di cui i decisori politici si possano avvalere per gestire le problematiche di questo settore. In questa direzione procede lo sviluppo di modelli matematici e computazionali alla base dei sistemi di supporto alle decisioni politiche; tali sistemi devono essere in grado di fornire una serie di scenari decisionali alternativi, con i quali è possibile aiutare il politico a svolgere il proprio compito, ma certamente senza sostituirvisi. Prendere le decisioni senza un supporto informatico è estremamente difficile poiché sia esse che le loro interconnessioni, ovvero impatti e conseguenze che ne derivano, sono moltissime e anche perché occorre prendere in considerazione diversi aspetti, da quelli economici a quelli ambientali e sociali che hanno un grado di complessità intrinseca molto elevato.

Come esempio, basti pensare alle valutazioni da fare per l'ottimizzazione di uno o più recettori ambientali, come la qualità dell'aria o delle acque. Invece, per quanto riguarda gli aspetti sociali, è necessario tenere conto di come la società reagirà alle politiche che si vogliono implementare: ad esempio se a fronte di determinati meccanismi incentivanti i cittadini o gli imprenditori investiranno in impianti di energia da fonti rinnovabili. Disporre quindi di un sistema che modelli dal punto di vista matematico le decisioni di un piano (locale, regionale, nazionale e così via) permette di prendere in considerazione tutti questi aspetti contemporaneamente, in modo da generare politiche che abbiano impatti economici, sociali e ambientali accettabili e controllati. 
\\*

Il lavoro da noi svolto è che ci accingiamo a illustrare rientra nell'ambito sopra esposto. In particolare rientra all'interno del progetto e-Policy, VII Programma quadro dell'Unione Europea, dedicato allo sviluppo di sistemi di supporto ai decisori per produrre politiche sostenibili dal punto di vista ambientale e socialmente accettate; la regione Emilia-Romagna è partner di questo progetto e lo sviluppo del piano regionale energetico ha fornito il caso di studio per e-Policy e il lavoro in seguito presentato. Da un punto di vista molto generale, il sistema per il supporto alle decisioni sviluppato è costituito da componenti che si avvalgono di metodi provenienti da settori diversi come l'intelligenza artificiale, la ricerca operativa, sociologia, economia, etc. Per quanto riguarda il lavoro qui descritto, l'ambito considerato è quello dell'intelligenza artificiale e i componenti studiati sono un simulatore ad agenti per la comprensione del comportamento dei cittadini in reazione alle politiche che si desidera implementare e un ottimizzatore che si occupa di modellare matematicamente e individuare un piano regionale energetico ottimo. 

Per questi motivi, nel primo capitolo di questa trattazione forniremo un quadro dettagliato del progetto e-Policy e delle problematiche relative alla pianificazione regionale. 
Successivamente nei capitoli secondo e terzo saranno mostrati rispettivamente il simulatore economico-sociale e l'analisi statistica dei risultati delle simulazioni. 
Nel quarto capitolo discuteremo la modellazione matematica e la fase di ottimizzazione, mentre nel quinto capitolo parleremo dell'interazione tra quest'ultima fase e quella di simulazione.

%\end{document}


% descrizione progetto ePolicy e introduzione agli strumenti incentivanti
%Andrea Borghesi
%Università degli studi di Bologna

%capitolo dedicato alla descrizione del progetto e-policy e ai meccanismi incentivanti 

\documentclass[12pt,a4paper,openright,twoside]{report}
\usepackage[italian]{babel}
\usepackage{indentfirst}
\usepackage[utf8]{inputenc}
\usepackage[T1]{fontenc}
\usepackage{fancyhdr}
\usepackage{graphicx}
\usepackage{titlesec,blindtext, color}
\usepackage[font={small,it}]{caption}
\usepackage{subfig}
\usepackage{listings}
\usepackage{color}
\usepackage{url}
\usepackage{textcomp}

%impostazioni generali per visualizzare codice
\definecolor{dkgreen}{rgb}{0,0.6,0}
\definecolor{gray}{rgb}{0.5,0.5,0.5}
\definecolor{mauve}{rgb}{0.58,0,0.82}
 
\lstset{ %
  basicstyle=\footnotesize,           % the size of the fonts that are used for the code
  backgroundcolor=\color{white},      % choose the background color. You must add \usepackage{color}
  numbers=left,                   % where to put the line-numbers
  numberstyle=\tiny\color{gray},  % the style that is used for the line-numbers
  numbersep=5pt,  
  showspaces=false,               % show spaces adding particular underscores
  showstringspaces=false,         % underline spaces within strings
  showtabs=false,                 % show tabs within strings adding particular underscores
  rulecolor=\color{black}, 
  tabsize=2,                      % sets default tabsize to 2 spaces
  breaklines=true,                % sets automatic line breaking
  breakatwhitespace=false,        % sets if automatic breaks should only happen at whitespace
  title=\lstname,                   % show the filename of files included with \lstinputlisting;
  frame=single,                   % adds a frame around the code
                                  % also try caption instead of title
  keywordstyle=\color{blue},          % keyword style
  commentstyle=\color{dkgreen},       % comment style
  stringstyle=\color{mauve},         % string literal style
  escapeinside={\%*}{*)},            % if you want to add LaTeX within your code
  morekeywords={*,...},              % if you want to add more keywords to the set
  deletekeywords={...}              % if you want to delete keywords from the given language
}

%per avere un bordo intorno alle figure
\usepackage{float}
\floatstyle{boxed} 
\restylefloat{figure}

%per poter poi impedire che certe parole vadano a capo
\usepackage{hyphenat}
\usepackage{listings}

%ridefinisco font per fancyhdr, per ottenere un'intestazione pulita
\newcommand{\changefont}{ \fontsize{9}{11}\selectfont }
\fancyhf{}
\fancyhead[LE,RO]{\changefont \slshape \rightmark} 	%section
\fancyhead[RE,LO]{\changefont \slshape \leftmark}	%chapter
\fancyfoot[C]{\changefont \thepage}					%footer

%titolo capitolo con "numero | titolo"
\definecolor{gray75}{gray}{0.75}
\newcommand{\hsp}{\hspace{20pt}}
\titleformat{\chapter}[hang]{\Huge\bfseries}{\thechapter\hsp\textcolor{gray75}{|}\hsp}{0pt}{\Huge\bfseries}


\oddsidemargin=30pt \evensidemargin=20pt

%sillabazioni non eseguite correttamente
\hyphenation{sil-la-ba-zio-ne pa-ren-te-si si-mu-la-to-re ge-ne-ra-re pia-no}

%interlinea
\linespread{1.15}  
\pagestyle{fancy}

%cartelle contenenti le immagini
\graphicspath{{/media/sda4/tesi/immagini/grafici/}{/media/sda4/tesi/immagini/grafici/incCompare/}{/media/sda4/tesi/immagini/grafici/rawData/}{/media/sda4/tesi/immagini/grafici/regressionAnalysis/}{/media/sda4/tesi/immagini/schemi/}{/media/sda4/tesi/immagini/simulazione/}{/media/sda4/tesi/immagini/epolicy/}}

%in modo che dopo il titolo di un paragrafo il testo vada a capo
\newcommand{\myparagraph}[1]{\paragraph{#1}\mbox{}\\}

\begin{document}
\chapter{\nohyphens{QUADRO GENERALE}}

%--------> introduzione


\section[E-POLICY]{PROGETTO E-POLICY}

Il progetto europeo \emph{ePolicy} (dall'inglese Engineering the Policy Making Life Cycle, cioè ingegnerizzare il processo di creazione delle politiche) ha come obiettivo la creazione di un sistema di supporto alle decisioni per la pianificazione regionale e la valutazione degli impatti sociali, economici e ambientali. Con l'espressione \emph{sistema di supporto alle decisioni} (a cui in seguito ci riferiremo anche utilizzando l'acronimo \emph{DSS}, dall'inglese Decision Support System) si intende una classe molto ampia di sistemi software che hanno come scopo aiutare a prendere decisioni in caso di gestione di problemi complessi, facilitando l'analisi di grandi quantità di dati e suggerendo strategie e  politiche da adottare.\\*
Avviato nell'Ottobre del 2011, il progetto è coordinato dall'Università di Bologna e coinvolge nove partner tra mondo dell'accademia e della ricerca, governi regionali e settore privato, distribuiti in cinque paesi diversi dell'Unione Europea.\\*\\*
I decisori politici devono prendere decisioni complesse valutando un notevole numero di variabili e vincoli, tenendo conto quindi degli impatti che le loro scelte avranno su diversi aspetti ambientali, economici e sociali. Al tempo stesso, si è osservata negli anni un sempre crescente desiderio da parte dei cittadini di contribuire alla creazione delle politiche attraverso mezzi come i social network e i blog.\\*\\*
L'intenzione del progetto, una volta concluso, è quella di permettere a coloro che effettuano le decisioni di disporre di un sistema integrato e user-friendly, in grado di creare e valutare piani alternativi altamente ottimizzati tra i quali poter scegliere sulla base di una dettagliata analisi dei costi e benefici degli stessi.\\*\\*
Oltre a esaminare gli aspetti teorici, il progetto ePolicy mira a applicare i suoi risultati a un caso pratico: la pianificazione energetica nella regione Emilia Romagna. In particolare, il governo regionale si è posto l'obiettivo di incrementare la produzione di energia da fonti rinnovabili, concentrandosi soprattutto sulle tecnologie fotovoltaiche (PV) e a biomassa. Di conseguenza, ePolicy punta a sviluppare un modello che fornirà supporto ai decisori politici della regione che stanno cercando di mettere in pratica il miglior meccanismo incentivante per stimolare la crescita della produzione energetica da alcune tecnologie rinnovabili.

\subsection[OTTIMIZZAZIONE E DSS]{\nohyphens{OTTIMIZZAZIONE E SUPPORTO ALLE DECISIONI}}

\subsection{SIMULAZIONE AD AGENTI}

\subsection{INTERAZIONE COMPONENTI}

\section{CASO DI STUDIO}

\subsection[PIANO REGIONALE]{\nohyphens{PIANO ENERGETICO REGIONALE}}

\subsection[IMPATTI PIANO]{\nohyphens{IMPATTI DEL PIANO ENERGETICO}}

\subsection[OBIETTIVI PIANO]{\nohyphens{OBIETTIVI DEL PIANO ENERGETICO}}

\subsection[PIANO REGIONALE]{\nohyphens{PIANO ENERGETICO REGIONALE}}

\section{\nohyphens{STRATEGIE IMPLEMENTATIVE}}

\subsection[INCENTIVI]{\nohyphens{TIPOLOGIE DI INCENTIVI}}

\myparagraph{MECCANISMI INCENTIVANTI}

\myparagraph{CONFRONTO DEI MECCANISMI D'INCENTIVAZIONE}

\subsection[INCENTIVI EUROPEI]{\nohyphens{INCENTIVI IN EUROPA}}

\myparagraph{TARIFFE DI INCENTIVAZIONE}

\myparagraph{QUOTE OBBLIGATORIE DA RINNOVABILI}

\myparagraph{SUSSIDI AGLI INVESTIMENTI}

\myparagraph{INCENTIVI O ESENZIONI PER LE TASSE}

\myparagraph{INCENTIVI FISCALI}


\subsection[INCENTIVI ITALIANI]{\nohyphens{INCENTIVI IN ITALIA}}

\myparagraph{TARIFFA INCENTIVANTE}

\myparagraph{TARIFFA INCENTIVANTE ONNICOMPRENSIVA}

\subsection[INCENTIVI REGIONALI]{\nohyphens{INCENTIVI IN EMILIA ROMAGNA}}

\myparagraph{MECCANISMI INCENTIVANTI REGIONALI}

\myparagraph{INCENTIVI FISCALI}

\end{document}


% descrizione simulatore
%Andrea Borghesi
%Università degli studi di Bologna

%capitolo dedicato alla simulazione 

\documentclass[12pt,a4paper,openright,twoside]{report}
\usepackage[italian]{babel}
\usepackage{indentfirst}
\usepackage[utf8]{inputenc}
\usepackage[T1]{fontenc}
\usepackage{fancyhdr}
\usepackage{graphicx}
\usepackage{titlesec,blindtext, color}
\usepackage[font={small,it}]{caption}
\usepackage{subfig}
\usepackage{listings}
\usepackage{color}

%impostazioni generali per visualizzare codice
\definecolor{dkgreen}{rgb}{0,0.6,0}
\definecolor{gray}{rgb}{0.5,0.5,0.5}
\definecolor{mauve}{rgb}{0.58,0,0.82}
 
\lstset{ %
  basicstyle=\footnotesize,           % the size of the fonts that are used for the code
  backgroundcolor=\color{white},      % choose the background color. You must add \usepackage{color}
  showspaces=false,               % show spaces adding particular underscores
  showstringspaces=false,         % underline spaces within strings
  showtabs=false,                 % show tabs within strings adding particular underscores
  tabsize=2,                      % sets default tabsize to 2 spaces
  breaklines=true,                % sets automatic line breaking
  breakatwhitespace=false,        % sets if automatic breaks should only happen at whitespace
  title=\lstname,                   % show the filename of files included with \lstinputlisting;
                                  % also try caption instead of title
  keywordstyle=\color{blue},          % keyword style
  commentstyle=\color{dkgreen},       % comment style
  stringstyle=\color{mauve},         % string literal style
  escapeinside={\%*}{*)},            % if you want to add LaTeX within your code
  morekeywords={*,...},              % if you want to add more keywords to the set
  deletekeywords={...}              % if you want to delete keywords from the given language
}

%per avere un bordo intorno alle figure
\usepackage{float}
\floatstyle{boxed} 
\restylefloat{figure}

%per poter poi impedire che certe parole vadano a capo
\usepackage{hyphenat}
\usepackage{listings}

%ridefinisco font per fancyhdr, per ottenere un'intestazione pulita
\newcommand{\changefont}{ \fontsize{9}{11}\selectfont }
\fancyhf{}
\fancyhead[LE,RO]{\changefont \slshape \rightmark} 	%section
\fancyhead[RE,LO]{\changefont \slshape \leftmark}	%chapter
\fancyfoot[C]{\changefont \thepage}					%footer

%titolo capitolo con "numero | titolo"
\definecolor{gray75}{gray}{0.75}
\newcommand{\hsp}{\hspace{20pt}}
\titleformat{\chapter}[hang]{\Huge\bfseries}{\thechapter\hsp\textcolor{gray75}{|}\hsp}{0pt}{\Huge\bfseries}


\oddsidemargin=30pt \evensidemargin=20pt

%sillabazioni non eseguite correttamente
\hyphenation{sil-la-ba-zio-ne pa-ren-te-si si-mu-la-to-re ge-ne-ra-re pia-no}

%interlinea
\linespread{1.15}  
\pagestyle{fancy}

%cartelle contenenti le immagini
\graphicspath{{/media/sda4/tesi/immagini/grafici/}{/media/sda4/tesi/immagini/schemi/}{/media/sda4/tesi/immagini/simulazione/}}


\begin{document}
\chapter{SIMULAZIONE}

In questo capitolo verrà illustrato in che modo abbiamo studiato le relazioni che legano i diversi aspetti del piano energetico regionale, concentrandoci particolarmente su come la creazione di meccanismi di incentivazione da parte della regione Emilia-Romagna possa influenzare la produzione di energia elettrica proveniente da impianti fotovoltaici.
Per capire queste relazioni abbiamo scelto un approccio basato sulle simulazioni, cioè di realizzare un modello dell'aspetto della realtà da noi preso in esame e attraverso esso esaminare le dinamiche del sistema di nostro interesse.
\\*
Ora saranno presentati gli strumenti di cui ci siamo serviti implementare il modello sopra citato e in seguito il simulatore vero e proprio, spiegandone caratteristiche, funzionalità e limitazioni.

\section{STRUMENTI}

Gli strumenti principali che abbiamo utilizzato per realizzare il simulatore sono due:\begin{itemize}
	\item \emph{Netlogo}, un software che offre un ambiente di sviluppo ideale per la realizzazione di modelli di simulazioni ad agenti, di networks e di sistemi dinamici (sviluppato nel presso il Center for Connected Learning and Computer-Based Modeling della Northwestern University);
	\item \emph{R}, ambiente di sviluppo open source specifico per l'analisi statistica dei dati, basato sull'omonimo linguaggio di programmazione.
\end{itemize}

\subsection{NETLOGO}

Netlogo è un linguaggio di programmazione e ambiente di sviluppo open source che permette la modellazione di sistemi complessi formati da molteplici agenti che interagiscono tra loro, studiandone l'evoluzione  e visualizzandola in tempo reale.
\\* L'ambiente di sviluppo è scritto in Java (con il vantaggio quindi di ottenere una grande portabilità del software stesso) e il linguaggio eredita ed estende le caratteristiche del linguaggio di programmazione multiparadigma Logo, realizzato negli anni '60 presso il Massachusetts Institute of Technology e caratterizzato dalla sua derivazione dal Lisp e le numerose applicazioni in ambito educativo. Il codice che definisce il comportamento degli agenti è interpretato senza necessità di essere compilato e questa caratteristica permette un'interazione a run time con il modello stesso (modificare parametri di controllo attraverso pulsanti e sliders, visualizzare dinamicamente variabili o grafici relativi alla simulazione in corso, etc..).

\paragraph{Ambiente di sviluppo}

All'interno di Netlogo un elemento fondamentale è il "mondo virtuale", ovvero l'ambiente della simulazione all'interno del quale i diversi agenti agiscono. Gli agenti, le entità che possono eseguire istruzioni, possono essere di quattro tipi (Fig.~\ref{netlogoWorld}): \begin{itemize}
\item \emph{turtles} (tartarughe), gli agenti che possono muoversi all'interno del mondo;
\item \emph{patches}, le aeree quadrate che costituisco il mondo bidimensionale di Netlogo e sopra le quali si possono spostare le turtles;
\item \emph{links}, i collegamenti tra due diverse turtles, non hanno una posizione né risiedono su patches e possono orientati o non orientati;
\item \emph{observer}, il quale rappresenta concettualmente la vista complessiva del modello visto da fuori e contiene tutte le informazioni macro del modello e, quindi, tutte le variabili globali che lo caratterizzano.
\end{itemize}

\begin{figure}[htb]
	\begin{center}
	\includegraphics[scale=0.3]{netlogoWorld}
	\end{center}
	\caption{Il mondo virtuale di Netlogo}
  	\label{netlogoWorld}
\end{figure}

Le tartarughe popolano il modello ed operano in parallelo, interagendo tra loro e con le patches su cui si muovono; possono essere specificate diverse tipologie di turtles che, nel linguaggio Logo, prendono il nome di breed, ovvero razza, caratteristica che in parte richiama il concetto di classe nel paradigma della programmazione ad oggetti, in quanto ogni razza possiede una lista di attributi e variabili proprietarie comuni solo agli agenti che vi appartengono.\\*\\*
L'ambiente di sviluppo è costituito da un'interfaccia grafica che consente di interagire intuitivamente con i parametri che regolano il modello o eseguire determinate azioni, attraverso l'uso di determinati pulsanti, sliders o altri elementi inseriti durante lo sviluppo del modello; questa interfaccia ha anche l'importante funzione di mostrare in tempo reale i movimenti degli agenti all'interno del mondo virtuale e presentare durante e dopo la simulazione le informazioni relative alla stessa, sotto forma di grafici, tabelle, etc...\\*
Accanto all'interfaccia grafica è ovviamente presente la sezione che riguarda il codice, il quale definisce i comportamenti delle entità  che agiscono dentro il mondo virtuale. Il codice della simulazione risiede tutto all'interno di un unico listato, suddiviso in diverse procedure che sono destinate all'esecuzione da parte degli agenti o, in maniera del tutto generale, di tutte le istanze che costituiscono il modello. Le procedure, in NetLogo, vengono suddivise in due diverse tipologie, ovvero \emph{commands}, azioni che devono essere portate a termine da un agente producendo un qualche risultato, e \emph{reporters}, istruzioni per calcolare un determinato valore che verrà riportato dall'agente a chiunque lo richieda. \\*
In Figura ~\ref{netlogoUI_code} sono mostrati un esempio di una parte di iterfaccia grafica con relativi selettori e rappresentazione del mondo virtuale e il codice relativo al pulsante \emph{setup} di tale interfaccia.

\begin{figure}[hbt]
	\centering
	\subfloat[Interfaccia Grafica]{\includegraphics[scale=0.4]{netlogoUI}\label{netlogoUI}}
	\quad 
	\subfloat[Codice]{\includegraphics[scale=0.45]{netlogoCode}\label{netlogoCode}}
	\caption{Esempio dell'ambiente di sviluppo di NetLogo}
	\label{netlogoUI_code}
\end{figure}

Altro elemento fondamentale dell'ambiente di sviluppo è rappresentato dalla gestione delle variabili aggregate e le relative statistiche. Poiché lo scopo di un modello di simulazione ad agenti è spesso rappresentato dalla necessità di comprendere a livello globale il comportamento di un sistema descritto nelle sue singole componenti ne deriva la necessità di calcolare e rappresentare l'andamento nel tempo di variabili aggregate per valutare qualitativamente il modello. In NetLogo ad ognuna delle variabili del modello è possibile associare un oggetto monitor, che ne mostra dinamicamente le variazioni, oppure un grafico, che ne raffigura l'andamento nel tempo. Gli  oggetti che permettono la gestione dei grafici sono \emph{plot} e \emph{histogram}, entrambi da definire sia a livello d'interfaccia grafica come avviene anche per le procedure, sia nel codice della simulazione.\\* 


\subsection{R}

Come già accennato, R è un linguaggio di programmazione open source e un ambiente software usato per la manipolazione di dati, calcolo e analisi statistica e presentazione grafica dei risultati. Il design R è stato ampiamente influenzato da due linguaggi preesistenti, S sviluppato da J.Chambers e colleghi presso i Bell Laboratories negli anni '70 e Scheme creato presso il MIT AI Lab sempre negli anni settanta da G.L.Steele e G.J.Sussman. \\*
Il nucleo di R consiste di un linguaggio interpretato a cui sono state aggiunte numerose funzionalità per un grande numero di procedure statistiche;  tra queste è possibile ricordarne alcune: modelli di regressione lineare, lineare generalizzata e non lineare, analisi di serie temporali, classici test parametrici e non, clustering, classificazione e altre. R è facilmente estendibile grazie alla presenza di numerosi pacchetti software creati dagli utenti e dedicati a specifiche aree di studio e possiede inoltre un grande insieme di funzioni indicate per una presentazione flessibile ed efficiente dei dati e la produzione di grafici di qualità.\\*
Per interagire con l'interprete del linguaggio R è possibile fornire le istruzioni direttamente da riga di comando oppure appoggiarsi a interfacce grafiche, ma per le nostre necessità è stato sufficiente utilizzare la riga di comando\\* \\* 
Per via della sua derivazione da S, R presenta alcune caratteristiche che lo fanno rientrare all'interno del paradigma dei linguaggi Object Oriented, almeno parzialmente, e al tempo stesso possiede alcuni aspetti che lo avvicinano alla natura dei linguaggi funzionali(come Scheme), come ad esempio la possibilità di trattare le funzioni stesse come oggetti.\\* Le principali strutture dati sono le seguenti: \begin{itemize}
\item \emph{vettori}, singole entità costituite da una collezione di valori di un certo tipo come ad esempio numerici,logici o caratteri;
\item \emph{matrici (arrays)}, generalizzazioni multi-dimensionali di vettori;
\item \emph{liste}, forme di vettori più generali nelle quali gli elementi non devono necessariamente essere dello stesso tipo;
\item \emph{fattori}, oggetti simili ai vettori usati per specificare una classificazione (raggruppamento) delle componenti di altri vettori con la stessa lunghezza;
\item \emph{data frames}, strutture simili alle matrici in cui le colonne possono essere di tipi diversi;
\item \emph{funzioni}, le quali sono esse stesse oggetti e forniscono così un modo semplice e flessibile di estendere R.
\end{itemize}
Come in ogni linguaggio di programmazione è poi ovviamente possibile manipolare queste strutture dati attraverso operatori, strutture di controllo, funzioni, etc...\\* \\*
Illustreremo ora un brevissimo esempio per far capire un possibile utilizzo di R per effettuare una semplice analisi statistica. Supponiamo di voler studiare la relazione che lega due variabili, \emph{a} e \emph{b}, i cui valori si trovano in un file di tipo \emph{Comma Separated Values}. Il primo passo è importare tali valori dal file e inserirli in una struttura dati, in questo caso una matrice con due colonne (una per ogni variabile) e ordinarli in base ai valori della prima variabile.

\lstset{language=R}
\begin{lstlisting}
> matrice.dati <- read.csv("file.csv")
> matrice.ordinata <- matrice_dati[order(matrice.dati$a),]
\end{lstlisting}

A questo punto sarebbe possibile svolgere diverse operazioni sui dati (ad esempio calcolare per ogni valore di ogni variabile i valori medi,...) ma ci limiteremo a effettuare una semplice regressione lineare.
\begin{lstlisting}
> modello.lineare <- lm(matrice$b ~ matrice$a)
\end{lstlisting}

\begin{figure}[hbt]
	\centering
	\subfloat[Grafico]{\includegraphics[scale=0.4]{example_r_graph}\label{example_r_graph}} 
	\quad
	\subfloat[Analisi statistica]{\includegraphics[scale=0.6]{example_r_result}\label{example_r_result}}
	\caption{Esempio di utilizzo di R}
	\label{example_r}
\end{figure}

R ci consente ora di effettuare analisi statistiche sul modello di regressione applicato per stabilirne validità e significatività in rapporto ai dati in nostro possesso e successivamente di presentare graficamente i risultati ottenuti.
\begin{lstlisting}
> #analisi statistica minima
> summary(modello.lineare)       
> #disegna i punti corrispondenti ai valori nella matrice
> plot(matrice$b ~ matrice$a,type="p",lwd=3,ylab="b",xlab="a")    
> #disegna la curva di regressione
> lines(matrice$a,predict(modello.lineare), lty="solid", col="darkred", lwd=2)    
\end{lstlisting}


In Figura ~\ref{example_r} sono stati riportati il grafico prodotto da questo esempio e i risultati ottenuti dalla semplicissima analisi statistica, tra i quali notiamo il coefficiente di determinazione (R-squared) e l'errore residuo ( 
Residual standard error).



\section{MODELLO AD AGENTI}


In generale esistono numerose tecniche di simulazione che si differenziano per i diversi metodi di formalizzazione dei modelli, i linguaggi simbolici usati e i gli ambiti di applicazione più indicati (tecniche matematiche, statistiche, sperimentali,..). In questo lavoro abbiamo scelto di sviluppare un modello basato sul paradigma della simulazione ad agenti, il quale trae origine dall'unione della teoria della complessità con l'intelligenza artificiale distribuita.\\*\\*
Possiamo iniziare con un definizione, spiegando quindi ciò che si intende col concetto di \emph{agente}, ovvero qualsiasi cosa possa essere vista come un sistema che percepisce il suo ambiente attraverso sensori e agisce su di esso mediante attuatori, in particolare un agente software è un'istanza di una classe del paradigma della programmazione ad oggetti, ovvero un oggetto indipendente  che incapsula proprietà e funzioni autonome ed interagisce, secondo protocolli di comunicazione, con altri oggetti/agenti altrettanto autonomi.\\*


\section{SIMULATORE BASE}


Dopo aver introdotto nei paragrafi precedenti gli strumenti software che abbiamo utilizzato e il tipo di metodologia scelta, è giunto ora il momento di illustrare dettagliatamente come il modello da simulare sia stato implementato. Ci concentreremo prima sulla versione iniziale della simulazione ad agenti che ha lo scopo di modellare, individuare ed analizzare le principali caratteristiche che influenzano la scelta di un investimento nel settore dell'energia da fonti rinnovabili sul territorio della regione Emilia-Romagna, ed in particolare nel fotovoltaico.\\*
Dalla prospettiva degli agenti (privati, aziende,...) prima di realizzare un investimento in questo settore è necessario effettuare alcuni studi relativi al luogo di installazione, all'irraggiamento solare, alla potenza dell'impianto ed al suo rendimento e contestualmente è di fondamentale importanza effettuare un'analisi approfondita che consenta di verificare con esattezza la convenienza ed il ritorno economico dell'investimento.\\*
La simulazione è stata pensata con lo scopo di analizzare le esigenze connesse allo sviluppo di un nuovo progetto  fotovoltaico in fase di pianificazione e verificare la fattibilità dell'idea, fornendo quindi un utile strumento di valutazione per i singoli investitori, ma allo stesso tempo consente di ricavare informazioni di natura più globale, come la quantità totale di energia prodotta con tecnologie fotovoltaiche o le spese sostenute dalla regione, grazie alle quali è possibile integrare i risultati delle simulazioni all'interno del problema di ottimizzazione che ha come obiettivo la creazione di un piano energetico regionale.


\subsection{METODO DI INCENTIVAZIONE}


\section{SIMULATORE ESTESO}

\subsection{MODALITÀ INCENTIVANTI}

\subsection{INTERAZIONE SOCIALE}


\section{LIMITI SIMULATORE}



\end{document}


% analisi dei risultati delle simulazioni
%Andrea Borghesi
%Università degli studi di Bologna

%capitolo dedicato all'analisi dei risultati delle simulazioni 

\documentclass[12pt,a4paper,openright,twoside]{report}
\usepackage[italian]{babel}
\usepackage{indentfirst}
\usepackage[utf8]{inputenc}
\usepackage[T1]{fontenc}
\usepackage{fancyhdr}
\usepackage{graphicx}
\usepackage{titlesec,blindtext, color}
\usepackage[font={small,it}]{caption}
\usepackage{subfig}
\usepackage{listings}
\usepackage{color}
\usepackage{url}
\usepackage{textcomp}

%impostazioni generali per visualizzare codice
\definecolor{dkgreen}{rgb}{0,0.6,0}
\definecolor{gray}{rgb}{0.5,0.5,0.5}
\definecolor{mauve}{rgb}{0.58,0,0.82}
 
\lstset{ %
  basicstyle=\footnotesize,           % the size of the fonts that are used for the code
  backgroundcolor=\color{white},      % choose the background color. You must add \usepackage{color}
  numbers=left,                   % where to put the line-numbers
  numberstyle=\tiny\color{gray},  % the style that is used for the line-numbers
  numbersep=5pt,  
  showspaces=false,               % show spaces adding particular underscores
  showstringspaces=false,         % underline spaces within strings
  showtabs=false,                 % show tabs within strings adding particular underscores
  rulecolor=\color{black}, 
  tabsize=2,                      % sets default tabsize to 2 spaces
  breaklines=true,                % sets automatic line breaking
  breakatwhitespace=false,        % sets if automatic breaks should only happen at whitespace
  title=\lstname,                   % show the filename of files included with \lstinputlisting;
  frame=single,                   % adds a frame around the code
                                  % also try caption instead of title
  keywordstyle=\color{blue},          % keyword style
  commentstyle=\color{dkgreen},       % comment style
  stringstyle=\color{mauve},         % string literal style
  escapeinside={\%*}{*)},            % if you want to add LaTeX within your code
  morekeywords={*,...},              % if you want to add more keywords to the set
  deletekeywords={...}              % if you want to delete keywords from the given language
}

%per avere un bordo intorno alle figure
\usepackage{float}
\floatstyle{boxed} 
\restylefloat{figure}

%per poter poi impedire che certe parole vadano a capo
\usepackage{hyphenat}
\usepackage{listings}

%ridefinisco font per fancyhdr, per ottenere un'intestazione pulita
\newcommand{\changefont}{ \fontsize{9}{11}\selectfont }
\fancyhf{}
\fancyhead[LE,RO]{\changefont \slshape \rightmark} 	%section
\fancyhead[RE,LO]{\changefont \slshape \leftmark}	%chapter
\fancyfoot[C]{\changefont \thepage}					%footer

%titolo capitolo con "numero | titolo"
\definecolor{gray75}{gray}{0.75}
\newcommand{\hsp}{\hspace{20pt}}
\titleformat{\chapter}[hang]{\Huge\bfseries}{\thechapter\hsp\textcolor{gray75}{|}\hsp}{0pt}{\Huge\bfseries}


\oddsidemargin=30pt \evensidemargin=20pt

%sillabazioni non eseguite correttamente
\hyphenation{sil-la-ba-zio-ne pa-ren-te-si si-mu-la-to-re ge-ne-ra-re pia-no}

%interlinea
\linespread{1.15}  
\pagestyle{fancy}

%cartelle contenenti le immagini
\graphicspath{{/media/sda4/tesi/immagini/grafici/}{/media/sda4/tesi/immagini/grafici/incCompare/}{/media/sda4/tesi/immagini/grafici/rawData/}{/media/sda4/tesi/immagini/grafici/regressionAnalysis/}{/media/sda4/tesi/immagini/schemi/}{/media/sda4/tesi/immagini/simulazione/}{/media/sda4/tesi/immagini/epolicy/}}

%in modo che dopo il titolo di un paragrafo il testo vada a capo
\newcommand{\myparagraph}[1]{\paragraph{#1}\mbox{}\\}

\begin{document}
\chapter{RISULTATI SIMULAZIONI}

Nel capitolo precedente abbiamo descritto il modello ad agenti implementato, evidenziandone le finalità e le caratteristiche fondamentali, accennando brevemente alle informazioni ricavabili dal simulatore.\\*
Lo scopo di questo capitolo sarà quindi la dettagliata analisi dei dati prodotti dalle simulazioni, l'individuazione delle relazioni che legano le grandezze in gioco all'interno dell'ambiente simulato, la presentazione e discussione dei risultati ottenuti.\\*
Inizialmente presenteremo gli strumenti utilizzati per effettuare l'analisi sopra descritta, per poi passare alla discussione vera e propria nei paragrafi successivi.

\section{STRUMENTI}
Per esaminare i dati prodotti dalle simulazioni effettuate e visualizzare i risultati ottenuti abbiamo utilizzato \emph{R} \cite{Rlanguage}, un ambiente di sviluppo specifico per l'analisi statistica dei dati, basato sull'omonimo linguaggio di programmazione.

\subsection{R}

R è un linguaggio di programmazione open source e un ambiente software usato per la manipolazione di dati, calcolo e analisi statistica e presentazione grafica dei risultati. Il design di R è stato ampiamente influenzato da due linguaggi preesistenti, S sviluppato da J.Chambers e colleghi presso i Bell Laboratories negli anni '70 e Scheme creato presso il MIT AI Lab sempre negli anni settanta da G.L.Steele e G.J.Sussman. \\*
Il nucleo di R consiste di un linguaggio interpretato a cui sono state aggiunte numerose funzionalità per un grande numero di procedure statistiche;  tra queste è possibile ricordarne alcune: modelli di regressione lineare, lineare generalizzata e non lineare, analisi di serie temporali, classici test parametrici e non, clustering, classificazione e altre. R è facilmente estendibile grazie alla presenza di numerosi pacchetti software creati dagli utenti e dedicati a specifiche aree di studio e possiede inoltre un grande insieme di funzioni indicate per una presentazione flessibile ed efficiente dei dati e la produzione di grafici di qualità.\\*
Per interagire con l'interprete del linguaggio R è possibile fornire le istruzioni direttamente da riga di comando oppure appoggiarsi a interfacce grafiche, ma per le nostre necessità è stato sufficiente utilizzare la riga di comando.\\* \\* 
Per via della sua derivazione da S, R presenta alcune caratteristiche che lo fanno rientrare all'interno del paradigma dei linguaggi Object Oriented, almeno parzialmente, e al tempo stesso possiede alcuni aspetti che lo avvicinano alla natura dei linguaggi funzionali(come Scheme), come ad esempio la possibilità di trattare le funzioni stesse come oggetti.\\* Le principali strutture dati sono le seguenti: \begin{itemize}
\item \emph{vettori}, singole entità costituite da una collezione di valori di un certo tipo come ad esempio numerici,logici o caratteri;
\item \emph{matrici (arrays)}, generalizzazioni multi-dimensionali di vettori;
\item \emph{liste}, forme di vettori più generali nelle quali gli elementi non devono necessariamente essere dello stesso tipo;
\item \emph{fattori}, oggetti simili ai vettori usati per specificare una classificazione (raggruppamento) delle componenti di altri vettori con la stessa lunghezza;
\item \emph{data frames}, strutture simili alle matrici in cui le colonne possono essere di tipi diversi;
\item \emph{funzioni}, le quali sono esse stesse oggetti e forniscono così un modo semplice e flessibile di estendere R.
\end{itemize}
Come in ogni linguaggio di programmazione è poi ovviamente possibile manipolare queste strutture dati attraverso operatori, strutture di controllo, funzioni, etc.\\* \\*
Illustreremo ora un brevissimo esempio per far capire un possibile utilizzo di R per effettuare una semplice analisi statistica. Supponiamo di voler studiare la relazione che lega due variabili, \emph{a} e \emph{b}, i cui valori si trovano in un file di tipo \emph{Comma Separated Values}. Il primo passo è importare tali valori dal file e inserirli in una struttura dati, in questo caso una matrice con due colonne (una per ogni variabile) e ordinarli in base ai valori della prima variabile.

\lstset{language=R}
\begin{lstlisting}
> matrice.dati <- read.csv("file.csv")
> matrice.ordinata <- matrice_dati[order(matrice.dati$a),]
\end{lstlisting}

A questo punto sarebbe possibile svolgere diverse operazioni sui dati (ad esempio calcolare per ogni valore di ogni variabile i valori medi,...) ma ci limiteremo a effettuare una semplice regressione lineare.
\begin{lstlisting}
> modello.lineare <- lm(matrice$b ~ matrice$a)
\end{lstlisting}

R ci consente ora di effettuare analisi statistiche sul modello di regressione applicato per stabilirne validità e significatività in rapporto ai dati in nostro possesso e successivamente di presentare graficamente i risultati ottenuti.
\begin{lstlisting}
> # analisi statistica minima
> summary(modello.lineare)       
> # disegna i punti corrispondenti ai valori nella matrice
> plot(matrice$b ~ matrice$a,type="p",lwd=3,ylab="b",xlab="a")    
> # disegna la curva di regressione
> lines(matrice$a,predict(modello.lineare), lty="solid", col="darkred", lwd=2)    
\end{lstlisting}


In Figura ~\ref{example_r} sono stati riportati il grafico prodotto da questo esempio e i risultati ottenuti dalla semplicissima analisi statistica, tra i quali notiamo il coefficiente di determinazione (R-squared) e l'errore residuo (
Residual standard error).

\begin{figure}[H]
	\centering
	\subfloat[Grafico]{\includegraphics[scale=0.4]{example_r_graph}\label{example_r_graph}} 
	\quad
	\subfloat[Analisi statistica]{\includegraphics[scale=0.6]{example_r_result}\label{example_r_result}}
	\caption{Esempio di utilizzo di R}
	\label{example_r}
\end{figure}


\section{METODOLOGIA ANALITICA}

Illustreremo ora con quali metodi sono stati analizzati i dati ricavati dalle simulazioni; questo comporta anche una rapida descrizione delle tecniche statistiche impiegate e della loro applicazione nel nostro contesto.

\subsection{ANALISI DI REGRESSIONE}
Una delle tecniche statistiche maggiormente utilizzate per stimare le relazioni tra variabili è l'\emph{analisi della regressione}; in questa categoria rientrano diversi metodi che hanno come obiettivo quello di trovare un modello che leghi una variabile dipendente ed una o più variabili indipendenti (in particolare l'analisi della regressione consente di capire come cambia il valore di una variabile dipendente al variare del valore di una variabile indipendente, mantenendo fisse le restanti); seguendo la terminologia di uso comune in seguito le variabili indipendenti saranno chiamate anche \emph{predittori}.  
I modelli con i quali si tenta di approssimare le relazioni studiate possono essere di numerosi tipi, tra i quali è possibile ricordare quelli parametrici (come la regressione lineare e più in generale tutte le forme di regressione polinomiale), nei quali la funzione di regressione è definita attraverso un certo numero di parametri stimati a partire dai dati, e quelli non parametrici, poi ancora modelli locali (regressione LOESS), Bayesiani, segmentati, etc..\\*
In genere la scelta del giusto modello da applicare ai propri dati è un procedimento empirico che prevede di tentare differenti tecniche di regressione sulla stessa serie di dati per poi valutare quale fosse la scelta migliore, ovvero quale sia il modello di regressione che presenta la maggiore bontà di adattamento (in inglese ''goodness of fit''), cioè una misura che riassume la discrepanza tra i valori osservati e i valori attesi sotto il modello in questione. 
Una volta scelto il modello migliore, questo può essere usato per fare predizioni, comprendere in che modo in che modo certe variabili o aspetti di un problema ne influenzino altri, essere integrato all'interno di un sistema informatico attraverso tecniche di apprendimento automatico (come sarà mostrato nel prosieguo d questa trattazione). 
\\*\\*
Un aspetto di grande importanza è quindi la validazione del modello, valutare cioè se è in accordo con i dati presi in esame. Tra i diversi metodi di validazione possibili alcuni prevedono metodi numerici, come ad esempio il calcolo del coefficiente di determinazione, altri richiedono l'uso di tecniche più qualitative, ad esempio l'analisi grafica dei valori residuali; in genere per effettuare una validazione completa e affidabile vengono impiegate tecniche appartenenti ad entrambe le categorie ed anche in questo lavoro abbiamo agito in questo modo.\\*\\*
Uno dei principali indicatori numerici usati per valutare la bontà di un modello di regressione è il calcolo del \emph{coefficiente di determinazione}, o $R^2$, un numero reale compreso tra 0 e 1 che misura la proporzione di variabilità della risposta dovuta al modello statistico; un valore vicino a 0 indica che la regressione scelta non si adatta ai dati, viceversa valori vicini a 1 indicano che il modello è buono. \\*
Un indicatore utile per giudicare se la regressione effettuata abbia significato dal punto di vista statistico è il  \emph{valore p} (\emph{p-value} in inglese), il quale viene in genere confrontato con il livello di significatività fissato (in genere indicato con la lettera $\alpha$ e con valori in genere tra 0.05 e 0.001); se il p-value risulta essere minore di $\alpha$ la regressione può essere considerata statisticamente significativa.\\*
Un altro metodo utilizzato per verificare la bontà della regressione effettuata, oltre che per altri diversi scopi, è cosiddetto il \emph{F-test}, che, molto brevemente, consente di valutare se la regressione abbia significatività statistica calcolando un valore associatole e confrontandolo con un valore critico di una particolare distribuzione di probabilità chiamata  distribuzione di Fisher-Snedecor, o \emph{F-distribution} (questo test è utile anche per confrontare tra loro diversi modelli di regressione applicati alle stesse serie di dati).\\*\\*
In genere i metodi numerici da soli non sono ritenuti sufficienti per stabilire la validità di un modello di regressione, in quanto tendono a concentrarsi troppo solamente su alcuni aspetti del rapporto tra modello e dati, tentando di comprimere quelle informazioni in un singolo numero o risultato di un test. Per questo motivo in genere  tali metodi vengono spesso affiancati da tecniche di tipo più qualitativo, con le quali è possibile stimare la bontà di adattamento di una regressione osservando determinati grafici relativi a certe caratteristiche del modello di regressione da valutare.\\*
Lo strumento primario per stabilire se un modello di regressione approssima in maniera significativa una serie di dati è il \emph{grafico di dispersione} (in inglese \emph{scatter-plot}) relativo alla distribuzione dei \emph{residuali} rispetto alla variabile usata come predittore. Col termine residuali di un modello si intendono le differenze
tra la risposta osservata (il valore della variabile dipendente) e la risposta prevista (stimata tramite regressione) per ogni valore della variabile indipendente appartenente al campione di dati. Se il modello si adattasse correttamente ai dati, allora i residuali approssimerebbero gli errori casuali che rendono la relazione tra variabile dipendente e indipendente una relazione statistica, quindi, in sostanza, se i residuali presentano un comportamento casuale, osservabile attraverso i grafici di dispersione, questo suggerisce che il modello abbia un buon adattamento ai dati; se viceversa fosse evidente una struttura non casuale nella distribuzione dei residuali, questo sarebbe un chiaro segnale che il modello non è una buona approssimazione.\\*
Un altro aspetto che è possibile valutare attraverso metodi grafici è se valga o meno \emph{l'assunzione di normalità}, ovvero stabilire se è lecito aspettarsi che gli errori casuali inerenti al processo statistico modellato con la regressione seguano una distribuzione normale; questa assunzione viene generalmente fatta perché spesso una distribuzione normale descrive in maniera ragionevolmente accurata la distribuzione effettiva degli errori casuali di serie di dati nel mondo reale. Per controllare quindi se l'assunzione di normalità della serie di dati sia valida (e potendo poi effettuare previsioni corrette tramite la regressione) vengono utilizzati i \emph{grafici di probabilità normale}, costruiti tracciando i valori ordinati dei residuali e confrontandoli con i valori di una distribuzione  normale standard; se i punti tracciati sul piano associati ai residuali giacciono vicini alla linea determinata dalla distribuzione normale, allora si può affermare che gli errori casuali seguono una distribuzione normale.


\subsection{IMPLEMENTAZIONE IN R}

Per studiare le relazioni di nostro interesse la metodologia scelta consiste nell'aver effettuato un grande numero di simulazioni controllate (ovvero fissando tutti i parametri non rilevanti e variando quelli di cui osservare il comportamento, come il budget regionale o il tipo di incentivo), dopodiché abbiamo effettuato un'analisi statistica dei dati e tentato di risalire alle curve relative all'andamento delle relazioni attraverso l'uso di tecniche di regressione lineare e non.\\*
Per effettuare l'analisi della regressione nel nostro caso, il primo passo è consistito nello scegliere le variabili di cui studiare la relazione; per ogni tipologia di incentivazione sono stati considerati tre casi (riportati in seguito in maniera estesa):
\begin{itemize}
\item relazione tra il budget che la regione dedica agli incentivi (variabile indipendente) e produzione energetica da impianti fotovoltaici (variabile dipendente);
\item relazione tra la sensibilità degli agenti simulati all'influenza dell'interazione sociale e la produzione energetica (variabile dipendente);
\item relazione tra il raggio dell'interazione sociale, predittore, e produzione energetica da fotovoltaico.
\end{itemize}

Spiegheremo ora in che modo siano state implementate le tecniche di regressione statistica e di verifica dei modelli all'interno dell'ambiente fornito da R (i frammento di codice mostrati sono relativi allo studio della relazione tra budget e produzione energetica, ma il modo di gestire le restanti relazione è stato assolutamente analogo).\\*\\*
I dati di nostro interesse generati dal simulatore sono serie di coppie di valori budget-produzione elettrica (facilmente rappresentabili in un grafico), vengono letti da file, inseriti in una matrice e ordinati. Dal momento che a causa degli elementi di casualità presenti nel simulatore i dati grezzi presentano un discreto rumore (una non trascurabile variabilità della produzione energetica a fronte degli stessi valori di budget) abbiamo calcolato la media della produzione energetica per ogni valore di budget, ottenendo i dati più puliti con cui tentare la regressione.

\begin{lstlisting}
>data.unsorted <- read.csv(nome_file)
># dati grezzi ordinati sul valore di budget
> data <- data.unsorted[order(data.unsorted$Budget),]
># aggdata rappresenta la matrice contenente le coppie costituite dal valore di Budget e relativo valore medio di produzione energetica (Out)
> aggdata <- tapply(data$Out,data$Budget,mean)
> aggdata <- as.data.frame(aggdata)  
> head(aggdata)
> aggdata$Budget <- seq(from=0, to=(length(aggdata[,1]))-1, 1)
\end{lstlisting}
 
Abbiamo tentato con diversi tipi di regressione, a partire dalla versione più semplice ovvero la lineare \cite{robustLinearRegression}(con la quale cerchiamo di fare previsione con una funzione lineare della variabile indipendente), fino a modelli più complessi come quella polinomiale \cite{Gergonne1974439,Stigler1974431} o ancora la cosiddetta regressione locale, o LOESS \cite{lowess}(la quale procede adattando modelli polinomiali di basso grado a sottoinsiemi locali della serie di dati, in pratica costruendo punto per punto la funzione di previsione); per i diversi modelli sono poi stati calcolati la bontà di adattamento ai dati sfruttando le informazioni quantitative fornite da R.

\begin{lstlisting}
> #modello lineare
> linearModelAgg <- lm(aggdata$Out ~ aggdata$Budget)
> # modello quadratico
> quadraticModelAgg <- lm(aggdata$Out ~ poly(aggdata$Budget, 2, raw=TRUE))
> # modello cubico
> cubicModelAgg <- lm(aggdata$Out ~ poly(aggdata$Budget, 3, raw=TRUE))
> # modello polinomiale di decimo grado
> highPolyModelAgg <- lm(aggdata$Out ~ poly(aggdata$Budget, 10, raw=TRUE))
> # LOESS model ( local regression )
> loessModelAgg <- loess(aggdata$Out ~ aggdata$Budget,span=0.65)
> my.count <- seq(from=0, to=(length(aggdata[,1]))-1, by=1)
> pred <- predict(loessModelAgg,my.count,se=TRUE)
>
> # stampa informazioni relative ai modelli, tra cui misure come il coefficiente di determinazione e F-test
> summary(linearModelAgg)
> summary(quadraticModelAgg)
> #  ... 
\end{lstlisting}

Come spiegato nel paragrafo precedente sono stati calcolati anche i residuali per poi analizzarli tramite grafico di dispersione e successivamente è stata verificato se la distribuzione degli errori casuali fosse normale.

\begin{lstlisting}
> # calcolo dei valori residuali
> modelResidLinear=resid(linearModelAgg)
> modelResidQuadratic=resid(quadraticModelAgg)
> #  ...
> # traccia i grafici di dispersione dei residuali per i diversi modelli
> plot(aggdata$Budget,modelResidLinear,type="p",lwd=3,ylab="Residuals", xlab="Budget Fotovoltaico ( milioni di Euro )",xlim=c(0,40),main="Linear Model") 
> plot(aggdata$Budget,modelResidQuadratic,type="p",lwd=3,ylab="Residuals", xlab="Budget Fotovoltaico ( milioni di Euro )",xlim=c(0,40),main="Quadratic Model") 
> #  ...
> # calcolo errori casuali relativi ai modelli
> lmstdres=rstandard(linearModelAgg)
> qmstdres=rstandard(quadraticModelAgg)
> #  ...
> # confronto grafico tra errori casuali e la distribuzione normale
> qqnorm(lmstdres, ylab="Standardized Residuals", xlab="Normal Scores",main="Linear Model") 
> qqline(lmstdres)
> #  ...
\end{lstlisting}

Dopo aver scelto quale sia il modello di regressione che meglio si adatta ai dati, è immediato presentare graficamente i risultati ottenuti.

\begin{lstlisting}
> # traccia il grafico con i dati grezzi
> plot(data$Out ~ data$Budget,type="n",lwd=3,ylab="Produzione Energetica ( kW )", xlab="Budget Fotovoltaico ( milioni di Euro )",cex.lab=0.9,xlim=c(0,30))
> points(data$Out ~ data$Budget,col="blue4",pch=1)
> # traccia il grafico per il modello di regressione scelto; ovviamente consente anche di confrontare tra loro i diversi modelli per avere un riscontro visivo della loro correttezza
> plot(aggdata$Out ~ aggdata$Budget,type="p",lwd=3,ylab="Produzione Energetica ( kW )", xlab="Budget Fotovoltaico ( milioni di Euro )",xlim=c(0,60)) 
> grid(lwd=2)
> # modello lineare
> points(data$Budget, predict(linearModel), type="l", col="red", lwd=2)
> # modello LOESS
> lines(aggdata$Budget,pred$fit, lty="solid", col="darkred", lwd=2)
\end{lstlisting}

\section[ANALISI RISULTATI]{ANALISI RISULTATI SIMULAZIONI}

Dopo aver implementato il simulatore descritto nel precedente capitolo, considerando in particolar modo la versione estesa, siamo passati ad analizzare le relazioni che legano la produzione di energia elettrica alle diverse metodologie di incentivi e relativi fondi stanziati dalla regione. Possiamo subito anticipare che, come era lecito attendersi, la presenza di un qualsiasi tipo di incentivo permette di ottenere una produzione energetica maggiore rispetto al caso di assenza di incentivi regionali e inoltre all'aumentare dei fondi stanziati per finanziare un tipo di incentivo la produzione di energia da impianti fotovoltaici tende ad aumentare.\\*
In un secondo momento siamo passati a studiare la relazione tra produzione energetica e interazione sociale (considerando sia variazioni del raggio che della sensibilità); in modo conforme alle nostre aspettative, anche in questo caso i risultati ottenuti indicano che una maggiore produzione energetica è associata ad un'interazione sociale più intensa.


\subsection{COMPORTAMENTO DEGLI INCENTIVI}

Il comportamento degli incentivi è stato studiato effettuando numerose simulazioni per ogni tipo di incentivo, variando la dimensione del fondo dedicato agli incentivi per il fotovoltaico. Il fondo è stato aumentato con incrementi di un milione di euro a partire da zero  fino a un massimo di 40 milioni (considerando un arco temporale di cinque anni); per ogni valore sono state effettuate 300 simulazioni, per un totale di 48000 simulazioni considerando tutti i diversi incentivi.\\*
Proseguiremo ora esaminando i singoli incentivi per poi concludere confrontandoli tra loro.

\myparagraph{Fondo Asta}

In Figura ~\ref{graphSimA} sono riportati sia i risultati ottenuti da tutte le simulazioni effettuate con l'incentivo Fondo Asta (Fig.~\ref{graphSimA_R}), sia la linea di regressione associata al modello che meglio vi si adattasse (Fig.~\ref{regr_graphSimA_R}), rappresentato in questo caso da una funzione quadratica.

\begin{figure}[H]
	\centering
	\subfloat[Simulazioni]{\includegraphics[scale=0.55]{graphSimA_R}\label{graphSimA_R}}
	\qquad
	\subfloat[Relazione]{\includegraphics[scale=0.55]{regr_graphSimA_R}\label{regr_graphSimA_R}}
	\caption{Fondo Asta}
	\label{graphSimA}
\end{figure}

\begin{figure}[hbt]
	\centering
	\includegraphics[scale=0.8]{residualsPlot_A}
	\caption{Fondo Asta, analisi dei residuali}
	\label{residualsPlot_A}
\end{figure}

Osservando il grafico possiamo notare quindi che la produzione energetica aumenta insieme all'aumentare del budget dedicato all'incentivazione, con una relazione quasi lineare per valori minori di trenta milioni di euro per poi continuare ad aumentare ma ad un ritmo minore; ciò è dovuto probabilmente al fatto che si raggiunge una specie di ''saturazione'' nell'accesso agli incentivi, cioè quando la somma stanziata per il Fonda Asta raggiunge valori sufficienti è possibile fornire la percentuale sul costo di costruzione dell'impianto fotovoltaico a quasi tutti gli agenti che ne fanno richiesta, di conseguenza la produzione energetica inizia a crescere più lentamente.\\* \\*
In questo caso, come anche con gli altri incentivi, la scelta del modello di regressione corretto è stata fatta sulla base dei valori numerici presentati nel paragrafo precedente, cercando comunque di scegliere il modello più semplice a parità di miglioramenti non significativi nelle misure della bontà di adattamento ai dati. Per il questo tipo di incentivo la regressione quadratica ha fornito risultati molto buoni, con un coefficiente di determinazione pari a 0.907 e valori di p-value e F-test che garantiscono la significatività statistica.\\*
In Figura ~\ref{residualsPlot_A} sono mostrati i grafici di dispersione dei residuali (asse della ordinate) rispetto al budget (la variabile indipendente, sull'asse delle ascisse) per alcuni dei modelli tentati; risulta evidente un andamento caratteristico della distribuzione solamente nel caso del modello lineare, mentre nei modelli restanti la distribuzione è molto simile e non strutturata, a indicare una maggiore adesione della regressione ai dati sottostanti.\\* \\*
In Figura ~\ref{normProbPlot_A} abbiamo infine presentato i grafici della distribuzione degli errori (rappresentati dai cerchi neri) di alcuni modelli, per confrontarli con una distribuzione normale standard (la retta). Come è facile osservare per tutti i modelli la distribuzione degli errori casuali segue piuttosto fedelmente una distribuzione normale, consentendoci di considerare vera l'assunzione di normalità; dal momento che questo avviene anche per tutti le restanti metodologie incentivanti, nei casi restanti, per non appesantire la trattazione, non abbiamo riportato ulteriori grafici di quest'ultimo tipo. 

\begin{figure}[H]
	\centering
	\includegraphics[scale=0.8]{normProbPlot_A}
	\caption{Fondo Asta, distribuzione errori}
	\label{normProbPlot_A}
\end{figure}

\myparagraph{Conto Interessi}

L'incentivo Conto Interessi (Fig.~\ref{graphSimCI}) mostra un andamento decisamente diverso rispetto al caso precedente, in quanto dopo una crescita molto veloce la produzione energetica si stabilizza e non aumenta a prescindere da quanto venga speso. \\* 

\begin{figure}[H]
	\centering
	\subfloat[Simulazioni]{\includegraphics[scale=0.55]{graphSimCI_R}\label{graphSimCI_R}}
	\qquad
	\subfloat[Relazione]{\includegraphics[scale=0.55]{regr_graphSimCI_R}\label{regr_graphSimCI_R}}
	\caption{Conto Interessi}
	\label{graphSimCI}
\end{figure}

Questo comportamento è in linea con quanto era lecito attendersi poiché il Conto Interessi è ampiamente la tipologia di incentivazione che richiede meno fondi e una volta che tutti i richiedenti sono stati soddisfatti, per un costo di circa tre milioni di euro, ulteriori aumenti di budget non corrispondono ad aumenti della produzione.

\begin{figure}[hbt]
	\centering
	\includegraphics[scale=0.8]{residualsPlot_CI}
	\caption{Conto Interessi, analisi dei residuali}
	\label{residualsPlot_CI}
\end{figure}

La tipologia di regressione che meglio approssima il comportamento di questo incentivo è costituita da una funzione polinomiale di grado elevato (in particolare il decimo), caratterizzato da un coefficiente di determinazione di 0.901; come in quello precedente in questo caso p-value e F-test garantiscono un'ottima significatività statistica.\\* 
Anche dalla Figura ~\ref{residualsPlot_CI} si può notare come l'analisi grafica dei residuali suggerisca che il metodo di regressione migliore sia quello scelto, infatti nei restanti modelli è facilmente riscontrabile una struttura non casuale.

\myparagraph{Fondo Rotazione}

\begin{figure}[hbt]
	\centering
	\includegraphics[scale=0.8]{residualsPlot_R}
	\caption{Fondo Rotazione, analisi dei residuali}
	\label{residualsPlot_R}
\end{figure}

\begin{figure}[H]
	\centering
	\subfloat[Simulazioni]{\includegraphics[scale=0.55]{graphSimR_R}\label{graphSimR_R}}
	\qquad
	\subfloat[Relazione]{\includegraphics[scale=0.55]{regr_graphSimR_R}\label{regr_graphSimR_R}}
	\caption{Fondo Rotazione}
	\label{graphSimR}
\end{figure}

La situazione nel caso del Fondo Rotazione (Fig.~\ref{graphSimR}) è simile a quella del Fondo Asta, infatti anche in questo caso il modello di regressione migliore è dato da una funzione quadratica (ancora con ottima significatività statistica e un coefficiente di determinazione pari a 0.983), con la differenza che la crescita della produzione in relazione ai finanziamenti resta più marcata anche per valori di budget più elevati e la pendenza della curva diminuisce più lentamente.\\*
Allo stesso modo la Figura ~\ref{residualsPlot_R} mostra come la distribuzione dei residuali per il modello scelto sia indicativa di un ottimo adattamento ai dati.

\myparagraph{Fondo Garanzia}

Infine consideriamo il caso dell'incentivo Fondo Garanzia, mostrato in Figura ~\ref{graphSimG}. Si nota nuovamente un andamento contraddistinto inizialmente da un aumento della produzione energetica al crescere dei fondi messi a disposizione con una curva quasi lineare (fino a circa quindici milioni di euro), per poi osservare una stabilizzazione nella produzione energetica dopo che un certo valore di budget è stato raggiunto (circa venti milioni di euro), probabilmente, in maniera simile al Conto Interessi, anche con questa metodologia incentivante è possibile soddisfare quasi tutti i richiedenti con una spesa sensibilmente minore rispetto ai casi Asta e Rotazione.\\*\\*
Il modello di regressione che è risultato essere il più opportuno con i dati del Fondo Garanzia, è il cosiddetto LOESS (modello locale); per questo tipo di modello non è possibile riportare dati come il coefficiente di determinazione in quanto non produce una funzione di regressione facilmente rappresentabile tramite una formula matematica, ma in ogni caso il confronto con altri tipi di modelli sulla base di analisi di tipo grafico ci ha fatto propendere verso la sua scelta (anche se lo stesso modello polinomiale usato per il Conto Interessi aveva mostrato un buon adattamento ai dati).\\*
In Figura ~\ref{residualsPlot_G} è possibile vedere anche per il Fondo Garanzia quali siano i grafici di dispersione dei residuali per alcuni modelli di regressione, dalla cui osservazione si può dedurre che il modello LOESS sia stato una buona scelta.

\begin{figure}[H]
	\centering
	\subfloat[Simulazioni]{\includegraphics[scale=0.55]{graphSimG_R}\label{graphSimG_R}}
	\qquad
	\subfloat[Relazione]{\includegraphics[scale=0.55]{regr_graphSimG_R}\label{regr_graphSimG_R}}
	\caption{Fondo Garanzia}
	\label{graphSimG}
\end{figure}

\begin{figure}[hbt]
	\centering
	\includegraphics[scale=0.8]{residualsPlot_G}
	\caption{Fondo Garanzia, analisi dei residuali}
	\label{residualsPlot_G}
\end{figure}

\myparagraph{Confronto Incentivi}

Infine in Figura ~\ref{incentCompare} sono confrontati i differenti comportamenti dei vari incentivi. Si nota facilmente come il Conto Interessi sia il tipo di incentivo migliore per quasi tutto l'intervallo considerato per il budget (che possiamo ritenere sensato in quanto compatibile con gli investimenti realmente effettuati dalla regione), leggermente superato dal Fondo Rotazione solamente con un fondo incentivi maggiore di quaranta milioni di euro. Il Fondo Garanzia e il Fondo Rotazione hanno un andamento equiparabile per finanziamenti non elevati, ma il secondo si comporta decisamente meglio in caso di forti investimenti (occorre comunque ricordare che in questo grafico non viene tenuto conto di quanta parte di budget viene effettivamente consumata dall'incentivo, fattore che viene invece considerato nella valutazione dell'efficacia impiegata nel modello a vincoli del problema). Il Fondo Asta risulta essere la metodologia di incentivo meno efficiente per la produzione di energia elettrica.

\begin{figure}[hbt]
	\centering
	\includegraphics[scale=0.65]{incentCompare}
	\caption{Confronto tra i diversi incentivi}
	\label{incentCompare}
\end{figure}

\subsection{EFFETTI DELL'INTERAZIONE SOCIALE}

Gli effetti dell'iterazione sociale sulla produzione energetica sono stati studiati agendo sui due parametri che possono influenzarla, il raggio dell'interazione (misurato in patches) e la sensibilità all'influenza (adimensionale) derivante dal comportamento dei vicini, ed effettuando numerose simulazioni controllate: per ricavare la relazione tra produzione energetica e raggio questo è stato fatto variare da 1 fino a 40 (valori espressi con un'unità di misura interna al simulatore, la patch del mondo virtuale), con incrementi di una un'unità e 200 prove per valore, per un totale di 32000 simulazioni; per la relazione tra produzione e sensibilità questa è stata fatta crescere da 0 fino a 20 a intervalli di 0.5, ancora con un totale di 32000 simulazioni.

\myparagraph{Sensibilità a interazione}

In questo paragrafo verranno mostrate le relazioni che legano la sensibilità degli agenti nei confronti dell'interazione sociale e la produzione energetica, distinguendo i quattro tipi di incentivi. Per ogni tipologia è stata effettuata un'analisi della regressione analoga a quelle effettuate in precedenza e, per maggiore concisione, esaminiamo ora direttamente i risultati ottenuti con il modello di regressione da noi ritenuto migliore, senza riportare i passaggi che ci hanno permesso di stabilire quale fosse (coefficiente di determinazione, analisi grafica dei residuali,...); il modello da noi scelto in quanto miglior approssimazione dei dati è stato quello LOESS, anche se la regressione con una funzione polinomiale di alto grado presentava una bontà di adattamento pressoché identica.\\* \\*
Osserviamo che  la produzione energetica è positivamente correlata all'incremento della sensibilità all'influenza sociale. Si possono notare alcune somiglianze nell'andamento delle curve di regressione dei Fondi Asta e Rotazione (Fig.~\ref{graphAsocS} e Fig.~\ref{graphRsocS}), infatti in entrambi i casi dopo una relazione quasi lineare per valori della sensibilità inferiori a 10, si nota una leggera diminuzione della pendenza della curva. \\*

\begin{figure}[H]
	\centering
	\includegraphics[scale=0.5]{graphSimA_socS}
	\caption{Fondo Asta, sensibilità a influenza sociale}
	\label{graphAsocS}
\end{figure}

\begin{figure}[hbt]
	\centering
	\includegraphics[scale=0.5]{graphSimCI_socS}
	\caption{Conto Interessi, sensibilità a influenza sociale}
	\label{graphCIsocS}
\end{figure}

\begin{figure}[H]
	\centering
	\includegraphics[scale=0.5]{graphSimR_socS}
	\caption{Fondo Rotazione, sensibilità a influenza sociale}
	\label{graphRsocS}
\end{figure}

\begin{figure}[hbt]
	\centering
	\includegraphics[scale=0.5]{graphSimG_socS}
	\caption{Fondo Garanzia, sensibilità a influenza sociale}
	\label{graphGsocS}
\end{figure}

Nel caso del Conto Interessi e Fondo Garanzia (Fig.~\ref{graphCIsocS} e Fig.~\ref{graphGsocS}) la linea di regressione presenta un comportamento più irregolare (specialmente nel caso del Fondo Garanzia) e sembra suggerire che aumentando la sensibilità oltre al valore 20 la corrispondente produzione energetica diminuisca, ma questa inversione di tendenza è apparente e con ulteriori simulazioni abbiamo provato che la funzione continua ad essere crescente, anche se poco e con notevoli oscillazioni, per valori di sensibilità maggiori.

\myparagraph{Raggio dell'interazione}

Come nel caso della sensibilità, anche per valutare la relazione tra produzione energetica e raggio dell'iterazione sociale abbiamo applicato i metodi di analisi della regressione visti in precedenza, quindi anche in questo caso abbiamo riportato direttamente i risultati con il modello migliore; la scelta è ricaduta nuovamente sul modello LOESS, nonostante la buona approssimazione fornita anche dal modello polinomiale (uno dei motivi che ci ha fatto propendere verso la nostra scelta è il fatto che una funzione di regressione definita da un polinomio di grado elevato fosse troppo sensibile alla presenza di valori anomali, outliers, e presentasse quindi un andamento più irregolare).\\* \\*
Per tutti tipi di incentivo possiamo osservare che la produzione energetica è positivamente correlata all'incremento del raggio dell'iterazione sociale, in particolare fino a un certo valore di raggio l'aumento di quest'ultimo provoca un rapido miglioramento della produzione, ma oltre tale valore la produzione incrementa più moderatamente; in ogni caso, pur con le dovute differenze tra le pendenze, le curve dei vari incentivi indicano che l'interazione sociale è influenzata maggiormente dall'aumento del raggio piuttosto che dalla sensibilità ad essa, anche se entrambi i fattori concorrono ad aumentare l'effetto positivo dell'interazione stessa (può essere utile anche notare che un un valore di 40 patches per il raggio in un mondo virtuale a forma di quadrato di lato 60 patches è chiaramente un valore non riscontrabile nel mondo reale).
%-------------> raggio iterazion 40 è grande

\begin{figure}[H]
	\centering
	\includegraphics[scale=0.5]{graphSimA_socR}
	\caption{Fondo Asta, sensibilità a raggio interazione}
	\label{graphAsocR}
\end{figure}

\begin{figure}[hbt]
	\centering
	\includegraphics[scale=0.5]{graphSimCI_socR}
	\caption{Conto Interessi, sensibilità a raggio interazione}
	\label{graphCIsocR}
\end{figure}

\begin{figure}[H]
	\centering
	\includegraphics[scale=0.5]{graphSimR_socR}
	\caption{Fondo Rotazione, sensibilità a raggio interazione}
	\label{graphRsocR}
\end{figure}

\begin{figure}[hbt]
	\centering
	\includegraphics[scale=0.5]{graphSimG_socR}
	\caption{Fondo Garanzia, sensibilità a raggio interazione}
	\label{graphGsocR}
\end{figure}


\nocite{*}
\bibliographystyle{plain}
\bibliography{bibliography}

\end{document}


% descrizione dell'ottimizzazione
%Andrea Borghesi
%Università degli studi di Bologna

%capitolo dedicato alla descrizione (breve) dell'ottimizzatore

%\documentclass[12pt,a4paper,openright,twoside]{report}
%\usepackage[italian]{babel}
%\usepackage{indentfirst}
%\usepackage[utf8]{inputenc}
%\usepackage[T1]{fontenc}
%\usepackage{fancyhdr}
%\usepackage{graphicx}
%\usepackage{titlesec,blindtext, color}
%\usepackage[font={small,it}]{caption}
%\usepackage{subfig}
%\usepackage{listings}
%\usepackage{color}
%\usepackage{url}
%\usepackage{textcomp}
%\usepackage{eurosym}
%\usepackage{amsmath}

%%impostazioni generali per visualizzare codice
%\definecolor{dkgreen}{rgb}{0,0.6,0}
%\definecolor{gray}{rgb}{0.5,0.5,0.5}
%\definecolor{mauve}{rgb}{0.58,0,0.82}
% 
%\lstset{ %
%  basicstyle=\footnotesize,           % the size of the fonts that are used for the code
%  backgroundcolor=\color{white},      % choose the background color. You must add \usepackage{color}
%  numbers=left,                   % where to put the line-numbers
%  numberstyle=\tiny\color{gray},  % the style that is used for the line-numbers
%  numbersep=5pt,  
%  showspaces=false,               % show spaces adding particular underscores
%  showstringspaces=false,         % underline spaces within strings
%  showtabs=false,                 % show tabs within strings adding particular underscores
%  rulecolor=\color{black}, 
%  tabsize=2,                      % sets default tabsize to 2 spaces
%  breaklines=true,                % sets automatic line breaking
%  breakatwhitespace=false,        % sets if automatic breaks should only happen at whitespace
%  title=\lstname,                   % show the filename of files included with \lstinputlisting;
%  frame=single,                   % adds a frame around the code
%                                  % also try caption instead of title
%  keywordstyle=\color{blue},          % keyword style
%  commentstyle=\color{dkgreen},       % comment style
%  stringstyle=\color{mauve},         % string literal style
%  escapeinside={\%*}{*)},            % if you want to add LaTeX within your code
%  morekeywords={*,...},              % if you want to add more keywords to the set
%  deletekeywords={...}              % if you want to delete keywords from the given language
%}

%%per avere un bordo intorno alle figure
%\usepackage{float}
%\floatstyle{boxed} 
%\restylefloat{figure}

%%per poter poi impedire che certe parole vadano a capo
%\usepackage{hyphenat}
%\usepackage{listings}

%%ridefinisco font per fancyhdr, per ottenere un'intestazione pulita
%\newcommand{\changefont}{ \fontsize{9}{11}\selectfont }
%\fancyhf{}
%\fancyhead[LE,RO]{\changefont \slshape \rightmark} 	%section
%\fancyhead[RE,LO]{\changefont \slshape \leftmark}	%chapter
%\fancyfoot[C]{\changefont \thepage}					%footer

%%titolo capitolo con "numero | titolo"
%\definecolor{gray75}{gray}{0.75}
%\newcommand{\hsp}{\hspace{20pt}}
%\titleformat{\chapter}[hang]{\Huge\bfseries}{\thechapter\hsp\textcolor{gray75}{|}\hsp}{0pt}{\Huge\bfseries}


%\oddsidemargin=30pt \evensidemargin=20pt

%%sillabazioni non eseguite correttamente
%\hyphenation{sil-la-ba-zio-ne pa-ren-te-si si-mu-la-to-re ge-ne-ra-re pia-no}

%%interlinea
%\linespread{1.15}  
%\pagestyle{fancy}

%%cartelle contenenti le immagini
%\graphicspath{{/media/sda4/tesi/immagini/grafici/}{/media/sda4/tesi/immagini/grafici/incCompare/}{/media/sda4/tesi/immagini/grafici/rawData/}{/media/sda4/tesi/immagini/grafici/regressionAnalysis/}{/media/sda4/tesi/immagini/schemi/}{/media/sda4/tesi/immagini/simulazione/}{/media/sda4/tesi/immagini/epolicy/}{/media/sda4/tesi/immagini/ottimizzazione/}}

%%in modo che dopo il titolo di un paragrafo il testo vada a capo
%\newcommand{\myparagraph}[1]{\paragraph{#1}\mbox{}\\}

%%per scrivere bene CLP(R) e CLP(FD)
%\newcommand{\clpr}{CLP({\ensuremath{\cal R}})}
%\newcommand{\clpfd}{CLP({\ensuremath{\cal FD}})}

%\begin{document}
\clearpage{\pagestyle{empty}\cleardoublepage}
\chapter{\nohyphens{Ottimizzazione}}
In aggiunta al modello di simulazione ad agenti, con la relativa prospettiva individuale, il progetto ePolicy ritiene necessario considerare anche una prospettiva globale (regionale nel caso della Regione), in grado di affrontare il problema della pianificazione e dalla valutazione degli impatti dei piani da un punto di vista più ampio, mantenendo al tempo stesso una stretta integrazione con il livello individuale.

La pianificazione delle attività regionale può essere vista come un complesso problema di ottimizzazione combinatoria; i decisori politici devono prendere decisioni e soddisfare un insieme di vincoli, tentando al tempo stesso di realizzare un certo numero di obiettivi, come ad esempio ridurre gli effetti negativi e incrementare i positivi su ambiente, società ed economia. La fase di valutazione - valutare quali siano gli impatti delle politiche scelte sull'ambiente ed in misura minora in ambito economico e sociale - è ora in genere effettuata in sequenza dopo la creazione di un piano, con lo svantaggio che se questo contenesse impatti negativi sull'ambiente potrebbero venire applicate solo delle contromisure correttive; per evitare ciò, nell'approccio di ePolicy la valutazione e la pianificazione sono condotte allo stesso tempo.\\*\\*
Per la valutazione ambientale sono stati proposti diversi metodi: un modello probabilistico \cite{logicDSSstrategicAss}, un modello fuzzy (la logica fuzzy prevede che si possa attribuire a una proposizione un grado di verità compreso tra 0 e 1) \cite{fuzzyLogicstrategicAss} e un modello lineare a vincoli (\emph{Constraint Logic Programming}, programmazione logica a vincoli, chiamata in seguito anche CLP) \cite{GavanelliEtAl}. Il motivo per sperimentare diversi tipi di modello è che le matrici usate dagli esperti ambientali si prestano a differenti interpretazioni, quindi era importante capire quale fosse la migliore scelta possibile. Il modello CLP è risultato essere il più veloce - a livello computazionale - in quanto per la programmazione lineare esistono tecniche di risoluzione molto efficienti. In secondo luogo, questo modello può essere facilmente esteso aggiungendo nuovi vincoli, per risolvere nuovi tipi di problemi; ad esempio se le attività da pianificare fossero variabili decisionali (invece che valori fissi) potremmo svolgere la pianificazione contemporaneamente alla valutazione ambientale. Dal momento che questo era uno degli scopi del progetto si è scelto di utilizzare l'approccio CLP.\\*\\*
In questo capitolo introdurremo molto brevemente la programmazione logica a vincoli, con una breve panoramica e citando gli strumenti software di cui ci siamo serviti, passeremo poi a presentare il modello sviluppato che incorpora al suo interno le attività di pianificazione e valutazione e infine mostreremo i risultati ottenuti applicando il modello al caso di studio scelto, ovvero il piano energetico 2011-2013 per la regione Emilia-Romagna.

\section[CLP]{\nohyphens{Programmazione logica a vincoli}}
Come abbiamo già ampiamente spiegato in precedenza il problema della creazione di un piano regionale può essere considerato come un problema caratterizzato da un insieme di vincoli e una funzione obiettivo. Nell'ambito dell'Intelligenza Artificiale i problemi per cui è richiesto soddisfare un insieme di vincoli sono definiti come \emph{Problemi di Soddisfacimento di Vincoli} (in inglese Constraint Satisfaction Problem, da cui l'acronimo \emph{CSP}). Un CSP è definito da una terna $<X,D,C>$, dove $X$ è un insieme di variabili $X=\{X_1,X_2...X_n\}$, $D$ è un dominio discreto per ogni variabile $D=\{D_1,D_2...D_n\}$ e $C$ è l'insieme di vincoli - un vincolo è una relazione tra variabili che definisce un sottoinsieme del prodotto cartesiano dei domini $D_1 \times D_2 \times ... \times D_n$; con queste premesse un la soluzione di un problema di soddisfacimento di vincoli è data da un assegnamento di valori alle variabili consistente con i vincoli \cite{cspFoundations}. Analogamente, un problema di ottimizzazione con vincoli (Constraint Optimization Problem, \emph{COP}) è definito da $<X,D,C,f>$, cioè un CSP più una \emph{funzione obiettivo} $f(X_1,X_2...X_n)$, la cui soluzione è un assegnamento di valori alle variabili compatibile con i vincoli del problema che ottimizza la funzione obiettivo. Le metodologie risolutive impiegate per risolvere problemi a vincoli attingono in buona parte alle tecniche di Ricerca Operativa e Intelligenza Artificiale, in particolare noi considereremo la \emph{Programmazione Logica a Vincoli}.\\*\\*
La programmazione logica a vincoli \cite{clpSurvey} (in inglese \emph{Constraint Logic Programming} da cui CLP) è una classe di linguaggi di programmazione che estendono la classica Programmazione Logica - il paradigma di programmazione basato sulla logica del primo ordine implementato da linguaggi come ad esempio Prolog \cite{Colmerauer,Kowalski,clocksin2003programming}, sviluppato nei primi anni settanta e ampiamente diffuso ancora oggi. Alle variabili possono essere assegnati sia termini (i tipi di dato e le strutture riconosciute in Prolog) sia valori interpretati, appartenenti a una determinata \emph{classe}, un parametro caratteristico dello specifico linguaggio CLP; per esempio è possibile avere \clpr \cite{clpR}, in grado di operare sui valori reali, oppure \clpfd, in cui le variabili appartengono a domini finiti. All'interno di una classe sono definite le funzioni interpretate (che possono essere nei domini numerici i soliti operatori $+$, $-$, $\times$, etc.) e i predicati (ad esempio, $<$, $\neq$, $\geq$, etc.), che sono chiamati \emph{vincoli}. La semantica dichiarativa consente l'interpretazione intuitiva per i vincoli e i termini, relativamente al dominio considerato: ad esempio, $1.3+2<5$ è \emph{vero} in \clpr; ciò è un'estensione molto significativa rispetto alla programmazione logica standard in quanto i linguaggi logici operano in domini non interpretati (``Universo di Herbrand'') e quindi le relazioni tra variabili possono essere solamente verificate a posteriori e non trattate come vincoli veri propri. La semantica operativa somiglia a quella di Prolog per atomi costruiti sui predicati usuali - quelli definiti da un insieme di clausole - ma conserva quelli da interpretare, i vincoli, in una struttura dati speciale, chiamata \emph{constraint store}; essa è in seguito interpretato e modificato da un meccanismo esterno, il \emph{constraint solver}, il risolutore dei vincoli. Il risolutore è in grado di controllare se una combinazione di vincoli è soddisfacibile o meno, e può modificare lo store, sperabilmente per semplificarne lo stato. Generalmente il solver non effettua una propagazione \emph{completa}: se una valutazione ha come risultati \emph{falso}, allora sicuramente la soluzione è impossibile, anche se in certi casi può capitare che il risolutore non rilevi l'impossibilità di un problema anche se non esistono soluzioni.\\*\\*  
\clpr è una classe di programmazione logica a vincoli in cui le variabili appartengo all'insieme dei numeri reali, i vincoli disponibili sono uguaglianze e disuguaglianze lineari e generalmente il risolutore è implementato tramite l'algoritmo del simplesso \cite{dantzig51}, molto veloce e completo per (dis)equazioni lineari (cioè è sempre in grado di restituire \emph{vero} o \emph{falso}). In alcuni sistemi, grazie alla disponibilità di risolutori efficienti per la programmazione lineare intera, certi vincoli non lineari sono accettati nel linguaggio, in particolare è possibile imporre che certe variabili assumano esclusivamente valori interi; in casi come questo la complessità del problema passa da P a NP-hard e il risolutore deve spesso ricorrere a tecniche di branch-and-bound. Ad ogni modo, l'utente può specificare anche una funzione obiettivo, un termine lineare che dovrebbe essere massimizzato o minimizzato garantendo al tempo stesso che tutti vincoli siano soddisfatti.


\section{Strumenti}
Al giorno d'oggi esistono numerose implementazioni di \clpr \cite{inclpR} e diverse versioni di Prolog dispongono di una propria libreria per \clpr. Per il progetto ePolicy si è scelto di adottare il software open source  ECL$^i$PS$^e$ \cite{clpEclipse,fromLPtoCLPeclipse}.

\subsection{ECL$^i$PS$^e$}
ECL$^i$PS$^e$ è un sistema software per lo sviluppo di applicazioni di programmazione a vincoli e indicata per lo studio di aspetti relativi alla risoluzione di problemi combinatori, come appunto la programmazione vincolata, modellazione di problemi, programmazione matematica, tecniche di ricerca di soluzioni, etc. Al suo interno sono contenuti librerie per risolutori a vincoli, un linguaggio di alto livello (derivato da Prolog), interfacce per risolutori esterni e altre funzionalità. In Figura~\ref{eclipseUI} mostriamo come si presenta l'interfaccia utente di ECL$^i$PS$^e$.

\begin{figure}[h]
	\centering
	\includegraphics[scale=0.28]{eclipseUI}
	\caption{ECL$^i$PS$^e$, Interfaccia Utente}
	\label{eclipseUI}
\end{figure}

Tra le diverse librerie disponibili, ne esiste una denominata \emph{Eplex} \cite{eplex} che interfaccia ECL$^i$PS$^e$ a un risolutore lineare intero, il quale può essere sia uno strumento commerciale, come CPLEX o Xpress-MP, che open source. A default Eplex nasconde molti dei dettagli del risolutore, ma nondimeno, quando richiesto, l'utente può regolare diversi parametri per migliorare le prestazioni ed esaminare lo stato interno del solver. Nell'ambito del progetto ePolicy ci siamo serviti di questa libreria, insieme alle altre funzionalità offerte da ECL$^i$PS$^e$, per modellare e risolvere i problemi relativi alla pianificazione energetica regionale.

Illustreremo ora due esempi di modellazione di problemi a vincoli sfruttando il linguaggio ECL$^i$PS$^e$ (nel primo caso considerando domini finiti e nel secondo valori reali, avvalendoci anche della libreria Eplex), anche per mostrare come possono essere strutturati i problemi di programmazione logica a vincoli; in questa trattazione supporremo noti i concetti elementari della programmazione logica (procedimenti risolutivi, definizioni di un termine, etc.), la cui discussione esula da questo lavoro.

\myparagraph{Esempio \clpfd}
Il cosiddetto \emph{Send More Money} puzzle è un esempio classico di programmazione a vincoli; le variabili $[S,E,N,D,M,O,R,Y]$ rappresentano cifre da 0 a 9 e lo scopo è assegnare alle variabili valori diversi in modo che l'operazione aritmetica di Figura~\ref{SendMoreMoney} risulti corretta - inoltre i numeri devono essere ben formati, da cui $S>0$ e $M>0$. 

\begin{figure}[h]
	\centering
	\includegraphics[width=0.215\textwidth]{sendMoreMoney}
	\caption{Send More Money puzzle}
	\label{SendMoreMoney}
\end{figure}

Con la programmazione convenzionale si avrebbe necessità di esprimere una strategia di ricerca in modo esplicito (senza contare possibili ottimizzazioni come cicli innestati), mentre con linguaggi logici come Prolog verrebbe sfruttata la ricerca fornita dal risolutore interno (il motore inferenziale), con il vantaggio di una programmazione estremamente facilitata ma col rischio di un'efficienza non elevata - a meno di programmi ottimizzati, i quali richiederebbero comunque maggiori tempo e abilità. 

Questo è in effetti il campo di applicazione ideale della programmazione logica a vincoli, in particolare nell'ambito  dei domini finiti \clpfd: le variabili possono assumere valori appartenenti ad un insieme finito di numeri interi, i vincoli sono facilmente esprimibili formalmente e occorre effettuare una certa quantità di ricerca nello spazio delle soluzioni. In questo problema sarebbe naturale usare le variabili del programma per rappresentare le diverse cifre e la soluzione finale dovrà essere un assegnamento di un valore unico per ogni variabile. 

Risolvere questo problema con Prolog comporta l'utilizzo della strategia di ricerca chiamata \emph{Generate and Test}, che prevede che prima la generazione di una soluzione e poi la verifica della consistenza dei vincoli e, nel caso che questa dia esito negativo, l'assegnamento di nuovi valori alle variabili seguita da nuova verifica e così via. In questo modo l'esplorazione dello spazio delle soluzioni è chiaramente inefficiente - per esempio la possibile implementazione in Prolog mostrata qui sotto, per quanto suscettibile a miglioramenti, deve gestire $\frac{10!}{2}$ possibili assegnamenti di valori alle variabili. 

\lstset{language=Prolog}
\begin{lstlisting}
% Send More Money puzzle in Prolog
smm :-
        X = [S,E,N,D,M,O,R,Y],           % variabili
        Digits = [0,1,2,3,4,5,6,7,8,9],	 % domini
        
        % predicato che assegna una soluzione
        assign_digits(X, Digits),
       	
       	%  verifica dei vincoli vincoli
        M > 0, 
        S > 0,
                  1000*S + 100*E + 10*N + D +
                  1000*M + 100*O + 10*R + E =:=
        10000*M + 1000*O + 100*N + 10*E + Y,
        write(X).

select(X, [X|R], R).
select(X, [Y|Xs], [Y|Ys]):- select(X, Xs, Ys).

assign_digits([], _List).
assign_digits([D|Ds], List):-
        select(D, List, NewList),
        assign_digits(Ds, NewList).
\end{lstlisting}

L'implementazione realizzata con ECL$^i$PS$^e$ presenta i vantaggi di semplificare ulteriormente la modellazione del problema e di appoggiarsi all'efficiente risolutore interno per l'esplorazione dello spazio delle soluzioni, in modo particolare il fatto che ogni volta che una variabile viene istanziata i vincoli vengono propagati per eliminare a priori strade inconsistenti, riducendo gli spazi delle soluzioni e prevenendo fallimenti sicuri. 

\begin{lstlisting}
% Send More Money puzzle in ECLiPSe
smm :-
     X = [S,E,N,D,M,O,R,Y],		% variabili
     X :: [0 .. 9],				% domini finiti
     
     % vincoli
     M #> 0,
     S #> 0,
               1000*S + 100*E + 10*N + D +
               1000*M + 100*O + 10*R + E #=
     10000*M + 1000*O + 100*N + 10*E + Y,
     alldistinct(X),
     
     % ricerca della soluzione
     labeling(X),
     write(X).

\end{lstlisting}

\myparagraph{Esempio \clpr - Eplex}

Presentiamo ora un esempio di un problema (Fig.~\ref{clpR_example}) che rientra nell'ambito dei \clpr e che fa uso della libreria Eplex, tratto dal manuale di ECL$^i$PS$^e$ \cite{eclipseTut}. Ci sono tre impianti, o fabbriche, (1-3) in grado di produrre un certo prodotto con capacità diverse e i cui prodotti devono essere trasportati a quattro clienti (A-D) con quantità richieste diverse; anche il costo unitario di trasporto ai clienti è variabile. L'obiettivo del problema è minimizzare i costi di trasporto soddisfacendo le esigenze dei clienti. 

\begin{figure}[h]
	\centering
	\includegraphics[scale=0.7]{clpR_example}
	\caption{Esempio di un problema \clpr. Fonte \cite{eclipseTut}}
	\label{clpR_example}
\end{figure}

Per formulare il problema definiamo la quantità di prodotto trasportata dall'impianto $N$ al cliente $p$ come variabile $N_p$ - ad esempio $A_1$ rappresenta il costo di trasporto dalla fabbrica $A$ al cliente $1$. I vincoli da considerare sono di due tipi (sempre facendo riferimento alla Figura~\ref{clpR_example}):
\begin{itemize}
\item La quantità di prodotto consegnata da tutti gli impianti a un cliente deve deve essere uguale alla domanda del cliente, ad esempio per il cliente $A$ che può essere rifornito dagli impianti 1-3, abbiamo che $A_1+A_2+A_3=21$
\item La quantità di prodotto in uscita da una fabbrica non può essere superiore alla sua capacità produttiva, ad esempio per l'impianto $1$ che invia prodotti ai clienti A-D si ha che $A_1+B_1+C_1+D_1 \leq 50$
\end{itemize}   
Poiché lo scopo è minimizzare i costi di trasporto, la funzione obiettivo è di minimizzare i costi combinati del trasporto dei prodotti dai tre impianti a tutti e quattro i clienti.

La formulazione del problema è quindi la seguente.\\*
Funzione obiettivo: 
\begin{equation}  \label{exampleClpRObiett}
	\min(10A_1+7A_2+200A_3+8B_1+5B_2+10B_3+5C_2+5C_2+8C_3+9D_1+3D_2+7D_3)
\end{equation}
Vincoli:
\begin{equation}  \label{exampleCons1}
	A_1+A_2+A_3=21
\end{equation}
\begin{equation}  \label{exampleCons2}
	B_1+B_2+B_3=40
\end{equation}
\begin{equation}  \label{exampleCons3}
	C_1+C_2+C_3=34
\end{equation}
\begin{equation}  \label{exampleCons4}
	D_1+D_2+D_3=10
\end{equation}
\begin{equation}  \label{exampleCons5}
	A_1+B_1+C_1+D_1 \leq 50
\end{equation}
\begin{equation}  \label{exampleCons6}
	A_2+B_2+C_2+D_2 \leq 30
\end{equation}
\begin{equation}  \label{exampleCons7}
	A_3+B_3+C_3+D_3 \leq 40
\end{equation}

Mostriamo ora come questo problema venga modellato sfruttando la libreria Eplex. In primo luogo occorre caricare la libreria Eplex di cui si dispone (in questo caso abbiamo sfruttato un risolutore esterno open source) e ottenerne un'\emph{istanza}, la quale rappresenta un singolo problema sotto forma di modulo, a cui possono essere riferiti vincoli e funzione obiettivo consentendo quindi al solver esterno di risolvere il problema. Il codice che segue mostra come il problema di Figura~\ref{clpR_example} sia stato trasposto all'interno di ECL$^i$PS$^e$.

\begin{lstlisting}
:- lib(eplex).		% caricamento della libreria Eplex
:- eplex_instance(prob).		% definizione dell'istanza - chiamata 'prob'

main(Cost, Vars) :-
		% dichiarazione delle variabili e definizione del loro dominio		
		Vars = [A1,A2,A3,B1,B2,B3,C1,C2,C3,D1,D2,D3],
		prob: (Vars $:: 0.0..1.0Inf),  % valori maggiori o uguali a 0
		
		% definizione dei vincoli applicati all'istanza eplex
		prob: (A1 + A2 + A3 $= 21), 
		prob: (B1 + B2 + B3 $= 40),
		prob: (C1 + C2 + C3 $= 34),
		prob: (D1 + D2 + D3 $= 10),

		prob: (A1 + B1 + C1 + D1 $=< 50),
		prob: (A2 + B2 + C2 + D2 $=< 30),
		prob: (A3 + B3 + C3 + D3 $=< 40),

		% inizializza il solver esterno con la funzione obiettivo
		prob: eplex_solver_setup(min(10*A1 + 7*A2 + 200*A3 + 
			8*B1 + 5*B2 + 10*B3 +
		 	5*C1 + 5*C2 + 8*C3 +
		 	9*D1 + 3*D2 + 7*D3)),

		% ---------- Fine Modellazione ----------

		% risoluzione del problema
		prob: eplex_solve(Cost).
\end{lstlisting}

Per usare un'istanza Eplex occorre prima dichiararla con \emph{eplex\_instance/1}; una volta dichiarata, l'istanza viene riferita tramite il nome specificato. 

Come primo passo creiamo le variabili del problema e imponiamo che possano assumere solamente valori non negativi e rendiamo noti all'istanza il loro dominio (\emph{\$::/2}). Successivamente imponiamo i vincoli che modellano il problema sotto forma di uguaglianze e disuguaglianze aritmetiche; per via del solver esterno scelto, gli unici tipi di vincoli accettati sono quelli lineari - che ovviamente consentono una maggiore efficienza nella risoluzione.

Occorre poi inizializzare il risolutore esterno con l'istanza eplex creata, in modo che questa possa essere risolta. Questo è fatto dal \emph{eplex\_solver\_setup/1}, che prende come argomento la funzione obiettivo, la quale può essere di minimizzazione o massimizzazione. Infine è possibile risolvere il problema modellato attraverso \emph{eplex\_solve/1}.

Quando un'istanza viene risolta, il solver prende in considerazione tutti i vincoli ad essa relativi, i valori che le variabili del problema possono assumere e la funzione obiettivo specificata. In questo caso è possibile ottenere una soluzione ottimale pari a 710.0: 
\begin{lstlisting}
?-	main(Cost, Vars).

Cost = 710.0
Vars = [A1{0.0 .. 1e+20 @ 0.0}, A2{0.0 .. 1e+20 @ 21.0}, ....]
\end{lstlisting}

\section{Modellazione problema}

Passeremo ora ad occuparci dell'implementazione dell'ottimizzatore impiegato nell'ambito del progetto ePolicy per fornire supporto alle decisioni per lo sviluppo del piano energetico regionale.

In Figura~\ref{ioDSS} è presentata una visione generale e di alto livello dei fattori che entrano in gioco durante la creazione la fase di pianificazione, come i costi e la capacità produttiva degli impianti, gli impatti economici e ambientali delle azioni intraprese, i vincoli di budget e gli obiettivi di produzione energetica, etc. (benché non siano oggetto di discussione in questo capitolo sono inclusi per completezza nella figura anche gli impatti sociali e le modalità di incentivazione, gestite rispettivamente dal componente di Opinion Mining e dal simulatore).

\begin{figure}[h]
	\centering
	\includegraphics[scale=0.45]{ioDSS}
	\caption{Panoramica di Input e Output del DSS di ePolicy}
	\label{ioDSS}
\end{figure}

\subsection[Approccio a vincoli]{\nohyphens{Perché un approccio basato sui vincoli}}
L'attività di pianificazione regionale è al momento svolta da esperti umani che costruiscono un singolo piano, considerando gli obbiettivi strategici regionali che seguono le direttive nazionali ed europee. Dopo che il piano è stato ideato l'ente per la protezione ambientale è chiamata a valutarne la sostenibilità dal punto di vista ambientale. In genere non c'è nessuna retroazione, la valutazione può solamente stabilire se il piano sia ecocompatibile o meno ma senza poterlo per modificare; in rari casi può proporre alcune misure correttive, le quali possono però solamente mitigare gli effetti negativi di decisioni di pianificazione sbagliate.

Oltre a ciò, sebbene le normative prevedano che una valutazione ambientale significativa debba confrontare due o più opzioni (piani differenti), questo è fatto raramente in Europa poiché la valutazione è tipicamente fatta a mano e richiede un lungo lavoro; anche nei pochi casi in cui due opzioni vengano considerate, solitamente una è il piano e l'altra è l'assenza di pianificazione.\\*\\*
La modellazione a vincoli supera le limitazioni dei processi manuali per diversi motivi. In primo luogo, essa fornisce uno strumento che automaticamente prende decisioni di pianificazione, tenendo in considerazione il budget allocato sulla base sia del piano operativo regionale che delle linee guida nazionali/europee.

Secondo, gli aspetti ambientali sono considerati durante la costruzione del piano, evitando di procedere per tentativi ed errori.

Come terza ragione, il ragionamento con i vincoli è uno strumento potente nelle mani di un decisore politico in quanto la generazione di scenari alternativi è estremamente semplificata ed il confronto e valutazione seguono naturalmente. Nel caso in cui i risultati non soddisfino coloro che stabiliscono le politiche o gli esperti ambientali gli aggiustamenti possono essere introdotti molto facilmente all'interno del modello; ad esempio, nel settore della pianificazione energetica regionale, cambiando i limiti della quantità di energia che ogni fonte può fornire, possiamo correggere il piano considerando l'andamento del mercato e anche la potenziale ricettività della regione.

\subsection{Modello CLP}
Il Piano Regionale è il risultato della principale attività di definizione delle politiche in cui le regioni europee siano coinvolte. Ogni regione dispone di un budget distribuito sulla base del Programma Operativo, il quale specifica le priorità di ogni regione per l'assegnamento dei fondi (nel campo dell'energia una esempio di priorità è l'incremento della produzione da fonti rinnovabili).

Partendo dalla struttura per il sistema di supporto alle decisioni descritta nel primo capitolo, ci accingiamo ora a illustrare il il modello per l'ottimizzazione globale ideato per prendere in considerazione la prospettiva regionale. 
\\* \\*
Nella fase di pianificazione occorre decidere quali attività debbano essere svolte, distinguendo a grandi linee sei tipologie: infrastrutture e impianti, edifici e trasformazioni sul territorio, estrazione delle risorse, modifiche del regime idrico, trasformazioni industriali, gestione dell'ambiente. Per ogni attività dovrebbe essere inoltre deciso un ordine di grandezza che descriva in quale quantità un'attività venga portata a termine.

Abbiamo quindi $N_a$ attività rappresentate con $a_i (i=1..N_a)$ e distinguiamo tra \emph{attività primarie} e \emph{attività secondarie}, nel caso del piano energetico esse sono rispettivamente quelle che producono direttamente energia e quelle che supportano le prime fornendo le necessarie infrastrutture. Possiamo quindi immaginare i seguenti tipi di input, collegati alla nozione di attività:
\begin{enumerate}
\item lista delle \emph{attività primarie} (impianti a biomassa, impianti fotovoltaici, impianti eolici, etc.) e \emph{quantità minime/massime} per ogni attività (es., aumento del 10\% la produzione di energia fotovoltaica);
\item \emph{funzione di costo}: consideriamo un vettore di costi $C=(c_1,...,c_{N_a})$ dove ogni elemento è associato ad una specifica attività e rappresenta il costo unitario della stessa - sostanzialmente si tratta di determinare il costo per MWatt di energia prodotta in base all'attività\footnote{Il costo di un impianto dipende principalmente dalla potenza installata: il costo di installazione di un impianto solare dipende dalla dimensione in metri quadri dei pannelli installati, da cui deriva di conseguenza una certa potenza massima. Notare come il costo considerato sia il costo totale per il sistema regionale, che non è lo stesso costo sostenuto dai contribuenti della regione Emilia-Romagna, poiché la regione può attuare le proprie politiche in diversi modi, convincendo i privati a investire nella produzione energetica. Ciò può essere fatto tramite una leva finanziaria oppure garantendo condizioni favorevoli (economiche i di altro tipo) agli investitori. Alcune fonti energetiche sono economicamente redditizie e quindi non richiedono sussidi da parte della regione. Ad esempio, le centrali a biomassa in Italia al giorno d'oggi sono economicamente vantaggiose per gli investitori, quindi i privati stanno proponendo diversi progetti per la costruzione di tali impianti; d'altra parte le biomasse producono anche inquinanti, non sono sempre sostenibili (vedere \cite{Cattafi} per una discussione ulteriore), quindi è probabile la nascita di comitati locali contrari alla costruzione di nuove centrali. Per queste ragioni, c'è un limite al numero di licenze che la regione concede agli investitori privati per la realizzazione di impianti a biomassa};
\item \emph{funzione di efficienza}: tramite un vettore per le produzioni energetiche $Out=(out_1,...,out_{N_a})$ determiniamo l'energia generata per ogni MWatt di potenza installata di ogni attività - questa funzione dipende dalla località geografica (ad esempio nell'Italia meridionale per via del clima soleggiato l'efficienza del fotovoltaico è maggiore rispetto alle aree settentrionali); 
\item \emph{la matrice delle attività primarie-secondarie} $D$, usata per stabilire quali attività secondarie e in quale misura sono necessarie per realizzare un determinato ammontare di attività primaria, ovvero una matrice di dipendenze tra le attività. 
\end{enumerate}
Ogni regione ha i propri obiettivi, alcuni intrinsecamente determinati dalla regione stessa altri conseguenze di decisioni politiche.
\begin{enumerate}
\item \emph{produzione richiesta/attesa}: in un piano energetico la produzione energetica attesa all'interno  dei confini regionali è basata sulle stime dei consumi energetici;
\item \emph{vincoli di budget}: ogni regione dispone di una certa quantità di fondi da destinare a incentivi con i quali indirizzare il mercato dell'energia nella direzione desiderata - dato un budget disponibile per il piano, questo è vincolato in termini di costo totale (questo vincolo può essere applicato sia al piano complessivo sia a parti di esso);
\item \emph{obiettivi politici}: sono forniti dai decisori politici, come ad esempio la priorità di conformarsi alla linea guida europea come l'iniziativa 20-20-20\footnote{L'iniziativa 20-20-20 mira a raggiungere tre obiettivi entro il 2020: ridurre del 20\% l'emissione di gas serra, produrre il 20\% dell'energia consumata attraverso fonti rinnovabili e migliorare del 20\% l'efficienza energetica.}, e possono essere traslati in vincoli che specificano la minima quantità di energia prodotta da fonti rinnovabili.
\end{enumerate} 

Infine occorre tenere in considerazione gli input relativi alla configurazione geofisica della regione e alla buona norma.
\begin{enumerate}
\item \emph{diversificazione delle fonti energetiche}: l'allocazione dei fondi non dovrebbe essere diretta verso un'unica fonte di energia ma coprire diverse risorse energetiche. Questo requisito è considerato come buona norma e supportato da numerose considerazioni, per esempio per assicurare ``robustezza'' nei confronti di fluttuazioni di prezzo e disponibilità delle varie risorse - queste indicazioni possono essere implementate sotto forma di vincoli che impongano che percentuali minime della produzione totale siano soddisfatte da ogni fonte energetica;
\item \emph{limiti sociali e geografici}: ogni regione ha le sue caratteristiche geofisiche (ad esempio, l'essere o meno ventose), che possono porre limiti alla quantità massima di energia generate da determinate risorse - le centrali idroelettriche possono essere costruite con un'attenta valutazione degli impatti ambientali che possono avere sul territorio, come, ovviamente, l'allagamento di vaste aree.
\end{enumerate} 

Un altro aspetto che i decisori politici devono considerare sono le ripercussioni su ambiente, società ed economia. Come già descritto nel primo capitolo uno degli strumenti per effettuare questa valutazione in Emilia-Romagna prevede l'impiego di matrici coassiali che definiscono le relazioni tra le attività primarie di un piano e gli impatti ambientali (o pressioni) positivi o negativi e in che modo tali attività influiscano sui recettori ambientali (anche essi introdotti in precedenza)\footnote{La matrice correntemente utilizzata per la valutazione ambientale contiene 93 attività, 29 impatti negativi, 19 positivi e 23 recettori e valuta 11 tipi di piano. Queste matrici prendono in esame anche in che modo le attività secondarie richieste impattino sui recettori}. 
\\*\\*
Lo scopo finale di un piano energetico può essere definito, più o meno, come il tentativo di guidare il mercato libero della produzione di energia verso la copertura del fabbisogno energetico regionale e il soddisfacimento di alcuni obiettivi politici. Per questo fine un piano consiste in un insieme di \emph{decisioni} riguardanti i seguenti aspetti:
\begin{enumerate}
\item quali tipi\footnote{Per la precisione i tipi di energia sono influenzati anche da alcuni obiettivi politici, presi come ingressi} (attività) di produzione energetica sia necessario impiegare e in che ordine di grandezza; 
\item quanti fondi debbano essere assegnati per spingere il mercato versa la produzione desiderata (attraverso l'implementazione di meccanismi incentivanti). Questa decisione può essere separata in due sotto-problemi:
	\begin{enumerate}
	\item la quantità di finanziamenti da allocare per promuovere ogni sistema di generazione di energia diverso, sotto forma di incentivi;
	\item i meccanismi per assegnare tali incentivi. 
	\end{enumerate}
\end{enumerate}

\subsection{Implementazione modello}
Illustreremo ora come sono stati modellate le problematiche introdotte precedentemente all'interno del paradigma della Programmazione Logica a Vincoli, in particolare sul dominio dei numeri reali.

\myparagraph{Matrici coassiali in \clpr}
Le matrici coassiali possono essere interpretate facilmente come modelli di programmazione lineare. Più in dettaglio, rappresentiamo le attività e i loro ordini di grandezza attraverso il vettore $A = (a_1,...,a_{N_a})$, dunque gli impatti ambientali causati dall'attività $i$ possono essere stimati col sistema di equazioni lineari:
\begin{equation}
\label{eq:impattiAtt}
	\forall j \in \{1,...,N_p\}  \quad p_j = m_j^i a_i
\end{equation}
Quando consideriamo un intero piano regionale, sommiamo i contributi di tutti le attività e otteniamo la stima dell'influenza esercitata su ogni impatto/pressione:
\begin{equation}
\label{eq:impattiAttTot}
	\forall j \in \{1,...,N_p\}  \quad  p_j = \sum_{i=1}^{N_a} m_j^i a_i
\end{equation}
Allo stesso modo, dato il vettore delle pressioni ambientali $P = (p_1,...,p_{N_p})$, è possibile calcolare l'influenza sul recettore ambientale $r_i$ per mezzo della matrice $N$, che collega pressioni e recettori:
\begin{equation}
\label{eq:pressRec}
	\forall j \in \{1,...,N_r\}  \quad  r_j = \sum_{i=1}^{N_p} n_j^i p_i
\end{equation}

L'obiettivo finale della valutazione di una politica ambientale è stimare l'impronta ambientale di un piano. Poiché il piano è definito da un insieme di valori che rappresentano l'ordine di grandezza delle attività previste, è possibile calcolare l'impronta ambientale $R = (r_1,...,r_{N_r})$ semplicemente applicando le equazioni (\ref{eq:impattiAttTot}) e (\ref{eq:pressRec}).
Un altro importante interrogativo che potrebbe interessare l'utente del sistema di supporto alle decisioni è sapere quali attività, tra quelle possibili, abbiano un maggiore impatto sul recettore $r_i$. Dal momento \clpr prevede di poter massimizzare o minimizzare una funzione obiettivo, il modello diventa:
\begin{equation}
\label{eq:impactOnGivenRec}
	\begin{aligned}
		max(r_i) \\
		s.t.  \quad (\ref{eq:impattiAttTot}) (\ref{eq:pressRec}) \\
		\sum_j a_j = 1 \\
		\forall j, a_j \mbox{ è intero} 
	\end{aligned}
\end{equation}

Infine, se ci sono leggi che impongono limiti per alcuni recettori (ad esempio sulla $CO_2$), è immediato imporre tali restrizioni sui recettori (es. $r_{CO_2} \leq limite_{CO_2}$), scoprendo quindi se un'attività può essere portata a termine o è necessario intraprendere misure correttive (ad esempio stimolare la crescita di nuove foreste per il $r_{CO_2}$).
\\* \\*
Nel casi in cui più attività alternative possano contribuire a soddisfare gli stessi bisogni, la regolamentazione prevede che le alternative debbano essere prese in esame e confrontate. Ad esempio, la necessità di energia elettrica aggiuntiva è soddisfatta dalla costruzione di una nuova centrale; in ogni caso, è possibile scegliere il tipo di impianto da realizzare in base alle condizioni ambientali. In un'area dall'elevato inquinamento atmosferico la costruzione di una centrale termoelettrica rischierebbe di aumentare il livello di inquinamento oltre la soglia imposta dalla legge, quindi sarebbe necessario progettare un tipo di impianto differente, ad esempio uno a energia solare. D'altro canto, un impianto solare potrebbe rivelarsi molto costoso e precludere dunque la realizzazione di altre attività indispensabili nell'aerea (costruire scuole, ospedali, etc.). In una simile circostanza il responsabile del piano potrebbe imporre un vincolo che afferma che sono necessari almeno $k$ MW di energia elettrica:
\begin{equation}
\label{eq:kEner}
	\sum_{i \in ImpiantiEnergetici} a_i \geq k
\end{equation}
(dove $ImpiantiEnergetici$ è l'insieme degli indici del vettore $A$ corrispondenti agli impianti che producono energia elettrica) e poi ottimizzare per uno dei recettori, es. $r_{CO_2}$, o qualche somma pesata dei ricettori d'interesse. O ancora, sarebbe possibile chiedere al DSS quale sia la massima produzione energetica possibile nella regione senza violare il limiti imposti dalla legge:  
\begin{equation}
\label{eq:maxOutLimLaw}
	\begin{aligned}
		max \sum_{{i \in ImpiantiEnergetici}}  a_i\\
		s.t. \quad (\ref{eq:impattiAttTot}) (\ref{eq:pressRec}) \\
		\forall i \in \{ 1,...,N_r \} \quad r_i \leq limite_i
	\end{aligned}
\end{equation}
In questo modo è possibile trovare il massimo numero di MW che possono essere prodotti e anche l'energia elettrica generata da ogni impianto. Il risolutore del problema CLP può inoltre trovare una soluzione che imponga l'esecuzione di attività di compensazione, come suggerito in precedenza.

\myparagraph{Analisi della sensibilità}
L'algoritmo del simplesso (applicabile grazie alla linearità dei vincoli del modello) fornisce in modo efficiente - nel nostro modello il tempo di computazione si è rivelato trascurabile - il valore ottimo della funzione obiettivo, l'assegnamento ottimo dei valori alle variabili decisionali, e altre informazioni utili per i decisori politici. In particolare consente di ricavare i cosiddetti \emph{costi ridotti} e la \emph{soluzione duale}. Questi indicatori danno informazioni preziose sulla sensibilità della soluzione trovata nei confronti dei parametri del modello e consentono di fare un tipo di analisi molto interessante per gli utenti del DSS.

La soluzione duale è un insieme di valori che corrispondono ai vincoli e può essere immaginata come la derivata della funzione obiettivo rispetto al lato destro dei vincoli (dall'inglese Right Hand Side, RHS). Ciò implica che attraverso la soluzione duale è immediato rendersi conto di quali siano i vincoli \emph{stretti}, ovvero quali modificherebbero il valore della funzione obiettivo se i coefficienti del lato destro cambiassero. Se ad esempio si cercasse di ottimizzare il numero $f_{MW}$ di MW di potenza elettrica con il vincolo $r_{CO_2} \leq limite_{CO_2}$ e il corrispondente valore duale $d_{CO_2}$ nella soluzione ottima fosse diverso da zero, questo significherebbe che:
\begin{equation}
\label{eq:sensAnal}
	d_{CO_2} = \frac{\partial f_{MW}}{\partial limite_{CO_2}}
\end{equation}
In altri termini, il valore della variabile $d_{CO_2}$ risponde alla domanda: \emph{``Quale sarebbe il decremento della produzione energetica nel caso in cui il limite di $CO_2$ si abbassasse di un punto?''}. Questo è un punto molto importante poiché le leggi tendono a cambiare, in genere diventando più severe. 

La stessa analisi potrebbe essere effettuata con il problema di ottimizzare alcuni recettori (o la loro somma pesata, dato un numero totale di impianti (o MW richiesti). In questo caso il valore duale associato ad un vincolo rappresenta quanto il recettore migliorerebbe se il vincolo fosse parzialmente rilassato (se il lato destro diventasse meno stringente). Ad esempio, supponiamo di ottimizzare le emissioni di ossidi di azoto ($NO_x$) e di avere un vincolo che imponga un limite superiore ($F$, es. in euro) al costo totale delle attività da portare a termine, dove il vettore $C$ specifica il costo di ogni attività: 
\begin{equation}
\label{eq:costoAtt}
	\sum_{i=1}^{N_a} c_i a_i \leq F
\end{equation}
Dopo aver ottenuto la soluzione ottima, il responsabile del piano potrebbe chiedersi: \emph{``Supponendo ora di disporre di più fondi, se aggiungessi un euro quanto diminuirebbe l'emissione di $NO_x$?''}. La risposta sarebbe data dal valore duale $d_e$ del vincolo \ref{eq:costoAtt}.

\myparagraph{Pianificazione con \clpr}

Dopo aver visto come effettuare la valutazione ambientale secondo il paradigma della \clpr, mostriamo ora come sia possibile integrare anche la fase di ideazione del piano regionale sfruttando le stesse metodologie di modellazione.
\\*\\*
Rappresentiamo nuovamente le attività tramite il vettore $A = (a_1,...,a_{N_a})$ e ad ognuna di esse associamo una variabile $G_i$ che ne descrive la grandezza; questa può essere rappresentata si in modo assoluto, come quantità di una data attività, che in modo relativo, come percentuale rispetto alla quantità esistente della stessa attività - in seguito faremo riferimento alla prima modalità.

Occorre distinguere tra attività primarie e secondarie: $A^P$ è l'insieme degli indici delle prime, mentre $A^S$ è l'insieme degli indici delle seconde. Le dipendenze tra attività primarie e secondarie sono espresse dal vincolo:
\begin{equation}
\label{eq:primSec}
	\forall j \in A^S  \quad  G_j = \sum_{i \in A^P} d_{ij} G_i
\end{equation}
Assegnato un valore $B_{Piano}$ al il budget disponibile a un determinato piano, si ha un equazione che limita il costo complessivo di un piano nel seguente modo:
\begin{equation}
\label{eq:costoTotPrimSec}
	\sum_{i=1}^{N_a} G_i c_i \leq B_{Piano}
\end{equation}  
Inoltre, data la produzione energetica attesa del piano $o_{Piano}$, abbiamo un vincolo che garantisce di raggiungere lo scopo prefissato:
\begin{equation}
\label{eq:expOut}
	\sum_{i=1}^{N_a} G_i o_i \geq o_{Piano}
\end{equation}  
Ad esempio in un piano energetico, il risultato desiderato potrebbe essere una maggiore disponibilità di energia in regione, così $o_{Piano}$ sarebbe l'incremento di energia elettrica (es. in MW) e $o_i$ sarebbe invece la produzione in MW di ogni attività $i$.

Per quanto riguarda gli obiettivi esistono diverse possibilità suggerite dagli esperti del settore. Da una prospettiva economica si potrebbe decidere di minimizzare il costo complessivo del piano (il quale sarebbe comunque soggetto ai vincoli sulla disponibilità finanziaria), ma chiaramente in questo caso potrebbero essere preferite le risorse energetiche più economiche, a prescindere dai loro potenziali effetti negativi sull'ambiente (nuovamente, questi potrebbero ad ogni modo essere vincolati). D'altro canto, si potrebbe fissare il budget e massimizzare invece la produzione energetica, prendendo in considerazione le risorse più efficienti. O ancora, si potrebbe decidere di realizzare un piano \emph{verde} e tenere in considerazione i recettori ambientali, ad esempio massimizzando la qualità dell'aria, delle acque superficiali, etc. Ovviamente in casi come questo le decisioni prodotte dalla soluzione ottima prodotta dall'ottimizzatore possono essere meno intuitive e proprio in circostanze simili il modello proposto si dimostra utile, infatti la relazione tra le decisioni riguardanti le attività primarie e secondarie e le conseguenze ambientali sono estremamente complesse da calcolare manualmente. \`E oggetto di ricerca la possibilità di creare funzioni obiettivo ancora più complicate combinando gli aspetti accennati sopra.
\\*\\*
Un aspetto importante da non sottovalutare durante l'ideazione di un piano regionale è la diversificazione delle fonti energetiche, cioè allocare le risorse economiche in fonti di energia diverse, sia rinnovabili che non rinnovabili, per prevenire fluttuazioni di prezzo e disponibilità. Per questo motivo è presente un vincolo sulla frazione minima $F_i$ dell'energia totale prodotta per ogni fonte $i$: 
\begin{equation}
\label{eq:fracEner}
	\forall i \in A^P \quad G_i o_i \geq F_i T^0
\end{equation}  
dove la produzione totale è ottenuta come: 
\begin{equation}
\label{eq:prodTot}
	T^0 = \sum_{j \in A^P} G_j o_j 
\end{equation}  
Oltre a ciò dobbiamo considerare le caratteristiche geofisiche della regione, come ad esempio il già citato caso delle centrali idroelettriche che possono essere costruite solamente prestando molta attenzione alle conseguenze sul territorio. Possiamo quindi imporre dei vincoli che limitino la massima energia $Max_i$ che può essere prodotta da una fonte energetica $i$:
\begin{equation}
\label{eq:maxEner}
	\forall i \in A^P \quad G_i o_i \leq Max_i
\end{equation}  
Infine, le priorità della regione dovrebbero essere conformi alle direttive e linee guida europee, come l'iniziativa 20-20-20. Per questa ragione, possono essere imposti vincoli sulla minima quantità di energia $Min_{rin}$ prodotta attraverso risorse rinnovabili, definendo $A^P_{rin}$ l'insieme di attività relative, ovvero:
\begin{equation}
\label{eq:minEnerRen}
	\sum_{i \in A^P_{rin}} G_i o_i \geq  Min_{rin}
\end{equation}  

\section{Il piano regionale 2011-2013}
Il modello a vincoli descritto in precedenza in questo capitolo è stato impiegato per la pianificazione del piano energetico 2011-2013 della regione Emilia-Romagna. Questo piano aveva come scopo quello di aprire la strada per il raggiungimento dell'obiettivo del 20-20-20, in particolare ottenere il 20\% dell'energia prodotta da fonti rinnovabili entro il 2020. Questa valore non si riferisce esclusivamente all'energia elettrica, ma prende in considerazione l'intero bilancio energetico della regione, includendo l'energia termica e i trasporti. 

Il settore dei trasporti può usare energia rinnovabile tramite l'uso di combustibili rinnovabili, come il biogas (metano prodotto dalla fermentazione di vegetali o scarti animali) o carburanti ottenuti da varie coltivazioni. L'energia termica può essere usata ad esempio per il riscaldamento delle abitazioni e in questo caso le fonti rinnovabili includono pannelli solari termici (i quali producono acqua calda per usi domestici), pompe geotermiche (impiegate per riscaldare o rinfrescare le case), impianti a biomassa (che producono acqua calda usata per riscaldare edifici vicini nei periodi invernali). 

Nel proseguo ci concentreremo solamente sull'energia elettrica e gli impianti considerati per la produzione di energia da fonti rinnovabili sono centrali idroelettriche, impianti fotovoltaici, impianti solari termodinamici, generatori eolici e, nuovamente, centrali a biomassa. Per ogni fonte energetica il piano dovrebbe fornire:
\begin{itemize}
\item la potenza installata in MW;
\item l'energia totale prodotta in un anno, in kTOE (TOE è una acronimo per Tonnes of Oil Equivalent, tonnellate di petrolio equivalenti);
\item il costo totale in M\euro.
\end{itemize} 
Il rapporto tra la potenza installata e l'energia totale prodotta è principalmente influenzato dalla disponibilità della risorsa: mentre una centrale a biomassa può produrre energia continuamente (almeno teoricamente), il sole è disponibile solo durante il giorno e il vento solo occasionalmente; per fonti energetiche non affidabili viene considerata una media annuale. 

Il costo dell'impianto dipende invece in maniera più significativa dalla potenza installata: un impianto solare ha un costo d'istallazione che dipende dalla superficie in metri quadri dei pannelli installati, i quali in cambio producono una certa potenza massima (potenza di picco).

I tecnici della regione hanno stimato (considerando i consumi di energia attuali, le previsioni di crescita, il risparmio energetico previsto) le richieste di energia complessiva per il 2020, di cui il 20\% dovrà essere generato a partire da fonti rinnovabili. Per questa quantità gli esperti hanno anche proposto la percentuale da ottenere con il piano 2011-2013, ovvero circa 177kTOE di energia elettrica e 296kTOE di energia termica. A partire da questi dati hanno poi redatto un piano per le energie elettrica e termica.

Abbiamo impiegato il modello a vincoli presentato precedentemente considerando inizialmente casi ``estremi'', nei quali venisse usata un'unica fonte energetica. Ad esempio se imponiamo di costruire solamente impianti a biomassa il modello fornisce i risultati presentati in Tabella~\ref{tab:energyPlanBiomassOnly}. Come detto in precedenza, oltre al piano viene fornita anche la valutazione ambientale.

\begin{table}[h]
\centering
	\begin{tabular}{ p {0.3\textwidth} | p {0.1\textwidth} p {0.1\textwidth}  p {0.1\textwidth} c }
	\hline \hline
	Impianti Elettrici & Potenza 2010 (MW) & Potenza 2013 (MW) & Energia 2013 (kTOE) & Investimenti (M\euro)\\
	\hline
	Idroelettrico & 300 & 300 & 67.06 & 0\\

	Fotovoltaico & 230 & 230 & 23.73 & 0\\
	
	Solare termodinamico & 0 & 0 & 0 & 0\\

	Eolico & 20 & 20 & 2.58 & 0\\

	Biomassa & 430 & 724.47 & 436.13 & 1030.64\\
	\hline
	Totale & 980 & 1274.47 & 529.5 & 1030.64\\
	\hline \hline
	\end{tabular}
	\caption{Esempio di piano energetico per l'elettricità realizzando unicamente centrali a biomassa}
	\label{tab:energyPlanBiomassOnly}	
\end{table}

Per meglio comprendere i contributi individuali delle varie forme di energia, in Figura~\ref{extremePlansPlot} sono stati tracciati graficamente i piani che usano una singola fonte energetica e quello sviluppato dagli esperti. Sull'asse delle $x$ è stato scelto il recettore \emph{Qualità dell'Aria}, in quanto tra i più sensibili per la regione Emilia-Romagna, mentre nell'asse delle $y$ è riportato il costo - i piani forniscono la stessa energia in kTOE ma richiedono l'istallazione di potenze diverse.

\begin{figure}[h]
	\centering
	\includegraphics[scale=0.45]{extremePlansPlot}
	\caption{Confronto tra il piano degli esperti e quelli estremi che sfruttano un'unica risorsa energetica}
	\label{extremePlansPlot}
\end{figure}

Notiamo subito che alcuni fonti energetiche migliorano la qualità dell'aria (valori positivi sull'asse delle $x$) mentre altre la peggiorano (valori negativi). Anche se , ovviamente, nessun impianto per la produzione di energia può migliorare la qualità dell'aria da solo (in quanto non può rimuovere le sostanze inquinanti dall'atmosfera), quello che accade è che gli impianti producono nuova energia elettrica senza introdurre ulteriori sostanze inquinanti, energia che, se non prodotta localmente, sarebbe dovuta essere importata da regioni vicine. In questo caso l'energia sarebbe stata prodotta sfruttando le stesse fonti energetiche della produzione nazionali, incluse quelle con emissioni inquinanti, quindi nel complesso il contributo è positivo per la qualità dell'aria. Sottolineiamo anche che diverse risorse energetiche hanno un diverso impatto sulla qualità dell'aria anche a causa della attività secondarie necessarie. 

Come ulteriore considerazione, occorre osservare che i piani ``estremi'' generalmente non sono fattibili, poiché nel nostro caso il vincolo sulla reale disponibilità di risorse energetiche è stato rilassato. A esempio le turbine eoliche forniscono una qualità dell'aria molto buona ad un costo contenuto, ma la quantità di produzione richiesta non è possibile in regione considerando la presenza media del vento e la disponibilità di aeree per l'istallazione.

Il piano proposto dagli esperti è più \emph{bilanciato}: tiene in considerazione la reale disponibilità della risorsa energetica nella regione e fornisce una combinazione di tutte le possibili fonti di energia - ciò è molto importante soprattutto per le risorse rinnovabili che spesso sono discontinue e quindi disporre di molteplici possibilità garantisce una fornitura elettrica costante e continuata.

\begin{figure}[h]
	\centering
	\includegraphics[scale=0.45]{paretoOpt}
	\caption{Frontiera ottima di Pareto della qualità dell'aria rispetto al costo}
	\label{paretoOpt}
\end{figure}

Oltre a valutare i piani proposti dagli esperti tramite il nostro modello è stato possibile crearne di alternativi. In particolare, un piano cercato è stato quello ottimale rispetto al costo e alla qualità dell'aria. Avendo una funzione con due obiettivi abbiamo tracciato il fronte ottimo di Pareto (riprendiamo qui la figura già brevemente descritta nel capitolo sul progetto ePolicy); ogni punto del fronte è tale che non è possibile migliorare uno dei due obiettivi senza peggiorare l'altro. Nel nostro caso la qualità dell'aria non può essere migliorata senza aumentare i costi e, viceversa, non è possibile ridurre il costo senza sacrificare il recettore considerato. La frontiera di Pareto è raffigurata in Figura~\ref{paretoOpt} insieme al piano realizzato dagli esperti. Benché questo piano sia vicino alla frontiera può comunque essere migliorato - in particolare si possono osservare sul fronte due punti che rappresentano due piani migliori, uno con costo uguale ma migliore qualità dell'aria, l'altro con qualità uguale ma costo minore. 

\begin{table}[h]
\centering
	\begin{tabular}{ p {0.3\textwidth} | p {0.1\textwidth} p {0.1\textwidth}  p {0.1\textwidth} c }
	\hline \hline
	Impianti Elettrici & Potenza 2010 (MW) & Potenza 2013 (MW) & Energia 2013 (kTOE) & Investimenti (M\euro)\\
	\hline
	Idroelettrico & 300 & 310 & 69.3 & 84\\

	Fotovoltaico & 230 & 850 & 87.7 & 2170\\
	
	Solare termodinamico & 0 & 10 & 1 & 45\\

	Eolico & 20 & 80 & 10.3 & 120\\

	Biomassa & 430 & 600 & 361.2 & 595\\
	\hline
	Totale & 980 & 1850 & 529.5 & 3014\\
	\hline \hline
	\end{tabular}
	\caption{Piano energetico ideato dagli esperti}
	\label{tab:expertPlan}	
\end{table}

\begin{table}[h]
\centering
	\begin{tabular}{ p {0.3\textwidth} | p {0.1\textwidth} p {0.1\textwidth}  p {0.1\textwidth} c }
	\hline \hline
	Impianti Elettrici & Potenza 2010 (MW) & Potenza 2013 (MW) & Energia 2013 (kTOE) & Investimenti (M\euro)\\
	\hline
	Idroelettrico & 300 & 303 & 67.74 & 25.2\\

	Fotovoltaico & 230 & 782.14 & 80.7 & 1932.51\\
	
	Solare termodinamico & 0 & 5 & 0.5 & 22.5\\

	Eolico & 20 & 140 & 18.03 & 240\\

	Biomassa & 430 & 602.23 & 362.54 & 602.8\\
	\hline
	Totale & 980 & 1832.37 & 529.5 & 2823\\
	\hline \hline
	\end{tabular}
	\caption{Piano energetico che domina quello degli esperti, con stessa qualità dell'aria ma costo inferiore}
	\label{tab:betterCostPlan}	
\end{table}

La Tabella~\ref{tab:expertPlan} mostra il piano degli esperti mentre la Tabella~\ref{tab:betterCostPlan} presenta un piano che domina quello degli esperti, con la stessa qualità dell'aria ad un costo inferiore. L'energia generata con le turbine eoliche è quasi raddoppiata (poiché forniscono un buon rapporto qualità-aria/costo, vedi Fig.~\ref{extremePlansPlot}), c'è un leggero incremento nella produzione da biomassa e le fonti restanti sono parzialmente ridotte di conseguenza.

Per quanto riguarda la valutazione ambientale, in Figura~\ref{receptors_pareto} sono mostrati i valori dei recettori in alcuni punti significativi della frontiera di Pareto. Ogni barra rappresenta un singolo recettore ambientale per un piano specifico sul fronte di Pareto di Fig.~\ref{paretoOpt}; la barra azzurra è associata al piano che garantisce la migliore qualità dell'aria mentre le barre nei restanti colori sono associate a piani di dal costo inferiore. I recettori mostrano tendenze differenti: alcuni migliorano muovendosi lungo la frontiera verso una maggiore qualità dell'aria (come la qualità del clima, benessere della popolazione, valore dei beni materiali), mentre migliorano procedendo verso soluzioni a costo inferiore (qualità del paesaggio, benessere della fauna selvatica, qualità dei terreni). Questo fatto è dovuto a diverse ragioni, che dipendono sia dal tipo di impianto installato che dalle attività secondarie richieste.

\begin{figure}[h]
	\centering
	\includegraphics[scale=0.65]{receptors_pareto}
	\caption{Valore dei recettori sulla frontiera di Pareto}
	\label{receptors_pareto}
\end{figure}

Ad esempio le turbine eoliche hanno un buon effetto sulla qualità dell'aria ma sono al tempo stesso considerato poco piacevoli dal punto di vista estetico, quindi non possono essere installati in determinate zone sensibili, come la cima delle colline, senza dover affrontare proteste dai parte dei residenti (recettore qualità del paesaggio); sfortunatamente, colline sono anche le aree più ventose in Emilia-Romagna. 

O ancora, poiché gli uccelli migratori seguono i percorsi dei venti per ridurre la fatica nei loro viaggi su lunghe distanze ma d'altro canto gli impianti eolici devono essere collocati in zone ventose per essere efficaci, si ha come conseguenza che durante la stagione delle migrazioni gli stormi rischierebbero di entrare in collisione con le grandi pale dei generatori; questo effetto non può essere ignorato, in particolare per specie a rischio (benessere della fauna selvatica).

%\nocite{*}
%\bibliographystyle{plain}
%\bibliography{bibliography}

%\end{document}


% interazione tra ottimizzatore e simulatore
%Andrea Borghesi
%Università degli studi di Bologna

%capitolo dedicato all'interazione tra ottimizzatore e simulatore

\documentclass[12pt,a4paper,openright,twoside]{report}
\usepackage[italian]{babel}
\usepackage{indentfirst}
\usepackage[utf8]{inputenc}
\usepackage[T1]{fontenc}
\usepackage{fancyhdr}
\usepackage{graphicx}
\usepackage{titlesec,blindtext, color}
\usepackage[font={small,it}]{caption}
\usepackage{subfig}
\usepackage{listings}
\usepackage{color}
\usepackage{url}
\usepackage{textcomp}
\usepackage{eurosym}
\usepackage{amsmath}
\usepackage{url}

%impostazioni generali per visualizzare codice
\definecolor{dkgreen}{rgb}{0,0.6,0}
\definecolor{gray}{rgb}{0.5,0.5,0.5}
\definecolor{mauve}{rgb}{0.58,0,0.82}
 
\lstset{ %
  basicstyle=\footnotesize,           % the size of the fonts that are used for the code
  backgroundcolor=\color{white},      % choose the background color. You must add \usepackage{color}
  numbers=left,                   % where to put the line-numbers
  numberstyle=\tiny\color{gray},  % the style that is used for the line-numbers
  numbersep=5pt,  
  showspaces=false,               % show spaces adding particular underscores
  showstringspaces=false,         % underline spaces within strings
  showtabs=false,                 % show tabs within strings adding particular underscores
  rulecolor=\color{black}, 
  tabsize=2,                      % sets default tabsize to 2 spaces
  breaklines=true,                % sets automatic line breaking
  breakatwhitespace=false,        % sets if automatic breaks should only happen at whitespace
  title=\lstname,                   % show the filename of files included with \lstinputlisting;
  frame=single,                   % adds a frame around the code
                                  % also try caption instead of title
  keywordstyle=\color{blue},          % keyword style
  commentstyle=\color{dkgreen},       % comment style
  stringstyle=\color{mauve},         % string literal style
  escapeinside={\%*}{*)},            % if you want to add LaTeX within your code
  morekeywords={*,...},              % if you want to add more keywords to the set
  deletekeywords={...}              % if you want to delete keywords from the given language
}

%per avere un bordo intorno alle figure
\usepackage{float}
\floatstyle{boxed} 
\restylefloat{figure}

%per poter poi impedire che certe parole vadano a capo
\usepackage{hyphenat}

%ridefinisco font per fancyhdr, per ottenere un'intestazione pulita
\newcommand{\changefont}{ \fontsize{9}{11}\selectfont }
\fancyhf{}
\fancyhead[LE,RO]{\changefont \slshape \rightmark} 	%section
\fancyhead[RE,LO]{\changefont \slshape \leftmark}	%chapter
\fancyfoot[C]{\changefont \thepage}					%footer

%titolo capitolo con "numero | titolo"
\definecolor{gray75}{gray}{0.75}
\newcommand{\hsp}{\hspace{20pt}}
\titleformat{\chapter}[hang]{\Huge\bfseries}{\thechapter\hsp\textcolor{gray75}{|}\hsp}{0pt}{\Huge\bfseries}


\oddsidemargin=30pt \evensidemargin=20pt

%sillabazioni non eseguite correttamente
\hyphenation{sil-la-ba-zio-ne pa-ren-te-si si-mu-la-to-re ge-ne-ra-re pia-no}

%interlinea
\linespread{1.15}  
\pagestyle{fancy}

%cartelle contenenti le immagini
\graphicspath{{/media/sda4/tesi/immagini/grafici/}{/media/sda4/tesi/immagini/grafici/incCompare/}{/media/sda4/tesi/immagini/grafici/rawData/}{/media/sda4/tesi/immagini/grafici/regressionAnalysis/}{/media/sda4/tesi/immagini/schemi/}{/media/sda4/tesi/immagini/simulazione/}{/media/sda4/tesi/immagini/epolicy/}{/media/sda4/tesi/immagini/ottimizzazione/}
{/media/sda4/tesi/immagini/interazione/}}

%in modo che dopo il titolo di un paragrafo il testo vada a capo
\newcommand{\myparagraph}[1]{\paragraph{#1}\mbox{}\\}

%per scrivere bene CLP(R) e CLP(FD)
\newcommand{\clpr}{CLP({\ensuremath{\cal R}})}
\newcommand{\clpfd}{CLP({\ensuremath{\cal FD}})}

\begin{document}

\chapter{\nohyphens{INTERAZIONE COMPONENTI}}
Nei capitoli precedenti abbiamo descritto gli elementi principali che concorrono a definire l'architettura del sistema ePolicy (certamente all'interno del progetto sono presente ulteriori componenti, come quello dedicato all'opinion mining, ma in questo caso ci riferiamo a quelli considerati in questo lavoro). Da un punto di vista generale essi sono il modello a vincoli che garantisce una ottimizzazione a livello globale e il simulatore che studia il comportamento delle modalità di incentivazione a livello locale/regionale. Un aspetto molto importante è quindi capire come gestire l'interazione tra ottimizzatore e simulatore in modo ottimale, in modo da integrare le prospettive globale e locale.

Possiamo illustrare la necessità di comprendere a fondo questa interazione con un esempio. Supponiamo che la fase di ottimizzazione abbia prodotto due scenari alternativi, il primo concentrandosi sulla creazione di impianti a biomassa e il secondo sostenendo la costruzione di centrali idroelettriche; entrambi i piani avrebbero un impatto non indifferente sui cittadini. La produzione energetica con la biomassa comporta un impatto sostanziale sulle aree boschive, potenziale inquinamento del suolo e delle coltivazioni, inquinamento dell'aria nelle aree urbane vicine alla centrale; d'altra parte, le centrali idroelettriche prevedono l'allagamento vaste porzioni di territorio. In ogni caso le strategie per implementare il piano, studiate tramite il simulatore, dovrebbero tenere in considerazione questi effetti sugli individui; le attività di implementazione implicherebbero quindi costi aggiuntivi che dovrebbero essere inseriti come nuovi vincoli all'interno dell'ottimizzatore, il quale potrebbe poi effettuare nuovamente la fase di pianificazione, con la possibilità di ottenere risultati diversi. 

Un approccio molto basilare sarebbe il semplice scambio di risultati tra i due livelli di pianificazione delle politiche, svolgendo anche diverse iterazioni, ma questo metodo rischierebbe di non garantire la convergenza. A un certo punto le iterazioni possono essere fermate quando un equilibrio è stato raggiunto o quando il decisore politico valuta che ulteriori aggiustamenti non siano più necessari o richiesti. Citando Clement Attlee\footnote{Fonte: A. Sampson, \emph{Anatomy of Britain}, Hodder \& Stoughton, 1962} \emph{``Democracy means government by discussion but it is only effective if you can stop people talking.''} - democrazia significa governo fondato sulla discussione, ma funziona solamente se si riesce a far smettere la gente di discutere.
\\*

Il tema dell'integrazione efficacie tra pianificazione regionale e simulazione è oggetto di intensa ricerca per ottenere una soluzione ottimale all'interno del progetto ePolicy (in modo particolare servendosi di metodologie appartenenti alla \emph{teoria dei giochi}); nel resto del capitolo verrà mostrato un possibile approccio, da noi implementato e messo alla prova con il problema dell'assegnazione dei fondi regionali per l'incentivazione della tecnologia fotovoltaica in Emilia-Romagna.

\section{INTEGRARE DSS E SIMULAZIONI}
Un primo approccio a cui è possibile pensare, consiste nello sfruttare i metodi e le tecniche dell'\emph{Apprendimento Automatico} (noto in letteratura anche come \emph{Machine Learning}), il quale rappresenta un'area dell'Intelligenza Artificiale che si occupa della realizzazione di sistemi e algoritmi che si basano su serie di osservazioni e dati per la sintesi di nuova conoscenza. Senza voler entrare nel dettaglio, possiamo comunque citare una definizione comunemente accettata di Apprendimento Automatico: \emph{``Un programma apprende da una certa esperienza E se, con rispetto a una classe di compiti T e una misura delle prestazioni P, le prestazioni P nello svolgere un compito dell'insieme T sono migliorate dall'esperienza E''} \cite{Mitchell}.
 
Nel nostro caso abbiamo sfruttato le tecniche di regressione viste nei capitoli precedenti per ricavare le relazioni che legano i fondi destinati agli incentivi regionali con la produzione elettrica di energia elettrica fotovoltaica, partendo dalle serie di dati forniti dal simulatore, affinché fosse poi possibile inserirle all'interno dell'ottimizzatore sotto forma di ulteriori vincoli, da tenere in considerazione per la fase di pianificazione. Il nostro fine è stato quello di estrarre dalla grande quantità di dati generata dal simulatore delle informazioni utili per migliorare il modello del problema da ottimizzare.

La Figura~\ref{schemaIterLearn} mostra lo schema dell'interazione tra il livello globale e il livello locale realizzata tramite l'approccio dell'Apprendimento Automatico. Nella parte alta osserviamo il sistema di supporto alle decisioni, l'ottimizzatore, che, a fronte delle possibili decisioni (l'allocazione di risorse per lo svolgimento di attività per la produzione di energia energetica nel rispetto dei diversi vincoli), produce un piano (oppure un insieme di piani o scenari). Il modello del DSS è arricchito con i vincoli che vengono appresi nella fase di \emph{Learning} a partire dai risultati prodotti dal simulatore; in ingresso al simulatore troviamo un insieme di piani interessanti per la relazione che stiamo tentando di apprendere - ad esempio per studiare il rapporto tra i fondi investiti nel metodo di incentivazione Conto Interessi è stato necessario effettuare simulazioni per un ampio numero di valori di budget. 
\begin{figure}[htb]
	\begin{center}
	\includegraphics[scale=0.65]{schemaIterLearn}
	\end{center}
	\caption{Modello di interazione basato su Apprendimento Automatico}
  	\label{schemaIterLearn}
\end{figure}

La fase di apprendimento, e quindi le simulazioni, devono essere effettuate prima della fase di pianificazione (\emph{offline}), in quanto occorre inserire all'interno del modello i nuovi vincoli appresi, i quali non saranno più modificati (se non nel caso in cui vengano sostituiti da altri ricavati da un nuovo processo di apprendimento). \`E necessario effettuare un grande numero di simulazioni per garantire un valore statistico alle relazioni apprese e fornire un buon insieme di dati tramite cui effettuare l'apprendimento; questo rappresenta sicuramente il maggiore limite di questo tipo di approccio, in quanto comporta che i vincoli appresi non possano essere modificati facilmente - effettuare un gran numero di simulazioni richiede molto tempo - e l'interazione avvenga sostanzialmente in una sola direzione, dall'simulatore verso il DSS. 
\\*\\*
Una seconda metodologia per integrare ottimizzazione e simulazione da noi considerata, nonostante non sia stata concretamente implementata a differenza della prima, è una tecnica classica di decomposizione dei problemi presa in prestito dall'ambito della Ricerca Operativa, la cosiddetta \emph{decomposizione di Benders} \cite{bendersDec}. Essa consiste in un metodo per risolvere problemi di ottimizzazione combinatoria che possono essere scomposti in due componenti, un problema master e un sotto-problema. Originariamente era stata concepita per il campo della Programmazione Lineare Intera ma è stata in seguito estesa per trattare risolutori più generali, \emph{Logic-Based Benders Decomposition} (decomposizione di Benders basata sulla Logica) \cite{Hooker95logic-basedbenders}. 

Nel caso da noi preso in esame il problema master è la definizione del piano energetico regionale che partizioni l'energia necessaria tra le diverse fonti energetiche rinnovabili e viene risolto tramite il modello a vincoli descritto nel capitolo precedente. Il sotto-problema consiste nella definizione della strategia di incentivazione per raggiungere la produzione energetica desiderata, in modo consistente con i vincoli regionali sul budget. Partendo dalle soluzioni ottenute con l'ottimizzazione, ovvero la produzione energetica attesa, per comprendere quale sia il budget corretto da allocare per gli incentivi vengono portate a termine diverse simulazioni - in numero comunque molto inferiore rispetto all'interazione basata sull'Apprendimento Automatico. Nel caso in cui gli incentivi non siano compatibili con il budget regionale viene generato un cosiddetto \emph{taglio di Benders} (chiamati anche \emph{no-good}), cioè un vincolo che va ad aggiungersi al modello del problema master, e successivamente una nuova soluzione viene generata dal DSS. 

A differenza del primo metodo, con questo approccio la comunicazione tra i due componenti qui considerati viene estesa ad un ciclo, come si può facilmente notare in Figura~\ref{schemaIterBend}.

\begin{figure}[htb]
	\begin{center}
	\includegraphics[scale=0.65]{schemaIterBend}
	\end{center}
	\caption{Modello di interazione basato su Decomposizione di Benders}
  	\label{schemaIterBend}
\end{figure}

L'interazione inizia dall'ottimizzatore che fornisce una soluzione per il problema master, soluzione che contiene la produzione energetica attesa da fotovoltaico e dei valori ipotetici della dimensione dei fondi da destinare agli incentivi regionali. Questi valori ipotetici sono passati al simulatore, il quale esegue delle simulazioni esclusivamente con tali parametri forniti dal DSS e produce le corrispondenti statistiche (il tempo di calcolo è di qualche ordine di grandezza minore rispetto all'approccio basato sull'apprendimento); queste ultime possono confermare o meno i valori ipotizzati in fase di ottimizzazione: se il valore (medio) di produzione energetica ottenuto dalle simulazioni è maggiore o uguale di quello atteso, l'iterazione può concludersi è il risultato è probabilmente ottimale \cite{bendersDec}. Viceversa se invece il valore atteso è maggiore di quello simulato un'altra iterazione è necessaria, quindi  all'ottimizzatore è comunicato un nuovo vincolo, il quale può essere visto come spiegazione del fatto che non è possibile ottenere la produzione energetica richiesta con i fondi agli incentivi ipotizzati. A questo punto il DSS inserisce il vincolo all'interno del modello del problema, ricerca nuovamente una soluzione ottimale e ipotizza nuovi valori da fornire in input al simulatore.

La sfida principale consiste nel determinare l'insieme dei vincoli che vengono trasferiti tra le due componenti: se venissero esclusi dall'insieme dei valori ammissibili solamente quelli ipotizzati - e trovati non adatti grazie alle simulazioni - si correrebbe il rischio di effettuare troppe iterazioni, arrivando al caso limite di effettuare una simulazione esaustiva per tutti i parametri (in pratica verrebbero nuovamente fatte delle simulazioni per ogni valore del budget per gli incentivi regionali); se invece dall'insieme dei valori  venissero esclusi (troppi) valori ulteriori il pericolo sarebbe quello di scartare delle soluzioni promettenti. Questo tema e l'implementazione effettiva di questo secondo approccio sono attualmente oggetto di ricerca.

\section{REGRESSIONE LINEARE A TRATTI}
%a33-cattafi -> pag. 8-9
Passiamo ora a discutere del modo in cui l'approccio basato sull'Apprendimento Automatico sia stato implementato nel nostro modello a vincoli.
\\*\\*
Come è stato descritto nel terzo capitolo, tramite un grande numero di simulazioni è stato possibile ottenere una grande quantità di dati dalla quale abbiamo successivamente ricavato le relazioni che legano i fondi per gli incentivi regionale alla produzione energetica da fotovoltaico e quest'ultima alla forza dell'interazione sociale tra gli agenti. Tali relazioni sono state espresse sotto forma di funzioni e corrispondenti curve, ottenute attraverso l'applicazione di tecniche di regressione. A questo punto la nostra intenzione è stata quella di integrare queste funzioni all'interno modello a vincoli del problema di ottimizzazione, aggiungendo cioè i nuovi vincoli appresi grazie alle simulazioni svolte; è sorto quindi un problema, poiché, come descritto nel capitolo precedente, il risolutore dai noi utilizzato l'ottimizzazione gestisce esclusivamente equazioni lineari - per motivi di efficienza. Dal momento che modificare questa caratteristica, ovvero impiegare un risolutore in grado di trattare le funzioni quadratiche e di grado anche superiore ricavate dalla regressione, avrebbe richiesto cambiamenti radicali nella struttura generale e nel codice dell'ottimizzatore, abbiamo ritenuto che fosse meglio procedere in un altro modo, che ci consentisse di preservare la linearità del modello a vincoli sviluppato. Per questo motivo abbiamo deciso di tentare di rendere lineari le relazioni  ottenute con la regressione sfruttando una tecnica matematica definita \emph{approssimazione lineare a tratti} \cite{piecewiseApprox,cattafi} (dall'inglese, \emph{piece-wise linear approximation}), che consiste nell'approssimare un'arbitraria funzione con un insieme di equazioni lineari con la massima accuratezza possibile. Possiamo ad esempio osservare in Figura~\ref{piecewiseApprox_example} l'approssimazione di una semplice funzione quadratica (in blu) attraverso cinque funzioni lineari (in rosso). 

\begin{figure}[htb]
	\begin{center}
	\includegraphics[scale=0.8]{piecewiseApprox_example}
	\end{center}
	\caption{Una funzione (in blu) e la sua approssimazione lineare a tratti (in rosso). Fonte {\tt http://commons.wikimedia.org/wiki/File:Finite\_element\_method\_1D\_illustration1.svg}}
  	\label{piecewiseApprox_example}
\end{figure}

Illustreremo adesso il funzionamento di questo metodo. Data una funzione (anche non lineare) $y=f(x)$, campioniamo la curva $g$ in $k$ punti $x_1,...,x_k$ e l'approssimazione lineare a tratti $y'=g'(x)\simeq g(x)$ è definita come
\begin{equation} 
\label{eq:aprroxEqX}
	x = \sum_{i=1}^k \lambda_i \cdot x_i,
\end{equation}
\begin{equation} 
\label{eq:aprroxEqY}
	y = \sum_{i=1}^k \lambda_i \cdot y_i,
\end{equation}
dove $\lambda_i \in [0..1]$ sono variabili continue soggette ai vincoli:
\begin{equation} 
\label{eq:aprroxEqLambda}
		\sum_{i=1}^k \lambda_i = 1
\end{equation}
Al massimo due $\lambda_i$ possono essere diverse da zero e in tal caso queste devono essere adiacenti.

Chiaramente queste ultime due condizioni non sono lineari, ma potrebbero essere modellate in un problema di Programmazione Logica Intera introducendo nuove 0-1 variabili intere, ma esiste una opzione più efficiente. In molti risolutori - compreso quello da noi impiegato - è possibile dichiarare $(\lambda_1,...,\lambda_k)$ come Special Order Set del secondo tipo (SOS2) \cite{bealeTomlin}, cioè un insieme ordinato di variabili utilizzato per specificare  determinate condizioni in problemi di ottimizzazione, e il risolutore sfrutterà questa informazione per ricercare una soluzione ottimale in modo più efficiente (in pratica, sapere che una variabile appartiene ad un certo insieme ordinato consente di usare in modo più intelligente gli algoritmi di branch-and-bound del solver).

\subsection{IMPLEMENTAZIONE IN R}

Le informazioni necessarie per poter inserire le equazioni (\ref{eq:aprroxEqX}), (\ref{eq:aprroxEqY}) e (\ref{eq:aprroxEqLambda}) all'interno del modello a vincoli sono le coordinate dei punti di campionamento. Per trovarle ci siamo serviti del precedentemente introdotto R e in particolare del pacchetto software \emph{Segmented} \cite{segmentedPackage}. Grazie ad esso è stato molto semplice trovare un'ottima approssimazione lineare per le funzioni che legavano il budget alla produzione energetica (una per ogni tipo di incentivo), come si può facilmente osservare nel codice qui presentato.

\lstset{language=R}
\begin{lstlisting}
> library(segmented)
> #  ...
> # inserisci i dati delle simulazioni in apposite strutture 
> #  ...
> # operazioni varie (ordina dati, etc.)
> #  ...
> # estrai un modello lineare a tratti per l'incentivo Conto Interessi
> modelloLineareATratti_CI <- segmented(modelloGrezzo_CI,seg.Z=~ Budget,psi=c(3))
> # estrai un modello lineare a tratti per l'incentivo Fondo Garanzia
> modelloLineareATratti_FG <- segmented(modelloGrezzo_FG,seg.Z=~ Budget,psi=c(12,30))
> #  ...
> # incentivi restanti
> #  ...
\end{lstlisting}

Una volta ricavate le approssimazioni lineari a tratti delle funzioni, è possibile visualizzare il risultato ottenuto, come riportato in Figura~\ref{incentCompare_piecewise}.

\begin{figure}[hbt]
	\centering
	\includegraphics[scale=0.6]{incentComparePiecewise}
	\caption{Confronto tra i diversi incentivi - Approssimazione lineare a tratti}
	\label{incentCompare_piecewise}
\end{figure}

\section{INTEGRAZIONE MODELLO}

\subsection{VARIABILI}

\subsection{VINCOLI}

\section{ASSEGNAZIONE FONDI}

\myparagraph{FONDO INCENTIVI – 2.5 MILIONI EURO}

\myparagraph{FONDO INCENTIVI – 5 MILIONI EURO}

\myparagraph{FONDO INCENTIVI – 10 MILIONI EURO}

\myparagraph{FONDO INCENTIVI – 15 MILIONI EURO}

\myparagraph{FONDO INCENTIVI – 20 MILIONI EURO}

\myparagraph{FONDO INCENTIVI – 40 MILIONI EURO}

\nocite{*}
\bibliographystyle{plain}
\bibliography{bibliography}

\end{document}


%Andrea Borghesi
%Università degli studi di Bologna

%conclusioni

%\documentclass[12pt,a4paper,openright,twoside]{book}
%\usepackage[italian]{babel}
%\usepackage{indentfirst}
%\usepackage[utf8]{inputenc}
%\usepackage[T1]{fontenc}
%\usepackage{fancyhdr}
%\usepackage{graphicx}
%\usepackage{titlesec,blindtext, color}
%\usepackage[font={small,it}]{caption}

%\begin{document}

\clearpage{\pagestyle{empty}\cleardoublepage}
\chapter*{Conclusioni}

\markboth{Conclusioni}{Conclusioni}
\addcontentsline{toc}{chapter}{Conclusioni}

In questo lavoro abbiamo considerato le problematiche relative alla realizzazione di un sistema per il supporto alle decisioni in grado di fornire ausilio ai decisori politici nel loro compito di effettuare scelte e prendere decisioni per conseguire determinati obiettivi, nel rispetto dei vincoli economici, ambientali e sociali. Il nostro scopo principale è stato quindi quello di ideare tecniche e metodologie (provenienti anche da diversi ambiti di ricerca) con le quali fosse possibile affrontare in modo efficacie le sfide poste durante la pianificazione e implementazione delle politiche, fornendo al tempo stesso uno strumento informatico che potesse essere utilizzato dai decisori politici stessi. 

Muovendoci all'interno dell'ambito del progetto europeo e-Policy, ci siamo occupati in particolare di studiare il comportamento di cittadini e imprenditori ai quali fossero offerti diversi strumenti incentivanti per la produzione di energia elettrica attraverso l'impiego di impianti fotovoltaici; questo studio è stato effettuato implementando un simulatore ad agenti in grado di ricreare la prospettiva economica e sociale dei singoli investitori e osservandone poi l'evoluzione nell'arco di un periodo temporale significativo. Accanto a questo primo elemento, un altro obiettivo raggiunto è stata la realizzazione di un modello matematico, sulla base del paradigma della programmazione a vicoli, tramite il quale fosse possibile ideare un piano energetico per la regione Emilia-Romagna. Il terzo aspetto su cui ci siamo concentrati è costituito dall'interazione tra le due componenti appena citate, ovvero abbiamo fatto in modo che dai risultati ottenuti tramite il simulatore potessero essere ricavate delle informazioni con le quali fosse possibile estendere e arricchire il modello a vincoli iniziale, garantendo quindi un'integrazione tra il livello globale considerato dalla fase di ottimizzazione e quello locale (cioè basato sul comportamento dei singoli individui/agenti) delle simulazioni.

I risultati principali che abbiamo ottenuto con questo lavoro sono stati: 
\begin{itemize}
\item l'analisi dettagliata e rigorosa delle relazioni tra le variabili in gioco all'interno del simulatore, che ci ha permesso di comprendere in che modo i cambiamenti di parametri come la disponibilità di fondi per i meccanismi incentivanti abbiano ripercussioni sul comportamento degli agenti;
\item l'apprendimento di vincoli in grado di esprimere tali tali relazioni e l'inserimento di tali vincoli all'interno del modello matematico che si occupa della pianificazione regionale, consentendo così alle varie componenti del sistema di supporto alle decisioni di interagire in modo efficacie.
\end{itemize}

Nonostante il fatto che gli scopi che ci fossimo prefissati siano stati raggiunti, il sistema sviluppato presenta certamente ancora qualche limite ed è suscettibile a diversi tipi di miglioramento prima di diventare uno strumento completo e pienamente sfruttabile dai decisori politici per la loro attività, infatti la ricerca prosegue in tutti gli ambiti coinvolti nel progetto e-Policy. Passiamo ora a illustrare possibili limiti e sviluppi futuri per le parti pertinenti a questo lavoro.
\\*

In primo luogo è utile sottolineare nuovamente che il simulatore implementato presenta al suo interno diverse assunzioni e approssimazioni effettuate per semplificarne l'implementazione e, pur consentendo al tempo stesso di effettuare uno studio accurato delle proprietà interessanti nell'ambito di questo lavoro, sarà quindi necessario migliorarlo. La direzione da seguire è quella di renderlo più realistico (potenziare la fase di valutazione di fattibilità degli investimenti compiuta dagli agenti), estenderlo affinché rifletta in maniera più accurata le dinamiche della società modellata (interazione sociale più realistica) e consenta di valutare il comportamento di ulteriori metodologie di incentivazione (implementare un meccanismo ad asta), modificarne i parametri in modo che produca in uscita valori ``reali'' (ad esempio la produzione energetica totale ottenuta da energia fotovoltaica è ora nell'ordine di grandezza di poche decine di MW, mentre nella realtà per la regione dell'Emilia-Romagna le grandezze in gioco siano più verso le centinaia di MW). Sempre per quanto riguarda il simulatore, una modifica molto importante sarà quella di consentire di simulare la presenza contemporanea di diversi meccanismi incentivanti, per osservare come le reciproche interazioni possano influenzare il risultato finale.

Un altro aspetto molto importante su cui sarà utile intervenire è quello riguardante le interazioni tra le componenti del sistema e-Policy, con un riferimento particolare al rapporto tra la fase di ottimizzazione e il simulatore. Come già accennato nel quinto capitolo, l'integrazione di queste componenti può essere ottenuta attraverso diverse tecniche, delle quali solamente una è stata concretamente implementata in questo lavoro. Da ciò segue che possibili sviluppi futuri dovrebbero andare nella direzione di sperimentare metodologie diverse per conseguire un'interazione efficiente e suggeriamo che l'impiego di metodi provenienti da molteplici aeree di ricerca potrebbe portare quasi sicuramente vantaggi per affrontare questa sfida.
\\*

Questo approccio multidisciplinare appena citato è forse una delle aspetti più importanti a caratterizzare il progetto e-Policy (sicuramente un elemento che, parlando a titolo personale, ha reso più interessante e affascinante affrontare le sfide presentatecisi), poiché, come abbiamo visto, realizzare un sistema in grado di fornire supporto alle decisioni in un settore altamente complesso come l'ideazione e l'implementazione delle politiche, è un compito che richiede l'utilizzo di molteplici competenze, tecniche e strumenti proprio a causa della natura intrinsecamente complessa della materia trattata. 

%\end{document}


\appendix
%Andrea Borghesi
%Università degli studi di Bologna

% appendice sulla programmazione logica a vincoli

\clearpage{\pagestyle{empty}\cleardoublepage}
\chapter{Esempi di CLP} 
\label{appendiceCLP} 

Illustreremo ora due esempi di modellazione di problemi a vincoli sfruttando il linguaggio ECL$^i$PS$^e$ (nel primo caso considerando domini finiti e nel secondo valori reali, avvalendoci anche della libreria Eplex), per mostrare come possono essere strutturati i problemi di programmazione logica a vincoli; in questa trattazione supporremo noti i concetti elementari della programmazione logica (procedimenti risolutivi, definizioni di un termine, etc.), la cui discussione esula da questo lavoro.

\myparagraph{Esempio \clpfd}
Il cosiddetto \emph{Send More Money} puzzle è un esempio classico di programmazione a vincoli; le variabili $[S,E,N,D,M,O,R,Y]$ rappresentano cifre da 0 a 9 e lo scopo è assegnare alle variabili valori diversi in modo che l'operazione aritmetica di Figura~\ref{SendMoreMoney} risulti corretta - inoltre i numeri devono essere ben formati, da cui $S>0$ e $M>0$. 

\begin{figure}[h]
	\centering
	\includegraphics[width=0.215\textwidth]{sendMoreMoney}
	\caption{Send More Money puzzle}
	\label{SendMoreMoney}
\end{figure}

Con la programmazione convenzionale si avrebbe necessità di esprimere una strategia di ricerca in modo esplicito (senza contare possibili ottimizzazioni come cicli innestati), mentre con linguaggi logici come Prolog verrebbe sfruttata la ricerca fornita dal risolutore interno (il motore inferenziale), con il vantaggio di una programmazione estremamente facilitata ma col rischio di un'efficienza non elevata - a meno di programmi ottimizzati, i quali richiederebbero comunque maggiori tempo e abilità. 

Questo è in effetti il campo di applicazione ideale della programmazione logica a vincoli, in particolare nell'ambito  dei domini finiti \clpfd: le variabili possono assumere valori appartenenti ad un insieme finito di numeri interi, i vincoli sono facilmente esprimibili formalmente e occorre effettuare una certa quantità di ricerca nello spazio delle soluzioni. In questo problema sarebbe naturale usare le variabili del programma per rappresentare le diverse cifre e la soluzione finale dovrà essere un assegnamento di un valore unico per ogni variabile. 

Risolvere questo problema con Prolog comporta l'utilizzo della strategia di ricerca chiamata \emph{Generate and Test}, che prevede che prima la generazione di una soluzione e poi la verifica della consistenza dei vincoli e, nel caso che questa dia esito negativo, l'assegnamento di nuovi valori alle variabili seguita da nuova verifica e così via. In questo modo l'esplorazione dello spazio delle soluzioni è chiaramente inefficiente - per esempio la possibile implementazione in Prolog mostrata qui sotto, per quanto suscettibile a miglioramenti, deve gestire $\frac{10!}{2}$ possibili assegnamenti di valori alle variabili. 

\lstset{language=Prolog}
\begin{lstlisting}
% Send More Money puzzle in Prolog
smm :-
        X = [S,E,N,D,M,O,R,Y],           % variabili
        Digits = [0,1,2,3,4,5,6,7,8,9],	 % domini
        
        % predicato che assegna una soluzione
        assign_digits(X, Digits),
       	
       	%  verifica dei vincoli vincoli
        M > 0, 
        S > 0,
                  1000*S + 100*E + 10*N + D +
                  1000*M + 100*O + 10*R + E =:=
        10000*M + 1000*O + 100*N + 10*E + Y,
        write(X).

select(X, [X|R], R).
select(X, [Y|Xs], [Y|Ys]):- select(X, Xs, Ys).

assign_digits([], _List).
assign_digits([D|Ds], List):-
        select(D, List, NewList),
        assign_digits(Ds, NewList).
\end{lstlisting}

L'implementazione realizzata con ECL$^i$PS$^e$ presenta i vantaggi di semplificare ulteriormente la modellazione del problema e di appoggiarsi all'efficiente risolutore interno per l'esplorazione dello spazio delle soluzioni, in modo particolare il fatto che ogni volta che una variabile viene istanziata i vincoli vengono propagati per eliminare a priori strade inconsistenti, riducendo gli spazi delle soluzioni e prevenendo fallimenti sicuri. 

\begin{lstlisting}
% Send More Money puzzle in ECLiPSe
smm :-
     X = [S,E,N,D,M,O,R,Y],		% variabili
     X :: [0 .. 9],				% domini finiti
     
     % vincoli
     M #> 0,
     S #> 0,
               1000*S + 100*E + 10*N + D +
               1000*M + 100*O + 10*R + E #=
     10000*M + 1000*O + 100*N + 10*E + Y,
     alldistinct(X),
     
     % ricerca della soluzione
     labeling(X),
     write(X).

\end{lstlisting}

\myparagraph{Esempio \clpr - Eplex}

Presentiamo ora un esempio di un problema (Fig.~\ref{clpR_example}) che rientra nell'ambito dei \clpr e che fa uso della libreria Eplex, tratto dal manuale di ECL$^i$PS$^e$ \cite{eclipseTut}. Ci sono tre impianti, o fabbriche, (1-3) in grado di produrre un certo prodotto con capacità diverse e i cui prodotti devono essere trasportati a quattro clienti (A-D) con quantità richieste diverse; anche il costo unitario di trasporto ai clienti è variabile. L'obiettivo del problema è minimizzare i costi di trasporto soddisfacendo le esigenze dei clienti. 

\begin{figure}[h]
	\centering
	\includegraphics[scale=0.6]{clpR_example}
	\caption{Esempio di un problema \clpr. Fonte \cite{eclipseTut}}
	\label{clpR_example}
\end{figure}

Per formulare il problema definiamo la quantità di prodotto trasportata dall'impianto $N$ al cliente $p$ come variabile $N_p$ - ad esempio $A_1$ rappresenta il costo di trasporto dalla fabbrica $A$ al cliente $1$. I vincoli da considerare sono di due tipi (sempre facendo riferimento alla Figura~\ref{clpR_example}):
\begin{itemize}
\item La quantità di prodotto consegnata da tutti gli impianti a un cliente deve essere uguale alla domanda del cliente, ad esempio per il cliente $A$ che può essere rifornito dagli impianti 1-3, abbiamo che $A_1+A_2+A_3=21$
\item La quantità di prodotto in uscita da una fabbrica non può essere superiore alla sua capacità produttiva, ad esempio per l'impianto $1$ che invia prodotti ai clienti A-D si ha che $A_1+B_1+C_1+D_1 \leq 50$
\end{itemize}   
Poiché lo scopo è minimizzare i costi di trasporto, la funzione obiettivo è di minimizzare i costi combinati del trasporto dei prodotti dai tre impianti a tutti e quattro i clienti.

La formulazione del problema è quindi la seguente.\\*
Funzione obiettivo: 
\begin{equation}  \label{exampleClpRObiett}
	\min(10A_1+7A_2+200A_3+8B_1+5B_2+10B_3+5C_2+5C_2+8C_3+9D_1+3D_2+7D_3)
\end{equation}
Vincoli:
\begin{equation}  \label{exampleCons1}
	A_1+A_2+A_3=21
\end{equation}
\begin{equation}  \label{exampleCons2}
	B_1+B_2+B_3=40
\end{equation}
\begin{equation}  \label{exampleCons3}
	C_1+C_2+C_3=34
\end{equation}
\begin{equation}  \label{exampleCons4}
	D_1+D_2+D_3=10
\end{equation}
\begin{equation}  \label{exampleCons5}
	A_1+B_1+C_1+D_1 \leq 50
\end{equation}
\begin{equation}  \label{exampleCons6}
	A_2+B_2+C_2+D_2 \leq 30
\end{equation}
\begin{equation}  \label{exampleCons7}
	A_3+B_3+C_3+D_3 \leq 40
\end{equation}

Mostriamo ora come questo problema venga modellato sfruttando la libreria Eplex. In primo luogo occorre caricare la libreria Eplex di cui si dispone (in questo caso abbiamo sfruttato un risolutore esterno open source) e ottenerne un'\emph{istanza}, la quale rappresenta un singolo problema sotto forma di modulo, a cui possono essere riferiti vincoli e funzione obiettivo consentendo quindi al solver esterno di risolvere il problema. Il codice che segue mostra come il problema di Figura~\ref{clpR_example} sia stato trasposto all'interno di ECL$^i$PS$^e$.

\begin{lstlisting}
:- lib(eplex).		% caricamento della libreria Eplex
:- eplex_instance(prob).		% definizione dell'istanza - chiamata 'prob'

main(Cost, Vars) :-
		% dichiarazione delle variabili e definizione del loro dominio		
		Vars = [A1,A2,A3,B1,B2,B3,C1,C2,C3,D1,D2,D3],
		prob: (Vars $:: 0.0..1.0Inf),  % valori maggiori o uguali a 0
		
		% definizione dei vincoli applicati all'istanza eplex
		prob: (A1 + A2 + A3 $= 21), 
		prob: (B1 + B2 + B3 $= 40),
		prob: (C1 + C2 + C3 $= 34),
		prob: (D1 + D2 + D3 $= 10),

		prob: (A1 + B1 + C1 + D1 $=< 50),
		prob: (A2 + B2 + C2 + D2 $=< 30),
		prob: (A3 + B3 + C3 + D3 $=< 40),

		% inizializza il solver esterno con la funzione obiettivo
		prob: eplex_solver_setup(min(10*A1 + 7*A2 + 200*A3 + 
			8*B1 + 5*B2 + 10*B3 +
		 	5*C1 + 5*C2 + 8*C3 +
		 	9*D1 + 3*D2 + 7*D3)),

		% ---------- Fine Modellazione ----------

		% risoluzione del problema
		prob: eplex_solve(Cost).
\end{lstlisting}

Per usare un'istanza Eplex occorre prima dichiararla con \emph{eplex\_instance/1}; una volta dichiarata, l'istanza viene riferita tramite il nome specificato. 

Come primo passo creiamo le variabili del problema e imponiamo che possano assumere solamente valori non negativi e rendiamo noti all'istanza il loro dominio (\emph{\$::/2}). Successivamente imponiamo i vincoli che modellano il problema sotto forma di uguaglianze e disuguaglianze aritmetiche; per via del solver esterno scelto, gli unici tipi di vincoli accettati sono quelli lineari - che ovviamente consentono una maggiore efficienza nella risoluzione.

Occorre poi inizializzare il risolutore esterno con l'istanza eplex creata, in modo che questa possa essere risolta. Questo è fatto dal \emph{eplex\_solver\_setup/1}, che prende come argomento la funzione obiettivo, la quale può essere di minimizzazione o massimizzazione. Infine è possibile risolvere il problema modellato attraverso \emph{eplex\_solve/1}.

Quando un'istanza viene risolta, il solver prende in considerazione tutti i vincoli ad essa relativi, i valori che le variabili del problema possono assumere e la funzione obiettivo specificata. In questo caso è possibile ottenere una soluzione ottimale pari a 710.0: 
\begin{lstlisting}
?-	main(Cost, Vars).

Cost = 710.0
Vars = [A1{0.0 .. 1e+20 @ 0.0}, A2{0.0 .. 1e+20 @ 21.0}, ....]
\end{lstlisting}


\backmatter
\nocite{*}
\bibliographystyle{plain}
\bibliography{bibliography}
\addcontentsline{toc}{chapter}{Bibliografia}

\end{document}
