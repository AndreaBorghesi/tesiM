%Andrea Borghesi
%Università degli studi di Bologna

% presentazione tesi laurea magistrale
\documentclass{beamer}
\usetheme{Warsaw}
\usecolortheme{whale}
\usecolortheme{orchid}

\usepackage[italian]{babel}
\usepackage{indentfirst}
\usepackage[utf8]{inputenc}
\usepackage[T1]{fontenc}
\usepackage{fancyhdr}
\usepackage{graphicx}
\usepackage[font={small,it}]{caption}
\usepackage{eurosym}
\usepackage{hyphenat}

%cartelle contenenti le immagini
\graphicspath{{/media/sda4/tesi/immagini/grafici/}{/media/sda4/tesi/immagini/grafici/incCompare/}{/media/sda4/tesi/immagini/grafici/rawData/}{/media/sda4/tesi/immagini/grafici/regressionAnalysis/}{/media/sda4/tesi/immagini/schemi/}{/media/sda4/tesi/immagini/simulazione/}{/media/sda4/tesi/immagini/epolicy/}{/media/sda4/tesi/immagini/ottimizzazione/}
{/media/sda4/tesi/immagini/interazione/}}

\newcommand{\clpr}{CLP({\ensuremath{\cal R}})}

\author[Andrea Borghesi]{Candidato:\qquad\qquad Andrea Borghesi}
\title[Tesi LM-Ing.Inf. \hspace{25mm} \insertframenumber/\inserttotalframenumber]{Integrazione di ottimizzazione e simulazioni per il piano energetico regionale dell'Emilia-Romagna}
\date{}

\begin{document}

	\begin{frame}[plain]
		\frametitle{\centerline{Alma Mater Studiorum - Universit\`a degli Studi di Bologna}}
		\framesubtitle{\centerline{Facolt\`a di Ingegneria}}
		\begin{center}
		   	{\small Corso di Laurea in Ingegneria Informatica M} \medskip\\
		  	{\small \em Tesi di Laurea in Sistemi Intelligenti\\} 
		\end{center}
		\titlepage
		\begin{center}
			\begin{tabular}{lcr}
				{Relatore:}&\qquad\qquad\qquad&{$Prof.^{ssa}$ Michela Milano}\\
				{Correlatore:}& & {$Prof.$ Marco Gavanelli}\\
			\end{tabular}
		\end{center}		
	\end{frame}

\section*{Outline}
\begin{frame}
	\tableofcontents
\end{frame}


\section{Introduzione}

	\begin{frame}
		\frametitle{I Processi Politici}
		\begin{itemize}
		\item Le politiche pubbliche sono estremamente complesse, avvengono in ambienti che cambiano rapidamente caratterizzati da incertezza e coinvolgono conflitti tra diversi interessi.
		\item L'ICT può svolgere un importante ruolo di supporto in tutte le fasi della pianificazione (ideazione, implementazione, etc.) delle politiche.
		\item Ci concentreremo sulle politiche energetiche a livello regionale.
		\end{itemize}
 	\end{frame}
 	
\subsection{Il Progetto e-Policy}
	\begin{frame}
		\frametitle{Il Progetto e-Policy}
		\begin{itemize}
		\item Progetto EU 7 Programma Quadro - sotto l'obiettivo: ICT solutions for Governance and Policy Modeling.
		\item Consorzio: Università di Bologna (Coordinatore), Regione Emilia-Romagna, Università di Ferrara, ASTER e altri 5 partner da 4 paesi europei.
		\end{itemize}
		\begin{block}{Finalità}
			\begin{enumerate}
			\item Supportare i decisori politici nel loro lavoro, ingegnerizzare il ciclo di vita del processo di creazione delle politiche.
			\item Valutare gli impatti sociali, economici e ambientali durante lo sviluppo delle politiche (sia a livello globale che individuale).
			\item Stabilire i probabili effetti sociali attraverso opinion mining.
			\item Fornire strumenti di visualizzazione efficaci.
			\end{enumerate}
		\end{block}
 	\end{frame}
 	
\subsection{Schema Generale}
	\begin{frame}
		\frametitle{Schema Generale}
		\begin{columns}
		\column{.57\textwidth}
			\begin{figure}[hbt]
				\centering
				\includegraphics[scale=0.3]{epolicyLifeCycle}
				\label{epolicyLifeCycle}
			\end{figure}
		\column{.43\textwidth}
		\begin{itemize}
			\item Ottimizzazione e supporto alle decisioni per la pianificazione con una prospettiva globale.
			\item Simulazione ad agenti per la prospettiva individuale.
			\item Tecniche per l'integrazione dei due livelli.
			\item Opinion mining per valutare gli impatti sociali delle politiche.
		\end{itemize}
		\end{columns}
	\end{frame}
	
	\begin{frame}
		\frametitle{Architettura e-Policy}
		\begin{columns}
		\column{.5\textwidth}
		\begin{block}{}
			\begin{itemize}
				\item Approccio multidisciplinare per affrontare la complessità delle problematiche.
				\item Occorre considerare aspetti economici, sociali ed ambientali, accanto alle esigenze di natura informatica e ingegneristica.
				\item Per l'ottimizzazione e la simulazione abbiamo impiegato tecniche del campo dell'Intelligenza Artificiale.
			\end{itemize}
		\end{block}

		\column{.5\textwidth}
			\begin{figure}[hbt]
				\centering
				\includegraphics[scale=0.3]{epolicyScheme}
			\label{epolicyScheme}
		\end{figure}
		\end{columns}
  	\end{frame}
	
\subsection{Obiettivi Tesi}
	\begin{frame}
		\frametitle{Obiettivi}
		Il lavoro svolto e qui presentato si è inserito nell'ambito del progetto e-Policy con questi obiettivi:
		\begin{itemize}
			\item[\checkmark] studiare metodi per gestire l'interazione tra la fase di ottimizzazione e le simulazioni;
			\item[\checkmark] estendere il simulatore ad agenti per analizzare l'efficacia di diversi meccanismi incentivanti per la produzione elettrica da energia fotovoltaica in Emilia-Romagna;
			\item[\checkmark] estendere l'ottimizzatore affinché consideri i risultati ottenuti per mezzo del simulatore e assegni i fondi regionali per gli incentivi in maniera ottimale.
		\end{itemize}	
	\end{frame}
  
  
\section{Descrizione Progetto}
  	
\subsection{Ottimizzazione}
	\begin{frame}
		\frametitle{Ottimizzazione}
		\begin{block}{}
			La pianificazione delle attività regionali è un complesso problema di ottimizzazione combinatoria:
			\begin{itemize}
				\item devono essere prese decisioni che soddisfino un insieme di vincoli (costo, impatti ambientali e sociali, etc.), 
				\item si cerca di raggiungere determinati obiettivi (minimizzare le spese, massimizzare la produzione energetica, etc.).
			\end{itemize}
		\end{block}
		\begin{alertblock}{\clpr}
			La natura del problema ha suggerito un approccio basato sulla Programmazione Logica a Vincoli, in particolare sul dominio dei reali (\clpr)
		\end{alertblock}
  	\end{frame}
  	\begin{frame}
		\frametitle{Modellazione Problema}
		\framesubtitle{Pianificazione Energetica Regionale}
		\begin{block}{}
			\begin{enumerate}
				\item La regione può finanziare attività in grado di garantire una certa produzione energetica (o anche attività di supporto).
				\item Ogni attività ha degli impatti su diversi recettori ambientali.
				\item I legami tra le diverse attività e con i relativi impatti sono rappresentati tramite matrici coassiali.
				\item La regione cerca di soddisfare determinati obiettivi rispettando diversi vincoli.
			\end{enumerate}
		\end{block}
		\begin{alertblock}{\clpr}
			Le attività sono rappresentate da variabili legate da vincoli lineari.
		\end{alertblock}
  	\end{frame}
  	
\subsection{Simulazione}
	\begin{frame}
		\frametitle{Simulazione}
		\begin{block}{}
			\begin{itemize}
				\item La simulazione fornisce la prospettiva ``locale''.
				\item Consente di studiare le reazioni delle persone alle decisioni politiche prese.
				\item Permette di definire le migliori strategie implementative.
				\item Il modello implementato contiene semplificazioni e si concentra sulla produzione di energia elettrica con impianti fotovoltaici.
			\end{itemize}
		\end{block}
		\begin{alertblock}{Realizzare il Simulatore}
			L'approccio scelto è basato su un modello ad agenti - i fenomeni macro emergono a partire dalle interazioni tra singole entità.
		\end{alertblock}
  	\end{frame}
  	\begin{frame}
		\frametitle{Simulazione ad Agenti}
			\begin{enumerate}
				\item Il mondo reale viene simulato attraverso agenti interagenti.
				\item Ogni agente (privati cittadini, aziende,..) decide se effettuare l'investimento sulla base di considerazioni economiche e sociali.
				\item Si interviene sul modello impostando parametri generali - numero agenti, budget regionale, presenza di incentivi, etc.
				\item A fine simulazione sono forniti numerosi valori - produzione energetica totale, percentuale di agenti che ha effettuato l'investimento, spesa effettiva sostenuta dalla regione e così via.
			\end{enumerate}
  	\end{frame}
  	
\subsection{Interazione Componenti}
	\begin{frame}
		\frametitle{Interazione Componenti}
		\begin{itemize}
			\item \`E fondamentale gestire l'interazione tra l'ottimizzazione (prospettiva globale) e simulazioni (prospettiva individuale).
			\item Una stretta integrazione tra i componenti permette di considerare la migliore implementazione delle politiche (simulatore) già in fase di pianificazione (ottimizzatore).
			\item Sono possibili diversi approcci e metodologie, da svariati aree come l'IA, Ricerca Operativa, Teoria dei Giochi.
			\item Problema ancora nuovo e aperto, oggetto di intensa ricerca.
		\end{itemize}
  	\end{frame}
  	\begin{frame}
		\frametitle{Modello di Apprendimento Automatico}
		\begin{columns}
		\column{.5\textwidth}
		\begin{block}{Schema Interazione}
			\begin{figure}[htb]
				\centering
				\includegraphics[scale=0.35]{schemaIterLearn}
			  	\label{schemaIterLearn}
			\end{figure}
		\end{block}
		\column{.5\textwidth}
		\begin{block}{}
			\begin{enumerate}
				\item Dai dati delle simulazioni è possibile ricavare le relazioni d'interesse - ad esempio i finanziamenti agli incentivi e la produzione energetica totale.
				\item Le relazioni sono inserite all'interno del modello dell'ottimizzatore sotto forma di vincoli, sfruttati poi per generare un nuovo piano.
			\end{enumerate}
		\end{block}
		\end{columns}
		
	\end{frame}
	\begin{frame}
		\frametitle{Apprendimento Modelli}
		\begin{columns}
		\column{.5\textwidth}
		\begin{block}{}
			\begin{itemize}
				\item Sono state eseguite numerose simulazioni (circa 80k) per disporre di dati sufficienti.
				\item Le relazioni tra le grandezze considerate sono state ricavate con tecniche di regressione.
				\item Le curve di regressione ottenute sono state approssimate con funzioni lineari (maggiore efficienza).
			\end{itemize}
		\end{block}
		\column{.5\textwidth}
		\begin{block}{Comportamento Incentivi}
			\begin{figure}[hbt]
				\centering
				\includegraphics[scale=0.3]{incentComparePiecewise}
				\label{incentComparePiecewise}
			\end{figure}
		\end{block}
		\end{columns}
  	\end{frame}
  	\begin{frame}
		\frametitle{Integrazione Modello}
		\begin{columns}
		\column{.45\textwidth}
		\begin{block}{}
			\begin{enumerate}
				\item Sono considerati 4 tipi di incentivo regionale.
				\item Le relazioni apprese tra incentivi e produzione energetica sono inserite nel modello a vincoli.
				\item \`E quindi possibile ottimizzare l'assegnazione dei fondi ai vari incentivi (fissato un budget limite).
			\end{enumerate}
		\end{block}
		\column{.55\textwidth}
		\begin{block}{Assegnazione Fondi}
			\begin{table}[h]
				\centering
				\begin{tabular}{ p{0.25\textwidth}  | p{0.15\textwidth} | p{0.15\textwidth}  }
					\hline \hline 
					\nohyphens{\emph{Tipo Incentivo}} & \nohyphens{\emph{Costo (M\euro)}} & \nohyphens{\emph{Energia Prodotta (MW)}} \\ \hline
					Asta &  0.00 & 0.000 \\ 
					\nohyphens{Conto Interessi} & 2.53 & 3.606 \\ 
					Rotazione & 1.00 & 0.286 \\ 
					Garanzia & 6.47 & 0.816 \\ \hline 
					Totale & 10.00 & 4.708 \\
					\hline \hline 
				\end{tabular}
				\label{tab:assegnFondi10M}	
			\end{table}
		\end{block}
		\end{columns}
  	\end{frame}
  
\section{Conclusioni}
	\begin{frame}
		\frametitle{Conclusioni}
		\begin{alertblock}{}
			\begin{itemize}
				\item[\checkmark] Analisi accurata dell'efficacia di quattro tipi di meccanismi incentivanti regionali e dell'effetto della componente sociale sul comportamento degli agenti simulati.
				\item[\checkmark] Implementazione di una tecnica di integrazione tra ottimizzazione e simulatore.
				\item[\checkmark] Prototipo funzionante che consente di assegnare in modo ottimale i fondi regionali disponibili per l'incentivazione.
			\end{itemize}
		\end{alertblock}
		\begin{block}{Sviluppi Futuri}
			\begin{itemize}
				\item Arricchire il simulatore e cercare di renderlo più realistico (in stretta collaborazione con la regione).
				\item Studiare e implementare ulteriori metodi di interazione (decomposizione di Benders,..).
			\end{itemize}
		\end{block}
	\end{frame}
  
\end{document}
