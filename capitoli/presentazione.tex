%Andrea Borghesi
%Università degli studi di Bologna

% presentazione tesi laurea magistrale
\documentclass{beamer}
\usetheme{Warsaw}
%\setbeamercolor*{palette primary}{use=structure,fg=white,bg=green!50!gray}
%\setbeamercolor*{palette quaternary}{fg=white,bg=green!30!black}
\usecolortheme{whale}
\usecolortheme{orchid}


\usepackage[italian]{babel}
\usepackage{indentfirst}
\usepackage[utf8]{inputenc}
\usepackage[T1]{fontenc}
\usepackage{fancyhdr}
\usepackage{graphicx}
\usepackage[font={small,it}]{caption}

%cartelle contenenti le immagini
\graphicspath{{/media/sda4/tesi/immagini/grafici/}{/media/sda4/tesi/immagini/grafici/incCompare/}{/media/sda4/tesi/immagini/grafici/rawData/}{/media/sda4/tesi/immagini/grafici/regressionAnalysis/}{/media/sda4/tesi/immagini/schemi/}{/media/sda4/tesi/immagini/simulazione/}{/media/sda4/tesi/immagini/epolicy/}{/media/sda4/tesi/immagini/ottimizzazione/}
{/media/sda4/tesi/immagini/interazione/}}

\title[titolo breve]{Integrazione di ottimizzazione e simulazioni per il piano energetico regionale dell'Emilia-Romagna}
\author{Andrea Borghesi}

\begin{document}
	\begin{frame}
		\titlepage
	\end{frame}

\section*{Outline}
\begin{frame}
	\tableofcontents
\end{frame}

\section{Introduzione}
	\begin{frame}
		\frametitle{I Processi Politici}
		descrizione problematiche decisioni politiche
 	\end{frame}
\subsection{Il Progetto e-Policy}
	\begin{frame}
		\frametitle{Il Progetto e-Policy}
		progetto e-policy a grandissime linee (immagine?)
 	\end{frame}
\subsection{Schema Generale}
	\begin{frame}
		\frametitle{Schema Generale}
		rapida descrizione componenti
		\begin{figure}[hbt]
			\centering
			\includegraphics[scale=0.3]{epolicyScheme}
			\label{epolicyScheme}
		\end{figure}
	\end{frame}
\subsection{Obiettivi Tesi}
	\begin{frame}
		\frametitle{Obiettivi}
		obiettivi del MIO lavoro e come siano inseriti dentro il progetto più ampio	
	\end{frame}
  
\section{Architettura}
	\begin{frame}
		\frametitle{Architettura e-Policy}
		testo sezione
  	\end{frame}
\subsection{Ottimizzazione}
	\begin{frame}
		\frametitle{Ottimizzazione}
		descrizione ottimizzatore in generale - problematiche e scopi
  	\end{frame}
  	\begin{frame}
		\frametitle{Programmazione Logica a Vincoli}
		descrizione CLP - eclipse
  	\end{frame}
\subsection{Simulazione}
	\begin{frame}
		\frametitle{Simulazione}
		descrizione simulatore - prospettiva individuale economica e sociale
  	\end{frame}
  	\begin{frame}
		\frametitle{Simulazione ad Agenti}
		descrizione simulatori ad agenti - netlogo
  	\end{frame}
\subsection{Interazione Componenti}
	\begin{frame}
		\frametitle{Interazione Componenti}
		Interazione tra ottimizzatore e simulatore
  	\end{frame}
  	\begin{frame}
		\frametitle{Modello di Apprendimento Automatico}
		modello basato sull apprendimento automatico
		\begin{figure}[htb]
			\centering
			\includegraphics[scale=0.4]{schemaIterLearn}
		  	\label{schemaIterLearn}
		\end{figure}
	\end{frame}
	\begin{frame}
		\frametitle{Apprendimento Modelli}
		descrizione regressione e simulazioni (R)
  	\end{frame}
  	\begin{frame}
		\frametitle{Apprendimento Modelli}
		\framesubtitle{Esempio - Comportamento Incentivi}
		\begin{figure}[hbt]
			\centering
			\includegraphics[scale=0.4]{incentCompare}
			\label{incentCompare}
		\end{figure}
  	\end{frame}
  
\section{Conclusioni}
	\begin{frame}
		\frametitle{Conclusioni}
		Conclusioni e sviluppi futuri
	\end{frame}
  
\end{document}
