%Andrea Borghesi
%Università degli studi di Bologna

% presentazione tesi laurea magistrale

\documentclass[12pt,a4paper,openright,twoside]{book}
\usepackage[italian]{babel}
\usepackage{indentfirst}
\usepackage[utf8]{inputenc}
\usepackage[T1]{fontenc}
\usepackage{fancyhdr}
\usepackage{graphicx}
\usepackage{titlesec,blindtext, color}
\usepackage[font={small,it}]{caption}

\documentclass{beamer}
\usetheme{Warsaw}

\begin{document}
  \begin{frame}
    \frametitle{This is the first slide}
    Le politiche pubbliche sono estremamente complessi, avvengono in ambienti che cambiano rapidamente caratterizzati da incertezza e coinvolgono conflitti tra diversi interessi. La nostra società è sempre più complessa a causa della globalizzazione, dell'ampliamento e del cambiamento delle situazioni geopolitiche. Questo implica che l'attività politica e la sua area di intervento si siano estese, rendendo più difficili da determinare gli effetti di tali interventi, mentre al tempo stesso diventa sempre più importante assicurarsi che le azioni intraprese affrontino in maniera efficace le sfide reali che la crescente complessità comporta.
  \end{frame}
  \begin{frame}
    \frametitle{This is the second slide}
    \framesubtitle{A bit more information about this}
    Questo implica che l'attività politica e la sua area di intervento si siano estese, rendendo più difficili da determinare gli effetti di tali interventi, mentre al tempo stesso diventa sempre più importante assicurarsi che le azioni intraprese affrontino in maniera efficace le sfide reali che la crescente complessità comporta.
  \end{frame}
% etc
\end{document}
