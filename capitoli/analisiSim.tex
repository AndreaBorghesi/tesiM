%Andrea Borghesi
%Università degli studi di Bologna

%capitolo dedicato all'analisi dei risultati delle simulazioni 

\documentclass[12pt,a4paper,openright,twoside]{report}
\usepackage[italian]{babel}
\usepackage{indentfirst}
\usepackage[utf8]{inputenc}
\usepackage[T1]{fontenc}
\usepackage{fancyhdr}
\usepackage{graphicx}
\usepackage{titlesec,blindtext, color}
\usepackage[font={small,it}]{caption}
\usepackage{subfig}
\usepackage{listings}
\usepackage{color}
\usepackage{url}
\usepackage{textcomp}

%impostazioni generali per visualizzare codice
\definecolor{dkgreen}{rgb}{0,0.6,0}
\definecolor{gray}{rgb}{0.5,0.5,0.5}
\definecolor{mauve}{rgb}{0.58,0,0.82}
 
\lstset{ %
  basicstyle=\footnotesize,           % the size of the fonts that are used for the code
  backgroundcolor=\color{white},      % choose the background color. You must add \usepackage{color}
  showspaces=false,               % show spaces adding particular underscores
  showstringspaces=false,         % underline spaces within strings
  showtabs=false,                 % show tabs within strings adding particular underscores
  tabsize=2,                      % sets default tabsize to 2 spaces
  breaklines=true,                % sets automatic line breaking
  breakatwhitespace=false,        % sets if automatic breaks should only happen at whitespace
  title=\lstname,                   % show the filename of files included with \lstinputlisting;
                                  % also try caption instead of title
  keywordstyle=\color{blue},          % keyword style
  commentstyle=\color{dkgreen},       % comment style
  stringstyle=\color{mauve},         % string literal style
  escapeinside={\%*}{*)},            % if you want to add LaTeX within your code
  morekeywords={*,...},              % if you want to add more keywords to the set
  deletekeywords={...}              % if you want to delete keywords from the given language
}

%per avere un bordo intorno alle figure
\usepackage{float}
\floatstyle{boxed} 
\restylefloat{figure}

%per poter poi impedire che certe parole vadano a capo
\usepackage{hyphenat}
\usepackage{listings}

%ridefinisco font per fancyhdr, per ottenere un'intestazione pulita
\newcommand{\changefont}{ \fontsize{9}{11}\selectfont }
\fancyhf{}
\fancyhead[LE,RO]{\changefont \slshape \rightmark} 	%section
\fancyhead[RE,LO]{\changefont \slshape \leftmark}	%chapter
\fancyfoot[C]{\changefont \thepage}					%footer

%titolo capitolo con "numero | titolo"
\definecolor{gray75}{gray}{0.75}
\newcommand{\hsp}{\hspace{20pt}}
\titleformat{\chapter}[hang]{\Huge\bfseries}{\thechapter\hsp\textcolor{gray75}{|}\hsp}{0pt}{\Huge\bfseries}


\oddsidemargin=30pt \evensidemargin=20pt

%sillabazioni non eseguite correttamente
\hyphenation{sil-la-ba-zio-ne pa-ren-te-si si-mu-la-to-re ge-ne-ra-re pia-no}

%interlinea
\linespread{1.15}  
\pagestyle{fancy}

%cartelle contenenti le immagini
\graphicspath{{/media/sda4/tesi/immagini/grafici/}{/media/sda4/tesi/immagini/schemi/}{/media/sda4/tesi/immagini/simulazione/}}

%in modo che dopo il titolo di un paragrafo il testo vada a capo
\newcommand{\myparagraph}[1]{\paragraph{#1}\mbox{}\\}

\begin{document}
\chapter{RISULTATI SIMULAZIONI}

Nel capitolo precedente abbiamo descritto il modello ad agenti implementato, evindenziandone le finalità e le caratteristiche fondamentali, accennando brevemente alle informazioni ricavabili dal simulatore.\\*
Lo scopo di questo capitolo sarà quindi la dettagliata analisi dei dati prodotti dalle simulazioni, l'individuazione delle relazioni che legano le grandezze in gioco all'interno dell'ambiente simulato, la presentazione e discussione dei risultati ottenuti.\\*
Inizialmente presenteremo gli strumenti utilizati per effettuare l'analisi sopra descritta, per poi passare alla discussione vera e propria nei paragrafi successivi.

\section{STRUMENTI}
Per esaminare i dati prodotti dalle simulazioni effettuate e visualizzare i risulatati ottenuti abbiamo utilizzato \emph{R}, un ambiente di sviluppo specifico per l'analisi statistica dei dati, basato sull'omonimo linguaggio di programmazione.

\subsection{R}

R è un linguaggio di programmazione open source e un ambiente software usato per la manipolazione di dati, calcolo e analisi statistica e presentazione grafica dei risultati. Il design R è stato ampiamente influenzato da due linguaggi preesistenti, S sviluppato da J.Chambers e colleghi presso i Bell Laboratories negli anni '70 e Scheme creato presso il MIT AI Lab sempre negli anni settanta da G.L.Steele e G.J.Sussman. \\*
Il nucleo di R consiste di un linguaggio interpretato a cui sono state aggiunte numerose funzionalità per un grande numero di procedure statistiche;  tra queste è possibile ricordarne alcune: modelli di regressione lineare, lineare generalizzata e non lineare, analisi di serie temporali, classici test parametrici e non, clustering, classificazione e altre. R è facilmente estendibile grazie alla presenza di numerosi pacchetti software creati dagli utenti e dedicati a specifiche aree di studio e possiede inoltre un grande insieme di funzioni indicate per una presentazione flessibile ed efficiente dei dati e la produzione di grafici di qualità.\\*
Per interagire con l'interprete del linguaggio R è possibile fornire le istruzioni direttamente da riga di comando oppure appoggiarsi a interfacce grafiche, ma per le nostre necessità è stato sufficiente utilizzare la riga di comando\\* \\* 
Per via della sua derivazione da S, R presenta alcune caratteristiche che lo fanno rientrare all'interno del paradigma dei linguaggi Object Oriented, almeno parzialmente, e al tempo stesso possiede alcuni aspetti che lo avvicinano alla natura dei linguaggi funzionali(come Scheme), come ad esempio la possibilità di trattare le funzioni stesse come oggetti.\\* Le principali strutture dati sono le seguenti: \begin{itemize}
\item \emph{vettori}, singole entità costituite da una collezione di valori di un certo tipo come ad esempio numerici,logici o caratteri;
\item \emph{matrici (arrays)}, generalizzazioni multi-dimensionali di vettori;
\item \emph{liste}, forme di vettori più generali nelle quali gli elementi non devono necessariamente essere dello stesso tipo;
\item \emph{fattori}, oggetti simili ai vettori usati per specificare una classificazione (raggruppamento) delle componenti di altri vettori con la stessa lunghezza;
\item \emph{data frames}, strutture simili alle matrici in cui le colonne possono essere di tipi diversi;
\item \emph{funzioni}, le quali sono esse stesse oggetti e forniscono così un modo semplice e flessibile di estendere R.
\end{itemize}
Come in ogni linguaggio di programmazione è poi ovviamente possibile manipolare queste strutture dati attraverso operatori, strutture di controllo, funzioni, etc...\\* \\*
Illustreremo ora un brevissimo esempio per far capire un possibile utilizzo di R per effettuare una semplice analisi statistica. Supponiamo di voler studiare la relazione che lega due variabili, \emph{a} e \emph{b}, i cui valori si trovano in un file di tipo \emph{Comma Separated Values}. Il primo passo è importare tali valori dal file e inserirli in una struttura dati, in questo caso una matrice con due colonne (una per ogni variabile) e ordinarli in base ai valori della prima variabile.

\lstset{language=R}
\begin{lstlisting}
> matrice.dati <- read.csv("file.csv")
> matrice.ordinata <- matrice_dati[order(matrice.dati$a),]
\end{lstlisting}

A questo punto sarebbe possibile svolgere diverse operazioni sui dati (ad esempio calcolare per ogni valore di ogni variabile i valori medi,...) ma ci limiteremo a effettuare una semplice regressione lineare.
\begin{lstlisting}
> modello.lineare <- lm(matrice$b ~ matrice$a)
\end{lstlisting}

R ci consente ora di effettuare analisi statistiche sul modello di regressione applicato per stabilirne validità e significatività in rapporto ai dati in nostro possesso e successivamente di presentare graficamente i risultati ottenuti.
\begin{lstlisting}
> #analisi statistica minima
> summary(modello.lineare)       
> #disegna i punti corrispondenti ai valori nella matrice
> plot(matrice$b ~ matrice$a,type="p",lwd=3,ylab="b",xlab="a")    
> #disegna la curva di regressione
> lines(matrice$a,predict(modello.lineare), lty="solid", col="darkred", lwd=2)    
\end{lstlisting}


In Figura ~\ref{example_r} sono stati riportati il grafico prodotto da questo esempio e i risultati ottenuti dalla semplicissima analisi statistica, tra i quali notiamo il coefficiente di determinazione (R-squared) e l'errore residuo ( 
Residual standard error).

\begin{figure}[H]
	\centering
	\subfloat[Grafico]{\includegraphics[scale=0.4]{example_r_graph}\label{example_r_graph}} 
	\quad
	\subfloat[Analisi statistica]{\includegraphics[scale=0.6]{example_r_result}\label{example_r_result}}
	\caption{Esempio di utilizzo di R}
	\label{example_r}
\end{figure}



\section[ANALISI RISULTATI]{ANALISI RISULTATI SIMULAZIONI}

\subsection{COMPORTAMENTO DEGLI INCENTIVI}

\myparagraph{Fondo Asta}

\myparagraph{Conto Interessi}

\myparagraph{Fondo Rotazione}

\myparagraph{Fondo Garanzia}

\subsection{EFFETTI DELL'INTERAZIONE SOCIALE}

\myparagraph{Sensibilità a interazione}

\myparagraph{Raggio dell'interazione}


\section[APPROSSIMAZIONE LINEARE]{APPROSSIMAZIONE LINEARE}

\subsection{REGRESSIONE LINEARE A TRATTI}


\nocite{*}
\bibliographystyle{plain}
\bibliography{bibliography}

\end{document}
