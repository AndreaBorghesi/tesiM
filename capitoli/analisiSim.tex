%Andrea Borghesi
%Università degli studi di Bologna

%capitolo dedicato all'analisi dei risultati delle simulazioni 

\documentclass[12pt,a4paper,openright,twoside]{report}
\usepackage[italian]{babel}
\usepackage{indentfirst}
\usepackage[utf8]{inputenc}
\usepackage[T1]{fontenc}
\usepackage{fancyhdr}
\usepackage{graphicx}
\usepackage{titlesec,blindtext, color}
\usepackage[font={small,it}]{caption}
\usepackage{subfig}
\usepackage{listings}
\usepackage{color}
\usepackage{url}
\usepackage{textcomp}

%impostazioni generali per visualizzare codice
\definecolor{dkgreen}{rgb}{0,0.6,0}
\definecolor{gray}{rgb}{0.5,0.5,0.5}
\definecolor{mauve}{rgb}{0.58,0,0.82}
 
\lstset{ %
  basicstyle=\footnotesize,           % the size of the fonts that are used for the code
  backgroundcolor=\color{white},      % choose the background color. You must add \usepackage{color}
  showspaces=false,               % show spaces adding particular underscores
  showstringspaces=false,         % underline spaces within strings
  showtabs=false,                 % show tabs within strings adding particular underscores
  tabsize=2,                      % sets default tabsize to 2 spaces
  breaklines=true,                % sets automatic line breaking
  breakatwhitespace=false,        % sets if automatic breaks should only happen at whitespace
  title=\lstname,                   % show the filename of files included with \lstinputlisting;
                                  % also try caption instead of title
  keywordstyle=\color{blue},          % keyword style
  commentstyle=\color{dkgreen},       % comment style
  stringstyle=\color{mauve},         % string literal style
  escapeinside={\%*}{*)},            % if you want to add LaTeX within your code
  morekeywords={*,...},              % if you want to add more keywords to the set
  deletekeywords={...}              % if you want to delete keywords from the given language
}

%per avere un bordo intorno alle figure
\usepackage{float}
\floatstyle{boxed} 
\restylefloat{figure}

%per poter poi impedire che certe parole vadano a capo
\usepackage{hyphenat}
\usepackage{listings}

%ridefinisco font per fancyhdr, per ottenere un'intestazione pulita
\newcommand{\changefont}{ \fontsize{9}{11}\selectfont }
\fancyhf{}
\fancyhead[LE,RO]{\changefont \slshape \rightmark} 	%section
\fancyhead[RE,LO]{\changefont \slshape \leftmark}	%chapter
\fancyfoot[C]{\changefont \thepage}					%footer

%titolo capitolo con "numero | titolo"
\definecolor{gray75}{gray}{0.75}
\newcommand{\hsp}{\hspace{20pt}}
\titleformat{\chapter}[hang]{\Huge\bfseries}{\thechapter\hsp\textcolor{gray75}{|}\hsp}{0pt}{\Huge\bfseries}


\oddsidemargin=30pt \evensidemargin=20pt

%sillabazioni non eseguite correttamente
\hyphenation{sil-la-ba-zio-ne pa-ren-te-si si-mu-la-to-re ge-ne-ra-re pia-no}

%interlinea
\linespread{1.15}  
\pagestyle{fancy}

%cartelle contenenti le immagini
\graphicspath{{/media/sda4/tesi/immagini/grafici/}{/media/sda4/tesi/immagini/grafici/incCompare/}{/media/sda4/tesi/immagini/grafici/rawData/}{/media/sda4/tesi/immagini/grafici/regressionAnalysis/}{/media/sda4/tesi/immagini/schemi/}{/media/sda4/tesi/immagini/simulazione/}}

%in modo che dopo il titolo di un paragrafo il testo vada a capo
\newcommand{\myparagraph}[1]{\paragraph{#1}\mbox{}\\}

\begin{document}
\chapter{RISULTATI SIMULAZIONI}

Nel capitolo precedente abbiamo descritto il modello ad agenti implementato, evidenziandone le finalità e le caratteristiche fondamentali, accennando brevemente alle informazioni ricavabili dal simulatore.\\*
Lo scopo di questo capitolo sarà quindi la dettagliata analisi dei dati prodotti dalle simulazioni, l'individuazione delle relazioni che legano le grandezze in gioco all'interno dell'ambiente simulato, la presentazione e discussione dei risultati ottenuti.\\*
Inizialmente presenteremo gli strumenti utilizzati per effettuare l'analisi sopra descritta, per poi passare alla discussione vera e propria nei paragrafi successivi.

\section{STRUMENTI}
Per esaminare i dati prodotti dalle simulazioni effettuate e visualizzare i risultati ottenuti abbiamo utilizzato \emph{R}, un ambiente di sviluppo specifico per l'analisi statistica dei dati, basato sull'omonimo linguaggio di programmazione.

\subsection{R}

R è un linguaggio di programmazione open source e un ambiente software usato per la manipolazione di dati, calcolo e analisi statistica e presentazione grafica dei risultati. Il design R è stato ampiamente influenzato da due linguaggi preesistenti, S sviluppato da J.Chambers e colleghi presso i Bell Laboratories negli anni '70 e Scheme creato presso il MIT AI Lab sempre negli anni settanta da G.L.Steele e G.J.Sussman. \\*
Il nucleo di R consiste di un linguaggio interpretato a cui sono state aggiunte numerose funzionalità per un grande numero di procedure statistiche;  tra queste è possibile ricordarne alcune: modelli di regressione lineare, lineare generalizzata e non lineare, analisi di serie temporali, classici test parametrici e non, clustering, classificazione e altre. R è facilmente estendibile grazie alla presenza di numerosi pacchetti software creati dagli utenti e dedicati a specifiche aree di studio e possiede inoltre un grande insieme di funzioni indicate per una presentazione flessibile ed efficiente dei dati e la produzione di grafici di qualità.\\*
Per interagire con l'interprete del linguaggio R è possibile fornire le istruzioni direttamente da riga di comando oppure appoggiarsi a interfacce grafiche, ma per le nostre necessità è stato sufficiente utilizzare la riga di comando\\* \\* 
Per via della sua derivazione da S, R presenta alcune caratteristiche che lo fanno rientrare all'interno del paradigma dei linguaggi Object Oriented, almeno parzialmente, e al tempo stesso possiede alcuni aspetti che lo avvicinano alla natura dei linguaggi funzionali(come Scheme), come ad esempio la possibilità di trattare le funzioni stesse come oggetti.\\* Le principali strutture dati sono le seguenti: \begin{itemize}
\item \emph{vettori}, singole entità costituite da una collezione di valori di un certo tipo come ad esempio numerici,logici o caratteri;
\item \emph{matrici (arrays)}, generalizzazioni multi-dimensionali di vettori;
\item \emph{liste}, forme di vettori più generali nelle quali gli elementi non devono necessariamente essere dello stesso tipo;
\item \emph{fattori}, oggetti simili ai vettori usati per specificare una classificazione (raggruppamento) delle componenti di altri vettori con la stessa lunghezza;
\item \emph{data frames}, strutture simili alle matrici in cui le colonne possono essere di tipi diversi;
\item \emph{funzioni}, le quali sono esse stesse oggetti e forniscono così un modo semplice e flessibile di estendere R.
\end{itemize}
Come in ogni linguaggio di programmazione è poi ovviamente possibile manipolare queste strutture dati attraverso operatori, strutture di controllo, funzioni, etc...\\* \\*
Illustreremo ora un brevissimo esempio per far capire un possibile utilizzo di R per effettuare una semplice analisi statistica. Supponiamo di voler studiare la relazione che lega due variabili, \emph{a} e \emph{b}, i cui valori si trovano in un file di tipo \emph{Comma Separated Values}. Il primo passo è importare tali valori dal file e inserirli in una struttura dati, in questo caso una matrice con due colonne (una per ogni variabile) e ordinarli in base ai valori della prima variabile.

\lstset{language=R}
\begin{lstlisting}
> matrice.dati <- read.csv("file.csv")
> matrice.ordinata <- matrice_dati[order(matrice.dati$a),]
\end{lstlisting}

A questo punto sarebbe possibile svolgere diverse operazioni sui dati (ad esempio calcolare per ogni valore di ogni variabile i valori medi,...) ma ci limiteremo a effettuare una semplice regressione lineare.
\begin{lstlisting}
> modello.lineare <- lm(matrice$b ~ matrice$a)
\end{lstlisting}

R ci consente ora di effettuare analisi statistiche sul modello di regressione applicato per stabilirne validità e significatività in rapporto ai dati in nostro possesso e successivamente di presentare graficamente i risultati ottenuti.
\begin{lstlisting}
> #analisi statistica minima
> summary(modello.lineare)       
> #disegna i punti corrispondenti ai valori nella matrice
> plot(matrice$b ~ matrice$a,type="p",lwd=3,ylab="b",xlab="a")    
> #disegna la curva di regressione
> lines(matrice$a,predict(modello.lineare), lty="solid", col="darkred", lwd=2)    
\end{lstlisting}


In Figura ~\ref{example_r} sono stati riportati il grafico prodotto da questo esempio e i risultati ottenuti dalla semplicissima analisi statistica, tra i quali notiamo il coefficiente di determinazione (R-squared) e l'errore residuo ( 
Residual standard error).

\begin{figure}[H]
	\centering
	\subfloat[Grafico]{\includegraphics[scale=0.4]{example_r_graph}\label{example_r_graph}} 
	\quad
	\subfloat[Analisi statistica]{\includegraphics[scale=0.6]{example_r_result}\label{example_r_result}}
	\caption{Esempio di utilizzo di R}
	\label{example_r}
\end{figure}


\section{METODOLOGIA ANALITICA}

Illustreremo ora con quali metodi sono stati analizzati i dati ricavati dalle simulazioni; questo comporta anche una rapida descrizione delle tecniche statistiche impiegate e della loro applicazione nel nostro contesto.

\subsection{ANALISI DI REGRESSIONE}
Una delle tecniche statistiche maggiormente utilizzate per stimare le relazioni tra variabili è l'\emph{analisi della regressione}; in questa categoria rientrano diversi metodi che hanno come obiettivo quello di trovare un modello che leghi una variabile dipendente ed una o più variabili indipendenti (in particolare l'analisi della regressione consente di capire come cambia il valore di una variabile dipendente al variare del valore di una variabile indipendente, mantenendo fisse le restanti); seguendo la terminologia di uso comune in seguito le variabili indipendenti saranno chiamate anche \emph{predittori}.  
I modelli con i quali si tenta di approssimare le relazioni studiate possono essere di numerosi tipi, tra i quali è possibile ricordare quelli parametrici (come la regressione lineare e più in generale tutte le forme di regressione polinomiale), nei quali la funzione di regressione è definita attraverso un certo numero di parametri stimati a partire dai dati, e quelli non parametrici, poi ancora modelli locali (regressione LOESS), Bayesiani, segmentati, etc..\\*
In genere la scelta del giusto modello da applicare ai propri dati è un procedimento empirico che prevede di tentare differenti tecniche di regressione sulla stessa serie di dati per poi valutare quale fosse la scelta migliore, ovvero quale sia il modello di regressione che presenta la maggiore bontà di adattamento (in inglese ''goodness of fit''), cioè una misura che riassume la discrepanza tra i valori osservati e i valori attesi sotto il modello in questione. 
Una volta scelto il modello migliore, questo può essere usato per fare predizioni, comprendere in che modo in che modo certe variabili o aspetti di un problema ne influenzino altri, essere integrato all'interno di un sistema informatico attraverso tecniche di apprendimento automatico (come sarà mostrato nel prosieguo d questa trattazione). 
\\*\\*
Un aspetto di grande importanza è quindi la validazione del modello, valutare cioè se è in accordo con i dati presi in esame. Tra i diversi metodi di validazione possibili alcuni prevedono metodi numerici, come ad esempio il calcolo del coefficiente di determinazione, altri richiedono l'uso di tecniche più qualitative, ad esempio l'analisi grafica dei valori residuali; in genere per effettuare una validazione completa e affidabile vengono impiegate tecniche appartenenti ad entrambe le categorie ed anche in questo lavoro abbiamo agito in questo modo.\\*\\*
Uno dei principali indicatori numerici usati per valutare la bontà di un modello di regressione è il calcolo del \emph{coefficiente di determinazione}, o $R^2$, un numero reale compreso tra 0 e 1 che misura proporzione di variabilità della risposta dovuta al modello statistico; un valore vicino a 0 indica che la regressione scelta non si adatta ai dati, viceversa valori vicini a 1 indicano che il modello è buono.

\subsection{IMPLEMENTAZIONE}

Per effettuare l'analisi della regressione nel nostro caso, il primo passo è consistito nello scegliere le variabili di cui studiare la relazione; per ogni tipologia di incentivazione sono stati considerati tre casi (riportati in seguito in maniera estesa):
\begin{itemize}
\item relazione tra il budget che la regione dedica agli incentivi e produzione energetica da impianti fotovoltaici, con il budget variabile indipendente e la produzione energetica variabile dipendente;
\item relazione tra la sensibilità degli agenti simulati all'influenza dell'interazione sociale e la produzione energetica, essendo ancora ques'ultima la variabile dipendente;
\item relazione tra il raggio dell'interazione sociale, predittore, e produzione energetica da fotovoltaico.
\end{itemize}

\section[ANALISI RISULTATI]{ANALISI RISULTATI SIMULAZIONI}

Dopo aver implementato il simulatore descritto nel precedente capitolo, considerando in particolar modo la versione estesa, siamo passati ad analizzare le relazioni che legano la produzione di energia elettrica alle diverse metodologie di incentivi e relativi fondi stanziati dalla regione. Possiamo subito anticipare che, come era lecito attendersi, la presenza di un qualsiasi tipo di incentivo permette di ottenere una produzione energetica maggiore rispetto al caso di assenza di incentivi e inoltre all'aumentare dei fondi stanziati per finanziare un tipo di incentivo la produzione di energia da impianti fotovoltaici aumenta di conseguenza.\\*
In un secondo momento siamo passati a studiare la relazione tra produzione energetica e interazione sociale (considerando sia variazioni del raggio che della sensibilità); in modo conforme alle nostre aspettative, anche in questo caso i risultati ottenuti indicano che una maggiore produzione energetica è associata ad un'interazione sociale più intensa.\\* \\*
Per studiare le relazioni di nostro interesse la metodologia scelta consiste nell'aver effettuato un grande numero di simulazioni controllate (ovvero fissando tutti i parametri non rilevanti e variando quelli di cui osservare il comportamento, come il budget regionale o il tipo di incentivo), dopodiché abbiamo effettuato una semplice analisi statistica dei dati e tentato di risalire alle curve relative all'andamento delle relazioni attraverso l'uso di tecniche di regressione lineare e non.


\subsection{COMPORTAMENTO DEGLI INCENTIVI}

Il comportamento degli incentivi è stato studiato effettuando numerose simulazioni per ogni tipo di incentivo, variando la dimensione del fondo dedicato agli incentivi per il fotovoltaico. Il fondo è stato aumentato con incrementi di un milione di euro a partire da zero  fino a un massimo di 40 milioni ( considerando un arco temporale di cinque anni ); per ogni valore sono state effettuate 300 simulazioni, per un totale di 48000 simulazioni considerando tutti i diversi incentivi.\\*
Proseguiremo ora esaminando i singoli incentivi per poi concludere confrontandoli tra loro.

\myparagraph{Fondo Asta}

\begin{figure}[H]
	\centering
	\subfloat[Simulazioni]{\includegraphics[scale=0.55]{graphSimA_R}\label{graphSimA_R}}
	\qquad
	\subfloat[Relazione]{\includegraphics[scale=0.55]{regr_graphSimA_R}\label{regr_graphSimA_R}}
	\caption{Fondo Asta}
	\label{graphSimA}
\end{figure}


\myparagraph{Conto Interessi}

\myparagraph{Fondo Rotazione}

\myparagraph{Fondo Garanzia}

\myparagraph{Confronto Incentivi}

\subsection{EFFETTI DELL'INTERAZIONE SOCIALE}

Gli effetti dell'iterazione sociale sulla produzione energetica sono stati studiati agendo sui due parametri che possono influenzarla, il raggio dell'interazione e la sensibilità all'influenza derivante dal comportamento dei vicini, ed effettuando numerose simulazioni controllate: per ricavare la relazione tra produzione energetica e raggio questo è stato fatto variare da 1 fino a 40 ( valori espressi con un'unità di misura interna al simulatore ), con incrementi di una un'unità e 200 prove per valore, per un totale di 32000 simulazioni; per la relazione tra produzione e sensibilità questa è stata fatta crescere da 0 fino a 20 a intervalli di 0.5, ancora con un totale di 32000 simulazioni.

\myparagraph{Sensibilità a interazione}

\myparagraph{Raggio dell'interazione}


\section[APPROSSIMAZIONE LINEARE]{APPROSSIMAZIONE LINEARE}

\subsection{REGRESSIONE LINEARE A TRATTI}


\nocite{*}
\bibliographystyle{plain}
\bibliography{bibliography}

\end{document}
